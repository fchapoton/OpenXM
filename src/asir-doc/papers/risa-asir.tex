\documentclass[12pt]{article}
\textheight 9.5in
\textwidth 6.5in
\topmargin -0.5in
\oddsidemargin -0.2in
\evensidemargin -0.2in
\usepackage{makeidx}  % allows index generation
\usepackage{graphicx} % standard LaTeX graphics tool
                      % for including eps-figure files
%\usepackage{subeqnar} % subnumbers individual equations
                      % within an array
%\usepackage{multicol} % used for the two-column index
%\usepackage{cropmark} % cropmarks for pages without
                      % pagenumbers
%\usepackage{math}     % placeholder for figures
\makeindex            % used for the subject index
                      % please use the style sprmidx.sty with
                      % your makeindex program

%upright Greek letters (example below: upright "mu")
\newcommand{\euler}[1]{{\usefont{U}{eur}{m}{n}#1}}
\newcommand{\eulerbold}[1]{{\usefont{U}{eur}{b}{n}#1}}
\newcommand{\umu}{\mbox{\euler{\char22}}}
\newcommand{\umub}{\mbox{\eulerbold{\char22}}}

%%%%%%%%%%%%%%%%%%%%%%%%%%%%%%%%%%%%%%%%%%%%%%%%%%%%%%%%%%%%%

%\twocolumn
%OPTIONAL%%%%%%%%%%%%%%%%%%%%%%%%%%%%%%%%%%%%%%%%%%%%%%%%%%%%
%
%\usepackage{amstex}   % useful for coding complex math
%\mathindent\parindent % needed in case "Amstex" is used
%
%%%%%%%%%%%%%%%%%%%%%%%%%%%%%%%%%%%%%%%%%%%%%%%%%%%%%%%%%%%%%

%AUTHOR_STYLES_AND_DEFINITIONS%%%%%%%%%%%%%%%%%%%%%%%%%%%%%%%
%
%Please reduce your own definitions and macros to an absolute
%minimum since otherwise the editor will find it rather
%strenuous to compile all individual contributions to a
%single book file
\usepackage{epsfig}
\def\cont{{\rm cont}}
\def\GCD{{\rm GCD}}
\def\Q{{\bf Q}}
%
%%%%%%%%%%%%%%%%%%%%%%%%%%%%%%%%%%%%%%%%%%%%%%%%%%%%%%%%%%%%%

\begin{document}
%
%\title*{A Computer Algebra System : Risa/Asir}
%
%
\title{A Computer Algebra System : Risa/Asir \footnote{This article is
a summary of two papers\cite{NORO}\cite{MAEKAWA}.}}
% allows explicit linebreak for the table of content
%
%
%\titlerunning{Risa/Asir}
% allows abbreviation of title, if the full title is too long
% to fit in the running head
%
\author{Masayuki Noro}
\date{}
%Kobe University, Rokko, Kobe 657-8501, Japan\\
%E-mail:noro@math.kobe-u.ac.jp}
%
%\authorrunning{Masayuki Noro}
% if there are more than two authors,
% please abbreviate author list for running head
%
%
%\institute{Kobe University, Rokko, Kobe 657-8501, Japan\\
%E-mail:noro@math.kobe-u.ac.jp}

\maketitle              % typesets the title of the contribution


\section{An overview of Risa/Asir}

Risa/Asir is a tool for performing various computations in
mathematics and engineering applications. 
The development of Risa/Asir started in 1989 at FUJITSU
and now the source code is also available at no cost.
Currently, Kobe distribution
is the most active branch in the development of Risa/Asir. 
Risa/Asir is characterized as follows:

\begin{enumerate}
\item An environment for large scale and efficient polynomial computation.

Risa/Asir consists of the Risa engine for performing operations on
mathematical objects and an interpreter for programs written in
the Asir user language. In Risa/Asir, polynomials are
represented in two different internal forms: 
the recursive representation and the distributed representation.
Polynomial factorization and GCD are based on the former representation,
and computations related to the Gr\"obner basis are based on the latter
representation.
%Ground fields of polynomial rings can be composed of the field of rationals,
%algebraic number fields and finite fields.
Ground fields of polynomial rings can be composed of 
The field of rationals, algebraic number fields and finite fields
are available as ground fields of polynomial rings.

\item A platform for parallel and distributed computation.

In order to  combine mathematical software systems,
we previously proposed the OpenXM 
(Open Message eXchange for Mathematics) 
protocol.
Risa/Asir acts as both an OpenXM client and an OpenXM server.
The Risa/Asir OpenXM client API provides a way to call 
functions in external OpenXM servers.
Using multiple Risa/Asir OpenXM servers,
one can perform parallel and distributed computation 
in an attempt to achieve linear speedup.
Conversely, other OpenXM clients can call functions in the Risa/Asir 
server. 

\item Open source software.

The source code of Risa/Asir is completely open, and algorithms and
implementations can be verified if necessary. The addition of new
codes is simple. 
\end{enumerate}

Risa/Asir runs on various platforms such as Linux, FreeBSD, Sun and
Windows. The whole source code is available from
{\tt http://www.math.kobe-u.ac.jp/OpenXM}.
The on-line help is available from the Risa/Asir command line or
from the menu bar on Windows. Manuals are located at 
{\tt http://www.math.kobe-u.ac.jp/OpenXM/Current/doc/index-doc.html}.

\section{Functionality of Risa/Asir}

Among various functions on algebraic computation,
we explain the implementations of two main functions: polynomial
factorization and Gr\"obner basis computation over various ground
fields.

\subsection{Operations over fields of characteristic 0}

Polynomial arithmetics, polynomial GCD and factorization are
implemented over our own implementation of integer and rational number
arithmetics.  In multivariate factorization, we implemented a
simplified version of Wang method to estimate the leading coefficient
of a factor to avoid the leading coefficient problem.

Risa/Asir provides a univariate factorization over algebraic number
fields represented by a successive extension.  As an application, the
splitting field computation of a univariate polynomial is implemented.
The implemented algorithm is an improvement of the algorithm proposed
by Trager. In the present algorithm, we utilize non square-free norms
to obtain partial (not necessarily irreducible) factorization. The
algorithm finally requires univariate factorization of a polynomial
over the rationals, however this factorization is often difficult
because the polynomial contains many spurious factors.  At present,
Risa/Asir is not optimized to factor polynomials of this type.  In
this case, a factorization in the PARI library implementing the
knapsack factorization algorithm can be used, and the splitting field
can be computed efficiently even in such cases
\footnote{A special configure option is required for linking the new version
of PARI library.}.

\subsection{Computation over finite fields}

In Risa/Asir finite fields are represented in several ways.
\begin{enumerate}
\item $GF(p)$ ($p$: a prime $p < 2^{29})$
\item $GF(p)$ ($p$: an arbitrary prime)
\item $GF(2^n)$ ($n$: small)
\item $GF(p^n)$ ($p$: an arbitrary prime, $n$: small integer)
\item $GF(p^s)$ ($p$: a small prime, $p^s < 2^{16})$
\end{enumerate}
The type 1 is used internally in various modular algorithms.  The type
2, 3 and 4 are general purpose representations.  Multiplication in
$GF(2^n)$ is implemented using Karatsuba algorithm extensively.  The
type 5 has been introduced for efficient implementation of
multivariate factorization over finite fields of small characteristic.
When we attempt to factor a polynomial over such a field, we often
have insufficient evaluation points.  In this case we have to extend
the ground field.  If the ground field and its extension are both
represented by a type 5 representation, we can expect that field
arithmetics are performed efficiently in both fields.  We presented a
new polynomial time algorithm to factor bivariate polynomials over a
finite field is presented. A combination of the Berlekamp-Hensel
method and the new algorithm under a type 5 representation enables us
to efficiently factor a certain class of polynomials, including
hard-to-factor polynomials.

\subsection{Gr\"obner basis computation}

We have incorporated various algorithms and improvements for Gr\"obner
basis computation. For the Buchberger algorithm, we have implemented
well-known criteria to detect unnecessary pairs, the sugar strategy,
trace algorithms by modular computation, stabilization of strategy by
combining homogenization and the trace algorithm, and efficient
content reduction during normal form computation.  We also have an
experimental implementation of $F_4$ algorithm.  Furthermore, we
implemented several types of change of ordering algorithms.  Among
these, {\tt tolex()} implements a modular FGLM algorithm, which avoids
intermediate coefficient swells during the ordinary FGLM algorithm and
realizes efficient computation of lexicographic Gr\"obner bases for
zero-dimensional ideals.

By supplying an option {\tt Demand} with a directory name, generated
basis elements are placed in the directory. Although this requires
additional cost in order to read the basis elements required for
normal form computations, the total amount of memory is smaller than
the case in which all basis elements are placed in memory, which may
enable a very large computation, generating numerous intermediate
basis elements.

As an application of Gr\"obner basis computation and polynomial
factorization, we implemented primary ideal decomposition and prime
decomposition of the radical of an ideal.  These functions can be used
to decompose solutions of systems of algebraic equations into
irreducible components. 

We implemented fundamental arithmetics, the Buchberger algorithm and
the minimal polynomial computation over Weyl algebra, the ring of
differential operators having polynomial coefficients.  Using these
arithmetics we implemented an efficient algorithm for computing the
$b$-function of a polynomial.

\subsection{Parallel and distributed computation via OpenXM}

Risa/Asir provides OpenXM API for executing funtions on other
mathematical software. In OpenXM, mathematical software systems are
wrapped as server stack machines. As an OpenXM client, Risa/Asir
dispatches a request to a server and receives the result. Inputs and
outputs are represented as CMO (Common Mathematical Object format)
objects.  The set of stack machine commands also contains various
control operations, such as server invocation, termination,
interruption and restarting.  Usually, each mathematical software
system has its own user language. OpenXM provides stack machine
commands for executing a command string and receiving a result as a
human readable character string, thus wrapping a mathematical software
system as an OpenXM server is relatively easy.  OpenXM protocol is
completely open, and any program can implement the protocol.  OpenXM
specifies a procedure for robust interruption and restarting of
execution, which enables clients to be reset safely from any state.

Communication between an OpenXM server and a client can be realized by various
methods: files, TCP/IP, MPI, PVM, RPC and linking a subroutine library.
The Risa/Asir subroutine library {\tt libasir.a} contains functions
simulating the stack machine commands supported in {\tt ox\_asir},
the OpenXM asir server.

\subsection{Integration of Risa/Asir and other OpenXM servers}

Asir OpenXM contrib (asir-contrib) is a collection of wrappers of
functions in external OpenXM servers. Using asir-contrib,
an external function can be called 
from Asir without knowing that the function is located in
an external server. Currently, the following functions (including
Asir built-in functions) are provided
in OpenXM:

\begin{itemize}
\item Operations on Integers

{\tt idiv},{\tt irem} (division with remainder),
{\tt ishift} (bit shifting),
{\tt iand},{\tt ior},{\tt ixor} (logical operations),
{\tt igcd},(GCD by various methods such as Euclid's algorithm and
the accelerated GCD algorithm),
{\tt fac} (factorial),
{\tt inv} (inverse modulo an integer),
{\tt random} (random number generator by the Mersenne twister algorithm).

\item Ground Fields

Arithmetics on various fields: the rationals,
${\bf Q}(\alpha_1,\alpha_2,\ldots,\alpha_n)$
($\alpha_i$ is algebraic over ${\bf Q}(\alpha_1,\ldots,\alpha_{\tt i-1})$),
$GF(p)$ ($p$ is a prime of arbitrary size), $GF(2^n)$.

\item Operations on Polynomials

{\tt sdiv }, {\tt srem } (division with remainder),
{\tt ptozp } (removal of the integer content),
{\tt diff } (differentiation),
{\tt gcd } (GCD over the rationals),
{\tt res } (resultant),
{\tt subst } (substitution),
{\tt umul} (fast multiplication of dense univariate polynomials
by a hybrid method with Karatsuba and FFT+Chinese remainder),
{\tt urembymul\_precomp} (fast dense univariate polynomial
division with remainder by the fast multiplication and
the precomputed inverse of a divisor),

\item Polynomial Factorization
{\tt fctr } (factorization over the rationals),
{\tt modfctr}, {\tt fctr\_ff } (univariate factorization over finite fields),
{\tt af } (univariate factorization over algebraic number fields),
{\tt sp} (splitting field computation).

\item Groebner basis

{\tt dp\_gr\_main }, {\tt nd\_gr\_trace} (Groebner basis computation of a polynomial ideal
over the rationals by the trace lifting),
{\tt dp\_gr\_mod\_main }, {\tt nd\_gr} (Groebner basis over small finite fields),
{\tt tolex } (Modular change of ordering for a zero-dimensional ideal),
{\tt tolex\_gsl } (Modular rational univariate representation
for a zero-dimensional ideal),
{\tt dp\_f4\_main}, {\tt dp\_f4\_mod\_main}, {\tt nd\_f4} 
($F_4$ over the rationals and small finite fields).

\item Ideal Decomposition

{\tt primedec}, {\tt primedec\_mod} (Prime decomposition of the radical),
{\tt primadec} (Primary decomposition of ideals by Shimoyama/Yokoyama algorithm).

\item Quantifier Elimination

{\tt qe} (real quantifier elimination in a linear and
quadratic first-order formula),
{\tt simpl} (heuristic simplification of a first-order formula).

\item Visualization of curves

{\tt plot} (plotting of a univariate function),
{\tt ifplot} (plotting zeros of a bivariate polynomial),
{\tt conplot} (contour plotting of a bivariate polynomial function).

\item Miscellaneous functions

{\tt det} (determinant),
{\tt qsort} (sorting of an array by the quick sort algorithm),
{\tt eval}, {\tt deval} (evaluation of a formula containing transcendental functions
such as
{\tt sin}, {\tt cos}, {\tt tan}, {\tt exp},
{\tt log})
{\tt pari(roots)} (finding all roots of a univariate polynomial),
{\tt pari(lll)} (computation of an LLL-reduced basis of a lattice).

\item $D$-modules ($D$ is the Weyl algebra)

{\tt sm1.gb } (Gr\"obner basis),
{\tt sm1.syz} (syzygy),
{\tt sm1.bfunction},{\tt bfunction} (the global $b$-function of a polynomial)
{\tt sm1.restriction} in the derived category of $D$-modules,
{\tt sm1.slope},
{\tt sm1.sm1(annfs)} (Annihilating ideal of $f^s$),
{\tt sm1.sm1(schreyer)} (free resolution by the Schreyer method),
%{\tt sm1.sm1(vMinRes)} (V-minimal free resolution),\\
{\tt sm1.sm1(characteristic)} (Characteristic variety),
{\tt sm1.sm1(integration)} in the derived category,
%{\tt sm1.sm1(tensor)}  in the derived category,
{\tt sm1.sm1(res-dual)} (Dual as a D-module).

\item Cohomology groups

{\tt deRham} (The de Rham cohomology groups of
${\bf C}^n \setminus V(f)$,
{\tt ext} (Ext modules for a holonomic $D$-module $M$
and the ring of formal power series).

\item Differential equations

Helping to derive and prove {\tt combinatorial} and
{special function identities},
{\tt sm1.gkz} (GKZ hypergeometric differential equations),
{\tt sm1.appell1}, {\tt sm1.appell4} (Appell's hypergeometric differential equations),
%{\tt indicial} (indicial equations),
{\tt sm1.generalized\_bfunction} (indicial equations),
{\tt sm1.rank} (Holonomic rank),
{\tt sm1.rrank} (Holonomic rank of regular holonomic systems),
%{\tt dsolv} (series solutions of holonomic systems).
{\tt dsolv\_dual}, {\tt dsolv\_starting\_terms} (series solutions of holonomic systems).

\item OpenMATH support

{\tt om\_xml} (CMO to OpenMATH XML),
{\tt om\_xml\_to\_cmo} (OpenMATH XML to CMO).

\item Homotopy Method

{\tt phc.phc} (Solving systems of algebraic equations by
numerical and polyhedral homotopy methods).

\item Toric ideal

{\tt tigers.tigers} (Enumerate all Gr\"obner basis of a toric ideal.
Finding test sets for integer program),
%{\tt arithDeg} (Arithmetic degree of a monomial ideal),
%{\tt stdPair} (Standard pair decomposition of a monomial ideal).

\item Communications

{\tt ox\_launch} (starting a server),
{\tt ox\_launch\_nox},
{\tt ox\_shutdown},
{\tt ox\_launch\_generic},\\
{\tt generate\_port},
{\tt try\_bind\_listen},
{\tt try\_connect},
{\tt try\_accept},
{\tt register\_server},\\
{\tt ox\_rpc},
{\tt ox\_cmo\_rpc},
{\tt ox\_execute\_string},
{\tt ox\_reset} (reset the server),
{\tt ox\_intr},\\
{\tt register\_handler},
{\tt ox\_push\_cmo},
{\tt ox\_push\_local},
{\tt ox\_pop\_cmo},
{\tt ox\_pop\_local},\\
{\tt ox\_push\_cmd},
{\tt ox\_sync},
{\tt ox\_get},
{\tt ox\_pops},
{\tt ox\_select},
{\tt ox\_flush},
{\tt ox\_get\_serverinfo}

\end{itemize}

\begin{thebibliography}{99}
\bibitem{NORO}
M. Noro,  A Computer Algebra System Risa/Asir.  Algebra, Geometry and Software, M. Joswig and N. Takayama (eds.), Springer, 147-162 (2002).
\bibitem{MAEKAWA}
M. Maekawa, M. Noro, N. Takayama and Y. Tamura
The Design and Implementation of OpenXM-RFC 100 and 101.
Computer Mathematics (Proc. ASCM2001), World Scientific, 102-111 (2001).
\end{thebibliography}

\end{document}
