\documentstyle[12pt,psfig,amssymb,isisrr]{article}
\newcommand{\msub}{/\kern-3pt/}
\rrno{5E}
\isismemo
\Email{sturm}
\abstract{Weispfenning has shown how to use test term methods for quantifier
elimination in linear and quadratic first-order formulas over real
closed fields. This paper describes the state of the implementation of
such methods in the computer algebra system Risa/Asir. The package
described here is entirely written in the C programming language. We
point on possible extensions of the package and give examples for
automatic quantifier eliminations performed by Risa/Asir.}
\keywords{real quantifier elimination, parametric constraint solving,
implementation, Asir package}
\begin{document}
\title{Real Quadratic Quantifier Elimination\\ in Risa/Asir}
\author{Thomas Sturm\thanks{Visiting researcher. Original affiliation: 
Fakult\"at f\"ur Mathematik und Informatik, Universit\"at Passau,
D-94030 Passau, Germany. Email: sturm@fmi.uni-passau.de.}}
\maketitle
%
\section{Introduction}
A {\em quantifier elimination procedure} is an algorithm that, given a
first order formula, computes an equivalent quantifier-free formula. A
quantifier elimination procedure for the reals has been given already
by Tarski~\cite{Tarski:48}. Due to the enormous power of such
algorithms efforts to implement them have been made from the very
beginning on. In fact the US RAND Corporation had soon tried to
implement the original Tarski procedure, which certainly failed at
those days.

The first complete implementation by Arnon was finished in 1981
\cite{Arnon:81}. It used the cylindrical algebraic decomposition
method by Collins~\cite{Collins:75}. This implementation of quantifier
elimination triggered the development of one of the first computer
algebra systems: SAC, now SAC-II/ALDES. Important improvements of the
CAD method have been made by Collins and Hong~\cite{CollinsHong:91}
resulting in the {\em partial} CAD implemented in Hong's QEPCAD
package~\cite{Hong:93}.

Weispfenning introduced an alternative approach to quantifier
elimination first for linear formulas \cite{Weispfenning:88}. This
could be extended to arbitrary degrees
\cite{LoosWeispfenning:93,Weispfenning:96,Weispfenning:94}. For
degrees $1$ and $2$, the methods have been successfully implemented in
the REDUCE package REDLOG written by the author and others
\cite{DolzmannSturm:95a,DolzmannSturm:96}. Despite the bad theoretical
complexity~\cite{Weispfenning:88,Davenport:88} and the degree
restrictions REDLOG turned out suitable for solving a number of
practical problems \cite{Weispfenning:95b,Weispfenning:96b}. Meanwhile
it is used commercially as part of an error diagnosis system for
physical networks.

We have now started a second implementation of the test term method in
the C language using the Risa/Asir computer algebra system
\cite{NoroTakeshima:92}. The aim of this reimplementation is twofold.
On one hand, we may expect to achieve a gain in efficiency due to both
the C language and the efficient polynomial algorithms present in
Risa. On the other hand we wish to make the methods available to
interested parties in industry where C is still the most widespread
programming language.

The purpose of this report is to summarize the state of the
reimplementation after an initial phase of eight weeks. We further
point on features to be added. All these features are already part of
the REDLOG package.
%
\section{Description of the implementation}
In the sequel, we assume the reader to be familiar with the theory of
quantifier elimination by test term methods as it has been described
by Weispfenning \cite{LoosWeispfenning:93,Weispfenning:96}.
Figure~\ref{modstruct} shows the module structure of the packege.
\begin{figure}
\centerline{\psfig{figure=qe2.eps,width=0.8\textwidth}}
\caption{Visualisation of the module interaction\label{modstruct}}
\end{figure}
\subsection{Functionality}
The current implementation can eliminate existential and universal
quantifiers from prenex first-order formulas, subject to the
restriction that with the elimination of each variable the latter
occurs at most quadratically within the quantified {\em matrix}
formula. This includes the technique described in
\cite{Weispfenning:96} where in a situation as
$$
\exists x\bigl(a_2x^2+a_1x+a_0=0\land\varphi\bigr)
$$
at least one of $a_2$, $a_1$, $a_0$ is a non-zero number. The
quantified variable $x$ may then occur with arbitrary degree in
$\varphi$. We refer to this special case as {\em Gauss elimination}
treated in Subsection~\ref{gaussnow}.

Furthermore, there are some minor tool functions operating on formulas
available. All functions are listed and described briefly in the
Appendix~\ref{doku}.

\subsection{Toplevel elimination}
The quantifier elimination routine is called with a prenex first-order
formula. This formula is simplified, and then turned into {\em
negation normal form}, i.e., a prenex formula the matrix of which
contains only $\land$ and $\lor$ as first-order operators. Next the
formula is split into a list of quantifier block and its matrix.

The elimination of universal quantifiers blocks is reduced to that of
existential ones using the equivalence $\forall
x\varphi\longleftrightarrow \lnot\exists x\lnot\varphi$. We shall
restrict our attention to existential quantifiers from now on.

For each block, beginning with the innermost, there is a routine
called that is designed to eliminate one prenex existential quantifier
block.
%
\subsection{Quantifier blocks}
A block of quantifiers $\exists x_n\ldots\exists x_1$ is eliminated
successesively beginning with the innermost quantifier. Let $E_1$ be
an elimination set for the innermost quantifier $\exists x_1$ we
obtain a formula
$$
\exists x_n\ldots\exists x_2\bigvee_{t\in E_1}\varphi[x\msub t].
$$
The remaining quantifiers can be moved inside the disjunction. The
branches obtained are eliminated independently decreasing the overall
complexity from doubly to singly exponential.

This technique is implemented by processing a {\em queue} of {\em
subproblems}. A subproblem is a pair $(V_{ij},\psi_{ij})$ consisting
of an (existentially quantified) variable list $V_{ij}$ and a matrix
formula $\psi_{ij}$. With the substitution of an elimination set for a
certain variable we do not construct a disjuntion but instead add the
results still containing quantified variables as new subproblems to a
problem queue called {\em container}. The completely eliminated
subproblems ($V_{nj}=\emptyset$) are, in contrast, added to a {\em
result box}.

The subproblems obtained this way form a tree by branching from each
subproblem to its child problems until its leafs eventually enter the
result box. The depth of this {\em elimination tree} is the number of
variables in the initial $V_{11}$. Its width at each level is the sum
of the sizes of the elimination sets obtained from the nodes of the
parent level.

Choosing a queue for storing subproblems amounts to computing the
elimination tree in a {\em breadth first search} (BFS) manner. The
advantage of BFS is that equal sibling nodes can be detected when
adding to the container. This prevents the recomputation of identical
subtrees.

The procedure terminates when the container gets empty. This happens
after finitely many steps according to K\"onig's Lemma. Finally a
disjunction is constructed from the content of the result box.

A subproblem $(V_{ij},\psi_{ij})$ is processed by first trying Gauss
elimination for $V_{ij}$ and $\psi_{ij}$. If this succeeds, an
elimination set is obtained from the Gauss elimination routine. Else a
translation is computed for the first (innermost) variable of $V_{ij}$
and the matrix $\psi_{ij}$. This translation is turned into an
elimination set by yet another routine. Elimination sets are
substituted elementwise. Each substitution result is simplified and
becomes a new subproblem with one variable less.

Notice that our container technique allows to eliminate the variables
in a different order for each branch of the tree. The Gauss
elimination already makes use of this feature.
%
\subsection{Gauss elimination}\label{gaussnow}
The Gauss elimination routine first checks alle variables trying to
find a possibility for a linear Gauss step, i.e. one in which the
coefficient $a_2$ of $x_2$ is zero. Such an elimination does not
increase the degree of the other variables. If this fails, it checks
for a quadratic Gauss. If the Gauss routine detects a Gauss situation,
it returns an elimination set to be substituted by the block
elimination routine. Else it signals its failure.
%
\subsection{Translation stage}
The test term technique ist based on testing endpoints of intervals
including $\pm\infty$ and possibly adding or subtracting some
infinitesimal $\epsilon$. In the translation phase all endpoint
candidates are collected and classified wrt.~their {\em bounding type}
and their {\em infinity type}.

The bounding type encodes the relevant information on the constraint
that has delivered the point. Table~\ref{bttab} collects the bounding
types currently used.
%
\begin{table}[t]
\begin{center}
\begin{tabular}{ll}
\hline
BTEQUAL & Equation: $p$\\
BTNEQ & Inequality: ${\Bbb R}\setminus\{p\}$\\
BTLEQ & Weak upper bound: $]-\infty,p]$\\
BTGEQ & Weak lower bound: $[p,\infty[$\\
BTLESSP & Strict upper bound: $]-\infty,p[$\\
BTGREATERP & Strict lower bound: $]p,\infty[$\\
BTWO & Weak ordering: Weak lower or upper bound\\
BTSO & Strict ordering: Strict lower or upper bound\\
\hline
\end{tabular}
\end{center}
\caption{Interpretation of an endpoint $p$ according to its bounding
type.\label{bttab}}
\end{table}
%
By the way of example, consider $4x-7\geq0$, which contributes $7/4$
as a lower bound on $x$, i.e., BTGEQ. In contrast, $ax-7\geq0$ does
not contribute a lower bound; for $a<0$ it is an upper bound. Thus the
bounding type for $7/a$ in the latter example is BTWO.

Table~\ref{ittab} collects the infinity types and their relation to
the bounding types.
%
\begin{table}[t]
\begin{center}
\begin{tabular}{lll}
\hline
infinity type & occurs with & interpretation\\
\hline
STD & BTEQUAL & $p$\\
EPS & BTSO, BTNEQ & $p$ plus {\em or} minus $\epsilon$\\
MEPS & BTLESSP & $p-\epsilon$\\
PEPS & BTGREATERP & $p+\epsilon$\\
PINF & --- & $\infty$\\
MINF & --- & $-\infty$\\
\hline
\end{tabular}
\end{center}
\caption{Interpretation of an endpoint $p$ according to its infinity
type. Infinity types are related to bounding types.\label{ittab}}
\end{table}
%
Points of infinity type $\epsilon$ are {\em unspecified}, all others
are {\em specified}. The types PINF and MINF are not used at the
translation stage.

Taking a point labeled with its infinity type and adding conditions
for its existence and its relevance yields a {\em guarded point}. 
Solutions $-c/b$ of linear contraints are guarded by $b\neq 0$. 
Solutions $(-b\pm\sqrt{b^2-4ac})/2a$ of quadratic contraints are
guarded by $a\neq0\land b^2-4ac\geq0$.

Finally, for a fixed variable a {\em translation} is an array, indexed
by bounding types, of sets of guarded points obtained from the
constraints in the matrix formula.

Actually, the endpoints are not stored as explicit terms but in the
form of minimal polynomials. Consequently, one point can stand for
several roots in case of quadratic constraint. Notice that all the
roots are assigned both the same infinity type and bounding type.

On the other hand, one constraint can in fact produce several guarded
points. For instance, consider the contraint $ax^2+5x+c\geq0$. This
yields a two-root guarded point
$$
\bigl(a\neq0\land 25-4ac\geq0,(ax^2+5x+c,\mbox{STD})\bigr)
$$
of bounding type BTWO. On the other hand, for $a=0$ it is a hidden
linear constraint. Hence we additionally obatin
$(5\neq0,(5x+c,\mbox{STD}))$ of bounding type BTGEQ.

The translation array is the input to the elimination set computation
routine.
%
\subsection{Elimination set computation}
An elimination set is a single set of guarded points non of which is
unspecified wrt.~its infinity type. Given a translation there are thus
three tasks to be performed.
\begin{enumerate}
\item Choose a small selection of guarded points for obtaining an
elimination set.
\item Specify each unspecified guarded point (of infinity type EPS) in
this selection to either PEPS or MEPS.
\item Add elements that are not induced by isolated contraints to the
elimination set.
\end{enumerate}
%
The current implementation decides between either taking all upper
bounds or all lower bounds depending on the number of such bounds
present. Accordingly, EPS is specified to either MEPS or PEPS, and
either $(0=0,(0,\mbox{PINF}))$ or $(0=0,(0,\mbox{MINF}))$ is added. If
there are no orderings at all, $(0=0,(x,\mbox{STD}))$, i.e., ``$0$''
is added instead of an infinite value. Mind that at least one point
has to be substituted for the case that none of the guards holds.

Summarizing, the bounding type provides information concerning the
selection of a proper subset of the translation. The infinity type
provides information concerning the substitution. The infinity type
cannot always be completely determined at the translation stage.
%
\subsection{Substitution}
In an object oriented style, there is one procedure that can
substitute a guarded point into a formula. The sustitution module
includes substitution of infinite and infinitesimal symbols and
substitution of root expressions within the language of ordered rings. 
The methods used are that described in \cite{Weispfenning:96}. We make
use of our minimal polynomial representation for the roots by
performing pseudo divison before symbolically substituting.
%
\subsection{Simplification}\label{currentsimpl}
Currently, the following simplification strategies are applied:
\begin{itemize}
\item
Evaluation of variable-free atomic formulas.
\item
Replacing equation and inequality right hand sides by their squarefree
part. Corresponding treatment of ordering atomic formulas: Here
factors of even multiplicity have to enter squared.
\item
Removing \verb!true! from conjunctions, \verb!false! from
disjunctions. Conjunctions containing \verb!false! and disjunctions
containing \verb!true! are replaced by the respective truth value.
Truth values are treated appropriately also with all other first-order
operators. 
\item
Removing equal subformulas from conjunctions and disjunctions. Atomic
subformulas are ordered canonically and placed before their complex
siblings. The order of atomic formulas is first wrt.~their left hand
side polynomial, and then wrt.~the order of relations as in
Table~\ref{atoptab}.
\end{itemize}
%
\section{Possible Extensions}\label{genopt}
We point on possible extension of the Asir quantifier elimination
code. All options mentioned here are already part of the REDLOG
package~\cite{DolzmannSturm:95a,DolzmannSturm:96}. They have been
extensively tested for their relevance.
\subsection{Functionality}
For many practical applications, it is necessary to add an option for
{\em extended quantifier elimination:} This variant keeps track of the
test points substituted. With the elimination of an existential
quantifier block, it provides instead of a disjunction
$\bigvee_{i=1}^n \psi_i$ a set of {\em guarded points}
$$
\bigl\{(\psi_1,S_1),\dots,(\psi_n,S_n)\bigr\}
$$
such that whenever an interpretation of the parameters satisfies some
$\psi_i$, the original $\exists$-quantified formula holds for this
interpretation, and $S_i$ provides some sample values for the
quantified variables possibly containing $\pm\epsilon$ or $\pm\infty$.

The extension of the package to {\em generic quantifier
elimination}~\cite{DSW:96} is of the same importance. There, the
quantifier elimination procedure is allowed to make certain assumption
of the form $t\neq0$ for parameter term $t$. These assumptions support
the elimination process leading to a much smaller result, which is
correct under~the assumptions. Of course, the list of assumptions made
is also returned to the caller. It has turned out that for the
majority of practical applications, the assumptions made are
``harmless''. They describe some degenerate cases, which are actually
not relevant.

Finally, the package should not be restricted to quantifier
elimination itself but also include other useful algorithms on
formulas such as CNF/DNF computation or advanced simplifiers for final
results.
%
\subsection{Toplevel elimination}
There should be code added for making formulas prenex such that the
quantifier elimination can be called with arbitrary formulas.
%
\subsection{Quantifier blocks}\label{blockopt}
There should be an option added to use a stack instead of a queue as
container. This amounts to a DFS computation of the elimination tree.
The advantage is that with (wlog.~existential) decision problems, one
has a good chance to detect a ``true'' leaf early. Computation can be
aborted then. For the elimination of several blocks in a decision
problem it is useful to switch dynamically from BFS to DFS.

A variable selection strategy has to be added. For instance, linear
variables have to be preferred because their elimination does not
increase the degree of the other variables.

Some techniques for reducing the degree of the quantified variables
take into account the whole formula \cite{DSW:96}. They should be
applied at this stage.
%
\subsection{Gauss elimination}
The Gauss elimination code already prefers linear variables. It should
check all possibilities yet more closely: among linear equations,
e.g., $x=0$ should be preferred to $ax+1=0$ since the former does not
introduce a guarding condition.

One should try to factorize polynomials of a degree greater than $2$.
%
\subsection{Translation stage}
At the translation stage, one also should try to factorize polynomials
of a degree greater than $2$. The current implementation aborts with
an error in such a case. Even if factorization fails, the computation
should be continued. The result can be requantified, and with an
(wlog.~existential) decision problem, one still has the chance to find
``true''.
%
\subsection{Elimination set computation}
There are numerous sophisticated techniques for constructing small
elimination sets from a translation.
%
\subsection{Simplification}
Due to the doubly exponential explosion in the number of atomic
fomulas, simplification plays an extremely important role with this
method. Appropriate simplifications are described in detail in
\cite{DolzmannSturm:95}.
%
\section{Computation examples}
All the following examples have been computed on a SUN Sparc-20.
\subsection{The Davenport--Heintz Example}
This example is taken from \cite{CollinsHong:91}, p.\,325. The formula
$$
\exists c \forall a \forall b(a=d \land b=c) \lor (a=c \land b=1)
\longrightarrow a^2=b))
$$
is a special case of more general formula used by Davenport and Heintz
\cite{Davenport:88} in order to show the time complexity of real
quantifier elimination. It is equivalent to $d=1\lor d=-1$. Asir
yields after $0.03$\,s the equivalent result
$$
\vbox{\hsize=0.9\textwidth\noindent$d^{4} - 1 = 0 \land d \neq 0 \land
(d = 0 \lor d \neq 0) \land (d= 0 \lor d \neq 0 \land (d^{2} - 1 \neq
0 \lor d \neq 0)) \land (d^{3} - d = 0 \lor d^{2}- d \neq 0 \land
d^{2} - 1 \neq 0) \land (d^{2} - 1 \neq 0 \lor d \neq 0) \land (d^{2} -
1 = 0 \lor d^{2} - d \neq 0 \land d^{2} - 1 \neq 0)$}
$$
The REDLOG standard simplifier can simplify this to $d \neq 0 \land (d
+ 1= 0 \lor d - 1 = 0)$. This points on the necessity of more
sophisticated simplification strategies.
%
\subsection{Transportation problem}
An example which should not suffer from missing simplification is the
2-dimensional planar $3\times 3$ transportation problem taken from
\cite{LoosWeispfenning:93}, pp.\,459/460. The input formula
$$
\exists x_{11}\exists x_{12}\exists x_{13}\exists x_{21}\exists
x_{22}\exists x_{23}\exists x_{31}\exists x_{32}\exists x_{33}\biggl(
\bigwedge_{i=1}^3\bigwedge_{j=1}^3 x_{ij}\geq0\land
\bigwedge_{k=1}^3\Bigl(\sum_{j=1}^3 x_{kj}=a_k\land \sum_{i=1}^3
x_{ik}=b_k\Bigr) \biggr)
$$
is known to be equivalent to $\sigma\equiv\bigwedge_i (a_i\geq0) \land
\bigwedge_j (b_j\geq0) \land \sum_i a_i=\sum_j b_j$. After $0.31$\,s
Asir yields a quantifier-free equivalent $\rho$ containing $84$ atomic
formulas. So does REDLOG. We automatically check $\forall a_1\forall
a_2\forall a_3\forall b_1\forall b_2 \forall
b_3(\rho\longleftrightarrow\sigma)$ obtaining ``true'' after
$47.83$\,s plus $36.04$\,s GC time.
%
\subsection{Kahan's Problem}
Write down conditions such that the ellipse $E(x,y)=0$ with
$$
E(x,y)=(x-c)^2/a^2+(y-d)^2/b^2-1
$$
is inside the circle $C(x,y)=0$ with $C(x,y)=x^2+y^2-1$
\cite{Lazard:88}. We treat the special case $d=0$. Eliminating
$$
\forall a\forall b\bigl(b^2 (x-c)^2+a^2 y^2-a^2 b^2 \neq 0 \lor
x^2+y^2-1 \leq 0\bigr)
$$
we obtain after $1.21$\,s plus $0.14$\,s GC time a quantifier-free formula 
containing $136$ atomic formulas. REDLOG obtains only $59$ atomic
formulas due degree decreasing techniques, better simplification, and
more sophisticated elimination set computation ($2.9$\,s).
%
\subsection{Operation amplifier}
This is a problem very close to practice taken from \cite{Henning:91}. 
\begin{figure}
\centerline{\psfig{figure=db15.ps}}
\caption{An inverting operation amplifier circuit\label{opampfig}}
\end{figure}
For the operation amplifier circuit shown in Figure~\ref{opampfig}, we
want to determine the output voltage $\rm V_{OUT}=v1$ as a function of
the input voltage $\rm V_{IN}=v\_3$. We obtain the following algebraic
formulation $\omega$ of the circuit:
$$\vbox{\hsize=0.9\textwidth\noindent$\rm -v\_2+v\_1+i\_v0\cdot r1=0
\land -v\_3\cdot r1+v\_2\cdot r1+v\_2\cdot r2-v\_1\cdot
r2-i\_pm\_op1\cdot r1\cdot r2=0 \land v\_3-v\_2+i\_og\_op1\cdot r2=0
\land v\_1=v1 \land v\_3-v\_og\_op1=0 \land - v\_pm\_op1-v\_2=0 \land
vs^2\cdot x\_op1^2+a\cdot v\_og\_op1^2=a\cdot vs^2 \land
v\_og\_op1=v\_pm\_op1\cdot x\_op1^2 \land i\_pm\_op1=0$}
$$
The variables to be (existentially) eliminated are
$$\rm
V:=\{i\_og\_op1, v\_2, i\_pm\_op1, v\_1, i\_v0, v\_pm\_op1, v\_og\_op1 ,
x\_op1\}
$$
For the values $\rm a=1000$, $\rm vs=10$, $\rm r1=1000$, and $\rm
r2=10000$ the translation $\omega$ specializes to the following
$\omega'$:
$$\vbox{\hsize=0.9\textwidth\noindent$\rm
10000\cdot i\_og\_op1 - v\_2 + v\_3 = 0 \land 10000\cdot i\_pm\_op1 +
10\cdot v\_1 - 11\cdot v\_2 + v\_3 = 0 \land i\_pm\_op1 = 0 \land
1000\cdot i\_v0 + v\_1 - v\_2 = 0 \land v1 - v\_1 = 0 \land v\_2 +
v\_pm\_op1 = 0 \land v\_3 - v\_og\_op1 = 0 \land 10\cdot v\_og\_op1^2
+ x\_op1^2 - 1000 = 0 \land v\_og\_op1 - v\_pm\_op1\cdot x\_op1^2 =
0$}
$$
A symbolic method usually used for eliminating the variables is the
computation of an {\em elimination ideal basis} wrt.~the variables. 
Using this method, we obtain by using Asir
$$\rm
f:=10\cdot v\_3^3+100\cdot v1\cdot v\_3^2-1011\cdot v\_3-10000\cdot v1
$$
as implicit description of our function.
\begin{figure}
\centerline{\psfig{figure=parsol.ps,width=0.8\textwidth}}
\caption{The operation amplifier's behaviour including parasitic
solutions for $\rm |v\_3|>10$.\label{solfig}}
\end{figure}
Figure~\ref{solfig} displays the set of {\em real} zeroes of $f$. 
However, only the part of the curve for $\rm -10\leq v\_3\leq10$ is
the correct solution. The other parts are {\em parasitic solutions}
caused by the fact that the elimination ideal is computed over an
algebraically closed field, such as $\Bbb C$ for our case. In general, 
it is a problem to distinguish between proper and parasitic solutions.

Since our quantifier elimination is a {\em real} method, we may expect
to get the correct result. The current Asir version is not able to
eliminate $\exists V(\omega')$. It would obtain a degree violation
when eliminating $\rm x\_op1$ as the final variable. Its degree will
be greater than $2$ then. We can, however, apply a special case of the
degree decreasing methods mentioned in Subsection~\ref{blockopt} by
hand: $\rm x\_op1$ occurs only quadratically. We replace $\rm
x\_op1^2$ by a new variable $z$ in $\omega'$ obtaining $\omega_z'$. 
Similarly we replace $\rm x\_op1$ by $z$ in $V$ yielding $V_z$. Then
we have the equivalence
$$
\exists V(\omega')\longleftrightarrow\exists V_z(\omega_z'\land
z\geq0),
$$
the right hand side of which can be eliminated. The result obtained
after $0.03$\,s contains the correct contraint on the range of $\rm
v\_3$ to exclude the parasitic solutions:
$$\rm \underbrace{10\cdot v\_3^3+100\cdot v1\cdot v\_3^2-1011\cdot
v\_3-10000\cdot v1}_{\mathit f}=0 \land v\_3^2-100 \leq 0.
$$

Inspection of this example shows that there is actually only Gauss
elimination performed, which is a little disappointing. For
demonstrating the power of the method, we show how to solve the
original parametric problem with REDLOG: For $\exists V(\omega)$, we
obtain $604$ atomic formulas after $2.69$\,s. A result of this size
might be useful for further automatic processing but it does not
immediately contribute to understanding the network.

We thus apply {\em generic quantifier elimination} as proposed in
Subsection~\ref{genopt}. The result obtained for $\exists V(\omega)$
after $0.27$\,s is:
$$\vbox{\hsize=0.9\textwidth\noindent$\rm
a\cdot r1\cdot v\_3^3 - a\cdot r1\cdot v\_3\cdot vs^2 + a\cdot r2\cdot
v1\cdot v\_3^2 - a\cdot r2\cdot v1\cdot vs^2 - r1\cdot v\_3\cdot vs^2 -
r2\cdot v\_3\cdot vs^2 = 0 \land a\cdot v\_3^2 - a\cdot vs^2 \leq 0 \land
vs \neq 0$},
$$
valid under the following conditions:
$$\rm
a \neq 0,\quad r1 \neq 0,\quad r2 \neq 0,\quad v_3 + vs \neq 0,\quad
v_3 - vs \neq 0,\quad v_3 \neq 0.
$$
None of the conditions is a problem: $\rm a$ is the amplification
factor, $\rm r1$ and $\rm r2$ are resistors, the absolute value of the 
output voltage $\rm v\_3$ can certainly never get equal to the supply
power $\rm vs$.
%
\begin{appendix}
\section{The Asir user interface}\label{doku}
\subsection{The formula data type}
For the purpose of quantifier-elimination formulas have been
introduced to Asir as a new data type. The numerical value
corresponding to formulas is $10$. Atomic formula operators are
collected in Table~\ref{atoptab}.
\begin{table}
\begin{center}
\begin{tabular}{c|c|c|c|c|c}
$=$ & $\neq$ & $\leq$ & $<$ & $\geq$ & $>$\\
\hline
\verb"@==" (\verb"@=") & \verb"@!=" & \verb"@<=" & \verb"@<" &
\verb"@>=" & \verb"@>"\\
\end{tabular}
\caption{Infix operators for the input of atomic formulas
(abbreviations for input).\label{atoptab}}
\end{center}
\end{table}
The First-order operators are displayed in Table~\ref{coptab}.
\begin{table}
\begin{center}
\begin{tabular}{c|c|c|c|c|c|c|c}
$\land$ & $\lor$ & $\lnot$ & $\longrightarrow$ & $\longleftarrow$ &
$\longleftrightarrow$ & $\exists x\ldots$ & $\forall x\ldots$\\
\hline
\verb"@&&" & \verb"@||" & \verb"@!" & \verb"@impl" & \verb"@repl" &
\verb"@equiv" & \verb"ex(x,...)" & \verb"all(x,...)" \\
(\verb"@&") & (\verb"@|") & & & & & &\\
\end{tabular}
\caption{Binary and infix operators for the input of first-order
formulas (abbreviations for input).\label{coptab}}
\end{center}
\end{table}

We illustrate by example the input of formulas:
\begin{small}
\begin{verbatim}
[1] F = m*x+b@==0 @&& 0@<=x @&& x@<100;
(m*x+b @== 0) @&& (-x @<= 0) @&& (x-100 @< 0)
[2] all(x,F @impl x@<50);
all(x,((m*x+b @== 0) @&& (-x @<= 0) @&& (x-100 @< 0) @impl x-50 @< 0))
[3] ex([x,m],F);
ex(x,ex(m,(m*x+b @== 0) @&& (-x @<= 0) @&& (x-100 @< 0)))
[4] all(@@);
all(b,ex(x,ex(m,(m*x+b @== 0) @&& (-x @<= 0) @&& (x-100 @< 0))))
\end{verbatim}
\end{small}
In atomic formulas, all right hand sides are subtracted to the left
hand sides at the parsing stage. Notice that the quantifier operators
accept also lists of variables. If no variables at all are given, the
existential or universal closure of the formula is constructed resp.,
i.e., all free variables are quantified.
\subsection{Functions for formulas}
\begin{description}
\item[simpl({\sl f})] A simplified equivalent of {\sl f}. The
simplification strategy is described in Subsection~\ref{currentsimpl}.
\item[qe({\sl f})] A quantifier-free formula equivalent to {\sl f}. 
The argument formulas has to obey certain degree restrictions:
Quantified variables must not occur with a degree greater than $2$. 
Notice that with the elimination of each quantifier, the degree in the
other variables may increase. It thus cannot be determined by
inspection of the input whether quantifier elimination will succeed.
\item[atnum({\sl f})] The number of atomic formulas contained in {\sl
f}.
\item[atl({\sl f})] The set of atomic formulas contained in {\sl f} as
a list.
\item[nnf({\sl f})] A negation normal form of {\sl f}. This is a
formula equivalent to {\sl f} which contains only \verb!ex!,
\verb!all!, \verb!@&&!, and \verb!@||! as first-order operators.
\item[subf({\sl f},{\sl x},{\sl t})] Substitute all ocurrences of
variable {\sl x} in formula {\sl f} by polynomial {\sl t}. This works
for quantifier-free formulas. Quantified variables are not yet treated
appropriately.
\end{description}
\end{appendix}
\bibliographystyle{alpha}
\bibliography{adts}
\end{document}
