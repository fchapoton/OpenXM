%$OpenXM: OpenXM/src/kxx/issac2000.tex,v 1.1 1999/12/14 09:31:40 takayama Exp $
%% You need acmconf.cls and flushend.sty to compile this file.
%% They may be obtained from 
%%  http://riksun.riken.go.jp/archives/tex-archive/macros/latex/contrib/supported/acmconf/
\documentclass[submit]{acmconf}
%% \CopyrightText{\copyright 2000, }
\IfFileExists{graphicx.sty}{\usepackage{graphicx}}{}
\ConferenceName{1. ISSAC 2000, St. Andrews, UK, 2000}
\ConferenceShortName{ISSAC2000}
\def\OpenXM{{\tt OpenXM}}

\begin{document}
\date{January 16, 2000}
\title{OpenXM 
      --- an Open System \\ to Integrate Mathematical Softwares}
\author{\Author{Masahide Maekawa}\\
         \Address{Kobe University}\\
         \Email{maekawa@math.kobe-u.ac.jp}\\
         \and
         \Author{Masayuki Noro}\\
         \Address{Fujitsu Labs}\\
         \Email{noryo@flab.fujitsu.co.jp}
         \and
         \Author{Katsuyoshi Ohara}\\
         \Address{Kanazawa University}\\
         \Email{ohara@alpha.s.kanazawa-u.ac.jp}
         \and
         \Author{Yukio Okutani}\\
         \Address{Kobe University}\\
         \Email{okutani@math.kobe-u.ac.jp}
         \and
         \Author{Nobuki Takayama}\\
         \Address{Kobe University}\\
         \Email{takayama@math.kobe-u.ac.jp}
         \and
         \Author{Yasushi Tamura}\\
         \Address{Kobe University}\\
         \Email{tamura@math.kobe-u.ac.jp}
       }
\maketitle

\begin{abstract}
  This is an abstract.
\end{abstract}

\begin{keywords}
OpenMath, MP, OpenXM. 
\end{keywords}

\section{Introduction}
\OpenXM is a free, or Open Source, infrastructure for mathematical
softwares.
It provides methods and protocols 
to integrate various mathematical softwares.
\OpenXM package is a set of softwares that supports \OpenXM protocols.
It is currently a collection of softwares
{\tt Risa/Asir} \cite{asir}, {\tt Kan/sm1} \cite{sm1}, {\tt PHC} pack, {\tt GNUPLOT},
{\tt Mathematica} interface, and
{\tt OpenMath} interface. 
More and more softwares are wrapped with the \OpenXM stackmachine;
they are getting a member of \OpenXM package.


\section{Design outline}

\section{Data Format}

\section{OpenXM Stackmachines}

\section{OpenXM Clients}

\section{Applications}

\begin{thebibliography}{X}
\bibitem[1]{NT} Noro and Takayama: {\em Design and Implementation
of OpenXM}, 1996 -- 2000.
\end{thebibliography}
\end{document}
\endinput
%%


%%Text may be set as \emph{emph}.\\
%%Text may be set as \texttt{texttt}.\\
%%Text may be set as \underline{unterline}.\\
%%Text may be set as \textbf{textbf}.\\
%%Text may be set as \textrm{textrm}.\\
%%Text may be set as {\tiny tiny}.\\

%%\begin{figure}
%%\hrule
%%Nice Postscript, isn't it?
%%\begin{center}
%%\IfFileExists{graphicx.sty}{
%%  \includegraphics{body.eps}
%%}{
%%  Sorry, package \texttt{graphicx} not present.
%%}
%%\end{center}
%%Same, a little bit smaller:
%%\begin{center}
%%\IfFileExists{graphicx.sty}{
%%  \includegraphics[scale=.5]{body.eps}
%%  }{
%%  Sorry, package \texttt{graphicx} not present.
%%}
%%\end{center}
%%\caption{\label{fig-1}This is a nice floating figure}
%%\hrule
%%\end{figure}

