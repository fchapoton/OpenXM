% $OpenXM: OpenXM/doc/issac2000/homogeneous-network.tex,v 1.9 2000/01/17 01:33:19 noro Exp $

\subsection{Distributed computation with homogeneous servers}
\label{section:homog}

One of the aims of OpenXM is a parallel speedup by a distributed computation
with homogeneous servers. As the current specification of OpenXM does
not include communication between servers, one cannot expect
the maximal parallel speedup. However it is possible to execute
several types of distributed computation as follows.

\subsubsection{Product of univariate polynomials}

Shoup \cite{Shoup} showed that the product of univariate polynomials
with large degrees and large coefficients can be computed efficiently
by FFT over small finite fields and Chinese remainder theorem.
It can be easily parallelized:

\begin{tabbing}
Input :\= $f_1, f_2 \in {\bf Z}[x]$ such that $deg(f_1), deg(f_2) < 2^M$\\
Output : $f = f_1f_2$ \\
$P \leftarrow$ \= $\{m_1,\cdots,m_N\}$ where $m_i$ is an odd prime, \\
\> $2^{M+1}|m_i-1$ and $m=\prod m_i $ is sufficiently large. \\
Separate $P$ into disjoint subsets $P_1, \cdots, P_L$.\\
for \= $j=1$ to $L$ $M_j \leftarrow \prod_{m_i\in P_j} m_i$\\
Compute $F_j$ such that $F_j \equiv f_1f_2 \bmod M_j$\\
\> and $F_j \equiv 0 \bmod m/M_j$ in parallel.\\
\> (The product is computed by FFT.)\\
return $\phi_m(\sum F_j)$\\
(For $a \in {\bf Z}$, $\phi_m(a) \in (-m/2,m/2)$ and $\phi_m(a)\equiv a \bmod m$)
\end{tabbing}

Figure \ref{speedup}
shows the speedup factor under the above distributed computation
on Risa/Asir. For each $n$, two polynomials of degree $n$
with 3000bit coefficients are generated and the product is computed.
The machine is Fujitsu AP3000,
a cluster of Sun connected with a high speed network and MPI over the
network is used to implement OpenXM.
\begin{figure}[htbp]
\epsfxsize=8.5cm
\epsffile{speedup.ps}
\caption{Speedup factor}
\label{speedup}
\end{figure}

If the number of servers is $L$ and the inputs are fixed, then the cost to
compute $F_j$ in parallel is $O(1/L)$, whereas the cost
to send and receive polynomials is $O(L)$ if {\tt ox\_push\_cmo()} and
{\tt ox\_pop\_cmo()} are repeatedly applied on the client.
Therefore the speedup is limited and the upper bound of
the speedup factor depends on the ratio of 
the computational cost and the communication cost for each unit operation.
Figure \ref{speedup} shows that 
the speedup is satisfactory if the degree is large and $L$
is not large, say, up to 10 under the above envionment.
If OpenXM provides operations for the broadcast and the reduction
such as {\tt MPI\_Bcast} and {\tt MPI\_Reduce} respectively, the cost of 
sending $f_1$, $f_2$ and gathering $F_j$ may be reduced to $O(log_2L)$
and we can expect better results in such a case.

\subsubsection{Competitive distributed computation by various strategies}

SINGULAR \cite{Singular} implements {\tt MP} interface for distributed
computation and a competitive Gr\"obner basis computation is
illustrated as an example of distributed computation.
Such a distributed computation is also possible on OpenXM.
The following Risa/Asir function computes a Gr\"obner basis by
starting the computations simultaneously from the homogenized input and
the input itself.  The client watches the streams by {\tt ox\_select()}
and the result which is returned first is taken. Then the remaining
server is reset.

\begin{verbatim}
/* G:set of polys; V:list of variables */
/* O:type of order; P0,P1: id's of servers */
def dgr(G,V,O,P0,P1)
{
  P = [P0,P1]; /* server list */
  map(ox_reset,P); /* reset servers */
  /* P0 executes non-homogenized computation */
  ox_cmo_rpc(P0,"dp_gr_main",G,V,0,1,O);
  /* P1 executes homogenized computation */
  ox_cmo_rpc(P1,"dp_gr_main",G,V,1,1,O);
  map(ox_push_cmd,P,262); /* 262 = OX_popCMO */
  F = ox_select(P); /* wait for data */
  /* F[0] is a server's id which is ready */
  R = ox_get(F[0]);
  if ( F[0] == P0 ) {
    Win = "nonhomo"; Lose = P1;
  } else {
    Win = "homo"; Lose = P0;
  }
  ox_reset(Lose); /* reset the loser */
  return [Win,R];
}
\end{verbatim}
