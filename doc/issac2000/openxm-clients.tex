% $OpenXM: OpenXM/doc/issac2000/openxm-clients.tex,v 1.2 2000/01/02 07:32:12 takayama Exp $

\section{OpenXM Clients}    
(noryo and Ohara)
MEMO: keywords:
Asir and Mathematica clients.

\subsection{Risa/Asir}

Risa/Asir provides a launcher to invoke an OpenXM server and to set up the
communication between the server and itself. It also provides primitives
for communication as built-in functions.

\subsubsection{{\tt ox\_launch}}
{\tt ox\_launch} is a general purpose launcher.  This application
invokes a server and initiates the server-client communication
according to the protocol stated in Section \ref{launcher}, then
itself becomes a control server.
Several facilities related to {{\tt ox\_launch}} are provided
as built-in functions of Risa/Asir: a function to invoke a server
automatically from a give host name and a server name, and a set 
of functions to execute the port generation, {\tt bind}, {\tt listen},
{\tt connect} and {\tt accept} operations on sockets individually.

\subsubsection{Manipulating servers}

Fundamental operations on OpenXM servers are sending and receiving
of {\tt OX} data and sending of {\tt OX} commands. The following functions
are provided to execute these primitive operations:
{\tt ox\_push\_cmo()} for pushing data to a server, 
{\tt ox\_push\_cmd()} for sending an {\tt SM} command to a server
and {\tt ox\_get()} for receiving data from a stream.

Some operations including the reset operation are realized by
combining these primitives.  Among them, frequently used ones are
provided as built-in functions. We show several ones.

\begin{itemize}
\item {\tt ox\_pop\_cmo()}

It requests a server to send data on the stack to the stream, then
it receives the data from the stream.

\item {\tt ox\_cmo\_rpc()}

After pushing the name of a function, arguments and the number of the
arguments to the stack of a server , it request the server to execute
the function. It does not wait the termination of the function call.

\item {\tt ox\_reset()}

After sending {\tt SM\_control\_reset\_connection} to a control server,
it completes the operations stated in Section \ref{control}.
\end{itemize}
Furthermore {\tt ox\_select()} is provided to detect streams ready for
reading. It is realized by the {\tt select()} system call and is used
to avoid blocking on read operations.
