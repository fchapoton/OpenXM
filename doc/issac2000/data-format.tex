% $OpenXM: OpenXM/doc/issac2000/data-format.tex,v 1.2 2000/01/02 07:32:11 takayama Exp $

\section{Data Format}   (Ohara)

OpenXM admits multiple math encodings such as OpenMath, MP, CMO
(Common Mathematical Object format).
OpenXM itself does not exhibit a bias towards a particular encodings 
as a main math data carrier and OpenXM compliant system do not need to
implement all possible data formats.
A server or a client does not necessarily implement full specifications
of OpenXM. 
However, if a program sends an OX data or an OX command unknown to its peer, 
an unrecoverable error may occur. 
To avoid such a case OpenXM provides a scheme not
to send an OX message unknown to peers. 
It is realized by exchanging a list of
supported CMO's, OpenXM stackmachine codes, and necessary extra informations. 
The list is called {\it mathcap}.
Mathcap is also defined as a CMO and the elements are 32bit integers
or strings.

Math encoding types of OX data are distinguished with tags
of OX messages.
For example,
an OX message with the tag 
OX\_DATA is followed by a CMO packet.
An OX message with the tag 
OX\_DATA\_OPENMATH\_XML is followed by 
an OpenMath XML string and its length.

All OpenXM compliant systems should implement 7 primitive data types
of the CMO.
These are used not only to express mathematical objects, but also
indispensable to carry control informations such as mathcap.
CMO is an original methematical data encoding system of the OpenXM project.
Let us explain the data format of CMO.
Any CMO packet consists of a header and a body.
The size of the header is 4 bytes that tags the data type of the body.
Data type tags are signed 32 bit integers which is called {\sl int32} in this
paper.
Following tags are registered in the OpenXM for now.
\begin{verbatim}
#define CMO_ERROR2                         0x7f000002
#define CMO_NULL                           1
#define CMO_INT32                          2
#define CMO_DATUM                          3
#define CMO_STRING                         4
#define CMO_MATHCAP                        5
#define CMO_LIST                           17
#define CMO_MONOMIAL32                     19
#define CMO_ZZ                             20
#define CMO_QQ                             21
#define CMO_ZERO                           22
#define CMO_DMS_GENERIC                    24
#define CMO_DMS_OF_N_VARIABLES             25
#define CMO_RING_BY_NAME                   26
#define CMO_RECURSIVE_POLYNOMIAL           27
#define CMO_LIST_R                         28
#define CMO_INT32COEFF                     30
#define CMO_DISTRIBUTED_POLYNOMIAL         31
#define CMO_POLYNOMIAL_IN_ONE_VARIABLE     33
#define CMO_RATIONAL                       34
#define CMO_64BIT_MACHINE_DOUBLE           40
#define CMO_ARRAY_OF_64BIT_MACHINE_DOUBLE  41
#define CMO_128BIT_MACHINE_DOUBLE          42
#define CMO_ARRAY_OF_128BIT_MACHINE_DOUBLE 43
#define CMO_BIGFLOAT                       50
#define CMO_IEEE_DOUBLE_FLOAT              51
#define CMO_INDETERMINATE                  60
#define CMO_TREE                           61
#define CMO_LAMBDA                         62
\end{verbatim}
The first seven types should be implemented on all OpenXM compliant systems.
The format of the first seven CMO's are as follows. \\
\begin{tabular}{|c|c|}
\hline
{\tt int32 CMO\_ERROR2} & {\sl CMObject} {\rm ob} \\ 
\hline
\end{tabular} \\
\begin{tabular}{|c|c|}
\hline
{\tt int32 CMO\_NULL}  \\ 
\hline
\end{tabular} \\
\begin{tabular}{|c|c|}
\hline
{\tt int32 CMO\_INT32}& {\tt int32} {\rm n}  \\ 
\hline
\end{tabular} \\
\begin{tabular}{|c|c|c|}
\hline
{\tt int32 CMO\_DATUM}& {\tt int32} {\rm n} & {\tt byte} {\rm  data[0]} \\
\hline
$\cdots$ & {\tt byte} {\rm  data[n-1]} \\ 
\cline{1-2}
\end{tabular} \\
\begin{tabular}{|c|c|c|}
\hline
{\tt int32 CMO\_STRING}& {\tt int32} {\rm n} & {\tt byte} {\rm data[0]} \\
\hline
$\cdots$ & {\tt byte} {\rm data[n-1]} \\ 
\cline{1-2}
\end{tabular} \\
\begin{tabular}{|c|c|}
\hline
{\tt int32 CMO\_MATHCAP} & {\it CMObject} {\rm ob} \\ 
\hline
\end{tabular} \\
\begin{tabular}{|c|c|c|}
\hline
{\tt int32 CMO\_LIST}& {\tt int32} {\rm n} & {\tt CMObject} {\rm ob[0]} \\
\hline
$\cdots$ & {\tt CMObject} {\rm ob[n-1]} \\ 
\cline{1-2}
\end{tabular} \\


A mathematical programmer who wants to implement CMO on a server proceeds 
as follows.
\begin{enumerate}
\item Look for the CMO list at the web cite \cite{openxm-web}.
If there is a CMO tht fits to her or his requirement, then use this CMO.     
\item If there is no suitable CMO, design a new CMO and register 
the new CMO to \cite{openxm-web} with a description and examples.
\end{enumerate}



