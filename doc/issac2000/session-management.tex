% $OpenXM: OpenXM/doc/issac2000/session-management.tex,v 1.7 2000/01/15 03:46:27 noro Exp $

\section{Session Management}
\label{secsession}
%MEMO: key words:
%Security (ssh PAM), initial negotiation of byte order,
%mathcap, interruption, debugging window, etc.
 
In this section we explain our control integration in
OpenXM.  We assume that various clients and servers
establish connections dynamically and communicate to each
other. Therefore it is necessary to give a dynamical and unified
method to start servers and to establish connections.
In addition to that, interruption of an exections and debugging
supports are necessary for intaractive distributed computation.

%\subsection{Interface of servers}
%
%A server has additional I/O streams for exchanging data between
%a client and itself other than ones for diagnostic
%messages. As the streams are for binary data,
%the byte order conversion is necessary when a
%client and a server have different byte orders. It is determined by
%exchanging the preferable byte order of each peer. If the preference
%does not coincide with each other,
%then the network byte order is used.
%This implies that all servers and clients should be able to
%handle the network byte
%order. Nevertheless it is necessary to negotiate the byte order to
%skip the byte order conversion because its cost is often dominant over
%fast networks.

\subsection{Invocation of servers}
\label{launcher}

An application called {\it launcher} is provided to start servers
and to establish connections as follows.

\begin{enumerate}
\item A launcher is invoked from a client.
When the launcher is invoked, the client 
informs the launcher of a port number for TCP/IP connection
and the name of a server.
\item The launcher and the client establish a connection with the
specified port number.
\item The launcher creates a process and executes the server after
setting the data channel appropriately.
\end{enumerate}

After finishing the above task as a launcher, the launcher process
acts as a control server and controls the server process created by
itself. As for a control server see Section \ref{control}.

As the data channel is used to exchange binary data,
the byte order conversion is necessary when a
client and a server have different byte orders. It is determined by
exchanging the preferable byte order of each peer. If the preference
does not coincide with each other,
then the network byte order is used.
This implies that all servers and clients should be able to
handle the network byte
order. Nevertheless it is necessary to negotiate the byte order to
skip the byte order conversion because its cost is often dominant over
fast networks.


\subsection{Control server}
\label{control}
In OpenXM we adopted the following simple and robust method to 
control servers.

An OpenXM server has logically two I/O channels: one for exchanging
data for computations and the other for controlling computations. The
control channel is used to send commands to control execution on the
server. The launcher introduced in Section \ref{launcher}
is used as a control process. We call such a process a {\it
control server}. In contrast, we call a server for computation an {\it
engine}. As the control server and the engine runs on the
same machine, it is easy to manipulate the engine, especially to
send a signal from the control server. A control server is also an
OpenXM stack machine and it accepts {\tt SM\_control\_*} commands
to send signals to a server or to terminate a server.

\subsection{Resetting a server}

A client can send a signal to an engine by using the control channel 
at any time. However, I/O operations are usually buffered,
which may cause troubles without care for remaining data in 
the buffers.  To reset a server safely the following are required.

\begin{enumerate}
\item A sending of an {\tt OX} message must be completed.

As an {\tt OX} message is sent as a combination of several {\tt CMO}
data, a global exit without sending all the data confuses the
subsequent communication.

\item After restarting a server, a request from a client 
must correctly corresponds to the response from the server.

An incorrect correspondence occurs if some data remain on the stream
after restarting a server.
\end{enumerate}

{\tt SM\_control\_reset\_connection} is an {\tt SM} command to
initiate a safe resetting of a server. We show the action of 
a server and a client from the initiation to the completion of
a resetting.

\centerline{\fbox{client}}

\begin{enumerate}
\item The client sends {\tt SM\_control\_reset\_connection} to the
control server. The control server sends {\tt SIGUSR1} to the engine.
\item The client enters the resetting state. It skips all {\tt
OX} messages from the engine until it receives {\tt OX\_SYNC\_BALL}.
\item After receiving {\tt OX\_SYNC\_BALL} the client sends 
{\tt OX\_SYNC\_BALL} to the engine and returns to the usual state.
\end{enumerate}

\centerline{\fbox{engine}}

\begin{enumerate}
\item After receiving {\tt SIGUSR1} from the control server,
the engine enters the resetting state.
\item The engine sends {\tt OX\_SYNC\_BALL} to the client.
We note that the operation does not block because
the client reads and skips {\tt OX} messages soon after sending
{\tt SM\_control\_reset\_connection}.
\item The engine skips all {\tt OX} messages from the engine until it
receives {\tt OX\_SYNC\_BALL}.
\item After receiving {\tt OX\_SYNC\_BALL} the engine returns to the
usual state.
\end{enumerate}

{\tt OX\_SYNC\_BALL} is used to mark the end of data remaining in the
I/O streams. After reading it, it is assured that each stream is empty
and that the subsequent request from a client correctly 
corresponds to the response from the server.
We note that we don't have to associate {\tt OX\_SYNC\_BALL} with
any special action to be executed by the server because it is
assured that the peer is in the resetting state when one has received
{\tt OX\_SYNC\_BALL}.

\subsection{Debugging supports}
Debugging is not easy for distributed computations.
If servers are executed on X window system, then the control server can
attach an {\tt xterm} to the standard outputs of the engine to display
diagnostic messages from the engine.
Furthermore, if the engine provides an interface to input commands,
then debugging of user defined programs will be
possible on the engine. For example {\tt ox\_asir}, which is
the OpenXM server of {\tt Risa/Asir}, can pop up a window to input
debug commands and the debugging similar to that on usual terminals is possible.
One can also send {\tt SIGINT} by using {\tt SM\_control\_to\_debug\_mode}
and it provides a similar functionality to entering the debugging
mode from a keyboard interruption.
