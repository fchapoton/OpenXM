% $OpenXM$
%%%%%%%%%%%%%%%%%%%%%%%%%%%%%%%%%%%%%%%%%%%%%%%%%%%%%%%%%%%%%%%
% This is a sample input file for your contribution to a multi-
% author book to be published by Springer Verlag.
%
% Please use it as a template for your own input, and please
% follow the instructions for the formal editing of your
% manuscript as described in the file "1readme".
%
% Please send the Tex and figure files of your manuscript
% together with any additional style files as well as the
% PS file to the editor of your book.
%
% He or she will collect all contributions for the planned
% book, possibly compile them all in one go and pass the
% complete set of manuscripts on to Springer.
%%%%%%%%%%%%%%%%%%%%%%%%%%%%%%%%%%%%%%%%%%%%%%%%%%%%%%%%%%%%%%%



%RECOMMENDED%%%%%%%%%%%%%%%%%%%%%%%%%%%%%%%%%%%%%%%%%%%%%%%%%%%

\documentclass[runningheads]{cl2emult}

\usepackage{makeidx}  % allows index generation
\usepackage{graphicx} % standard LaTeX graphics tool
                      % for including eps-figure files
\usepackage{subeqnar} % subnumbers individual equations
                      % within an array
\usepackage{multicol} % used for the two-column index
\usepackage{cropmark} % cropmarks for pages without
                      % pagenumbers
\usepackage{math}     % placeholder for figures
\makeindex            % used for the subject index
                      % please use the style sprmidx.sty with
                      % your makeindex program

%upright Greek letters (example below: upright "mu")
\newcommand{\euler}[1]{{\usefont{U}{eur}{m}{n}#1}}
\newcommand{\eulerbold}[1]{{\usefont{U}{eur}{b}{n}#1}}
\newcommand{\umu}{\mbox{\euler{\char22}}}
\newcommand{\umub}{\mbox{\eulerbold{\char22}}}

%%%%%%%%%%%%%%%%%%%%%%%%%%%%%%%%%%%%%%%%%%%%%%%%%%%%%%%%%%%%%


%OPTIONAL%%%%%%%%%%%%%%%%%%%%%%%%%%%%%%%%%%%%%%%%%%%%%%%%%%%%
%
%\usepackage{amstex}   % useful for coding complex math
%\mathindent\parindent % needed in case "Amstex" is used
%
%%%%%%%%%%%%%%%%%%%%%%%%%%%%%%%%%%%%%%%%%%%%%%%%%%%%%%%%%%%%%

%AUTHOR_STYLES_AND_DEFINITIONS%%%%%%%%%%%%%%%%%%%%%%%%%%%%%%%
%
%Please reduce your own definitions and macros to an absolute
%minimum since otherwise the editor will find it rather
%strenuous to compile all individual contributions to a
%single book file
\usepackage{epsfig}
\def\cont{{\rm cont}}
\def\GCD{{\rm GCD}}
%
%%%%%%%%%%%%%%%%%%%%%%%%%%%%%%%%%%%%%%%%%%%%%%%%%%%%%%%%%%%%%

\begin{document}
%
\title*{A Computer Algebra System Risa/Asir and OpenXM}
%
%
\toctitle{A Computer Algebra System Risa/Asir and OpenXM}
% allows explicit linebreak for the table of content
%
%
\titlerunning{Risa/Asir and OpenXM}
% allows abbreviation of title, if the full title is too long
% to fit in the running head
%
\author{Masayuki Noro\inst{1}}
%
%\authorrunning{Masayuki Noro}
% if there are more than two authors,
% please abbreviate author list for running head
%
%
\institute{Kobe University, Rokko, Kobe 657-8501, Japan}

\maketitle              % typesets the title of the contribution

\begin{abstract}
OpenXM \cite{OPENXM} is an infrastructure for exchanging mathematical
data.  It defines a client-server architecture for parallel and
distributed computation.  Risa/Asir is software for polynomial
computation. It has been developed for testing new algorithms, and now
it acts as both a client and a server in the OpenXM package. In this
article we present an overview of Risa/Asir and its performances on
several functions.  We also show Risa/Asir's OpenXM interfaces and
examples of usages of them.
\end{abstract}

\section{A computer algebra system Risa/Asir}

\subsection{What is Risa/Asir?}

Risa/Asir \cite{RISA} is software mainly for polynomial
computation. Its major functions are polynomial factorization and
Groebner basis computation, whose core parts are implemented as
builtin functions.  Some higher algorithms such as primary ideal
decomposition or Galois group computation are built on them by the
user language.  The user language is called Asir language. Asir
language can be regarded as C language without type declaration of
variables, with list processing, and with automatic garbage
collection. A builtin {\tt gdb}-like user language debugger is
available. It is open source and the source code and binaries are
available via ftp or CVS.
Risa/Asir is not only an standalone computer algebra system but also a
main component in OpenXM package \cite{OPENXM}, which is a collection
of software comliant to OpenXM protocol specification.  OpenXM is an
infrastructure for exchanging mathematical data and Risa/Asir has
three kind of OpenXM intefaces : an inteface as a server, as a cllient
and as a subroutine library. We will explain them in the later
section.

Our goals of developing Risa/Asir are as follows:

\begin{enumerate}
\item Providing a test bed of new algorithms

Risa/Asir has been a platform for testing experimental algorithms in
polynomial factorization, computation related to Groebner basis,
cryptography and quantifier elimination. As to Groebner basis, we have
been mainly interested in problems over {\bf Q} and we tried applying
various modular techniques to overcome difficulties caused by huge
intermediate coefficients. We have had several results and they have
been implemented in Risa/Asir.

\item Gereral purpose open system

We need a lot of functions to make Risa/Asir a general purpose
computer algebra system.  In recent years we can obtain various high
performance applications or libraries as free software. We wrapped
such software as OpenXM servers and we started to release a collection
of such servers and cleints as OpenXM package in 1997. Risa/Asir is
now a main client in the package.

\item Environment for parallel and distributed computation

The origin of OpenXM is a protocol for doing parallel distributed
compuatations by connecting multiple Risa/Asir. OpenXM is also
designed to provide an enviroment efficient parallel distributed
computation. Currently only client-server communication is possible,
but we are preparing a specification OpenXM-RFC 102 allowing
client-client communication, which will enable us to execute
wider range of parallel algorithms efficiently.
\end{enumerate}

\subsection{Groebner basis and the related computation}

Currently Risa/Asir can only deal with polynomial ring. Operations on
modules over polynomial rings have not yet supported.  However, both
commutative polynomial rings and Weyl algebra are supported and one
can compute Groebner basis in both rings over the rationals, fields of
rational functions and finite fields. In the early stage of our
development, our effort was mainly devoted to improve the efficiency
of computation over the rationals. Our main tool is modular
computation. For Buchberger algorithm we adopted the trace lifting
algorithm by Traverso \cite{TRAV} and elaborated it by applying our
theory on a correspondence between Groebner basis and its modular
image \cite{NOYO}. We also combine the trace lifting with
homogenization to stabilize selection strategies, which enables us to
compute several examples efficiently which is hard to compute without
such a combination.  Our modular method can be applied to the change
of ordering algorithm and rational univariate representation.  We also
made a test implementation of $F_4$ algorithm \cite{F4}. Later we will
show timing data on Groebner basis computation.

\subsection{Polynomial factorization}

Here we briefly review functions on polynomial factorization.  For
univariate factorization over {\bf Q}, the classical
Berlekamp-Zassenhaus algorithm is implemented.  Efficient algorithms
recently proposed have not yet implemented.  For Univariate factorizer
over algebraic number fields, Trager's algorithm \cite{TRAGER} is
implemented with some modifications.  Its major applications are
splitting field and Galois group computation of polynomials over the
rationals. For such purpose a tower of simple extensions are suitable
because factors represented over a simple extension often have huge
coefficients \cite{ANY}.  For univariate factorization over finite
fields, equal degree factorization + Cantor-Zassenhaus algorithm is
implemented. We can use various representation of finite fields:
$GF(p)$ with a machine integer prime $p$, $GF(p)$, $GF(p^n)$ with any
odd prime $p$, $GF(2^n)$ with a bit representation of polynomials over
$GF(2)$ and $GF(p^n)$ with small $p^n$ represented by a primitive
root.  For multivariate factorization over the rationals, the
classical EZ(Extented Zassenhaus) type algorithm is implemented.

\subsection{Other functions}
By applying Groebner basis computation and polynomial factorization,
we have implemented several higher level algorithms. A typical
application is primary ideal decomposition of polynomial ideals over
{\bf Q}, which needs both functions.  Shimoyama-Yokoyama algorithm
\cite{SY} for primary decompsition is written in the user language.
Splitting field and Galois group computation are closely related and
are also important applications of polynomial factorization.  Our
implementation of Galois group computation algorithm \cite{ANY}
requires splitting field computation, which is written in the
user language.

\section{Techniques for efficient Groebner basis computation over {\bf Q}}
\label{gbtech}

In this section we review several practical techniques to improve
Groebner basis computation over {\bf Q}, which are easily
implemented but may not be well known.
We use the following notations.
\begin{description}
\item $\phi_p$ : the canonical projection from ${\bf Z}$ onto $GF(p)$
\item $HT(f)$ : the head term of a polynomail with respect to a term order
\item $HC(f)$ : the head coefficient of a polynomail with respect to a term order
\end{description}

\subsection{Combination of homogenization and trace lifting}

Traverso's trace lifting algorithm can be
formulated in an abstract form as follows \cite{FPARA}.
\begin{tabbing}
Input : a finite subset $F \subset {\bf Z}[X]$\\
Output : a Groebner basis $G$ of $Id(F)$ with respect to a term order $<$\\
do \= \\
\> $p \leftarrow$ a new prime\\
\>Guess \= a Groebner basis candidate $G \subset Id(F)$ 
such that $\phi_p(G)$ \\
\>\> is a Groebner basis of $Id(\phi_p(F))$ in ${GF(p)}[X]$\\
\>Check that $G$ is a Groebner basis of $Id(G)$ and $F \subset Id(G)$\\
\>If $G$ passes the check return $G$\\
end do
\end{tabbing}
We can apply various methods for {\tt guess} part of the above
algorithm.  Originally we guess the candidate by replacing zero normal
form checks over {\bf Q} with those over $GF(p)$ in the Buchberger
algorithm, which we call {\it tl\_guess}. In Asir one can specify
another method {\it tl\_h\_guess\_dh}, which is a combination of
{\it tl\_guess} and homogenization.
\begin{tabbing}
$tl\_h\_guess\_dh(F,p)$\\
Input : $F\subset {\bf Z}[X]$, a prime $p$\\
Output : a Groebner basis candidate\\
$F_h \leftarrow$ the homogenization of $F$\\
$G_h \leftarrow tl\_guess(F_h,p)$ under an appropriate term order\\
$G \leftarrow$ the dehomogenization of $G_h$\\
$G \leftarrow G \setminus \{g \in G| \exists h \in G \setminus \{g\}$
such that $HT(h)|HT(g)$ \}
\end{tabbing}
The input is homogenized to suppress intermediate coefficient swells
of intermediate basis elements.  The number of zero normal forms may
increase by the homogenization, but they are detected over
GF(p). Finally, by dehomogenizing the candidate we can expect that
lots of redundant elements can be removed.  We will show later that this is
surely efficient for some input polynomial sets.

\subsection{Minimal polynomial computation by modular method}
Let $I$ be a zero-dimensional ideal in $R={\bf Q}[x_1,\ldots,x_n]$.
Then the minimal polynomial $m(x_i)$ of a variable $x_i$ in $R/I$ can
be computed by a partial FGLM \cite{FGLM}, but it often takes long
time if one searches $m(x_i)$ incrementally over {\bf Q}.  In this
case we can apply a simple modular method to compute the minimal
polynomial.
\begin{tabbing}
Input : a Groebner basis $G$ of $I$, a variable $x_i$\\
Output : the minimal polynomial of $x$ in $R/I$\\
do \= \\
\> $p \leftarrow$ a new prime such that $p \not{|} HC(g)$ for all $g \in G$\\
\> $m_p \leftarrow$ the minimal polynomial of $x_i$ in $GF(p)[x_1,\ldots,x_n]/Id(\phi_p(G))$\\
\> If there exists $m(x_i) \in I$ such that $\phi_p(m) = m_p$ and $\deg(m)=\deg(m_p)$\\
\> then return $m(x_i)$\\
end do
\end{tabbing}
In this algorithm, $m_p$ can be obtained by a partial FGLM over
$GF(p)$ because $\phi_p(G)$ is a Groebner basis. Once we know the
candidate of $\deg(m(x_i))$, $m(x_i)$ can be determined by solving a
system of linear equations via the method of indeterminate
coefficient. Arguments on \cite{NOYO} ensures that $m(x_i)$ is what we
want if it exists. Note that the full FGLM can also be computed by the
same method.

\subsection{Integer contents reduction}

In some cases the cost to remove integer contents during nomal form
computations is dominant. We can remove the content of an integral
polynomial $f$ efficiently by the following method \cite{REPL}.
\begin{tabbing}
Input : an integral polynomial $f$\\
Output : a pair $(\cont(f),f/\cont(f))$\\
$g_0 \leftarrow$ an estimate of $\cont(f)$ such that $\cont(f)|g_0$\\
Write $f$ as $f = g_0q+r$ by division with remainder for each coefficient\\
If $r = 0$ then return $(g_0,q)$\\
else return $(g,g_0/g \cdot q + r/g)$, where $g = \GCD(g_0,\cont(r))$
\end{tabbing}
By serataing the set of coefficients of $f$ into two subsets and by
computing GCD of sums in the elements in the subsets we can estimate
$g_0$ with high accuracy. Then other components are easily computed.

%\subsection{Demand loading of reducers}
%An execution of the Buchberer algorithm may produce vary large number
%of intermediate basis elements. In Asir, we can specify that such
%basis elements should be put on disk to enlarge free memory space.
%This does not reduce the efficiency so much because all basis elements
%are not necessarily used in a single normal form computation, and the
%cost for reading basis elements from disk is often negligible because
%of the cost for coefficient computations. 

\section{Risa/Asir performance}

We show timing data on Risa/Asir for polynomial factorization
and Groebner basis computation. The measurements were made on
a PC with PentiumIII 1GHz and 1Gbyte of main memory. Timings
are given in seconds. In the tables `---' means it was not
measured.

\subsection{Groebner basis computation}

Table \ref{gbmod} and Table \ref{gbq} shows timing data for Groebner
basis compuation over $GF(32003)$ and over {\bf Q} respectively.
$C_n$ is the cyclic $n$ system and $K_n$ is the Katsura $n$ system,
both are famous bench mark problems.  We also measured the timing for
$McKay$ system over {\bf Q} \cite{REPL}.  the term order is graded
reverse lexicographic order.  In the both tables, the first three rows
are timings for the Buchberger algorithm, and the last two rows are
timings for $F_4$ algorithm. As to the Buchberger algorithm over
$GF(32003)$, Singular\cite{SINGULAR} shows the best performance among
the three systems. $F_4$ implementation in Risa/Asir is faster than
the Buchberger algorithm implementation in Singluar, but it is still
several times slower than $F_4$ implemenataion in FGb \cite{FGB}.  In
Table \ref{gbq}, $C_7$ and $McKay$ can be computed by the Buchberger
algorithm with the methods described in Section \ref{gbtech}.  It is
obvious that $F_4$ implementation in Risa/Asir over {\bf Q} is too
immature. Nevertheless the timing of $McKay$ is greatly reduced.
Why is $F_4$ efficient in this case? The answer is in the right
half of Fig. \ref{f4vsbuch}. During processing S-polynomials of degree
16, the Buchberger algorithm produces intermediate polynomials with
huge coefficients, but if we compute normal forms of these polynomials
by using all subsequently generated basis elements, then their
coefficients will be reduced after removing contents. As $F_4$
algorithm automatically produces the reduced basis elements, the
degree 16 computation is quite easy in $F_4$.


\begin{table}[hbtp]
\begin{center}
\begin{tabular}{|c||c|c|c|c|c|c|c|} \hline
		& $C_7$ & $C_8$ & $K_7$ & $K_8$ & $K_9$ & $K_{10}$ & $K_{11}$ \\ \hline
Asir $Buchberger$	& 31 & 1687  & 2.6  & 27 & 294  & 4309 & --- \\ \hline
Singular & 8.7 & 278 & 0.6 & 5.6 & 54 & 508 & 5510 \\ \hline
CoCoA 4 & 241 & $>$ 5h & 3.8 & 35 & 402 &7021  & --- \\ \hline\hline
Asir $F_4$	& 5.3 & 129 & 0.5  & 4.5 & 31  & 273 & 2641 \\ \hline
FGb(estimated)	& 0.9 & 23 & 0.1 & 0.8 & 6 & 51 & 366 \\ \hline
\end{tabular}
\end{center}
\caption{Groebner basis computation over $GF(32003)$}
\label{gbmod}
\end{table}

\begin{table}[hbtp]
\begin{center}
\begin{tabular}{|c||c|c|c|c|c|} \hline
		& $C_7$ & $Homog. C_7$ & $K_7$ & $K_8$ & $McKay$ \\ \hline
Asir $Buchberger$ 	& 389 & 594 & 29 & 299 & 34950 \\ \hline
Singular & --- & 15247 & 7.6 & 79 & $>$ 20h \\ \hline
CoCoA 4 & --- & 13227 & 57 & 709 & --- \\ \hline\hline
Asir $F_4$ 	&  989 & 456 & 90 & 991 & 4939 \\ \hline
FGb(estimated)	& 8 &11 & 0.6 & 5 & 10 \\ \hline
\end{tabular}
\end{center}
\caption{Groebner basis computation over {\bf Q}}
\label{gbq}
\end{table}

\begin{figure}[hbtp]
\begin{center}
\epsfxsize=12cm
\epsffile{blenall.ps}
\end{center}
\caption{Maximal coefficient bit length of intermediate bases}
\label{f4vsbuch}
\end{figure}

\subsection{Polynomial factorization}

Table \ref{unifac} shows timing data for univariate factorization over
{\bf Q}.  $N_{i,j}$ is an irreducible polynomial which are hard to
factor by the classical algorithm. $N_{i,j}$ is a norm of a polynomial
and $\deg(N_i) = i$ with $j$ modular factors. Risa/Asir is
disadvantageous in factoring polynomials of this type because the
algorithm used in Risa/Asir has exponential complexity. In contrast,
CoCoA 4\cite{COCOA} and NTL-5.2\cite{NTL} show nice performances
because they implement recently developed algorithms.

\begin{table}[hbtp]
\begin{center}
\begin{tabular}{|c||c|c|c|c|} \hline
		& $N_{105,23}$ & $N_{120,20}$ & $N_{168,24}$ & $N_{210,54}$ \\ \hline
Asir 	& 0.86	& 59 & 840 & hard \\ \hline
Asir NormFactor & 1.6 	& 2.2& 6.1& hard \\ \hline
%Singular& hard?	& hard?& hard? & hard? \\ \hline
CoCoA 4 & 0.2 	& 7.1	& 16 & 0.5 \\ \hline\hline
NTL-5.2	& 0.16 	& 0.9 	& 1.4 & 0.4 \\ \hline
\end{tabular}
\end{center}
\caption{Univariate factorization over {\bf Q}}
\label{unifac}
\end{table}

Table \ref{multifac} shows timing data for multivariate
factorization over {\bf Q}.
$W_{i,j,k}$ is a product of three multivariate polynomials 
$Wang[i]$, $Wang[j]$, $Wang[k]$ given in a data file
{\tt fctrdata} in Asir library directory. It is also included
in Risa/Asir source tree and located in {\tt asir2000/lib}.
For these examples Risa/Asir shows reasonable performance
compared with other famous systems. 

\begin{table}[hbtp]
\begin{center}
\begin{tabular}{|c||c|c|c|c|c|} \hline
	& $W_{1,2,3}$ & $W_{4,5,6}$ & $W_{7,8,9}$ & $W_{10,11,12}$ & $W_{13,14,15}$ \\ \hline
variables & 3 & 5 & 5 & 5 & 4 \\ \hline
monomials & 905 & 41369 & 51940 & 30988 & 3344 \\ \hline\hline
Asir 	& 0.2 & 4.7 & 14 & 17 & 0.4 \\ \hline
%Singular& $>$15min 	& ---	& ---& ---& ---\\ \hline
CoCoA 4 & 5.2 & $>$15min 	& $>$15min & $>$15min & 117 \\ \hline\hline
Mathematica 4& 0.2 	& 16 	& 23 & 36 & 1.1 \\ \hline
Maple 7& 0.5 	& 18 	& 967  & 48 & 1.3 \\ \hline
\end{tabular}
\end{center}
\caption{Multivariate factorization over {\bf Q}}
\label{multifac}
\end{table}

\section{OpenXM and Risa/Asir OpenXM interfaces}

\subsection{OpenXM overview}

OpenXM stands for Open message eXchange protocol for Mathematics.
Form the viewpoint of protocol design, it is a child of OpenMath
\cite{OPENMATH}.  However our approch is somewhat different. Our main
purpose is to provide an environment for integrating {\it existing}
mathematical software systems. OpenXM RFC-100 \cite{RFC100} defines a
client-server architecture.  Under this specification, a client
invokes an OpenXM (OX) server.  The client can send OpenXM (OX)
messages to the server.  OX messages consist of {\it data} and {\it
command}. Data is encoded according to the common mathematical object
(CMO) format which defines serialized representation of mathematical
objects.  An OX server is a stackmachine. If data is sent as an OX
message, the server pushes the data onto its stack. There is a common
set of stackmachine commands and all OX server understands its subset. 
The command set includes commands for manipulating the stack and
requests for execution of a procedure. In addition, a server may
accept its own command sequences if the server wraps some interactive
software. That is the server may be a hybrid server.

OpenXM RFC-100 also defines methods for session management. In particular
the method to reset a server is carefully designed and it provides
a robust way of using servers both for interactive and non-interactive
purposes.

\subsection{OpenXM client interface of {\tt asir}}

Risa/Asir is a main client in OpenXM package.  The application {\tt
asir} can access to OpenXM servers via several builtin interface
functions. and various inferfaces to existing OpenXM servers are
prepared as user defined functions written in Asir language.  We show
a typical OpenXM session.

\begin{verbatim}
[1] P = ox_launch();  /* invoke an OpenXM asir server */
0
[2] ox_push_cmo(P,x^10-y^10);
/* push a polynomial onto the stack */
0
[3] ox_execute_function(P,"fctr",1);  /* call factorizer */
0
[4] ox_pop_cmo(P);  /* get the result from the stack */
[[1,1],[x^4+y*x^3+y^2*x^2+y^3*x+y^4,1],
[x^4-y*x^3+y^2*x^2-y^3*x+y^4,1],[x-y,1],[x+y,1]]
[5] ox_cmo_rpc(P,"fctr,",x^10000-2^10000*y^10000); 
/* call factorizer; an utility function */
0
[6] ox_reset(P); /* reset the computation in the server */
1
[7] ox_shutdown(P); /* shutdown the server */
0
\end{verbatim}

\subsection{OpenXM server {\tt ox\_asir}}

An application {\tt ox\_asir} is a wrapper of {\tt asir} and provides
all the functions of {\tt asir} to OpenXM clients. It completely
implements the OpenXM reset protocol and also provides remote
debugging of user programs running on the server. We show a program
for checking whether a polynomial set is a Groebner basis or not. A
client executes {\tt gbcheck()} and servers execute {\tt
sp\_nf\_for\_gbcheck()} which is a simple normal form computation of a
S-polynomial. First of all the client collects all critical pairs
necessary for the check. Then the client requests normal form
computations to idling servers. If there are no idling servers the
clients waits for some servers to return results by {\tt
ox\_select()}, which is a wrapper of UNIX {\tt select()}. If we have
large number of critcal pairs to be processed, we can expect
good load balancing by {\tt ox\_select()}.

\begin{verbatim}
def gbcheck(B,V,O,Procs) {
  map(ox_reset,Procs);
  dp_ord(O); D = map(dp_ptod,B,V);  
  L = dp_gr_checklist(D); DP = L[0]; Plist = L[1];
  /* register DP in servers */
  map(ox_cmo_rpc,Procs,"register_data_for_gbcheck",vtol(DP));
  /* discard return value in stack */
  map(ox_pop_cmo,Procs);
  Free = Procs; Busy = [];
  while ( Plist != [] || Busy != []  )
    if ( Free == [] || Plist == [] ) {
      /* someone is working; wait for data */
      Ready = ox_select(Busy);
	  /* update Busy list and Free list */
      Busy = setminus(Busy,Ready); Free = append(Ready,Free);
      for ( ; Ready != []; Ready = cdr(Ready) )
        if ( ox_get(car(Ready)) != 0 ) {
		  /* a normal form is non zero */
          map(ox_reset,Procs); return 0;
        }
    } else {
	  /* update Busy list and Free list */
      Id = car(Free); Free = cdr(Free); Busy = cons(Id,Busy);
	  /* take a pair */
	  Pair = car(Plist); Plist = cdr(Plist);
	  /* request a normal form computation */
      ox_cmo_rpc(Id,"sp_nf_for_gbcheck",Pair);
      ox_push_cmd(Id,262); /* 262 = OX_popCMO */
    }
  map(ox_reset,Procs); return 1;
}
\end{verbatim}

\subsection{Asir OpenXM library {\tt libasir.a}}

Asir OpenXM library {\tt libasir.a} includes functions simulating the
stack machine commands supported in {\tt ox\_asir}.  By linking {\tt
libasir.a} an application can use the same functions as in {\tt
ox\_asir} without accessing to {\tt ox\_asir} via TCP/IP.

\section{Concluding remarks}
We have shown the current status of Risa/Asir and its OpenXM
interfaces. As a result of our policy of development, it is true that
Risa/Asir does not have abundant functions. However it is a completely
open system and its total performance is not bad. As OpenXM interface
specification is completely documented, we can add another function to
Risa/Asir by wrapping an existing software system as an OX server and
vice versa. User program debugger can be used both for local and
remote debugging. By combining the debugger and the function to reset
servers, one will be able to enjoy parallel and distributed
computation.
%
\begin{thebibliography}{7}
%
\addcontentsline{toc}{section}{References}

\bibitem{ANY}
Anay, H., Noro, M., Yokoyama, K. (1996)
Computation of the Splitting fields and the Galois Groups of Polynomials.
Algorithms in Algebraic geometry and Applications, 
Birkh\"auser (Proceedings of MEGA'94), 29--50.

\bibitem{FPARA}
Jean-Charles Faug\`ere (1994)
Parallelization of Groebner basis.
Proceedings of PASCO'94, 124--132.

\bibitem{F4}
Jean-Charles Faug\`ere (1999)
A new efficient algorithm for computing Groebner bases  ($F_4$).
Journal of Pure and Applied Algebra (139) 1-3 , 61--88.

\bibitem{FGLM}
Faug\`ere, J.-C. et al. (1993)
Efficient computation of zero-dimensional Groebner bases by change of ordering.
Journal of Symbolic Computation 16, 329--344.

\bibitem{RFC100}
M. Maekawa, et al. (2001)
The Design and Implementation of OpenXM-RFC 100 and 101.
Proceedings of ASCM2001, World Scientific, 102--111.

\bibitem{RISA}
Noro, M. et al. (1994-2001)
A computer algebra system Risa/Asir.
{\tt http://www.openxm.org}, {\tt http://www.math.kobe-u.ac.jp/Asir/asir.html}.

\bibitem{REPL}
Noro, M., McKay, J. (1997)
Computation of replicable functions on Risa/Asir.
Proceedings of PASCO'97, ACM Press, 130--138.

\bibitem{NOYO}
Noro, M., Yokoyama, K. (1999)
A Modular Method to Compute the Rational Univariate
Representation of Zero-Dimensional Ideals.
Journal of Symbolic Computation, 28, 1, 243--263.

\bibitem{OPENXM}
OpenXM committers (2000-2001)
OpenXM package.
{\tt http://www.openxm.org}.

\bibitem{SY}
Shimoyama, T., Yokoyama, K. (1996)
Localization and Primary Decomposition of Polynomial Ideals.
Journal of Symbolic Computation, 22, 3, 247--277.

\bibitem{TRAGER}
Trager, B.M. (1976)
Algebraic Factoring and Rational Function Integration.
Proceedings of SYMSAC 76, 219--226.

\bibitem{TRAV}
Traverso, C. (1988)
Groebner trace algorithms.
LNCS {\bf 358} (Proceedings of ISSAC'88), Springer-Verlag, 125--138.

\bibitem{COCOA}
{\tt http://cocoa.dima.unige.it/}.

\bibitem{FGB}
{\tt http://www-calfor.lip6.fr/\~\,jcf/}.

\bibitem{NTL}
{\tt http://www.shoup.net/}.

\bibitem{OPENMATH}
{\tt http://www.openmath.org/}.

\bibitem{SINGULAR}
{\tt http://www.singular.uni-kl.de/}.

\end{thebibliography}

%INDEX%%%%%%%%%%%%%%%%%%%%%%%%%%%%%%%%%%%%%%%%%%%%%%%%%%%%%%%%%%%%%%%
\clearpage
\addcontentsline{toc}{section}{Index}
\flushbottom
\printindex
%%%%%%%%%%%%%%%%%%%%%%%%%%%%%%%%%%%%%%%%%%%%%%%%%%%%%%%%%%%%%%%%%%%%%

\end{document}

