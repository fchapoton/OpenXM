%  $OpenXM$ 

\documentclass{slides}
\usepackage{epsfig}

\begin{document}
http://www.openxm.org \\
takayama@openxm.org, takayama@math.kobe-u.ac.jp \\

Open message eXchange protocol for Mathematics.

\fbox{Aim of the project}
\begin{enumerate}
\item An experiment to connect mathematical software each other
\begin{enumerate}
\item Efficient and easy distributed computation
\item Use other system as a subroutine or a remote object
\item New generation of mathematical software
(old regime = maple, mathematica, ...)
\end{enumerate}
\item Provide a package of high level of mathematics and
documentation on the OpenXM protocol infrastructure.
\item Attractive to mathematicians.
\item Presentation on web.
\end{enumerate}

\fbox{Release schedule}

\fbox{Other projects}

\newpage

\fbox{Design outline}
\begin{enumerate}
\item Communication is an exchange of messages. The messages are classified into
three types:
DATA, COMMAND, and SPECIAL.
They are called OX (OpenXM) messages.
Among the three types,
{\it OX data messages} wrap mathematical data.
We use standards of mathematical data formats such as OpenMath and MP
as well as our own data format {\it CMO}
({\it Common Mathematical Object format}).
\item Servers, which provide services to other processes, are stack machines.
The stack machine is called the
{\it OX stack machine}.
Existing mathematical software are wrapped with this stack machine.
Minimal requirements for a target wrapped with the OX stack machine
are as follows:
\begin{enumerate}
\item The target must have a serialized interface such as a character based
interface.
\item An output of the target must be understandable for computer programs;
it should follow a grammar that can be parsed with other software tools.
\end{enumerate}
\item Any server may have a hybrid interface;
it may accept and execute not only stack machine commands, 
but also its original command sequences.
For example,
if we send the following string to the {\tt ox\_asir} server  \\
\verb+        " fctr(x^100-y^100); "      + \\
and call the stack machine command  \\
\verb+        SM_executeStringByLocalParser    + \\ 
then the server executes the asir command \\
\verb+ fctr(x^100-y^100); + 
(factorize $x^{100}-y^{100}$ over ${\bf Q}$)
and pushes the result onto the stack.
\item Network transparent supports for controlling servers are provided.
For example OpenXM defines a robust reset procedure to restart computations
without any confusion in I/O buffers.
It is very useful for debugging programs running on distributed environment.
\end{enumerate}

\newpage

\fbox{Components of version 1.1.2}
\begin{description}
\item{\tt ox\_asir}
A server for Risa/Asir, a general-purpose computer algebra
system. It provides almost
all functionalities of Risa/Asir such as polynomial factorization over algebraic numbers,
Gr\"obner basis computation, primary ideal decomposition,
and efficient computation over finite fields.
(Noro, Yokoyama, Shimoyama, Takeshima)
\item{\tt ox\_pari}  {\footnotesize ftp://megrez.math.u-bordeaux.fr/pub/pari}\\
pari is contained in {\tt ox\_asir}.
(C. Batut, K. Belabas, D. Bernardi, H. Cohen and M. Olivier)
\item{\tt ox\_sm1}  
A server for Kan/sm1, a system for computation in 
the ring of differential operators including computation of Gr\"obner bases
and cohomology groups.
(Oaku, Takayama)
\item {\tt ox\_phc}  {\footnotesize http://www.math.msu.edu/~jan/download.html} \\
A server for PHC pack, a general-purpose solver for
polynomial systems by homotopy continuation. 
(Verschelde)
\item {\tt ox\_tigers} {\footnotesize         http://www.math.tamu.edu/~rekha/programs.html } \\
A server for TiGERS, a system to enumerate
all Gr\"obner bases of affine toric ideals.
It can be used to determine the state polytope
of a given affine toric ideal.
(Hubert, Rekha)
\item {\tt ox\_gnuplot}  {\footnotesize ftp://ftp.gnuplot.vt.edu/pub/gnuplot/}\\
\item {\tt ox\_math}
A server for Mathematica.
(Ohara)
\item {\tt OMproxy}  {\footnotesize cf. http://www.openmath.org} \\
A server for translation between CMO and OpenMath/XML expressions.
It is written in Java.
This module provides Java classes OXmessage, CMO, and SM
for the OpenXM protocol, too.
(Tamura)
\end{description}
In addition to these servers, Risa/Asir, Kan/sm1 and Mathematica
can act as clients. 
\newpage
\fbox{OX}
{\footnotesize
\begin{verbatim}
#define	OX_COMMAND               513  // COMMAND
#define	OX_DATA	                 514  // DATA
#define OX_SYNC_BALL             515  // SPECIAL
#define OX_DATA_WITH_LENGTH      521  // DATA
#define OX_DATA_OPENMATH_XML     523  // DATA
#define OX_DATA_OPENMATH_BINARY  524  // DATA
#define OX_DATA_MP               525  // DATA
\end{verbatim} }
\newpage

\fbox{CMO}
{\footnotesize
\begin{verbatim}
#define CMO_ERROR2                         0x7f000002
#define CMO_NULL                           1
#define CMO_INT32                          2
#define CMO_DATUM                          3
#define CMO_STRING                         4
#define CMO_MATHCAP                        5
#define CMO_LIST                           17

#define CMO_MONOMIAL32                     19
#define CMO_ZZ                             20
#define CMO_QQ                             21
#define CMO_ZERO                           22
#define CMO_DMS_GENERIC                    24
#define CMO_DMS_OF_N_VARIABLES             25
#define CMO_RING_BY_NAME                   26
#define CMO_RECURSIVE_POLYNOMIAL           27
#define CMO_LIST_R                         28
#define CMO_INT32COEFF                     30
#define CMO_DISTRIBUTED_POLYNOMIAL         31
#define CMO_POLYNOMIAL_IN_ONE_VARIABLE     33
#define CMO_RATIONAL                       34
#define CMO_64BIT_MACHINE_DOUBLE           40
#define CMO_ARRAY_OF_64BIT_MACHINE_DOUBLE  41
#define CMO_BIGFLOAT                       50
#define CMO_IEEE_DOUBLE_FLOAT              51
#define CMO_INDETERMINATE                  60
#define CMO_TREE                           61
#define CMO_LAMBDA                         62
\end{verbatim} }

\fbox{SM COMMAND}
{\footnotesize
\begin{verbatim}
#define SM_popSerializedLocalObject               258
#define SM_popCMO                                 262
#define SM_popString                              263
#define SM_mathcap                                264
#define SM_pops                                   265
#define SM_setName                                266
#define SM_evalName                               267
#define SM_executeStringByLocalParser             268
#define SM_executeFunction                        269
#define SM_beginBlock                             270
#define SM_endBlock                               271
#define SM_shutdown                               272
#define SM_setMathCap                             273
#define SM_executeStringByLocalParserInBatchMode  274
#define SM_getsp                                  275
#define SM_dupErrors                              276
#define SM_control_kill                          1024
#define SM_control_to_debug_mode                 1025
#define SM_control_exit_debug_mode               1026
#define SM_control_reset_connection              1030
\end{verbatim} }
\newpage

\fbox{Programming in asir client}	


\begin{verbatim}
#define SM_executeFunction  269
def ex1() {
  P = ox_launch();
  ox_push_cmo(P,13);
  ox_push_cmo(P,8);
  ox_push_cmo(P,ntoint32(2));
  ox_push_cmo(P,"igcd");
  ox_push_cmd(P,SM_executeFunction);
  return(ox_pop_cmo(P));
}
\end{verbatim}
{\footnotesize
\begin{verbatim}
igcd(I1,I2)
     :: The integer greatest common divisor of I1 and I2.
\end{verbatim}
}

\begin{verbatim}
  ox_rpc(P,"igcd",13,8);
  ox_pop_cmo(P);
\end{verbatim}

A simple interface to servers by using strings:
\begin{verbatim}
ox_execute_string(P,"igcd(13,8);");
\end{verbatim}
\newpage

Error packet of ox\_asir server:
\begin{verbatim}
[340] ox_launch();
0
[341] ox_rpc(0,"hoge",0);
0
[342] ox_pop_cmo(0);
error([8,executeFunction : the function hoge not found])
\end{verbatim}

Error packet of ox\_sm1 server:
\begin{verbatim}
[340] ox_launch(0,getenv("OpenXM_HOME")+"/bin/ox_sm1");
0
[341] ox_cmo_rpc(0,"hoge",0);
0
[342] ox_pop_cmo(0);
error([7,4294967295,executeString: 
 Warning: identifier is not in the dictionaries])
\end{verbatim}

\newpage
Generate a table of dilog by calling pari (server).
\begin{verbatim}
def ex2() {
  P = ox_launch();
  Ans = [ ];
  for (I=0; I<1.0; I=I+0.1) {
    /* ox_execute_string(P,"pari(dilog,0.3);"); */
    ox_execute_string(P,"pari(dilog,"+rtostr(I)+");");
    Ans = append(Ans,[ox_pop_local(P)]);
  }
  return(Ans);
}
\end{verbatim}
\begin{verbatim}
[345] ex2();
[0, 0.102617791099391136905, 0.21100377543970478482,
 0.32612951007547605591,0.44928297447128169231,0.58224052646501250462,0.72758630771633335369,0.88937762428603865937,1.074794600008248448403,1.29971472300495878170,1.64493406684822643609]
[346] 
\end{verbatim}
\newpage

Computing the dimensions of de Rham cohomology groups
of ${\bf C}^n \setminus V(f)$ by calling kan/sm1.
{\footnotesize
 Oaku, Takayama, An algorithm for de Rham cohomology groups of the
  complement of an affine variety via D-module computation,
  Journal of pure and applied algebra 139 (1999), 201-233.}
\begin{verbatim}
[344] P = sm1_start();
[345] sm1_deRham([x^3-y^2*z^2,[x,y,z]]|proc=P);
[1,1,0,0]
[346] 
\end{verbatim}
{\footnotesize
\begin{verbatim}
def sm1_deRham(A) {
  SM1_FIND_PROC(P);
  P = sm1_check_server(P);
  sm1_push_int0(P,A);
  sm1(P," deRham ");
  B = sm1_pop(P);
  ox_check_errors2(P);
  return(B);
}
\end{verbatim}
}

\begin{enumerate}
\item Free resolution in $D$ (ox\_sm1).
\item Annihilating ideal of $f^{-1}$ (ox\_sm1).
\item Factorization  (ox\_asir).
\end{enumerate}

\newpage

\fbox{ Distributed computation by a job pool}
{\footnotesize \begin{verbatim}
#define SM_popCMO 262
def pool() {
  P = [ox_launch(), ox_launch(), ox_launch()];
  map(ox_push_cmo,P,0);
  map(ox_push_cmd,P,SM_popCMO);
  /* map(ox_get,P); mistake */

  I = 50;
  Jobs = [x^(I)-y^(I), x^(I+1)-y^(I+1),
          x^(I+2)-y^(I+2), x^(I+3)-y^(I+3),
          x^(I+4)-y^(I+4), x^(I+5)-y^(I+5)];
  N = length(Jobs);
  Ans = [ ];
  print("------  started -----");
  /* while (length(Jobs) != 0) { mistake */
  while (length(Ans) != N) {
    Q = ox_select(P)[0];
    F = ox_get(Q);
    /* print([Q,F]); */
    if (F != 0) {
        Ans = append(Ans,[F]); 
        print(Q);
    }
    if (length(Jobs) > 0) {
      Job = car(Jobs); Jobs=cdr(Jobs);
      ox_rpc(Q,"fctr",Job); ox_push_cmd(Q,SM_popCMO);
    }
  }
  return(Ans);
}


[345] pool();
------  started -----
1 2 1 0 1 2

[[[-1,1],[y^16+x*y^15+x^2*y^14+x^3*y^13 ---- snip ----

\end{verbatim}}
\newpage

\fbox{Finding series solution} \\
\fbox{of holonomic system to the direction $w$}

{\footnotesize
\begin{verbatim}
def dsolv_initial(F1,V1,W1) {
  extern Dsolv_message_initial;
  S=[F1,V1,W1];
  F=S[0];
  V=S[1];  
  DV=map(sm1_d,V);
  W=S[2];
  N = length(V);
  G = sm1_gb([F,V,dsolv_consw(V,W)]);
  In = G[1]; 
  In = map(subst,In,h,1);
  In = sm1_gb([In,V])[0];  /* Computing the reduced basis. */
  Ans = [ ];
  for (I=0; I<length(In); I++) {
    D = sm1_distraction([In[I],V,V,DV,V]);
    Ans = append(Ans,[D]);
  }
  return(Ans);
}
\end{verbatim}

--- SNIP ---

}

\begin{enumerate}
\item Clean code
\item Two days for our implementation.
\end{enumerate}
\begin{enumerate}
\item Argument check.
\item Debug window for servers.
\item We have to write $n$-help messages.
\end{enumerate}

\newpage
\fbox{ Product of univariate polynomials (Shoup)}

\begin{tabbing}
Input :\= $f_1, f_2 \in {\bf Z}[x]$ such that $deg(f_1), deg(f_2) < 2^M$\\
Output : $f = f_1f_2$ \\
$P \leftarrow$ \= $\{m_1,\cdots,m_N\}$ where $m_i$ is an odd prime, \\
\> $2^{M+1}|m_i-1$ and $m=\prod m_i $ is sufficiently large. \\
Separate $P$ into disjoint subsets $P_1, \cdots, P_L$.\\
for \= $j=1$ to $L$ $M_j \leftarrow \prod_{m_i\in P_j} m_i$\\
Compute $F_j$ such that $F_j \equiv f_1f_2 \bmod M_j$\\
\> and $F_j \equiv 0 \bmod m/M_j$ in parallel.\\
\> (The product is computed by FFT.)\\
return $\phi_m(\sum F_j)$\\
(For $a \in {\bf Z}$, $\phi_m(a) \in (-m/2,m/2)$ \\
and $\phi_m(a)\equiv a \bmod m$)
\end{tabbing}


\epsfxsize=17cm
\begin{center}
\epsffile{../calc2000/speedup.ps}
\end{center}

\newpage

\fbox{Solving algebraic equations}

\begin{verbatim}
[287] phc(katsura(7));
The detailed output is in the file tmp.output.*
The answer is in the variable Phc.
0
[290] B=map(first,Phc)$
[291] gnuplot_plotDots([],0)$
[292] gnuplot_plotDots(B,0)$
\end{verbatim}

\epsfxsize=17cm
\begin{center}
\epsffile{../calc2000/katsura7.ps}
\end{center}
The first components of the solutions to the system of algebraic equations Katsura 7.
\newpage

\end{document}