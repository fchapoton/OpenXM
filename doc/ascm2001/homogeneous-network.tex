% $OpenXM: OpenXM/doc/ascm2001/homogeneous-network.tex,v 1.3 2001/03/07 08:12:56 noro Exp $

\subsection{Distributed computation with homogeneous servers}
\label{section:homog}

One of the aims of OpenXM is a parallel speedup by a distributed computation
with homogeneous servers. As the current specification of OpenXM does
not include communication between servers, one cannot expect
the maximal parallel speedup. However it is possible to execute
several types of distributed computation as follows.

\subsubsection{Product of univariate polynomials}

Shoup \cite{Shoup} showed that the product of univariate polynomials
with large degrees and large coefficients can be computed efficiently
by FFT over small finite fields and Chinese remainder theorem.
It can be easily parallelized:

\begin{tabbing}
Input :\= $f_1, f_2 \in {\bf Z}[x]$ such that $deg(f_1), deg(f_2) < 2^M$\\
Output : $f = f_1f_2$ \\
$P \leftarrow$ \= $\{m_1,\cdots,m_N\}$ where $m_i$ is an odd prime, \\
\> $2^{M+1}|m_i-1$ and $m=\prod m_i $ is sufficiently large. \\
Separate $P$ into disjoint subsets $P_1, \cdots, P_L$.\\
for \= $j=1$ to $L$ $M_j \leftarrow \prod_{m_i\in P_j} m_i$\\
Compute $F_j$ such that $F_j \equiv f_1f_2 \bmod M_j$\\
\> and $F_j \equiv 0 \bmod m/M_j$ in parallel.\\
\> (The product is computed by FFT.)\\
return $\phi_m(\sum F_j)$\\
(For $a \in {\bf Z}$, $\phi_m(a) \in (-m/2,m/2)$ and $\phi_m(a)\equiv a \bmod m$)
\end{tabbing}

Figure \ref{speedup}
shows the speedup factor under the above distributed computation
on Risa/Asir. For each $n$, two polynomials of degree $n$
with 3000bit coefficients are generated and the product is computed.
The machine is FUJITSU AP3000,
a cluster of Sun workstations connected with a high speed network 
and MPI over the network is used to implement OpenXM.
\begin{figure}[htbp]
\epsfxsize=8.5cm
\epsffile{speedup.ps}
\caption{Speedup factor}
\label{speedup}
\end{figure}

If the number of servers is $L$ and the inputs are fixed, then the cost to
compute $F_j$ in parallel is $O(1/L)$, whereas the cost
to send and receive polynomials is $O(L)$ if {\tt ox\_push\_cmo()} and
{\tt ox\_pop\_cmo()} are repeatedly applied on the client.
Therefore the speedup is limited and the upper bound of
the speedup factor depends on the ratio of 
the computational cost and the communication cost for each unit operation.
Figure \ref{speedup} shows that 
the speedup is satisfactory if the degree is large and $L$
is not large, say, up to 10 under the above environment.
If OpenXM provides collective operations for broadcast and reduction
such as {\tt MPI\_Bcast} and {\tt MPI\_Reduce} respectively, the cost of 
sending $f_1$, $f_2$ and gathering $F_j$ may be reduced to $O(\log_2L)$
and we can expect better results in such a case. In order to implement
such operations we need new specifications for inter-sever communication
and the session management, which will be proposed as OpenXM-RFC 102.
We note that preliminary experiments show the collective operations
work well on OpenXM.

\subsubsection{Competitive distributed computation by various strategies}

SINGULAR \cite{Singular} implements {\it MP} interface for distributed
computation and a competitive Gr\"obner basis computation is
illustrated as an example of distributed computation.
Such a distributed computation is also possible on OpenXM.
The following Risa/Asir function computes a Gr\"obner basis by
starting the computations simultaneously from the homogenized input and
the input itself.  The client watches the streams by {\tt ox\_select()}
and the result which is returned first is taken. Then the remaining
server is reset.

\begin{verbatim}
/* G:set of polys; V:list of variables */
/* O:type of order; P0,P1: id's of servers */
def dgr(G,V,O,P0,P1)
{
  P = [P0,P1]; /* server list */
  map(ox_reset,P); /* reset servers */
  /* P0 executes non-homogenized computation */
  ox_cmo_rpc(P0,"dp_gr_main",G,V,0,1,O);
  /* P1 executes homogenized computation */
  ox_cmo_rpc(P1,"dp_gr_main",G,V,1,1,O);
  map(ox_push_cmd,P,262); /* 262 = OX_popCMO */
  F = ox_select(P); /* wait for data */
  /* F[0] is a server's id which is ready */
  R = ox_get(F[0]);
  if ( F[0] == P0 ) {
    Win = "nonhomo"; Lose = P1;
  } else {
    Win = "homo"; Lose = P0;
  }
  ox_reset(Lose); /* reset the loser */
  return [Win,R];
}
\end{verbatim}

\subsubsection{Nesting of client-server communication}

Under OpenXM-RFC 100 an OpenXM server can be a client of other servers.
Figure \ref{tree} illustrates a tree-like structure of an OpenXM
client-server communication.
\begin{figure}
\label{tree}
\begin{center}
\begin{picture}(200,140)(0,0)
\put(70,120){\framebox(40,15){client}}
\put(20,60){\framebox(40,15){server}}
\put(70,60){\framebox(40,15){server}}
\put(120,60){\framebox(40,15){server}}
\put(0,0){\framebox(40,15){server}}
\put(50,0){\framebox(40,15){server}}
\put(135,0){\framebox(40,15){server}}

\put(90,120){\vector(-1,-1){43}}
\put(90,120){\vector(0,-1){43}}
\put(90,120){\vector(1,-1){43}}
\put(40,60){\vector(-1,-2){22}}
\put(40,60){\vector(1,-2){22}}
\put(140,60){\vector(1,-3){14}}
\end{picture}
\caption{Tree-like structure of client-server communication}
\end{center}
\end{figure}
Such a computational model is useful for parallel implementation of
algorithms whose task can be divided into subtasks recursively.  A
typical example is {\it quicksort}, where an array to be sorted is
partitioned into two sub-arrays and the algorithm is applied to each
sub-array. In each level of recursion, two subtasks are generated
and one can ask other OpenXM servers to execute them. 
Though it makes little contribution to the efficiency in the case of
quicksort, we present an Asir program of this distributed quicksort
to demonstrate that OpenXM gives an easy way to test this algorithm.
In the program, a predefined constant {\tt LevelMax} determines
whether new servers are launched or whole subtasks are done on the server.

\begin{verbatim}
#define LevelMax 2
extern Proc1, Proc2;
Proc1 = -1$ Proc2 = -1$

/* sort [A[P],...,A[Q]] by quicksort */
def quickSort(A,P,Q,Level) {
  if (Q-P < 1) return A;
  Mp = idiv(P+Q,2); M = A[Mp]; B = P; E = Q;
  while (1) {
    while (A[B] < M) B++;
    while (A[E] > M && B <= E) E--;
    if (B >= E) break;
    else { T = A[B]; A[B] = A[E]; A[E] = T; E--; }
  }
  if (E < P) E = P;
  if (Level < LevelMax) {
   /* launch new servers if necessary */
   if (Proc1 == -1) Proc1 = ox_launch(0);
   if (Proc2 == -1) Proc2 = ox_launch(0);
   /* send the requests to the servers */
   ox_rpc(Proc1,"quickSort",A,P,E,Level+1);
   ox_rpc(Proc2,"quickSort",A,E+1,Q,Level+1);
   if (E-P < Q-E) {
     A1 = ox_pop_local(Proc1);
     A2 = ox_pop_local(Proc2);
   }else{
     A2 = ox_pop_local(Proc2);
     A1 = ox_pop_local(Proc1);
   }
   for (I=P; I<=E; I++) A[I] = A1[I];
   for (I=E+1; I<=Q; I++) A[I] = A2[I];
   return(A);
  }else{
   /* everything is done on this server */
   quickSort(A,P,E,Level+1);
   quickSort(A,E+1,Q,Level+1);
   return(A);
  }
}
\end{verbatim}

Another example is a parallelization of the Cantor-Zassenhaus
algorithm for polynomial factorization over finite fields.
It is a recursive algorithm similar to quicksort.
At each level of the recursion, a given polynomial can be
divided into two non-trivial factors with some probability by using 
a randomly generated polynomial as a {\it separator}.
In the following program, one of the two factors generated on a server
is sent to another server and the other factor is factorized on the server
itself. 
\begin{verbatim}
/* factorization of F */
/* E = degree of irreducible factors in F */
def c_z(F,E,Level)
{
  V = var(F); N = deg(F,V);
  if ( N == E ) return [F];
  M = field_order_ff(); K = idiv(N,E); L = [F];
  while ( 1 ) {
    /* generate a random polynomial */
    W = monic_randpoly_ff(2*E,V);
    /* compute a power of the random polynomial */
    T = generic_pwrmod_ff(W,F,idiv(M^E-1,2));
    if ( !(W = T-1) ) continue;
    /* G = GCD(F,W^((M^E-1)/2)) mod F) */
    G = ugcd(F,W);
    if ( deg(G,V) && deg(G,V) < N ) {
      /* G is a non-trivial factor of F */
      if ( Level >= LevelMax ) {
        /* everything is done on this server */
        L1 = c_z(G,E,Level+1);
        L2 = c_z(sdiv(F,G),E,Level+1);
      } else {
        /* launch a server if necessary */
        if ( Proc1 < 0 ) Proc1 = ox_launch();
        /* send a request with Level = Level+1 */
        /* ox_c_z is a wrapper of c_z on the server */
        ox_cmo_rpc(Proc1,"ox_c_z",lmptop(G),E,
            setmod_ff(),Level+1);
        /* the rest is done on this server */
        L2 = c_z(sdiv(F,G),E,Level+1);
        L1 = map(simp_ff,ox_pop_cmo(Proc1));
      }
      return append(L1,L2);
    }
  }
}
\end{verbatim}







