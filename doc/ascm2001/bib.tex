%$OpenXM: OpenXM/doc/ascm2001/bib.tex,v 1.2 2001/03/07 06:54:40 takayama Exp $

\begin{thebibliography}{X}
\bibitem{OpenMath}
The OpenMath Esprit Consortium 
(Caprotti, O. and Cohen, A.M. Editors),
The OpenMath Standard. D1.3.2a (Public) \\
{\footnotesize \tt http://www.nag.co.uk/projects/OpenMath},
February, 1999.
\bibitem{CLO}
Cox, D., Little, J.,  O'Shea,
{\it Using Algebraic Geometry}, Springer, 1998.
\bibitem{GKW}
Gray, S., Kajler, N. and Wang, P. S.,
Design and Implementation of MP, a Protocol for Efficient
  Exchange of Mathematical Expressions,
{\sl Journal of Symbolic Computation}, 1996.
\bibitem{Macaulay2}
Grayson, D. and Stillman, M.,
Macaulay2, a software system for research in algebraic geometry,
available at {\tt http://www.math.uiuc.edu/Macaulay2}.
\bibitem{Singular}
Greuel, G.-M. et al., SINGULAR : a computer algebra system for polynomial
computations, \\
{\tt http://www.mathematik.uni-kl.de/\~\,zca/Singular/}.
\bibitem{tigers}
Hubert, B. and Thomas, R.,
{\tt TiGERS} --- computing Gr\"obner fans of toric
ideals, 1998.
{\footnotesize
{\tt http://www.math.tamu.edu/\~\,rekha/programs.html}}
\bibitem{gap}
Linton, S. and Solomon, A.,
OpenMath, IAMC and {\tt GAP},
preprint, 1999.
\bibitem{MPI} Message Passing Interface,
{\tt http://www.mpi-forum.org} 
\bibitem{netsolve}
NetSolve, {\tt http://www.cs.utk.edu/netsolve}
\bibitem{asir} 
Noro, M. et al., 
A Computer Algebra System {\tt Risa/Asir},  1993, 1995, 2000\\
{\tt ftp://archives.cs.ehime-u.ac.jp/pub/asir2000/}
\bibitem{noro-mckay}
Noro, M. and McKay, J.,
Computation of replicable functions on Risa/Asir.
Proceedings of the Second International Symposium on
Symbolic Computation PASCO'97, ACM Press, 130-138 (1997).
\bibitem{ox-rfc-100}
Noro, M and Takayama, N., Design and Implementation
of OpenXM Client-Server Model and Common Mathematical Object Format
(OpenXM-RFC 100), 1996, 1997, 1998, 1999, 2000, 2001.
%%
\bibitem{oaku-advance}
Oaku, T.,
Algorithms for $b$-functions, restrictions, and algebraic local cohomology
groups of $D$-modules.
Advances in Applied Mathematics, {\bf 61}, 61--105, 1997.
%%
\bibitem{ox-rfc-101}
Ohara, K.,
Protocol to Start Engines (OpenXM-RFC 101),
2000.
\bibitem{openxm-web}
{\footnotesize {\tt http://www.math.kobe-u.ac.jp/OpenXM}}
or 
{\footnotesize {\tt http://www.openxm.org}}
\bibitem{sst-book}
Saito, M., Sturmfels, B. and Takayama, N.,
{\it Gr\"obner Deformations of Hypergeometric Differential Equations}.
Algorithms and Computation in Mathematics {\bf 6}. Springer, 1999.
\bibitem{schefstrom}
Schefstr\"om, D.,
Building a highly integrated development environment using
preexisting parts.
In IFIP 11th World Computer Congress, San Francisco, USA.
\bibitem{Shoup}
Shoup, V., 
A new polynomial factorization algorithm and 
its implementation,
{\sl Journal of Symbolic Computation}, 20, 364-397, 1996.
\bibitem{kan}
	Takayama, N.,
	{\em Kan: A system for computation in
	algebraic analysis,} 1991 version 1,
        1994 version 2, the latest version is 2.991106. 
	{\tt \small ftp.math.kobe-u.ac.jp/pub/kan}
\bibitem{phc}
Verschelde, J.,
PHCpack: A general-purpose solver for polynomial systems by
homotopy continuation.  ACM Transaction on Mathematical Softwares, 25(2) 
251-276, 1999.
\bibitem{iamc}
Wang, P.,
Design and Protocol for Internet Accessible Mathematical Computation.
Technical Report ICM-199901-001, ICM/Kent State University, 1999.
\bibitem{mathlink}
Wolfram, S.,
{\it The Mathematica Book, Fourth Edition}.
1999, Cambridge University Press.
\end{thebibliography}
