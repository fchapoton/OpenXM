%$OpenXM: OpenXM/doc/ascm2001/ascm2001.tex,v 1.3 2001/03/07 06:54:40 takayama Exp $
%% You need acmconf.cls and flushend.sty to compile this file.
%% They may be obtained from 
%%  http://riksun.riken.go.jp/archives/tex-archive/macros/latex/contrib/supported/acmconf/
\documentclass[submit]{acmconf}
%\documentclass{article}
%% \CopyrightText{\copyright 2000, }
\IfFileExists{graphicx.sty}{\usepackage{graphicx}}{}
\IfFileExists{epsfig.sty}{\usepackage{epsfig}}{}
\ConferenceName{ASCM 2001, Ehime, Japan, 2001}
\ConferenceShortName{ASCM 2001}
\def\OpenXM{{\rm OpenXM\ }}

\begin{document}
\date{March 7, 2001}
\title{The Design and Implementation of OpenXM-RFC 100 and 101}
\author{\Author{Masahide Maekawa}\\
         \Address{Kobe University}\\
         \Email{maekawa@math.kobe-u.ac.jp}\\
         \and
         \Author{Masayuki Noro}\\
         \Address{Kobe University}\\
         \Email{noro@math.kobe-u.ac.jp}
         \and
         \Author{Katsuyoshi Ohara}\\
         \Address{Kanazawa University}\\
         \Email{ohara@air.s.kanazawa-u.ac.jp}
         \and
         \Author{Nobuki Takayama}\\
         \Address{Kobe University}\\
         \Email{takayama@math.kobe-u.ac.jp}
         \and
         \Author{Yasushi Tamura}\\
         \Address{Kobe University}\\
         \Email{tamura@math.kobe-u.ac.jp}
       }
\maketitle

\begin{abstract}
\OpenXM is a free, or Open Source, infrastructure for mathematical
software systems.
It provides methods and protocols 
for interactive distributed computation and
for integrating mathematical software systems.
\OpenXM package version 1.1.3 
is a set of software systems that support \OpenXM-RFC 100 and
101 proposed standard protocols.
It is currently a collection of software systems
{\tt Asir} \cite{asir}, 
{\tt GNUPLOT}, 
{\tt Kan/sm1} \cite{kan}, 
{\tt Macaulay2} \cite{Macaulay2},
{\tt PHC} pack \cite{phc}, 
{\tt TiGERS}  \cite{tigers},
{\tt Mathematica} interface, and
{\tt OpenMath}/XML \cite{OpenMath} translator.
These are wrapped with the \OpenXM stack machine
to connect each other.
{\it Availability}: The OpenXM package and OpenXM-RFC's 
have been obtainable at \cite{openxm-web}
from January 24, 2000.
\end{abstract}

\begin{keywords}
Asir,
IAMC, Interactive Distributed Computation, 
Integration of Mathematical Software,
MP, OpenMath, OpenXM, OpenXM-RFC. 
\end{keywords}

%\section{Introduction}

% $OpenXM: OpenXM/doc/ascm2001p/design-outline.tex,v 1.1 2001/06/19 07:32:58 noro Exp $

\section{Design Outline and OpenXM Request for Comments (OpenXM-RFC)} 

As Schefstr\"om clarified in \cite{schefstrom},
integration of tools and software has three dimensions:
data, control, and user interface.

Data integration concerns with the exchange of data between different
software or same software.
OpenMath \cite{OpenMath} and MP (Multi Protocol) \cite{GKW} are,
for example, general purpose mathematical data protocols.
They provide standard ways to express mathematical objects.
For example,
\begin{verbatim}
 <OMOBJ>  <OMI> 123 </OMI> </OMOBJ>
\end{verbatim}
means the (OpenMath) integer $123$ in OpenMath/XML expression.

Control integration concerns with the establishment and management of
inter-software communications.
Control involves, for example, a way to ask computations to other processes
and a method to interrupt computations on servers from a client.
RPC, HTTP, MPI, PVM are regarded as a general purpose control protocols or
infrastructures.
MCP (Mathematical Communication Protocol)
by Wang \cite{iamc} and OMEI \cite{omei} are such protocols for mathematics.

Although data and control are orthogonal to each other,
real world requires both.
NetSolve \cite{netsolve}, OpenMath$+$MCP, MP$+$MCP \cite{iamc},
and MathLink \cite{mathlink} provide both data and control integration.
Each integration method has their own features determined by their
own design goals.
OpenXM (Open message eXchange protocol for Mathematics)
is a project aiming to integrate data, control and user interfaces
with design goals motivated by the followings.
\begin{enumerate}
\item We should test the proposed standards mentioned above on
various mathematical software systems, but the testing has not been
enough.
\item Noro has been involved in the development of 
a computer algebra system Risa/Asir \cite{asir}.
An interface for interactive distributed computations was introduced
to Risa/Asir 
%% version 950831 released 
in 1995.
The model of computation was RPC (remote procedure call).
A robust interruption protocol was provided 
by  two communication channels
like the File Transfer Protocol (ftp).
As an application of this protocol,
a parallel speed-up was achieved for a Gr\"obner basis computation
to determine all odd order replicable functions 
(Noro and McKay \cite{noro-mckay}).
However, the protocol was local in Asir and we thought that we should
design an open protocol.
\item Takayama has developed
a special purpose system Kan/sm1 \cite{kan},
which is a Gr\"obner engine for the ring of differential operators $D$. 
In order to implement algorithms in $D$-modules due to Oaku 
(see, e.g., \cite{sst-book}),
factorizations and primary ideal decompositions are necessary.
Kan/sm1 does not have an implementation for these and called
Risa/Asir as a UNIX external program.
This approach was not satisfactory.
Especially, we could not write a clean interface code between these
two systems.
We thought that it is necessary to provide a data and control protocol
for Risa/Asir to work as a server of factorization and primary ideal
decomposition.
\item We have been profited from increasing number 
of mathematical software.
These are usually ``expert'' systems in one area of mathematics
such as ideals, groups, numbers, polytopes, and so on.
They have their own interfaces and data formats,
which are fine for intensive users of these systems.
However, a unified system will be more convenient.
%for users who want to explore a new area of mathematics with these
%software or users who need these systems only occasionally.

\item  We believe that an open integrated system is a future of mathematical
software.
However, it might be just a dream without realizability.
We want to build a prototype of such an open system by using
existing standards, technologies and several mathematical software.
We want to see how far we can go with this approach.
\end{enumerate}

Motivated with these, we started the OpenXM project with the following
fundamental architecture, which is currently described in
OpenXM-RFC 100  proposed standard %% ``draft standard'' and ``standard''
``Design and Implementation of OpenXM client-server model and common
mathematical object format'' \cite{ox-rfc-100}.
\begin{enumerate}
\item Communication is an exchange of messages. The messages are classified into
three types:
DATA, COMMAND, and SPECIAL.
They are called OX (OpenXM) messages.
Among the three types,
{\it OX data messages} wrap mathematical data.
We use standards of mathematical data formats such as OpenMath and MP
as well as our own data format {\it CMO}
({\it Common Mathematical Object format}),
which can be expressed in terms of XML.
\item Servers, which provide services to other processes, are stack machines.
The stack machine is called the
{\it OX stack machine}.
Existing mathematical software systems are wrapped with this stack machine.
Minimal requirements for a target software wrapped with the OX stack machine
are as follows:
\begin{enumerate}
\item The target must have a serialized interface such as a character based
interface.
\item An output of the target must be understandable for computer programs;
it should follow a grammar that can be parsed with other software.
\end{enumerate}
\item Any server may have a hybrid interface;
it may accept and execute not only stack machine commands, 
but also its original command sequences.
For example,
if we send the following string to the {\tt ox\_asir} server 
(OpenXM server of Risa/Asir) \\
\verb+        " fctr(x^100-y^100); "      + \\
and call the stack machine command  \\
\verb+        SM_executeStringByLocalParser    + \\ 
then the server executes the asir command \\
\verb+ fctr(x^100-y^100); + 
(factorize $x^{100}-y^{100}$ over ${\bf Q}$)
and pushes the result onto the stack.
\end{enumerate}
OpenXM package  implements the OpenXM-RFC 100 \cite{ox-rfc-100}
and 101 \cite{ox-rfc-101} based on
the above fundamental architecture.
In this paper, we discuss mainly on systems implementing
OpenXM-RFC 100 and 101 on TCP/IP.
For example, the following is a command sequence to ask $1+1$ from
the Asir client to the {\tt ox\_sm1} server through TCP/IP:
\begin{verbatim}
  P = sm1_start();
  ox_push_cmo(P,1); ox_push_cmo(P,1);
  ox_execute_string(P,"add"); ox_pop_cmo(P);
\end{verbatim}
Here, {\tt ox\_sm1} is an OpenXM server of Kan/sm1.

Our project of integrating mathematical software
systems is taking the ``RFC'' approach, which has been
used to develop internet protocols.
We think that ``RFC'' approach is an excellent way and
we hope that other groups, who are working on standard protocols,
take this ``RFC'' approach, too.

The OpenXM on MPI \cite{MPI} is currently running on Risa/Asir
as we will see in Section \ref{section:homog}.
We are now preparing the OpenXM-RFC 102 ``Mathematical communication
on MPI'' (draft protocol)
based on our experiments on MPI.

In the rest of the paper, we abbreviate
OpenXM-RFC 100 and 101 to OpenXM if no confusion arises.











%%$OpenXM: OpenXM/doc/issac2000/ox-messages.tex,v 1.4 2000/01/13 10:57:10 ohara Exp $

\section{OX messages}

An OX message for TCP/IP is a byte stream consisting of
a header and a body.
\begin{center}
\begin{tabular}{|c|c|}
\hline
Header	& \hspace{10mm} Body \hspace{10mm} \\
\hline
\end{tabular}
\end{center}
The header consists of two signed 32 bit integers.
The first one is an OX tag 
and the second one is a serial number of the OX message.
Negative numbers are expressed by the two's complement.
Several byte orders including the network byte order
are allowed and the byte order is determined as a part of
the establishment of a connection. See Section \ref{secsession} for details.

The OX messages are classified into three types:
DATA, COMMAND, and SPECIAL.
We have the following main tags for the OX messages.
\begin{verbatim}
#define	OX_COMMAND               513  // COMMAND
#define	OX_DATA	                 514  // DATA
#define OX_SYNC_BALL             515  // SPECIAL
#define OX_DATA_WITH_LENGTH      521  // DATA
#define OX_DATA_OPENMATH_XML     523  // DATA
#define OX_DATA_OPENMATH_BINARY  524  // DATA
#define OX_DATA_MP               525  // DATA
\end{verbatim}

New OX tags may be added.
The new tag should be classified into DATA or COMMAND.
For example, \verb+ OX_DATA_ASIR_LOCAL_BINARY +  was added a few month ago
to send internal serialized objects of asir via the OpenXM protocol.
This is a tag classified to DATA.
See the web page of OpenXM to add a new tag.
The server is a stack machine (see Section~\ref{sec:ox-stackmachines}
for detail).
{\it OX data} message sent by the client
are pushed onto the stack of the server. 
If the server gets an {\it OX command} message, then the server extracts
a stack machine code in the OX command message and interprets the code.
For example, in case of SM\_executeFunction, some data are popped from
the stack and they are used as arguments of a function call.

We explain an implementation of handling OX messages.
For example, the asir command {\tt ox\_push\_cmo(P,1)}
(push integer $1$ onto the server $P$)
sends an OX data message
{\tt (OX\_DATA,(CMO\_ZZ,1))} to the server $P$.
Here,
OX\_DATA stands for OX\_DATA header and 
{\tt (CMO\_ZZ,1)} is a body standing for $1$ expressed 
in the CMO data encoding format.
The server tranlates $(CMO\_ZZ, 1)$ to its own internal object fotrmat
for integers and pushs the object onto the stack.

%An OpenXM client admit that its own command sends some OX messages
%sequentially at once.  
%
%For example, the asir command
%{\tt ox\_execute\_string(P, "Print[x+y]")} sends an OX data message
%{\tt (OX\_DATA, (CMO\_STRING, "Print[x+y]"))} and an OX command message
%{\tt (OX\_COMMAND, (SM\_executeStringByLocalParser))} to an OpenXM
%server.


% $OpenXM: OpenXM/doc/issac2000/data-format.tex,v 1.5 2000/01/13 10:58:16 ohara Exp $

\section{Data Format}   (Ohara)

OpenXM admits multiple mathematical encodings such as OpenMath, MP, CMO
(Common Mathematical Object format).
OpenXM itself does not exhibit a bias towards a particular encodings 
as a main mathematical data carrier and an OpenXM compliant system do not need to
implement all possible data formats.
A server or a client does not necessarily implement full specifications
of OpenXM. However  they should at least implement seven primitive 
data types of the CMO, which are necessary to 
carry several control informations such as a {\it mathcap}.
Mathcap is a list of supported CMO's, OpenXM stack machine codes, 
and necessary extra informations.
If a program sends an OX messages to its peer, 
an unrecoverable error may occur. 
By exchanging mathcaps a program knows its peer's capability 
and such an error can be avoided.
Mathcap is also defined as a CMO and the CMO has a structure of a nest
of lists.  its leafs of the end are also a CMO which tags with
CMO\_INT32 or CMO\_STRING.

Mathematical encoding types of OX data are distinguished with tags
of OX messages.
For example,
an OX message with the tag 
OX\_DATA is followed by a CMO packet.
An OX message with the tag 
OX\_DATA\_OPENMATH\_XML is followed by 
an OpenMath XML string and its length.

Let us explain the data format of CMO.
Any CMO packet consists of a header and a body.
The size of the header is 4 bytes that tags the data type of the body.
Data type tags are signed 32 bit integers which is called {\sl int32} in this
paper.
Following tags are registered in the OpenXM for now.
\begin{verbatim}
#define CMO_ERROR2                         0x7f000002
#define CMO_NULL                           1
#define CMO_INT32                          2
#define CMO_DATUM                          3
#define CMO_STRING                         4
#define CMO_MATHCAP                        5
#define CMO_LIST                           17

#define CMO_MONOMIAL32                     19
#define CMO_ZZ                             20
#define CMO_QQ                             21
#define CMO_ZERO                           22
#define CMO_DMS_GENERIC                    24
#define CMO_DMS_OF_N_VARIABLES             25
#define CMO_RING_BY_NAME                   26
#define CMO_RECURSIVE_POLYNOMIAL           27
#define CMO_LIST_R                         28
#define CMO_INT32COEFF                     30
#define CMO_DISTRIBUTED_POLYNOMIAL         31
#define CMO_POLYNOMIAL_IN_ONE_VARIABLE     33
#define CMO_RATIONAL                       34
#define CMO_64BIT_MACHINE_DOUBLE           40
#define CMO_ARRAY_OF_64BIT_MACHINE_DOUBLE  41
#define CMO_128BIT_MACHINE_DOUBLE          42
#define CMO_ARRAY_OF_128BIT_MACHINE_DOUBLE 43
#define CMO_BIGFLOAT                       50
#define CMO_IEEE_DOUBLE_FLOAT              51
#define CMO_INDETERMINATE                  60
#define CMO_TREE                           61
#define CMO_LAMBDA                         62
\end{verbatim}
The first seven types should be implemented on all OpenXM compliant systems.
The format of the first seven CMO's are as follows. \\
\begin{tabular}{|c|c|}
\hline
{\sl int32} {\tt CMO\_ERROR2} & {\sl CMObject} {\rm ob} \\ 
\hline
\end{tabular} \\
\begin{tabular}{|c|c|}
\hline
{\sl int32} {\tt CMO\_NULL}  \\ 
\hline
\end{tabular} \\
\begin{tabular}{|c|c|}
\hline
{\sl int32} {\tt CMO\_INT32}& {\sl int32} {\rm n}  \\ 
\hline
\end{tabular} \\
\begin{tabular}{|c|c|c|}
\hline
{\sl int32} {\tt CMO\_DATUM}& {\sl int32} {\rm n} & {\sl byte} {\rm  data[0]} \\
\hline
$\cdots$ & {\sl byte} {\rm  data[n-1]} \\ 
\cline{1-2}
\end{tabular} \\
\begin{tabular}{|c|c|c|}
\hline
{\sl int32} {\tt CMO\_STRING}& {\sl int32} {\rm n} & {\sl byte} {\rm data[0]} \\
\hline
$\cdots$ & {\sl byte} {\rm data[n-1]} \\ 
\cline{1-2}
\end{tabular} \\
\begin{tabular}{|c|c|}
\hline
{\sl int32} {\tt CMO\_MATHCAP} & {\sl CMObject} {\rm ob} \\ 
\hline
\end{tabular} \\
\begin{tabular}{|c|c|c|}
\hline
{\sl int32} {\tt CMO\_LIST}& {\sl int32} {\rm n} & {\sl CMObject} {\rm ob[0]} \\
\hline
$\cdots$ & {\sl CMObject} {\rm ob[n-1]} \\ 
\cline{1-2}
\end{tabular} \\

A mathematical programmer who wants to implement CMO on a server proceeds 
as follows.
\begin{enumerate}
\item Look for the CMO list at the web cite \cite{openxm-web}.
If there is a CMO that fits to her or his requirement, then use this CMO.     
\item If there is no suitable CMO, design a new CMO and register 
the new CMO to \cite{openxm-web} with a description and examples.
\end{enumerate}


% $OpenXM: OpenXM/doc/issac2000/openxm-stackmachines.tex,v 1.2 2000/01/02 07:32:12 takayama Exp $

\section{OpenXM Stackmachines}   (Tamura)

In OpenXM specification, all servers are stackmachines.
These are called OX stachmachines.
When a server ox\_xyz gets an OX data message,
it translates the data into its own object and push the object
on the stack.
The translation scheme together with definitions of 
mathematical operations
of the system ox\_xyz is called the {\it phrase dictionary} of
ox\_xyz following the OpenMath specification.

Any OX command message starts with the int32 tag OX\_COMMAND.
The body is OX stackmachine operation code expressed by int32.
The codes are listed below.
\begin{verbatim}
#define SM_popSerializedLocalObject               258
#define SM_popCMO                                 262
#define SM_popString                              263
#define SM_mathcap                                264
#define SM_pops                                   265
#define SM_setName                                266
#define SM_evalName                               267
#define SM_executeStringByLocalParser             268
#define SM_executeFunction                        269
#define SM_beginBlock                             270
#define SM_endBlock                               271
#define SM_shutdown                               272
#define SM_setMathCap                             273
#define SM_executeStringByLocalParserInBatchMode  274
#define SM_getsp                                  275
#define SM_dupErrors                              276
#define SM_DUMMY_sendcmo                          280
#define SM_sync_ball                              281
#define SM_control_kill                          1024
#define SM_control_to_debug_mode                 1025
#define SM_control_exit_debug_mode               1026
#define SM_control_reset_connection              1030
\end{verbatim}

OpenXM does not have a standard for mathematical operation sets
while it is a work in progress in \cite{gap}.
Each OpenXM server has its own mathematical operation set.
Mathematical operations are performed as follows.
Mathematical operator name, such as fctr (asir factorization command),
is pushed as a string,
the stackmachine command
SM\_executeFunction (269) pops the operator name, the number of arguments
and arguments, and
the OX stackmachine evaluates the operator, and pushes the result on the stack.
For example, the following code factorizes $x^{100}-1$ by calling
ox\_asir from asir.
\begin{verbatim}
P = ox_launch(); 
ox_push_cmo(P,x^100-1); ox_push_cmo(P,ox_int32(1));
ox_push_cmd(P,269); 
Ans = ox_pop_cmo(P);
\end{verbatim}

When an error has occurred on an OpenXM server,
an error object is pushed to the stack instead of a result of the computation.
The error object consists of the serial number of the OX message
which caused the error, and an error message.
\begin{verbatim}
[341] ox_rpc(0,"fctr",1.2*x)$
[342] ox_pop_cmo(0);
error([8,fctr : invalid argument])
\end{verbatim}

Errors are not sent to the client except a SM\_pop* command is received.
OX stackmachines works in the asynchronous mode which is similar 
to X servers.
For servers for graphic applications, it is an advantageous feature.
It is also easy to emulate RPC and a web server for MCP \cite{iamc} 
on our asynchronous OX stackmachines.







% $OpenXM: OpenXM/doc/issac2000/session-management.tex,v 1.12 2000/01/17 07:15:52 noro Exp $

\section{Session Management}
\label{secsession}
%MEMO: key words:
%Security (ssh PAM), initial negotiation of byte order,
%mathcap, interruption, debugging window, etc.
 
In this section we explain our control integration in
OpenXM.  We assume that various clients and servers
establish connections dynamically and communicate to each
other. Therefore it is necessary to give a dynamical and unified
method to start servers and to establish connections.
In addition to that, interruption of execution and 
debugging facilities
are necessary for interactive distributed computation.

%\subsection{Interface of servers}
%
%A server has additional I/O streams for exchanging data between
%a client and itself other than ones for diagnostic
%messages. As the streams are for binary data,
%the byte order conversion is necessary when a
%client and a server have different byte orders. It is determined by
%exchanging the preferable byte order of each peer. If the preference
%does not coincide with each other,
%then the network byte order is used.
%This implies that all servers and clients should be able to
%handle the network byte
%order. Nevertheless it is necessary to negotiate the byte order to
%skip the byte order conversion because its cost is often dominant over
%fast networks.

\subsection{Invocation of servers}
\label{launcher}

An application called {\it launcher} is provided to start servers
and to establish connections as follows.

\begin{enumerate}
\item A launcher is invoked from a client.
When the launcher is invoked, the client 
informs the launcher of a port number for TCP/IP connection
and the name of a server.
\item The launcher and the client establish a connection with the
specified port number. One time password may be used to prevent
launcher spoofing.
\item The launcher creates a process and executes the server after
setting the data channel appropriately.
\end{enumerate}

After finishing the above task as a launcher, the launcher process
acts as a control server and controls the server process created by
itself. As to the details of the control server see Section \ref{control}.

As the data channel is used to exchange binary data,
the byte order conversion is necessary when a
client and a server have different byte orders. It is determined by
exchanging the preferable byte order of each peer. If the preference
does not coincide with each other,
then the network byte order is used.
This implies that all servers and clients should be able to
handle the network byte
order. Nevertheless it is necessary to negotiate the byte order to
skip the byte order conversion because its cost is often dominant over
fast networks.


\subsection{Control server}
\label{control}
In OpenXM we adopted the following simple and robust method to 
control servers.

An OpenXM server has logically two I/O channels: one for exchanging
data for computations and the other for controlling computations. The
control channel is used to send commands to control execution on the
server. The launcher introduced in Section \ref{launcher}
is used as a control process. We call such a process a {\it
control server}. In contrast, we call a server for computation an {\it
engine}. As the control server and the engine runs on the
same machine, it is easy to send a signal from the control server. 
A control server is also an
OpenXM stack machine and it accepts {\tt SM\_control\_*} commands
to send signals to a server or to terminate a server.

\subsection{Resetting an engine}

A client can send a signal to an engine by using the control channel 
at any time. However, I/O operations are usually buffered,
which may cause troubles.
To reset an engine safely the following are required.

\begin{enumerate}
\item Any OX message must be a synchronized object in the sense of Java.

As an OX message is sent as a combination of several {\tt CMO}
data, a global exit without sending all may generate broken data.

\item After restarting an engine, a request from a client 
must correctly corresponds to the response from the engine.

An incorrect correspondence occurs if some data remain on the stream
after restarting an engine.
\end{enumerate}

{\tt SM\_control\_reset\_connection} is a stack machine command to
initiate a safe resetting of an engine.
The control server sends {\tt SIGUSR1} to the engine if it receives
{\tt SM\_control\_reset\_connection} from the client.
Under the OpenXM reset protocol, an engine and a client act as follows.

\vskip 2mm
\noindent
{\it Client side} 
\begin{enumerate}
\item After sending {\tt SM\_control\_reset\_connection} to the
control server, the client enters the resetting state. It discards all {\tt
OX} messages from the engine until it receives {\tt OX\_SYNC\_BALL}.
\item After receiving {\tt OX\_SYNC\_BALL} the client sends 
{\tt OX\_SYNC\_BALL} to the engine and returns to the usual state.
\end{enumerate}

\noindent
{\it Engine side}
\begin{enumerate}
\item 
After receiving {\tt SIGUSR1} from the control server,
the engine enters the resetting state.
The engine sends {\tt OX\_SYNC\_BALL} to the client.
The operation does not block because
the client is now in the resetting state.
\item The engine discards all OX messages from the engine until it
receives {\tt OX\_SYNC\_BALL}. After receiving {\tt OX\_SYNC\_BALL} 
the engine returns to the usual state.
\end{enumerate}

\begin{figure}[htbp]
\epsfxsize=8.5cm
\epsffile{reset.eps}
\caption{OpenXM reset procedure}
\label{reset}
\end{figure}

Figure \ref{reset} illustrates the flow of data.
{\tt OX\_SYNC\_BALL} is used to mark the end of data remaining in the
I/O streams. After reading it, it is assured that each stream is empty
and that the subsequent request from a client correctly 
corresponds to the response from the engine.
We note that we don't have to associate {\tt OX\_SYNC\_BALL} with
any special action to be executed by the engine because it is
assured that the engine is in the resetting state when it has received
{\tt OX\_SYNC\_BALL}.

\subsection{Debugging facilities}
Debugging is sometimes very hard for distributed computations.
We provide two methods to help debugging on X window system:
1. the diagnostic messages from the engine are displayed in a {\tt xterm}
window;
2. the engine can pop up a window to input debug commands.
For example {\tt ox\_asir}, which is
the OpenXM server of Risa/Asir, can pop up a window to input
debug commands and the debugging similar to that on usual terminals is possible.
One can also send {\tt SIGINT} by using {\tt SM\_control\_to\_debug\_mode}
and it provides a similar functionality to entering the debugging
mode from a keyboard interruption.


% $OpenXM: OpenXM/doc/issac2000/openxm-clients.tex,v 1.1 1999/12/23 10:25:08 takayama Exp $

\section{OpenXM Clients}    
(noryo and Ohara)
MEMO: keywords:
Asir and Mathematica clients.


%% $OpenXM$
\section{ 1077 functions are available on our servers and libraries}

This is a list of examples and functions which our
servers provide.
For details, see manuals of each system.

\noindent
\fbox{\large {Operations on Integers}}

\noindent
{idiv},{irem} (division with remainder),
{ishift} (bit shifting),
{iand},{ior},{ixor} (logical operations),
{igcd},(GCD by various methods such as Euclid's algorithm and
the accelerated GCD algorithm),
{fac} (factorial),
{inv} (inverse modulo an integer),
{random} (random number generator by the Mersenne twister algorithm).

\medbreak
\noindent
\fbox{\large {Ground Fields}}

\noindent
Arithmetics on various fields: the rationals,
${\bf Q}(\alpha_1,\alpha_2,\ldots,\alpha_n)$
($\alpha_i$ is algebraic over ${\bf Q}(\alpha_1,\ldots,\alpha_{i-1})$),
$GF(p)$ ($p$ is a prime of arbitrary size), $GF(2^n)$.

\medbreak
\noindent
\fbox{\large {Operations on Polynomials}}

\noindent
{sdiv }, {srem } (division with remainder),
{ptozp } (removal of the integer content),
{diff } (differentiation),
{gcd } (GCD over the rationals),
{res } (resultant),
{subst } (substitution),
{umul} (fast multiplication of dense univariate polynomials
by a hybrid method with Karatsuba and FFT+Chinese remainder),
{urembymul\_precomp} (fast dense univariate polynomial
division with remainder by the fast multiplication and
the precomputed inverse of a divisor),

\noindent
\fbox{\large {Polynomial Factorization}}
{fctr } (factorization over the rationals),
{fctr\_ff } (univariate factorization over finite fields),
{af } (univariate factorization over algebraic number fields),
{sp} (splitting field computation).

\medbreak
\noindent
\fbox{\large {Groebner basis}}

\noindent
{dp\_gr\_main } (Groebner basis computation of a polynomial ideal
over the rationals by the trace lifting),
{dp\_gr\_mod\_main } (Groebner basis over small finite fields),
{tolex } (Modular change of ordering for a zero-dimensional ideal),
{tolex\_gsl } (Modular rational univariate representation
for a zero-dimensional ideal),
{dp\_f4\_main } ($F_4$ over the rationals),
{dp\_f4\_mod\_main } ($F_4$ over small finite fields).

\medbreak
\noindent
\fbox{\large {Ideal Decomposition}}

\noindent
{primedec} (Prime decomposition of the radical),
{primadec} (Primary decomposition of ideals by Shimoyama/Yokoyama algorithm).

\medbreak
\noindent
\fbox{\large {Quantifier Elimination}}

\noindent
{qe} (real quantifier elimination in a linear and
quadratic first-order formula),
{simpl} (heuristic simplification of a first-order formula).

%%$
{\scriptsize
\begin{verbatim}
[0] MTP2 = ex([x11,x12,x13,x21,x22,x23,x31,x32,x33],
x11+x12+x13 @== a1 @&& x21+x22+x23 @== a2 @&& x31+x32+x33 @== a3
@&& x11+x21+x31 @== b1 @&& x12+x22+x32 @== b2 @&& x13+x23+x33 @== b3
@&& 0 @<= x11 @&& 0 @<= x12 @&& 0 @<= x13 @&& 0 @<= x21
@&& 0 @<= x22 @&& 0 @<= x23 @&& 0 @<= x31 @&& 0 @<= x32 @&& 0 @<= x33)$
[1] TSOL= a1+a2+a3@=b1+b2+b3 @&& a1@>=0 @&& a2@>=0 @&& a3@>=0
@&& b1@>=0 @&& b2@>=0 @&& b3@>=0$
[2] QE_MTP2 = qe(MTP2)$
[3] qe(all([a1,a2,a3,b1,b2,b3],QE_MTP2 @equiv TSOL));
@true
\end{verbatim}}

\medbreak
\noindent
\fbox{\large {Visualization of curves}}

\noindent
{plot} (plotting of a univariate function),
{ifplot} (plotting zeros of a bivariate polynomial),
{conplot} (contour plotting of a bivariate polynomial function).

\medbreak
\noindent
\fbox{\large {Miscellaneous functions}}

\noindent
{det} (determinant),
{qsort} (sorting of an array by the quick sort algorithm),
{eval} (evaluation of a formula containing transcendental functions
such as
{sin}, {cos}, {tan}, {exp},
{log})
{roots} (finding all roots of a univariate polynomial),
{lll} (computation of an LLL-reduced basis of a lattice).

\medbreak
\noindent
\fbox{\large {$D$-modules}} ($D$ is the Weyl algebra)

\noindent
{gb } (Gr\"obner basis),
{syz} (syzygy),
{annfs} (Annhilating ideal of $f^s$),
{bfunction},
{schreyer} (free resolution by the Schreyer method),
{vMinRes} (V-minimal free resolution),
{characteristic} (Characteristic variety),
{restriction} in the derived category of $D$-modules,
{integration} in the derived category,
{tensor}  in the derived category,
{dual} (Dual as a D-module),
{slope}.

\medbreak
\noindent
\fbox{\large {Cohomology groups}}

\noindent
{deRham} (The de Rham cohomology groups of
${\bf C}^n \setminus V(f)$,
{ext} (Ext modules for a holonomic $D$-module $M$
and the ring of formal power series).

\medbreak
\noindent
\fbox{\large {Differential equations}}

\noindent
Helping to derive and prove {combinatorial} and
{special function identities},
{gkz} (GKZ hypergeometric differential equations),
{appell} (Appell's hypergeometric differential equations),
{indicial} (indicial equations),
{rank} (Holonomic rank),
{rrank} (Holonomic rank of regular holonomic systems),
{dsolv} (series solutions of holonomic systems).

\medbreak
\noindent
\fbox{\large {OpenMATH support}}

\noindent
{om\_xml} (CMO to OpenMATH XML),
{om\_xml\_to\_cmo} (OpenMATH XML to CMO).

\medbreak
\noindent
\fbox{\large {Homotopy Method}}

\noindent
{phc} (Solving systems of algebraic equations by
numerical and polyhedral homotopy methods).

\medbreak
\noindent
\fbox{\large {Toric ideal}}

\noindent
{tigers} (Enumerate all Gr\"obner basis of a toric ideal.
Finding test sets for integer program),
{arithDeg} (Arithmetic degree of a monomial ideal),
{stdPair} (Standard pair decomposition of a monomial ideal).

\medbreak
\noindent
\fbox{\large {Communications}}

\noindent
{ox\_launch} (starting a server),
{ox\_launch\_nox},
{ox\_shutdown},
{ox\_launch\_generic},
{generate\_port},
{try\_bind\_listen},
{try\_connect},
{try\_accept},
{register\_server},
{ox\_rpc},
{ox\_cmo\_rpc},
{ox\_execute\_string},
{ox\_reset} (reset the server),
{ox\_intr},
{register\_handler},
{ox\_push\_cmo},
{ox\_push\_local},
{ox\_pop\_cmo},
{ox\_pop\_local},
{ox\_push\_cmd},
{ox\_sync},
{ox\_get},
{ox\_pops},
{ox\_select},
{ox\_flush},
{ox\_get\_serverinfo}

\medbreak
\noindent
In addition to these functions, {Mathematica functions}
can be called as server functions.

\medbreak
\noindent
\fbox{\large {Examples}}
{\footnotesize
\begin{verbatim}
[345] sm1_deRham([x^3-y^2*z^2,[x,y,z]]);
[1,1,0,0]
/* dim H^i = 1 (i=0,1), =0 (i=2,3) */
\end{verbatim}}

%%\noindent
%%{\footnotesize \begin{verbatim}
%%[287] phc(katsura(7)); B=map(first,Phc)$
%%[291] gnuplot_plotDots(B,0)$
%%\end{verbatim} }

% \epsfxsize=3cm
% \begin{center}
% %\epsffile{../calc2000/katsura7.ps}
% \epsffile{katsura7.ps}
% \end{center}
%%The first components of the solutions to the system of algebraic equations Katsura 7.

\medbreak
\noindent
\fbox{ {Authors}}
Castro-Jim\'enez, Dolzmann, Hubert, Murao, Noro, Oaku, Okutani,
Shimoyama, Sturm, Takayama, Tamura, Verschelde, Yokoyama.


%$OpenXM$


% $OpenXM: OpenXM/doc/calc2000/heterotic-network.tex,v 1.1.1.1 2000/04/24 04:20:11 noro Exp $
\section{Applications}

\subsection{Heterogeneous Servers}

\def\pd#1{ \partial_{#1} }

By using OpenXM, we can treat OpenXM servers essentially 
like a subroutine.
Since OpenXM provides a universal stack machine which does not
depend on each server, 
it is relatively easy to install new servers.
We can build a new computer math system by assembling
different OpenXM servers.
It is similar to building a toy house by LEGO blocks.

We will see two examples of custom-made systems
built by OpenXM servers.

\subsubsection{Computation of annihilating ideals by kan/sm1 and ox\_asir}

Let $D = {\bf Q} \langle x_1, \ldots, x_n , \pd{1}, \ldots, \pd{n} \rangle$
be the ring of differential operators.
For a given polynomial
$ f \in {\bf Q}[x_1, \ldots, x_n] $,
the annihilating ideal of $f^{-1}$ is defined as
$$ {\rm Ann}\, f^{-1} = \{ \ell \in D \,|\,
  \ell \bullet f^{-1} = 0 \}.
$$
Here, $\bullet$ denotes the action of $D$ to functions.
The annihilating ideal can be regarded as the maximal differential
equations for the function $f^{-1}$.
An algorithm to determine generators of the annihilating ideal
was given by Oaku (see, e.g., \cite[5.3]{sst-book}).
His algorithm reduces the problem to computations of Gr\"obner bases
in $D$ and to find the minimal integral root of a polynomial.
This algorithm (the function {\tt annfs}) is implemented by
kan/sm1 \cite{kan}, for Gr\"obner basis computation in $D$, and
{\tt ox\_asir}, to factorize polynomials to find the integral
roots.
These two OpenXM compliant systems are integrated by
the OpenXM protocol.

For example, the following is a sm1 session to find the annihilating
ideal for $f = x^3 - y^2 z^2$.
\begin{verbatim}
sm1>[(x^3-y^2 z^2) (x,y,z)] annfs ::
Starting ox_asir server.
Byte order for control process is network byte order.
Byte order for engine process is network byte order.
[[-y*Dy+z*Dz, 2*x*Dx+3*y*Dy+6, -2*y*z^2*Dx-3*x^2*Dy, 
-2*y^2*z*Dx-3*x^2*Dz, -2*z^3*Dx*Dz-3*x^2*Dy^2-2*z^2*Dx], 
 [-1,-139968*s^7-1119744*s^6-3802464*s^5-7107264*s^4
     -7898796*s^3-5220720*s^2-1900500*s-294000]] 
\end{verbatim}
The last polynomial is factored as
$-12(s+1)(3s+5)(3s+4)(6s+5)(6s+7)$
and the minimal integral root is $-1$
as shown in the output.

Similarly, 
an algorithm to stratify singularity 
\cite{oaku-advance}
is implemented by
kan/sm1 \cite{kan}, for Gr\"obner basis computation in $D$, and
{\tt ox\_asir}, for primary ideal decompositions.

\subsubsection{A Course on Solving Algebraic Equations}

Risa/Asir \cite{asir} is a general computer algebra system
which can be used for Gr\"obner basis computations for zero dimensional ideal
with ${\bf Q}$ coefficients.
However, it is not good at graphical presentations and
numerical methods.
We integrated Risa/Asir, ox\_phc (based on PHC pack by Verschelde \cite{phc}
for the polyhedral homotopy method) and
ox\_gnuplot (GNUPLOT) servers
to teach a course on solving algebraic equations.
This course was presented with the text book \cite{CLO},
which discusses 
on the Gr\"obner basis method and the polyhedral homotopy method
to solve systems of algebraic equations.
We taught the course
with a unified environment
controlled by the Asir user language, which is similar to C.
The following is an Asir session to solve algebraic equations by calling
the PHC pack (Figure \ref{katsura} is the output of {\tt [292]}):
\begin{verbatim}
[287] phc(katsura(7));
The detailed output is in the file tmp.output.*
The answer is in the variable Phc.
0
[290] B=map(first,Phc)$
[291] gnuplot_plotDots([],0)$
[292] gnuplot_plotDots(B,0)$
\end{verbatim}

\begin{figure}[htbp]
\epsfxsize=8cm
\begin{center}
\epsffile{katsura7.ps}
\end{center}
\caption{The first components of the solutions to the system of algebraic equations Katsura 7.}
\label{katsura}
\end{figure}





% $OpenXM: OpenXM/doc/issac2000/homogeneous-network.tex,v 1.10 2000/01/17 07:06:53 noro Exp $

\subsection{Distributed computation with homogeneous servers}
\label{section:homog}

One of the aims of OpenXM is a parallel speedup by a distributed computation
with homogeneous servers. As the current specification of OpenXM does
not include communication between servers, one cannot expect
the maximal parallel speedup. However it is possible to execute
several types of distributed computation as follows.

\subsubsection{Product of univariate polynomials}

Shoup \cite{Shoup} showed that the product of univariate polynomials
with large degrees and large coefficients can be computed efficiently
by FFT over small finite fields and Chinese remainder theorem.
It can be easily parallelized:

\begin{tabbing}
Input :\= $f_1, f_2 \in {\bf Z}[x]$ such that $deg(f_1), deg(f_2) < 2^M$\\
Output : $f = f_1f_2$ \\
$P \leftarrow$ \= $\{m_1,\cdots,m_N\}$ where $m_i$ is an odd prime, \\
\> $2^{M+1}|m_i-1$ and $m=\prod m_i $ is sufficiently large. \\
Separate $P$ into disjoint subsets $P_1, \cdots, P_L$.\\
for \= $j=1$ to $L$ $M_j \leftarrow \prod_{m_i\in P_j} m_i$\\
Compute $F_j$ such that $F_j \equiv f_1f_2 \bmod M_j$\\
\> and $F_j \equiv 0 \bmod m/M_j$ in parallel.\\
\> (The product is computed by FFT.)\\
return $\phi_m(\sum F_j)$\\
(For $a \in {\bf Z}$, $\phi_m(a) \in (-m/2,m/2)$ and $\phi_m(a)\equiv a \bmod m$)
\end{tabbing}

Figure \ref{speedup}
shows the speedup factor under the above distributed computation
on Risa/Asir. For each $n$, two polynomials of degree $n$
with 3000bit coefficients are generated and the product is computed.
The machine is Fujitsu AP3000,
a cluster of Sun connected with a high speed network and MPI over the
network is used to implement OpenXM.
\begin{figure}[htbp]
\epsfxsize=8.5cm
\epsffile{speedup.ps}
\caption{Speedup factor}
\label{speedup}
\end{figure}

If the number of servers is $L$ and the inputs are fixed, then the cost to
compute $F_j$ in parallel is $O(1/L)$, whereas the cost
to send and receive polynomials is $O(L)$ if {\tt ox\_push\_cmo()} and
{\tt ox\_pop\_cmo()} are repeatedly applied on the client.
Therefore the speedup is limited and the upper bound of
the speedup factor depends on the ratio of 
the computational cost and the communication cost for each unit operation.
Figure \ref{speedup} shows that 
the speedup is satisfactory if the degree is large and $L$
is not large, say, up to 10 under the above envionment.
If OpenXM provides operations for the broadcast and the reduction
such as {\tt MPI\_Bcast} and {\tt MPI\_Reduce} respectively, the cost of 
sending $f_1$, $f_2$ and gathering $F_j$ may be reduced to $O(log_2L)$
and we can expect better results in such a case.

\subsubsection{Competitive distributed computation by various strategies}

SINGULAR \cite{Singular} implements {\it MP} interface for distributed
computation and a competitive Gr\"obner basis computation is
illustrated as an example of distributed computation.
Such a distributed computation is also possible on OpenXM.
The following Risa/Asir function computes a Gr\"obner basis by
starting the computations simultaneously from the homogenized input and
the input itself.  The client watches the streams by {\tt ox\_select()}
and the result which is returned first is taken. Then the remaining
server is reset.

\begin{verbatim}
/* G:set of polys; V:list of variables */
/* O:type of order; P0,P1: id's of servers */
def dgr(G,V,O,P0,P1)
{
  P = [P0,P1]; /* server list */
  map(ox_reset,P); /* reset servers */
  /* P0 executes non-homogenized computation */
  ox_cmo_rpc(P0,"dp_gr_main",G,V,0,1,O);
  /* P1 executes homogenized computation */
  ox_cmo_rpc(P1,"dp_gr_main",G,V,1,1,O);
  map(ox_push_cmd,P,262); /* 262 = OX_popCMO */
  F = ox_select(P); /* wait for data */
  /* F[0] is a server's id which is ready */
  R = ox_get(F[0]);
  if ( F[0] == P0 ) {
    Win = "nonhomo"; Lose = P1;
  } else {
    Win = "homo"; Lose = P0;
  }
  ox_reset(Lose); /* reset the loser */
  return [Win,R];
}
\end{verbatim}


%$OpenXM: OpenXM/doc/calc2000/bib.tex,v 1.2 2000/04/24 05:01:29 noro Exp $

\begin{thebibliography}{X}
\bibitem{OpenMath}
The OpenMath Esprit Consortium 
(Caprotti, O. and Cohen, A.M. Editors),
The OpenMath Standard. D1.3.2a (Public), February, 1999.
{\tt http://www.nag.co.uk/projects/OpenMath}
\bibitem{CLO}
Cox, D., Little, J.,  O'Shea,
{\it Using Algebraic Geometry}, Springer, 1998.
\bibitem{omimp}
Dalmas, S., Ga\"etano, M., Watt, S., An OpenMath 1.0 Implementation.
Proceedings of ISSAC'97, ACM Press, 241-248 (1997).
\bibitem{GKW}
Gray, S., Kajler, N. and Wang, P. S.,
Design and Implementation of MP, a Protocol for Efficient
  Exchange of Mathematical Expressions,
{\sl Journal of Symbolic Computation}, 1996.
\bibitem{Singular}
Greuel, G.-M. et al., SINGULAR : a computer algebra system for polynomial
computations.\\
{\tt http://www.mathematik.uni-kl.de/\~\,zca/Singular/}
\bibitem{tigers}
Huber, B., Thomas, R., Computing Gr\"obner Fans of Toric Ideals, 1999.
{\tt http://www.math.tamu.edu/\~\,rekha/programs.html/}
\bibitem{pseware}
Lakshman, Y.N., Char, B, Johnson, J., Software Components using
Symbolic Computation for Problem Solving Environments.
Proceedings of ISSAC'98, ACM Press, 46-53 (1998).
\bibitem{gap}
Linton, S. and Solomon, A.,
OpenMath, IAMC and {\tt GAP},
preprint, 1999.
\bibitem{MPI} Message Passing Interface.
{\tt http://www.mpi-forum.org} 
%\bibitem{netsolve}
%NetSolve, {\tt http://www.cs.utk.edu/netsolve}
\bibitem{asir} 
Noro, M. et al., 
A Computer Algebra System {\tt Risa/Asir},  1993, 1995, 2000.
{\tt ftp://archives.cs.ehime-u.ac.jp/pub/asir2000/}
\bibitem{noro-mckay}
Noro, M. and McKay, J.,
Computation of replicable functions on Risa/Asir.
Proceedings of the Second International Symposium on
Symbolic Computation PASCO'97, ACM Press, 130-138 (1997).
\bibitem{noro-takayama}
Noro, M and Takayama, N., Design and Implementation
of OpenXM, 1996, 1997, 1998, 1999, 2000.
\bibitem{oaku-advance}
Oaku, T.,
Algorithms for $b$-functions, restrictions, and algebraic local cohomology
groups of $D$-modules.
Advances in Applied Mathematics, {\bf 61}, 61--105, 1997.
\bibitem{openxm-web}
{\tt http://www.math.kobe-u.ac.jp/OpenXM}
\bibitem{sst-book}
Saito, M., Sturmfels, B. and Takayama, N.,
{\it Gr\"obner Deformations of Hypergeometric Differential Equations}.
Algorithms and Computation in Mathematics {\bf 6}. Springer, 1999.
\bibitem{schefstrom}
Schefstr\"om, D.,
Building a highly integrated development environment using
preexisting parts.
In IFIP 11th World Computer Congress, San Francisco, USA.
\bibitem{Shoup}
Shoup, V., 
A new polynomial factorization algorithm and 
its implementation,
{\sl Journal of Symbolic Computation}, 20, 364-397, 1996.
\bibitem{kan}
Takayama, N.,
{\em Kan: A system for computation in
algebraic analysis,} 1991 version 1,
1994 version 2, the latest version is 2.991106. 
{\tt ftp://ftp.math.kobe-u.ac.jp/pub/kan}
\bibitem{phc}
Verschelde, J.,
PHCpack: A general-purpose solver for polynomial systems by
homotopy continuation.  ACM Transaction on Mathematical Software, 25(2) 
251-276, 1999.
\bibitem{iamc}
Wang, P.,
Design and Protocol for Internet Accessible Mathematical Computation.
Technical Report ICM-199901-001, ICM/Kent State University, 1999.
\bibitem{mathlink}
Wolfram, S.,
{\it The Mathematica Book, Fourth Edition}.
1999, Cambridge University Press.
\end{thebibliography}

\end{document}
\endinput
%%


%%Text may be set as \emph{emph}.\\
%%Text may be set as \texttt{texttt}.\\
%%Text may be set as \underline{unterline}.\\
%%Text may be set as \textbf{textbf}.\\
%%Text may be set as \textrm{textrm}.\\
%%Text may be set as {\tiny tiny}.\\

%%\begin{figure}
%%\hrule
%%Nice Postscript, isn't it?
%%\begin{center}
%%\IfFileExists{graphicx.sty}{
%%  \includegraphics{body.eps}
%%}{
%%  Sorry, package \texttt{graphicx} not present.
%%}
%%\end{center}
%%Same, a little bit smaller:
%%\begin{center}
%%\IfFileExists{graphicx.sty}{
%%  \includegraphics[scale=.5]{body.eps}
%%  }{
%%  Sorry, package \texttt{graphicx} not present.
%%}
%%\end{center}
%%\caption{\label{fig-1}This is a nice floating figure}
%%\hrule
%%\end{figure}




