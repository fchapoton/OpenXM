% $OpenXM: OpenXM/doc/ascm2001p/heterotic-network.tex,v 1.5 2001/06/21 00:15:34 takayama Exp $
\section{Applications}

\subsection{Heterogeneous Servers}

\def\pd#1{ \partial_{#1} }

By using OpenXM, we can treat OpenXM servers essentially 
like a subroutine.
Since OpenXM provides a universal stack machine which does not
depend on each server, 
it is relatively easy to install new servers.
We can build a new mathematical software system by assembling
different OpenXM servers.
OpenXM servers currently provide 1077 functions
\cite{openxm-1077}.
We can use these as building blocks for a new system.

%We present an example of a custom-made system
%built by OpenXM servers.

%\subsubsection{Computation of annihilating ideals by kan/sm1 and ox\_asir}
%
%Let $D = {\bf Q} \langle x_1, \ldots, x_n , \pd{1}, \ldots, \pd{n} \rangle$
%be the ring of differential operators.
%For a given polynomial
%$ f \in {\bf Q}[x_1, \ldots, x_n] $,
%the annihilating ideal of $f^{-1}$ is defined as
%$$ {\rm Ann}\, f^{-1} = \{ \ell \in D \,|\,
%  \ell \bullet f^{-1} = 0 \}.
%$$
%Here, $\bullet$ denotes the action of $D$ to functions.
%The annihilating ideal can be regarded as the maximal differential
%equations for the function $f^{-1}$.
%An algorithm to determine generators of the annihilating ideal
%was given by Oaku (see, e.g., \cite[5.3]{sst-book}).
%His algorithm reduces the problem to computations of Gr\"obner bases
%in $D$ and to find the minimal integral root of a polynomial.
%This algorithm (the function {\tt annfs}) is implemented by
%kan/sm1 \cite{kan}, for Gr\"obner basis computation in $D$, and
%{\tt ox\_asir}, to factorize polynomials to find the integral
%roots.
%These two OpenXM compliant systems are integrated by
%the OpenXM protocol.

%For example, the following is a sm1 session to find the annihilating
%ideal for $f = x^3 - y^2 z^2$.
%\begin{verbatim}
%sm1>[(x^3-y^2 z^2) (x,y,z)] annfs ::
%Starting ox_asir server.
%Byte order for control process is network byte order.
%Byte order for engine process is network byte order.
%[[-y*Dy+z*Dz, 2*x*Dx+3*y*Dy+6, -2*y*z^2*Dx-3*x^2*Dy, 
%-2*y^2*z*Dx-3*x^2*Dz, -2*z^3*Dx*Dz-3*x^2*Dy^2-2*z^2*Dx], 
% [-1,-139968*s^7-1119744*s^6-3802464*s^5-7107264*s^4
%     -7898796*s^3-5220720*s^2-1900500*s-294000]] 
%\end{verbatim}
%The last polynomial is factored as
%$-12(s+1)(3s+5)(3s+4)(6s+5)(6s+7)$
%and the minimal integral root is $-1$
%as shown in the output.
%
%Similarly, 
%an algorithm to stratify singularity 
%\cite{oaku-advance}
%is implemented by
%kan/sm1 \cite{kan}, for Gr\"obner basis computation in $D$, and
%{\tt ox\_asir}, for primary ideal decompositions.

%\subsubsection{A Course on Solving Algebraic Equations}
%
%Risa/Asir \cite{asir} is a general computer algebra system
%which can be used for Gr\"obner basis computations for zero dimensional ideal
%with ${\bf Q}$ coefficients.
%However, it is not good at graphical presentations and
%numerical methods.
%We integrated Risa/Asir, ox\_phc (based on PHC pack by Verschelde \cite{phc}
%for the polyhedral homotopy method) and
%ox\_gnuplot (GNUPLOT) servers
%to teach a course on solving algebraic equations.
%This course was presented with the text book \cite{CLO},
%which discusses 
%on the Gr\"obner basis method and the polyhedral homotopy method
%to solve systems of algebraic equations.
%We taught the course
%with a unified environment
%controlled by the Asir user language, which is similar to C.
%The following is an Asir session to solve algebraic equations by calling
%the PHC pack (Figure \ref{katsura} is the output of {\tt [292]}):
%\begin{verbatim}
%[287] phc(katsura(7));
%The detailed output is in the file tmp.output.*
%The answer is in the variable Phc.
%0
%[290] B=map(first,Phc)$
%[291] gnuplot_plotDots([],0)$
%[292] gnuplot_plotDots(B,0)$
%\end{verbatim}
%
%\begin{figure}[htbp]
%\epsfxsize=8.5cm
%\epsffile{katsura7.ps}
%\caption{The first components of the solutions to the system of algebraic equations Katsura 7.}
%\label{katsura}
%\end{figure}

Asir-contrib/Hypergeometric package is an 
example of a custom-made system built by OpenXM servers.
GKZ hypergeometric system is a system of linear partial differential
equations associated to $A=(a_{ij})$  
(an integer $d\times n$-matrix of rank $d$)
and $\beta \in {\bf C}^d$.
The book by Saito, Sturmfels and Takayama \cite{sst-book}
discusses algorithmic methods to construct series solutions of the GKZ
system.
The current Asir-contrib/Hypergeoemtric package is built in order to implement
these algorithms.
What we need for the implementation are mainly
(1) Gr\"obner basis computation both in the ring of polynomials
and in the ring of differential operators,
(2) enumeration of all the Gr\"obner bases of toric ideals,
and 
(3) primary ideal decomposition.
Asir and kan/sm1 provide functions for (1),
{\tt TiGERS} provides a function for (2),
and
Asir provides a function for (3).
These software systems communicate with each other 
by the OpenXM-RFC 100 protocol
and form a unified package for GKZ hypergeometric systems.
See the chapter of {\tt dsolv} of Asir Contrib User's manual \cite{openxm-web}
for details.

%Let us see an example how to construct series solution of a GKZ hypergeometric
%system.
%The function
%{\tt dsolv\_starting\_term} finds the leading terms of series solutions
%to a given direction.
%\begin{enumerate}
%\item Generate the GKZ hypergeometric system associated to
%                    $\pmatrix{ 1&1&1&1&1 \cr
%                              1&1&0&-1&0 \cr
%                              0&1&1&-1&0 \cr}$
%by the function {\tt sm1\_gkz}.
%\begin{verbatim}
%[1076] F = sm1_gkz( 
%      [ [[1,1,1,1,1],
%         [1,1,0,-1,0],
%         [0,1,1,-1,0]], [1,0,0]]);
%[[x5*dx5+x4*dx4+x3*dx3+x2*dx2+x1*dx1-1,
%  -x4*dx4+x2*dx2+x1*dx1,
%  -x4*dx4+x3*dx3+x2*dx2,
%  -dx2*dx5+dx1*dx3,dx5^2-dx2*dx4],
% [x1,x2,x3,x4,x5]]
%\end{verbatim}
%\item Find the leading terms of this system to the direction
%$(1,1,1,1,0)$.
%\begin{verbatim}
%[1077] A= dsolv_starting_term(F[0],F[1],
%                            [1,1,1,1,0])$
%Computing the initial ideal.
%Done.
%Computing a primary ideal decomposition.
%Primary ideal decomposition of 
%the initial Frobenius ideal
%to the direction [1,1,1,1,0] is
%[[[x5+2*x4+x3-1,x5+3*x4-x2-1,
%   x5+2*x4+x1-1,3*x5^2+(8*x4-6)*x5-8*x4+3,
%   x5^2-2*x5-8*x4^2+1,x5^3-3*x5^2+3*x5-1],
% [x5-1,x4,x3,x2,x1]]]
%
%----------- root is [ 0 0 0 0 1 ]
%----------- dual system is
%[x5^2+(-3/4*x4-1/2*x3-1/4*x2-1/2*x1)*x5+1/8*x4^2
% +(1/4*x3+1/4*x1)*x4+1/4*x2*x3-1/8*x2^2+1/4*x1*x2,
% x4-2*x3+3*x2-2*x1,x5-x3+x2-x1,1]
%\end{verbatim}
%\item From the output, we can see that we have four possible 
%leading terms.
%Factoring these leading terms, we get the following simpler expressions.
%The third entry
%{\tt [[1,1],[x5,1],[-log(x1)+log(x2)-log(x3)+log(x5),1]], }
%means that there exists a series solution which starts with
%\[
%x_5 (-\log x_1 + \log x_2 - \log x_3 + \log x_5) =
%   x_5 \log \frac{x_2 x_5}{x_1 x_3}
%\]
%\begin{verbatim}
%[1078] A[0];
%[[ 0 0 0 0 1 ]]
%[1079] map(fctr,A[1][0]);
%[[[1/8,1],[x5,1],[log(x2)+log(x4)-2*log(x5),1],
%   [2*log(x1)-log(x2)+2*log(x3)+log(x4)-4*log(x5)
%    ,1]],
% [[1,1],[x5,1],
%   [-2*log(x1)+3*log(x2)-2*log(x3)+log(x4),1]],
% [[1,1],[x5,1],
%   [-log(x1)+log(x2)-log(x3)+log(x5),1]],
% [[1,1],[x5,1]]]
%\end{verbatim}
%\end{enumerate}








