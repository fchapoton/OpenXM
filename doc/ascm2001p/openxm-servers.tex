%% $OpenXM$
\section{ 1077 functions are available on our servers and libraries}

This is a list of examples and functions which our
servers provide.
For details, see manuals of each system.

\noindent
\fbox{\large {Operations on Integers}}

\noindent
{idiv},{irem} (division with remainder),
{ishift} (bit shifting),
{iand},{ior},{ixor} (logical operations),
{igcd},(GCD by various methods such as Euclid's algorithm and
the accelerated GCD algorithm),
{fac} (factorial),
{inv} (inverse modulo an integer),
{random} (random number generator by the Mersenne twister algorithm).

\medbreak
\noindent
\fbox{\large {Ground Fields}}

\noindent
Arithmetics on various fields: the rationals,
${\bf Q}(\alpha_1,\alpha_2,\ldots,\alpha_n)$
($\alpha_i$ is algebraic over ${\bf Q}(\alpha_1,\ldots,\alpha_{i-1})$),
$GF(p)$ ($p$ is a prime of arbitrary size), $GF(2^n)$.

\medbreak
\noindent
\fbox{\large {Operations on Polynomials}}

\noindent
{sdiv }, {srem } (division with remainder),
{ptozp } (removal of the integer content),
{diff } (differentiation),
{gcd } (GCD over the rationals),
{res } (resultant),
{subst } (substitution),
{umul} (fast multiplication of dense univariate polynomials
by a hybrid method with Karatsuba and FFT+Chinese remainder),
{urembymul\_precomp} (fast dense univariate polynomial
division with remainder by the fast multiplication and
the precomputed inverse of a divisor),

\noindent
\fbox{\large {Polynomial Factorization}}
{fctr } (factorization over the rationals),
{fctr\_ff } (univariate factorization over finite fields),
{af } (univariate factorization over algebraic number fields),
{sp} (splitting field computation).

\medbreak
\noindent
\fbox{\large {Groebner basis}}

\noindent
{dp\_gr\_main } (Groebner basis computation of a polynomial ideal
over the rationals by the trace lifting),
{dp\_gr\_mod\_main } (Groebner basis over small finite fields),
{tolex } (Modular change of ordering for a zero-dimensional ideal),
{tolex\_gsl } (Modular rational univariate representation
for a zero-dimensional ideal),
{dp\_f4\_main } ($F_4$ over the rationals),
{dp\_f4\_mod\_main } ($F_4$ over small finite fields).

\medbreak
\noindent
\fbox{\large {Ideal Decomposition}}

\noindent
{primedec} (Prime decomposition of the radical),
{primadec} (Primary decomposition of ideals by Shimoyama/Yokoyama algorithm).

\medbreak
\noindent
\fbox{\large {Quantifier Elimination}}

\noindent
{qe} (real quantifier elimination in a linear and
quadratic first-order formula),
{simpl} (heuristic simplification of a first-order formula).

%%$
{\scriptsize
\begin{verbatim}
[0] MTP2 = ex([x11,x12,x13,x21,x22,x23,x31,x32,x33],
x11+x12+x13 @== a1 @&& x21+x22+x23 @== a2 @&& x31+x32+x33 @== a3
@&& x11+x21+x31 @== b1 @&& x12+x22+x32 @== b2 @&& x13+x23+x33 @== b3
@&& 0 @<= x11 @&& 0 @<= x12 @&& 0 @<= x13 @&& 0 @<= x21
@&& 0 @<= x22 @&& 0 @<= x23 @&& 0 @<= x31 @&& 0 @<= x32 @&& 0 @<= x33)$
[1] TSOL= a1+a2+a3@=b1+b2+b3 @&& a1@>=0 @&& a2@>=0 @&& a3@>=0
@&& b1@>=0 @&& b2@>=0 @&& b3@>=0$
[2] QE_MTP2 = qe(MTP2)$
[3] qe(all([a1,a2,a3,b1,b2,b3],QE_MTP2 @equiv TSOL));
@true
\end{verbatim}}

\medbreak
\noindent
\fbox{\large {Visualization of curves}}

\noindent
{plot} (plotting of a univariate function),
{ifplot} (plotting zeros of a bivariate polynomial),
{conplot} (contour plotting of a bivariate polynomial function).

\medbreak
\noindent
\fbox{\large {Miscellaneous functions}}

\noindent
{det} (determinant),
{qsort} (sorting of an array by the quick sort algorithm),
{eval} (evaluation of a formula containing transcendental functions
such as
{sin}, {cos}, {tan}, {exp},
{log})
{roots} (finding all roots of a univariate polynomial),
{lll} (computation of an LLL-reduced basis of a lattice).

\medbreak
\noindent
\fbox{\large {$D$-modules}} ($D$ is the Weyl algebra)

\noindent
{gb } (Gr\"obner basis),
{syz} (syzygy),
{annfs} (Annihilating ideal of $f^s$),
{bfunction},
{schreyer} (free resolution by the Schreyer method),
{vMinRes} (V-minimal free resolution),
{characteristic} (Characteristic variety),
{restriction} in the derived category of $D$-modules,
{integration} in the derived category,
{tensor}  in the derived category,
{dual} (Dual as a D-module),
{slope}.

\medbreak
\noindent
\fbox{\large {Cohomology groups}}

\noindent
{deRham} (The de Rham cohomology groups of
${\bf C}^n \setminus V(f)$,
{ext} (Ext modules for a holonomic $D$-module $M$
and the ring of formal power series).

\medbreak
\noindent
\fbox{\large {Differential equations}}

\noindent
Helping to derive and prove {combinatorial} and
{special function identities},
{gkz} (GKZ hypergeometric differential equations),
{appell} (Appell's hypergeometric differential equations),
{indicial} (indicial equations),
{rank} (Holonomic rank),
{rrank} (Holonomic rank of regular holonomic systems),
{dsolv} (series solutions of holonomic systems).

\medbreak
\noindent
\fbox{\large {OpenMATH support}}

\noindent
{om\_xml} (CMO to OpenMATH XML),
{om\_xml\_to\_cmo} (OpenMATH XML to CMO).

\medbreak
\noindent
\fbox{\large {Homotopy Method}}

\noindent
{phc} (Solving systems of algebraic equations by
numerical and polyhedral homotopy methods).

\medbreak
\noindent
\fbox{\large {Toric ideal}}

\noindent
{tigers} (Enumerate all Gr\"obner basis of a toric ideal.
Finding test sets for integer program),
{arithDeg} (Arithmetic degree of a monomial ideal),
{stdPair} (Standard pair decomposition of a monomial ideal).

\medbreak
\noindent
\fbox{\large {Communications}}

\noindent
{ox\_launch} (starting a server),
{ox\_launch\_nox},
{ox\_shutdown},
{ox\_launch\_generic},
{generate\_port},
{try\_bind\_listen},
{try\_connect},
{try\_accept},
{register\_server},
{ox\_rpc},
{ox\_cmo\_rpc},
{ox\_execute\_string},
{ox\_reset} (reset the server),
{ox\_intr},
{register\_handler},
{ox\_push\_cmo},
{ox\_push\_local},
{ox\_pop\_cmo},
{ox\_pop\_local},
{ox\_push\_cmd},
{ox\_sync},
{ox\_get},
{ox\_pops},
{ox\_select},
{ox\_flush},
{ox\_get\_serverinfo}

\medbreak
\noindent
In addition to these functions, {Mathematica functions}
can be called as server functions.

\medbreak
\noindent
\fbox{\large {Examples}}
{\footnotesize
\begin{verbatim}
[345] sm1_deRham([x^3-y^2*z^2,[x,y,z]]);
[1,1,0,0]
/* dim H^i = 1 (i=0,1), =0 (i=2,3) */
\end{verbatim}}

%%\noindent
%%{\footnotesize \begin{verbatim}
%%[287] phc(katsura(7)); B=map(first,Phc)$
%%[291] gnuplot_plotDots(B,0)$
%%\end{verbatim} }

% \epsfxsize=3cm
% \begin{center}
% %\epsffile{../calc2000/katsura7.ps}
% \epsffile{katsura7.ps}
% \end{center}
%%The first components of the solutions to the system of algebraic equations Katsura 7.

\medbreak
\noindent
\fbox{ {Authors}}
Castro-Jim\'enez, Dolzmann, Hubert, Murao, Noro, Oaku, Okutani,
Shimoyama, Sturm, Takayama, Tamura, Verschelde, Yokoyama.
