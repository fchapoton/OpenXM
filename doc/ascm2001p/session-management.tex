% $OpenXM$

\section{Session Management}
\label{secsession}
%MEMO: key words:
%Security (ssh PAM), initial negotiation of byte order,
%mathcap, interruption, debugging window, etc.
 
In this section we show our control integration in
OpenXM, which gives a dynamical and unified
method to start servers, to establish connections and to reset
computations.
%method to start servers and to establish connections.
%We assume that various clients and servers
%establish connections dynamically and communicate to each
%other. Therefore it is necessary to give a dynamical and unified
%method to start servers and to establish connections.
%In addition to that, interruption of execution and 
%debugging facilities
%are necessary for interactive distributed computation.

%\subsection{Interface of servers}
%
%A server has additional I/O streams for exchanging data between
%a client and itself other than ones for diagnostic
%messages. As the streams are for binary data,
%the byte order conversion is necessary when a
%client and a server have different byte orders. It is determined by
%exchanging the preferable byte order of each peer. If the preference
%does not coincide with each other,
%then the network byte order is used.
%This implies that all servers and clients should be able to
%handle the network byte
%order. Nevertheless it is necessary to negotiate the byte order to
%skip the byte order conversion because its cost is often dominant over
%fast networks.

\subsection{Invocation of servers}
\label{launcher}

Under OpenXM-RFC 100, servers are invoked as follows.
An application called {\it launcher} is invoked from a client.
The launcher sets up two communication channels: the data channel 
and the control. Then it creates a server process which can communicate
with the client via the data channel. Finally the launcher
acts as a control server and controls the server process.
As to the details of the control server see Section \ref{control}.
%As the data channel is used to exchange binary data,
%the byte order conversion is necessary when a
%client and a server have different byte orders. It is determined by
%exchanging the preferable byte order of each peer. If the preference
%does not coincide with each other,
%then the network byte order is used.
%This implies that all servers and clients should be able to
%handle the network byte
%order. Nevertheless it is necessary to negotiate the byte order to
%skip the byte order conversion because its cost is often dominant over
%fast networks.
OpenXM-RFC 101 \cite{ox-rfc-101} gives more secure and
efficient way to start servers.
It uses ``ssh'' to launch a control server and 
the control server can launch one or more than one
engines.
See \cite{ox-rfc-101} for details.

\subsection{Control server}
\label{control}
%In OpenXM we adopted the following simple and robust method to 
%control servers.
%
%An OpenXM server has logically two I/O channels: one for exchanging
%data for computations and the other for controlling computations. The
%control channel is used to send commands to control execution on the
%server. The launcher introduced in Section \ref{launcher}
%is used as a control process. We call such a process a {\it
%control server}. In contrast, we call a server for computation an {\it
%engine}. As the control server and the engine runs on the
%same machine, it is easy to send a signal from the control server. 
%A control server is also an
%OpenXM stack machine and it accepts {\tt SM\_control\_*} commands
%to send signals to a server or to terminate a server.
%
%\subsection{Resetting an engine}

{\tt SM\_control\_reset\_connection} sent to the control server
initiates a safe resetting of an engine.
After receiving the command, the control server sends {\tt SIGUSR1} 
to the engine and the resetting procedure starts.
Figure \ref{reset} illustrates the flow of data under the OpenXM
reset protocol.
\begin{figure}[htbp]
\begin{center}
\epsfxsize=8cm
\epsffile{reset.eps}
\caption{OpenXM reset procedure}
\label{reset}
\end{center}
\end{figure}
{\tt OX\_SYNC\_BALL} is used to mark the end of data remaining in the
I/O streams. After reading it, it is assured that each stream is empty
and that the subsequent request from a client correctly 
corresponds to the response from the engine.
%We note that we don't have to associate {\tt OX\_SYNC\_BALL} with
%any special action to be executed by the engine because it is
%assured that the engine is in the resetting state when it has received
%{\tt OX\_SYNC\_BALL}.

%\subsection{Debugging facilities}
%Debugging is sometimes very hard for distributed computations.
%We provide two methods to help debugging on X window system:
%1. the diagnostic messages from the engine are displayed in a {\tt xterm}
%window;
%2. the engine can pop up a window to input debug commands.
%For example {\tt ox\_asir}, which is
%the OpenXM server of Risa/Asir, can pop up a window to input
%debug commands and the debugging similar to that on usual terminals is possible.
%One can also send {\tt SIGINT} by using {\tt SM\_control\_to\_debug\_mode}
%and it provides a similar functionality to entering the debugging
%mode from a keyboard interruption.
