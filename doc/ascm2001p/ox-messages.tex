%%$OpenXM$

\section{OX messages}

An OX message for TCP/IP is a byte stream consisting of
a header and a body.
\begin{center}
\begin{tabular}{|c|c|}
\hline
Header	& \hspace{10mm} Body \hspace{10mm} \\
\hline
\end{tabular}
\end{center}
The header consists of two signed 32 bit integers.
The first one is an OX tag 
and the second one is a serial number of the OX message.

%Negative numbers are expressed by the two's complement.
%Several byte orders including the network byte order
%are allowed and the byte order is determined as a part of
%the establishment of a connection. See Section \ref{secsession} for details.

The OX messages are classified into three types:
DATA, COMMAND, and SPECIAL.
%We have currently the following general tags for the OX messages:
Typical OX tags are 
\verb+OX_COMMAND+ (COMMAND), \verb+OX_DATA+ (DATA) and \verb+OX_SYNC_BALL+ (SPECIAL).

%\begin{verbatim}
%#define	OX_COMMAND               513  // COMMAND
%#define	OX_DATA	                 514  // DATA
%#define OX_SYNC_BALL             515  // SPECIAL
%#define OX_DATA_WITH_LENGTH      521  // DATA
%#define OX_DATA_OPENMATH_XML     523  // DATA
%#define OX_DATA_OPENMATH_BINARY  524  // DATA
%#define OX_DATA_MP               525  // DATA
%\end{verbatim}

%A new OX tag may be added.
%The new tag should be classified into DATA or COMMAND.
%For example, \verb+ OX_DATA_ASIR_LOCAL_BINARY +  was added recently
%to send internal serialized objects of Asir via the OpenXM protocol.
%This is a tag classified to DATA.
%See the web page of OpenXM to add a new tag \cite{openxm-web}.

%An OpenXM client admit that its own command sends some OX messages
%sequentially at once.  
%
%For example, the asir command
%{\tt ox\_execute\_string(P, "Print[x+y]")} sends an OX data message
%{\tt (OX\_DATA, (CMO\_STRING, "Print[x+y]"))} and an OX command message
%{\tt (OX\_COMMAND, (SM\_executeStringByLocalParser))} to an OpenXM
%server.
