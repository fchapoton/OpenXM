%% $OpenXM: OpenXM/doc/OpenXM-specs/howtowrite.tex,v 1.1.1.1 2000/01/20 08:52:46 noro Exp $
//&jp \section{ どのように協調機能にすぐれた Math Soft を書くのか? }

/*&jp
いくつかの経験的なプログラム作法をメモとしてまとめておく.
\begin{enumerate}
\item 文字列を読み込み, 計算して, また文字列を結果として出力する
プログラムおよび関数は協調機能にすぐれたプログラムの基本である.
グラフィカルインターフェースのみのプログラムは他のソフトと協調
させるのがむずかしい.
\item Output を読むのは人間とは限らない. 
Output の読者は, 他のソフトかもしれない.
したがって, Output もやはり, 文法にしたがって出力すべきである.
エラー出力も文法にしたがうべきである.
\end{enumerate}

\subsection{open gnuplot の実装}
書いてない.

\subsection{open phc の実装}
書いてない.

\subsection{open M2 の実装}
書いてない.

この節の補足は,
{\tt this/open/ohp} を見よ (Open XM day, 1999, 7/12 Mon の原稿)
*/
