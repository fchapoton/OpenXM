%% $OpenXM: OpenXM/doc/OpenXM-specs/cmo-register.tex,v 1.3 2002/01/20 09:26:21 takayama Exp $
//&jp \section{ 新しい CMO の登録 }
//&eg \section{ Registering a new CMO }

//&jp \subsection{新しい CMO を作るときの約束}
//&eg \subsection{Requirement for a new CMO}

/*&jp
CMO では, 再帰的なデータ表現をゆるしているので,
Object を増やした場合, もとからある CMO の動作も拡張される.
そのシステムが正しく動作するのか確認する必要がある.
計算することはできなくても, 通信の中断なしに,
データの受渡しをできないといけない.
*/
/*&eg
CMO data types are defined recursively.
Thus, if one introduces a new CMO, then old CMO's may be 
also extended.
*/

//&jp \subsection{OpenXM プロジェクトに参加するには?}
//&eg \subsection{How to join in the OpenXM project}

/*&jp
OpenXM にパッケージを加えてくれるのは大歓迎です.
takayama@math.kobe-u.ac.jpに連絡下さい.
新しい CMO を必要なら加えて下さい.
加えた場合は
\begin{enumerate}
\item 定義した CMObject の形式的な定義と説明.
\item システム xxx が, この CMObject に関してどのように振舞うかの説明.
\item この CMObject, システム xxx に関する URL.
\end{enumerate}
をおくって下さい.
必要ならディスカッションをおこない, 確定版の仕様を作成し, 確定した段階で
CMO のタグを発行し, この CMObject に関連する URL を
OpenXM ホームページ \cite{openxxx} よりリンクします.
*/

/*&eg
You are welcome to add packages to OpenXM.
Ask takayama@math.kobe-u.ac.jp for details.
You may introduce new CMO's if necessary.
If you have defined a new CMO, send 
\begin{enumerate}
\item the formal definition and an explanation of the CMO,
\item an explanation of the behavior of a system xxx for the CMO,
\item URL's related to the CMO or xxx.
\end{enumerate}
After discussing on the new CMO, we will fix the specification.
Then we will issue the tag for the new CMO and create links to
the URL related to the CMO from the OpenXM home page \cite{openxxx}.
*/
