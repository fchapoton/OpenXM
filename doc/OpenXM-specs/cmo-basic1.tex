%% $OpenXM: OpenXM/doc/OpenXM-specs/cmo-basic1.tex,v 1.15 2016/08/22 05:38:27 takayama Exp $
//&jp \section{ 数, 多項式 の  CMO 表現 }
//&eg \section{ CMOexpressions for numbers and polynomials }
\label{sec:basic1}
/*&C
@../SSkan/plugin/cmotag.h
\begin{verbatim}
#define     CMO_MONOMIAL32  19
#define     CMO_ZZ          20 
#define     CMO_QQ          21
#define     CMO_ZERO        22
#define     CMO_DMS_GENERIC  24
#define     CMO_DMS_OF_N_VARIABLES  25
#define     CMO_RING_BY_NAME   26
#define     CMO_DISTRIBUTED_POLYNOMIAL 31
#define     CMO_RATIONAL       34
#define     CMO_COMPLEX       35

#define     CMO_BIGFLOAT32    52

#define     CMO_INDETERMINATE  60
#define     CMO_TREE           61
#define     CMO_LAMBDA         62    /* for function definition */
\end{verbatim}

*/

/*&jp
以下, グループ CMObject/Basic, CMObject/Tree 
および CMObject/DistributedPolynomial
に属する CMObject の形式を説明する.

\noindent
{\tt OpenXM/src/ox\_toolkit} にある {\tt bconv} をもちいると
CMO expression を binary format に変換できるので,
これを参考にするといい.
*/
/*&eg
In the sequel, we will explain on the groups
CMObject/Basic, CMObject/Tree 
and CMObject/DistributedPolynomial.

\noindent
The program {\tt bconv} at {\tt OpenXM/src/ox\_toolkit}
translates 
CMO expressions into binary formats.
It is convinient to understand the binary formats explained in
this section.
*/

/*&C
\noindent Example:
\begin{verbatim}
bash$ ./bconv
> (CMO_ZZ,123123);
00 00 00 14 00 00 00 01 00 01 e0 f3 
\end{verbatim}
*/
/*&jp

\bigbreak
\noindent
Group CMObject/Basic requires CMObject/Primitive. \\
ZZ, QQ, Zero, Rational, Indeterminate $\in$ CMObject/Basic. \\
\begin{eqnarray*}
\mbox{Zero} &:& ({\tt CMO\_ZERO}) \\ 
& & \mbox{ --- ユニバーサルな ゼロを表す. } \\
\mbox{ZZ}         &:& ({\tt CMO\_ZZ},{\sl int32}\, {\rm f}, {\sl byte}\, \mbox{a[1]}, \ldots , 
{\sl byte}\, \mbox{a[$|$f$|$]} ) \\
&:& \mbox{ --- bignum をあらわす. a[i] についてはあとで説明}\\
\mbox{QQ}        &:& ({\tt CMO\_QQ}, 
                      {\sl int32}\, {\rm m}, {\sl byte}\, \mbox{a[1]}, \ldots, {\sl byte}\, \mbox{a[$|$m$|$]},
                      {\sl int32}\, {\rm n}, {\sl byte}\, \mbox{b[1]}, \ldots, {\sl byte}\, \mbox{b[$|$n$|$]})\\
& & \mbox{ --- 有理数 $a/b$ を表す. } \\
\mbox{Rational}        &:& ({\tt CMO\_RATIONAL}, {\sl CMObject}\, {\rm a}, {\sl CMObject}\, {\rm b}) \\
& & \mbox{ ---  $a/b$ を表す. } \\
\mbox{Bigfloat32}         &:& ({\tt CMO\_BIGFLOAT32},
{\sl int32}\, {\rm prec}, {\sl int32}\, {\rm sign}, {\sl int32}\, {\rm exp}, 
{\sl int32}\, \mbox{a[1]}, \ldots , {\sl int32}\, \mbox{a[k]} ) \\
&:& \mbox{ --- bigfloat をあらわす. a[i], k についてはあとで説明}\\
\mbox{Complex}        &:& ({\tt CMO\_COMPLEX}, {\sl CMObject}\, {\rm re}, {\sl CMObject}\, {\rm im}) \\
& & \mbox{ ---  $a+b\sqrt{-1}$ を表す. } \\
\mbox{Indeterminate}        &:& ({\tt CMO\_INDETERMINATE}, {\sl Cstring}\, {\rm v}) \\
& & \mbox{ --- 変数名 $v$ . } \\
\end{eqnarray*}
*/


/*&eg

\bigbreak
\noindent
Group CMObject/Basic requires CMObject/Primitive. \\
ZZ, QQ, Zero, Rational, Indeterminate $\in$ CMObject/Basic. \\
\begin{eqnarray*}
\mbox{Zero} &:& ({\tt CMO\_ZERO}) \\ 
& & \mbox{ --- Universal zero } \\
\mbox{ZZ}         &:& ({\tt CMO\_ZZ},{\sl int32}\, {\rm f}, {\sl byte}\, \mbox{a[1]}, \ldots ,
{\sl byte}\, \mbox{a[$|$m$|$]} ) \\
&:& \mbox{ --- bignum. The meaning of a[i] will be explained later.}\\
\mbox{QQ}        &:& ({\tt CMO\_QQ}, 
                      {\sl int32}\, {\rm m}, {\sl byte}\, \mbox{a[1]}, \ldots, {\sl byte}\, \mbox{a[$|$m$|$]},
                      {\sl int32}\, {\rm n}, {\sl byte}\, \mbox{b[1]}, \ldots, {\sl byte}\, \mbox{b[$|$n$|$]})\\
& & \mbox{ --- Rational number $a/b$. } \\
\mbox{Rational}        &:& ({\tt CMO\_RATIONAL}, {\sl CMObject}\, {\rm a}, {\sl CMObject}\, {\rm b}) \\
& & \mbox{ ---  Rational expression $a/b$. } \\
\mbox{Bigfloat32}         &:& ({\tt CMO\_BIGFLOAT32},
{\sl int32}\, {\rm prec}, {\sl int32}\, {\rm sign}, {\sl int32}\, {\rm exp}, 
{\sl int32}\, \mbox{a[1]}, \ldots , {\sl int32}\, \mbox{a[k]} ) \\
&:& \mbox{ --- bigfloat. The meaning of a[i], k will be explained later.}\\
\mbox{Complex}        &:& ({\tt CMO\_COMPLEX}, {\sl CMObject}\, {\rm re}, {\sl CMObject}\, {\rm im}) \\
& & \mbox{ ---  Complex number $a+b\sqrt{-1}$. } \\
\mbox{Indeterminate}        &:& ({\tt CMO\_INDETERMINATE}, {\sl Cstring}\, {\rm v}) \\
& & \mbox{ --- Variable name $v$ . } \\
\end{eqnarray*}
*/
/*&C

*/
/*&C

*/

/*&jp
Indeterminate は変数名をあらわす.
v はバイト列であればなにを用いてもよいが,
システム毎に変数名として用いられるバイト列は制限がある.
各システム xxx は任意の文字列を各システム固有の変数名へ1対1に変換できるように
実装しないといけない.
(これを
{\tt Dx} は {\tt \#dx} と変換するなどの
escape sequence を用いて実現するのは, 無理があるようである.
テーブルを作成する必要があるであろう.)
*/
/*&eg
The name of a variable should be expressed by using Indeterminate.
v may be any sequence of bytes, but each system has its own
restrictions on the names of variables.
Indeterminates of CMO and internal variable names must be translated
in one-to-one correspondence.
*/


/*&jp 
\subsection{Indeterminate および Tree}

\noindent
Group CMObject/Tree requires CMObject/Basic. \\
Tree, Lambda $\in$ CMObject/Tree. \\
\begin{eqnarray*}
\mbox{Tree}        &:& ({\tt CMO\_TREE}, {\sl Cstring}\, {\rm name},
 {\sl List}\, {\rm attributes}, {\sl List}\, {\rm leaves}) \\
& & \mbox{ --- 名前 name の定数または関数. 関数の評価はおこなわない. } \\
& & \mbox{ --- attributes は空リストでなければ name の属性を保持している. }\\
& & \mbox{ --- 属性リストは, key と 値のペアである. }\\
\mbox{Lambda}        &:& ({\tt CMO\_LAMBDA}, {\sl List}\, {\rm args},
                          {\sl Tree} {\rm body}) \\                          
& & \mbox{ --- body を args を引数とする関数とする. } \\
\end{eqnarray*}
*/
/*&eg 
\subsection{Indeterminate and Tree}

\noindent
Group CMObject/Tree requires CMObject/Basic. \\
Tree, Lambda $\in$ CMObject/Tree. \\
\begin{eqnarray*}
\mbox{Tree}        &:& ({\tt CMO\_TREE}, {\sl Cstring}\, {\rm name},
 {\sl List}\, {\rm attributes}, {\sl List}\, {\rm leaves}) \\
& & \mbox{ --- ``name'' is the name of the node of the tree. } \\
& & \mbox{ --- Attributes may be a null list. If it is not null, it is a list of}\\
& & \mbox{ --- key and value pairs. } \\
\mbox{Lambda}        &:& ({\tt CMO\_LAMBDA}, {\sl List}\, {\rm args},
                          {\sl Tree} {\rm body}) \\                          
& & \mbox{ --- a function with the arguments body. } \\
\end{eqnarray*}
*/

/*&jp
数式を処理するシステムでは, Tree 構造が一般にもちいられる.
たとえば, $\sin(x+e)$ は,
{\tt (sin, (plus, x, e))}
なる Tree であらわすのが一般的である.
Tree の表現を スタックマシンのレベルでおこなうとすると,
{\tt ox\_BEGIN\_BLOCK}, {\tt ox\_END\_BLOCK} で評価を抑制するのが
一つの方法である (cf. Postscript の {\tt \{ }, {\tt \} }).
たとえば上の方法では 
{\tt x, e, plus, sin } を begin block, end block でかこめばよろしい.
われわれはスタックマシンの実装をなるべく簡単にするという立場をとりたい,
また数学オブジェクトを OX スタックマシンと CMObject を混在して表現したく
ない.
したがって,
Tree 構造は Open Math 風の表現をもちいた CMO を導入することにした.
またこのほうが, われわれの想定するシステム xxx において, Open XM 対応が
はるかに容易である.
なお, Tree は, Open Math では, Symbol, Application のメカニズムに相当する.
*/
/*&eg
In many computer algebra systems, mathematical expressions are usually
expressed in terms of a tree structure.
For example,
$\sin(x+e)$ is expressed as
{\tt (sin, (plus, x, e))}
as a tree.
Tree may be expressed by putting the expression between
{\tt SM\_beginBlock} and {\tt SM\_endBlock}, which are
stack machine commands for delayed evaluation.
(cf. {\tt \{ }, {\tt \} } in PostScript).
However it makes the implementation of stack machines complicated.
It is desirable that CMObject is independent of OX stack machine.
Therefore we introduce an OpenMath like tree representation for CMO 
Tree object.
This method allows us to implement tree structure easily 
on individual OpenXM systems.
Note that CMO Tree corresponds to Symbol and Application in OpenMath.
*/


/*&C

*/
/*&jp
Lambda は関数を定義するための関数である.
Lisp の Lambda 表現と同じ.
*/
/*&eg
Lambda is used to define functions.
The notion ``lambda'' is borrowed from the language Lisp.
*/

\noindent
//&jp 例: $sin(x+e)$ の表現.
//&eg Example: the expression of $sin(x+e)$.
\begin{verbatim}
(CMO_TREE, (CMO_STRING, "sin"), 
    (CMO_LIST,[size=]1,(CMO_LIST,[size=]2,(CMO_STRING, "cdname"),
                                          (CMO_STRING,"basic")))
    (CMO_LIST,[size=]1, 
        (CMO_TREE, (CMO_STRING, "plus"), (CMO_STRING, "basic"),
            (CMO_LIST,[size=]2, (CMO_INDETERMINATE,"x"),
//&jp                  (CMO_TREE,(CMO_STRING, "e"),  自然対数の底
//&eg                  (CMO_TREE,(CMO_STRING, "e"),  the base of natural logarithms
    (CMO_LIST,[size=]1,(CMO_LIST,[size=]2,(CMO_STRING, "cdname"),
                                          (CMO_STRING,"basic")))
        ))
    )
)
\end{verbatim}
//&jp  Leave の成分には, 多項式を含む任意のオブジェクトがきてよい.
//&eg  Elements of the leave may be any objects including polynomials.

\noindent
Example:
\begin{verbatim}
sm1> [(plus) [[(cdname) (basic)]] [(123).. (345)..]] [(class) (tree)] dc ::
Class.tree [$plus$ , [[$cdname$ , $basic$ ]], [ 123 , 345 ]  ] 
\end{verbatim}

\noindent
Example:
\begin{verbatim}
asir
[753] taka_cmo100_xml_form(quote(sin(x+1)));
<cmo_tree>  <cmo_string>"sin"</cmo_string> 
 <cmo_list><cmo_int32 for="length">1</cmo_int32>   
   <cmo_list><cmo_int32 for="length">2</cmo_int32>    
     <cmo_string>"cdname"</cmo_string> 
     <cmo_string>"basic"</cmo_string> 
   </cmo_list> </cmo_list> 
<cmo_tree>    <cmo_string>"plus"</cmo_string> 
  <cmo_list><cmo_int32 for="length">1</cmo_int32>     
    <cmo_list><cmo_int32 for="length">2</cmo_int32>      
      <cmo_string>"cdname"</cmo_string> 
      <cmo_string>"basic"</cmo_string> 
    </cmo_list> </cmo_list> 
 <cmo_indeterminate> <cmo_string>"x"</cmo_string>  </cmo_indeterminate>
 <cmo_zz>1</cmo_zz>
</cmo_tree></cmo_tree>
\end{verbatim}


\bigbreak
//&jp 次に, 分散表現多項式に関係するグループを定義しよう.
/*&eg
Let us define a group for distributed polynomials. In the following,
DMS stands for Distributed Monomial System.
*/

\medbreak
\noindent
Group CMObject/DistributedPolynomials requires CMObject/Primitive,
CMObject/Basic. \\
Monomial, Monomial32, Coefficient, Dpolynomial, DringDefinition,
Generic DMS ring, RingByName, DMS of N variables $\in$ 
CMObject/DistributedPolynomials. \\
/*&jp
\begin{eqnarray*}
\mbox{Monomial} &:& \mbox{Monomial32}\, |\, \mbox{Zero} \\
\mbox{Monomial32}&:& ({\tt CMO\_MONOMIAL32}, {\sl int32}\, n,
{\sl int32}\, \mbox{e[1]}, \ldots,
{\sl int32}\, \mbox{e[n]}, \\
& & \ \mbox{Coefficient}) \\
& & \mbox{ --- e[i] で, $n$ 変数 monomial 
$x^e = x_1^{e_1} \cdots x_n^{e_n}$ の各指数 $e_i$
をあらわす.} \\
\mbox{Coefficient}&:& \mbox{ZZ} | \mbox{Integer32} \\
\mbox{Dpolynomial}&:& \mbox{Zero} \\
& & |\ ({\tt CMO\_DISTRIBUTED\_POLYNOMIAL},{\sl int32}\, m, \\
& & \ \ \mbox{DringDefinition},
[\mbox{Monomial32}|\mbox{Zero}], \\
& &\ \ 
\{\mbox{Monomial32}\}) \\
& &\mbox{--- m はモノミアルの個数である.}\\
\mbox{DringDefinition}
&:& \mbox{DMS of N variables} \\
& & |\ \mbox{RingByName} \\
& & |\ \mbox{Generic DMS ring} \\
& & \mbox{ --- 分散表現多項式環の定義. } \\
\mbox{Generic DMS ring}
&:& \mbox{({\tt CMO\_DMS\_GENERIC}) --- 新版はこちら}\\
\mbox{RingByName}&:& ({\tt CMO\_RING\_BY\_NAME}, {\sl Cstring}\  {\rm s}) \\
& & \mbox{ --- 名前 s で, 格納された ring 定義.} \\
\mbox{DMS of N variables}
&:& ({\tt CMO\_DMS\_OF\_N\_VARIABLES}, \\
& & \ ({\tt CMO\_LIST}, {\sl int32}\, \mbox{m},
{\sl Integer32}\,  \mbox{n}, {\sl Integer32}\,\mbox{p} \\
& & \ \ [,{\sl object}\,\mbox{s}, {\sl Cstring}\,\mbox{c}, 
          {\sl List}\, \mbox{vlist},
{\sl List}\, \mbox{wvec}, {\sl List}\, \mbox{outord}]) \\
& & \mbox{ --- m はあとに続く要素の数} \\
& & \mbox{ --- n は変数の数, p は 標数} \\
& & \mbox{ --- s は ring の名前} \\
& & \mbox{ --- c は係数環, QQ, ZZ の場合は文字列で QQ, ZZ と書く.} \\
& & \mbox{ --- vlist は Indeterminate のリスト(新版). 多項式環の変数リスト} \\
& & \mbox{ --- wvec は order をきめる weight vector,} \\
& & \mbox{ --- outord は出力するときの変数順序.} \\
\end{eqnarray*}
*/
/*&eg
\begin{eqnarray*}
\mbox{Monomial} &:& \mbox{Monomial32}\, |\, \mbox{Zero} \\
\mbox{Monomial32}&:& ({\tt CMO\_MONOMIAL32}, {\sl int32}\, n,
                      {\sl int32}\, \mbox{e[1]}, \ldots,
                      {\sl int32}\, \mbox{e[n]}, \\
                 & & \ \mbox{Coefficient}) \\
                 & & \mbox{ --- e[i] is the exponent $e_i$ of the monomial 
                      $x^e = x_1^{e_1} \cdots x_n^{e_n}$. } \\
\mbox{Coefficient}&:& \mbox{ZZ} | \mbox{Integer32} \\
\mbox{Dpolynomial}&:& \mbox{Zero} \\
                 & & |\ ({\tt CMO\_DISTRIBUTED\_POLYNOMIAL},{\sl int32}\, m, \\
                 & & \ \ \mbox{DringDefinition}, [\mbox{Monomial32}|\mbox{Zero}], \\
                 & &\ \ 
                    \{\mbox{Monomial32}\})  \\
                 & &\mbox{--- m is equal to the number of monomials.}\\
\mbox{DringDefinition}
                 &:& \mbox{DMS of N variables} \\
                 & & |\ \mbox{RingByName} \\
                 & & |\ \mbox{Generic DMS ring} \\
                 & & \mbox{ --- definition of the ring of distributed polynomials. } \\
\mbox{Generic DMS ring}
                 &:& ({\tt CMO\_DMS\_GENERIC}) \\
\mbox{RingByName}&:& ({\tt CMO\_RING\_BY\_NAME}, {\sl Cstring} s) \\
                 & & \mbox{ --- The ring definition referred by the name ``s''.} \\
\mbox{DMS of N variables}
                 &:& ({\tt CMO\_DMS\_OF\_N\_VARIABLES}, \\
                 & & \ ({\tt CMO\_LIST}, {\sl int32}\, \mbox{m},
                  {\sl Integer32}\,  \mbox{n}, {\sl Integer32}\, \mbox{p} \\
                 & & \ \ [,{\sl Cstring}\,\mbox{s}, {\sl List}\, \mbox{vlist},
                          {\sl List}\, \mbox{wvec}, {\sl List}\, \mbox{outord}]) \\
                 & & \mbox{ --- m is the number of elements.} \\
                 & & \mbox{ --- n is the number of variables, p is the characteristic} \\
                 & & \mbox{ --- s is the name of the ring, vlist is the list of variables.} \\
                 & & \mbox{ --- wvec is the weight vector.} \\
                 & & \mbox{ --- outord is the order of variables to output.} \\
\end{eqnarray*}
*/

/*&jp
RingByName や DMS of N variables はなくても, DMS を定義できる.
したがって, これらを実装してないシステムで DMS を扱うものが
あってもかまわない.

以下, 以上の CMObject  にたいする,
xxx = asir, kan の振舞いを記述する.
*/
/*&eg
Note that it is possible to define DMS without RingByName and 
DMS of N variables.

In the following we describe how the above CMObjects 
are implemented on Asir and Kan.
*/

\subsection{ Zero}
/*&jp
CMO では ゼロの表現法がなんとおりもあることに注意.
%% どのようなゼロをうけとっても,
%% システムのゼロに変換できるべきである.
*/
/*&eg
Note that CMO has various representations of zero.
*/


//&jp \subsection{ 整数 ZZ }
//&eg \subsection{ Integer ZZ }

\begin{verbatim}
#define     CMO_ZZ          20 
\end{verbatim}

/*&jp
この節ではOpen xxx 規約における任意の大きさの整数(bignum)の扱いについ
て説明する.  Open XM 規約における多重精度整数を表すデータ型 CMO\_ZZ は 
GNU MPライブラリなどを参考にして設計されていて, 符号付き絶対値表現を用
いている.  (cf. {\tt kan/sm1} の配布ディレクトリのなかの {\tt
plugin/cmo-gmp.c}) CMO\_ZZ は次の形式をとる.
*/
/*&eg
We describe the bignum (multi-precision integer) representation 
{\tt CMO\_ZZ} in OpenXM.
The format is similar
to that in GNU MP. (cf. {\tt plugin/cmo-gmp.c} in the {\tt kan/sm1} 
distribution). CMO\_ZZ is defined as follows.
*/

\begin{tabular}{|c|c|c|c|c|}
\hline
{\tt int32 CMO\_ZZ} & {\tt int32 $f$} & {\tt int32 $b_0$} & $\cdots$ &
{\tt int32 $b_{n}$} \\
\hline
\end{tabular}

/*&jp
$f$ は32bit整数である.  $b_0, \ldots, b_n$ は unsigned int32 である.
$|f|$ は $n+1$ である.  この CMO の符号は $f$ の符号で定める.  前述し
たように, 32bit整数の負数は 2 の補数表現で表される.

Open xxx 規約では上の CMO は以下の整数を意味する. ($R = 2^{32}$)
*/
/*&eg
$f$ is a 32bit integer. $b_0, \ldots, b_n$ are unsigned 32bit integers.
$|f|$ is equal to $n+1$. 
The sign of $f$ represents that of the above integer to be expressed. 
As stated in Section
\ref{sec:basic0}, a negative 32bit integer is represented by
two's complement.

In OpenXM the above CMO represents the following integer. ($R = 2^{32}$.)
*/

\[
\mbox{sgn}(f)\times (b_0 R^{0}+ b_1 R^{1} + \cdots + b_{n-1}R^{n-1} + b_n R^n).
\]

/*&jp
\noindent  例: 
{\tt int32} を network byte order で表現
しているとすると,例えば, 整数 $14$ は CMO\_ZZ で表わすと,
*/
/*&eg
\noindent Example:
If we express {\tt int32} by the network byte order,
a CMO\_ZZ $14$ is expressed by
*/
\[
\mbox{(CMO\_ZZ, 1, 0, 0, 0, e)},
\]
//&jp と表わす. これはバイト列では
//&eg The corresponding byte sequence is
\[
\mbox{\tt 00 00 00 14 00 00 00 01 00 00 00 0e}
\]
//&jp となる.


//&jp なお ZZ の 0 ( (ZZ) 0 と書く ) は, {\tt (CMO\_ZZ, 00,00,00,00)} と表現する.
//&eg Note that CMO\_ZZ 0 is expressed by {\tt (CMO\_ZZ, 00,00,00,00)}.


//&jp \subsection{ 分散表現多項式 Dpolynomial }
//&eg \subsection{ Distributed polynomial Dpolynomial }

/*&jp
環とそれに属する多項式は次のような考えかたであつかう.

Generic DMS ring に属する元は,
変数を $n$ 個持つ 適当な係数集合 $K$ を持つ多項式環 $K[x_1, \ldots, x_n]$ 
の元である.
係数集合 $K$ がなにかは, 実際データを読み込み, Coefficient を見た段階で
わかる.
この環に属する多項式を CMO 形式でうけとった場合, 各サーバはその
サーバの対応する Object  に変換しないといけない. 
この変換の仕方は, 各サーバ毎にきめる.

Asir の場合は, $K[x_1, \ldots, x_n]$ の元は分散表現多項式に変換される.
\noroa{ でも, order はどうなるの? }

{\tt kan/sm1} の場合は事情は複雑である.
{\tt kan/sm1} は, Generic DMS ring にあたる クラスをもたない.
つまり, Default で存在する, $n$ 変数の分散表現多項式環は存在しないわけである.
したがって, {\tt kan/sm1} では, DMS of N variables が来た場合,
これを CurrentRing の元として読み込む.  CurrentRing の変数の数が $n'$
で, $n' < n$ だと新しい多項式環を生成してデータを読み込む.
Order その他の optional 情報はすべて無視する.

DMS の 2 番目のフィールドで,
Ring by Name を用いた場合, 現在の名前空間で変数 yyy に格納された ring object
の元として, この多項式を変換しなさいという意味になる.
{\tt kan/sm1} の場合, 環の定義は ring object として格納されており,
この ring object を 変数 yyy で参照することにより CMO としてうけとった
多項式をこの ring の元として格納できる.
*/

/*&eg
We treat polynomial rings and their elements as follows.

Generic DMS ring is an $n$-variate polynomial ring $K[x_1, \ldots, x_n]$,
where $K$ is a coefficient set. $K$ is unknown in advance
and it is determined when coefficients of an element are received.
When a server has received an element in Generic DMS ring,
the server has to translate it into the corresponding local object
on the server. Each server has its own translation scheme.
In Asir such an element are translated into a distributed polynomial.
In {\tt kan/sm1} things are complicated.
{\tt kan/sm1} does not have any class corresponding to Generic DMS ring.
{\tt kan/sm1} translates a DMS of N variables into an element of
the CurrentRing. 
If the CurrentRing is $n'$-variate and $n' < n$, then
an $n$-variate polynomial ring is newly created. 


If RingByName ({\tt CMO\_RING\_BY\_NAME}, yyy) 
is specified as the second field of DMS,
it requests a sever to use a ring object whose name is yyy
as the destination ring for the translation.
*/

\medbreak \noindent
//&jp {\bf Example}: (すべての数の表記は 16 進表記)
//&eg {\bf Example}: (all numbers are represented in hexadecimal notation)
{\footnotesize \begin{verbatim}
Z/11Z [6 variables]
(kxx/cmotest.sm1) run
[(x,y) ring_of_polynomials ( ) elimination_order 11 ] define_ring ;
(3x^2 y). cmo /ff set ;
[(cmoLispLike) 1] extension ;
ff ::
Class.CMO CMO StandardEncoding: size = 52, size/sizeof(int) = 13, 
tag=CMO_DISTRIBUTED_POLYNOMIAL 

  0  0  0 1f  0  0  0  1  0  0  0 18  0  0  0 13  0  0  0  6
  0  0  0  0  0  0  0  2  0  0  0  0  0  0  0  0  0  0  0  1
  0  0  0  0  0  0  0  2  0  0  0  3

ff omc ::
 (CMO_DISTRIBUTED_POLYNOMIAL[1f],[size=]1,(CMO_DMS_GENERIC[18],),
  (CMO_MONOMIAL32[13],3*x^2*y),),
\end{verbatim} }
/*&jp
$ 3 x^2 y$ は 6 変数の多項式環の 元としてみなされている.
*/
/*&eg
$3 x^2 y$ is regarded as an element of a six-variate polynomial ring.
*/


//&jp \subsection{再帰表現多項式の定義} 
//&eg \subsection{Recursive polynomials} 

\begin{verbatim}
#define CMO_RECURSIVE_POLYNOMIAL        27
#define CMO_POLYNOMIAL_IN_ONE_VARIABLE  33
\end{verbatim}

Group CMObject/RecursivePolynomial requires CMObject/Primitive, CMObject/Basic.\\
Polynomial in 1 variable, Coefficient, Name of the main variable,
Recursive Polynomial, Ring definition for recursive polynomials
$\in$ CMObject/RecursivePolynomial \\

/*&jp
\begin{eqnarray*}
\mbox{Polynomial in 1 variable} &:& 
\mbox{({\tt CMO\_POLYNOMIAL\_IN\_ONE\_VARIABLE},\, {\sl int32}\, m, } \\
& & \quad \mbox{ Name of the main variable }, \\
& & \quad \mbox{ \{ {\sl int32} e, Coefficient \}} ) \\
& & \mbox{ --- m はモノミアルの個数. } \\
& & \mbox{ --- e, Coefficieint はモノミアルを表現している. } \\
& & \mbox{ --- 順序の高い順にならべる. 普通は巾の高い順.} \\
& & \mbox{ ---  e は 1変数多項式の巾をあらわす. } \\
\mbox{Coefficient} &:& \mbox{ ZZ} \,|\, \mbox{ QQ } \,|\, 
\mbox{ integer32  } \,|\,
\mbox{ Polynomial in 1 variable } \\
& & \quad \,|\, \mbox{Tree} \,|\, \mbox{Zero} \,|\,\mbox{Dpolynomial}\\
\mbox{Name of the main variable } &:& 
\mbox{ {\sl int32} v }   \\
& & \mbox{ --- v は 変数番号 (0 からはじまる) を表す. } \\
\mbox{Recursive Polynomial} &:& 
\mbox{ ( {\tt CMO\_RECURSIVE\_POLYNOMIAL}, } \\
& & \quad \mbox{ RringDefinition, } \\
& & \quad
\mbox{ Polynomial in 1 variable}\, | \, \mbox{Coefficient} )  \\
\mbox{RringDefinition} 
& : &  \mbox{ {\sl List} v } \\
& & \quad \mbox{ --- v は, 変数名(indeterminate) または Tree のリスト. } \\
& & \quad \mbox{ --- 順序の高い順. } \\
\end{eqnarray*}
*/
/*&eg
\begin{eqnarray*}
\mbox{Polynomial in 1 variable} &:& 
\mbox{({\tt CMO\_POLYNOMIAL\_IN\_ONE\_VARIABLE},\, {\sl int32}\, m, } \\
& & \quad \mbox{ Name of the main variable }, \\
& & \quad \mbox{ \{ {\sl int32} e, Coefficient \}} ) \\
& & \mbox{ --- m is the number of monomials. } \\
& & \mbox{ --- A pair of e and Coefficient represents a monomial. } \\
& & \mbox{ --- The pairs of e and Coefficient are sorted in the } \\
& & \mbox{ \quad decreasing order, usually with respect to e.} \\
& & \mbox{ ---  e denotes an exponent of a monomial with respect to } \\
& & \mbox{ \quad the main variable. } \\
\mbox{Coefficient} &:& \mbox{ ZZ} \,|\, \mbox{ QQ } \,|\, 
\mbox{ integer32  } \,|\,
\mbox{ Polynomial in 1 variable } \\
& & \quad \,|\, \mbox{Tree} \,|\, \mbox{Zero} \,|\,\mbox{Dpolynomial}\\
\mbox{Name of the main variable } &:& 
\mbox{ {\sl int32} v }   \\
& & \mbox{ --- v denotes a variable number. } \\
\mbox{Recursive Polynomial} &:& 
\mbox{ ( {\tt CMO\_RECURSIVE\_POLYNOMIAL}, } \\
& & \quad \mbox{ RringDefinition, } \\
& & \quad
\mbox{ Polynomial in 1 variable}\, | \, \mbox{Coefficient} )  \\
\mbox{RringDefinition} 
& : &  \mbox{ {\sl List} v } \\
& & \quad \mbox{ --- v is a list of names of indeterminates or trees. } \\
& & \quad \mbox{ --- It is sorted in the decreasing order. } \\
\end{eqnarray*}
*/
\bigbreak
\noindent
Example:
\begin{verbatim}
(CMO_RECURSIEVE_POLYNOMIAL, ("x","y"),
(CMO_POLYNOMIAL_IN_ONE_VARIABLE, 2,      0,  <--- "x"
  3, (CMO_POLYNOMIAL_IN_ONE_VARIABLE, 2, 1,  <--- "y"
       5, 1234,
       0, 17),
  1, (CMO_POLYNOMIAL_IN_ONE_VARIABLE, 2, 1,  <--- "y"
       10, 1,
       5, 31)))
\end{verbatim}
//&jp これは,
//&eg This represents
$$   x^3 (1234 y^5 + 17 ) +  x^1 (y^{10} + 31 y^5)  $$
/*&jp
をあらわす.
%%非可換多項式もこの形式であらわしたいので, 積の順序を上のように
%%すること. つまり, 主変数かける係数の順番.
*/
/*&eg
%%We intend to represent non-commutative polynomials with the
%%same form. In such a case, the order of products are defined
%%as above, that is a power of the main variable $\times$ a coeffcient.

*/

\noindent
\begin{verbatim}
sm1
sm1>(x^2-h). [(class) (recursivePolynomial)] dc /ff set ;
sm1>ff ::
Class.recursivePolynomial h * ((-1)) + (x^2  * (1))
\end{verbatim}

//&jp \subsection{CPU依存の double } 
//&eg \subsection{CPU dependent double} 

\begin{verbatim}
#define CMO_64BIT_MACHINE_DOUBLE   40
#define CMO_ARRAY_OF_64BIT_MACHINE_DOUBLE  41
#define CMO_128BIT_MACHINE_DOUBLE   42
#define CMO_ARRAY_OF_128BIT_MACHINE_DOUBLE  43
\end{verbatim}

\noindent
Group CMObject/MachineDouble requires CMObject/Primitive.\\
64bit machine double, Array of 64bit machine double
128bit machine double, Array of 128bit machine double
$\in$ CMObject/MachineDouble \\

/*&jp
\begin{eqnarray*}
\mbox{64bit machine double} &:& 
\mbox{({\tt CMO\_64BIT\_MACHINE\_DOUBLE}, } \\
& & \quad \mbox{ {\sl byte} s1 , \ldots , {\sl byte} s8})\\
& & \mbox{ --- s1, $\ldots$, s8 は {\tt double} (64bit). } \\
& & \mbox{ --- この表現はCPU依存である.}\\
&&  \mbox{\quad\quad byte order negotiation を用いる.} \\
\mbox{Array of 64bit machine double} &:& 
\mbox{({\tt CMO\_ARRAY\_OF\_64BIT\_MACHINE\_DOUBLE}, {\sl int32} m, } \\
& & \quad \mbox{ {\sl byte} s1[1] , \ldots , {\sl byte}}\, s8[1], \ldots , {\sl byte}\, s8[m])\\
& & \mbox{ --- s*[1], $\ldots$ s*[m] は m 個の double (64bit) である. } \\
& & \mbox{ --- この表現はCPU依存である.}\\
& & \mbox{ \quad\quad byte order negotiation を用いる.} \\
\mbox{128bit machine double} &:& 
\mbox{({\tt CMO\_128BIT\_MACHINE\_DOUBLE}, } \\
& & \quad \mbox{ {\sl byte} s1 , \ldots , {\sl byte} s16})\\
& & \mbox{ --- s1, $\ldots$, s16 は {\tt long double} (128bit). } \\
& & \mbox{ --- この表現はCPU依存である.}\\
&&  \mbox{\quad\quad byte order negotiation を用いる.} \\
\mbox{Array of 128bit machine double} &:& 
\mbox{({\tt CMO\_ARRAY\_OF\_128BIT\_MACHINE\_DOUBLE}, {\sl int32} m, } \\
& & \quad \mbox{ {\sl byte} s1[1] , \ldots , {\sl byte} s16[1], \ldots , {\sl byte} s16[m]})\\
& & \mbox{ --- s*[1], $\ldots$ s*[m] は m 個の long double (128bit) である. } \\
& & \mbox{ --- この表現はCPU依存である.}\\
& & \mbox{ \quad\quad byte order negotiation を用いる.} 
\end{eqnarray*}
*/
/*&eg
\begin{eqnarray*}
\mbox{64bit machine double} &:& 
\mbox{({\tt CMO\_64BIT\_MACHINE\_DOUBLE}, } \\
& & \quad \mbox{ {\sl byte} s1 , \ldots , {\sl byte} s8})\\
& & \mbox{ --- s1, $\ldots$, s8 は {\tt double} (64bit). } \\
& & \mbox{ --- Encoding depends on CPU.}\\
&&  \mbox{\quad\quad Need the byte order negotiation.} \\
\mbox{Array of 64bit machine double} &:& 
\mbox{({\tt CMO\_ARRAY\_OF\_64BIT\_MACHINE\_DOUBLE}, {\sl int32} m, } \\
& & \quad \mbox{ {\sl byte} s1[1] , \ldots , {\sl byte}}\, s8[1], \ldots , {\sl byte}\, s8[m])\\
& & \mbox{ --- s*[1], $\ldots$ s*[m] are 64bit double's. } \\
& & \mbox{ --- Encoding depends on CPU.}\\
& & \mbox{\quad\quad Need the byte order negotiation.} \\
\mbox{128bit machine double} &:& 
\mbox{({\tt CMO\_128BIT\_MACHINE\_DOUBLE}, } \\
& & \quad \mbox{ {\sl byte} s1 , \ldots , {\sl byte} s16})\\
& & \mbox{ --- s1, $\ldots$, s16 は {\tt long double} (128bit). } \\
& & \mbox{ --- Encoding depends on CPU.}\\
& & \mbox{\quad\quad Need the byte order negotiation.} \\
\mbox{Array of 128bit machine double} &:& 
\mbox{({\tt CMO\_ARRAY\_OF\_128BIT\_MACHINE\_DOUBLE}, {\sl int32} m, } \\
& & \quad \mbox{ {\sl byte} s1[1] , \ldots , {\sl byte} s16[1], \ldots , {\sl byte} s16[m]})\\
& & \mbox{ --- s*[1], $\ldots$ s*[m] are 128bit long double's. } \\
& & \mbox{ --- Encoding depends on CPU.}\\
& & \mbox{\quad\quad Need the byte order negotiation.} \\
\end{eqnarray*}
*/

\bigbreak

\begin{verbatim}
#define CMO_IEEE_DOUBLE_FLOAT 51
\end{verbatim}

/*&jp
IEEE 準拠の float については, IEEE 754 double precision floating-point
format (64 bit) の定義を見よ.

256.100006 の Intel Pentium の double64 での内部表現は
{\tt cd 0c 80 43 } \\
256.100006 の PowerPC (Mac) の double64 での内部表現は
{\tt 43 80 0c cd }.
この例でみるように byte の順序が逆である.
エンジンスタートの時の byte order negotiation で byte の順序を指定する.


*/
/*&eg
See IEEE 754 double precision floating-point (64 bit) for the details of 
float compliant to the IEEE standard.

The internal expression of 256.100006 in the Intel Pentium  is
{\tt cd 0c 80 43 } \\
The internal expression of 256.100006 in the PowerPC (Mac) is
{\tt 43 80 0c cd }.
As you have seen in this example,
the orders of the bytes are opposite each other.
The byte order is specified by the byte order negotiation protocol
when the engine starts.

*/

\subsection{Bigfloat32}
/*&jp
int32 を基本とした bigfloat の表現方法について述べる.
この形式は mpfr を 32bit CPU で使用した時の内部表現と共通である.
*/
/*&eg
This subsection describes our format for bigfloat in terms of the int32.
This format is identical to the internal format of mpfr on 32 bit CPU's.
*/
Ref: {\tt OpenXM/src/mpfr/bfsize/bfsize.c}

\begin{verbatim}
#define     CMO_BIGFLOAT32          52 
\end{verbatim}


/*&jp
Bigfloat32 は次の形式の int32 の配列である.
*/
/*&eg
The bigfloat32 is an array of int32 numbers of the following format.
*/

\begin{center}
{\sl int32}\, {\rm prec}, {\sl int32}\, {\rm sign}, {\sl int32}\, {\rm exp}, 
{\sl int32}\, \mbox{a[1]}, \ldots , {\sl int32}\, \mbox{a[k]}  
\end{center}

/*&jp
p=prec は精度, s=sign は符号(1 が正の数,  -1 (2の補数表現)が負の数) ,
E=exp は指数部で, 上のデータは数
$$ s (a[k]/B + a[k-1]/B^2 + ... + a[1]/B^k) 2^E  $$
を表す.
ここで $B=2^{32}$,
$k = \lceil p/32 \rceil$
である.
*/
/*&eg
p=prec is the precision, s=sign is the sign(1 means positive,  -1 (expressed by two's complement) is nevative),
E=exp is the exponent, and the data above expresses the number
$$ s (a[k]/B + a[k-1]/B^2 + ... + a[1]/B^k) 2^E.  $$
Here, $B=2^{32}$,
$k = \lceil p/32 \rceil$.
*/

