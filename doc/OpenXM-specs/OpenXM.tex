%%  $OpenXM: OpenXM/doc/OpenXM-specs/OpenXM.tex,v 1.22 2016/08/22 09:08:50 takayama Exp $
\documentclass{article}
\IfFileExists{epsfig.sty}{\usepackage{epsfig}}{}
\usepackage{html}
\begin{document}
/*&jp
\title{{\bf Open XM の設計と実装: クライアントサーバモデルと数学共通表現 (OpenXM-RFC 100, proposed standard)} \\
 --- Open message eXchange protocol for Mathematics }
\author{ 野呂正行, 
         高山信毅\thanks{神戸大学理学部数学教室,\htmladdnormallink{http://www.math.kobe-u.ac.jp/$\tilde{\ }$taka}{http://www.math.kobe-u.ac.jp/\tilde{\ }taka}}}
\date{ 1997年, 11月20日 --- 2000年, 11月17日, \\
       2001年, 8月27日, 2002年, 1月20日 (小修正),\\
        これ以後の修正は changeLog を見よ.}
*/
/*&eg
\title{{\bf Design and Implementation of OpenXM client server model and
common mathematical object format (OpenXM-RFC 100, proposed standard)} \\
 --- Open message eXchange protocol for Mathematics 
}
\author{ Masayuki Noro, 
         Nobuki Takayama\thanks{Department of Mathematics, Kobe University,\htmladdnormallink{http://www.math.kobe-u.ac.jp/$\tilde{\ }$taka}{http://www.math.kobe-u.ac.jp/\tilde{\ }taka}}}
\date{ November 20, 1997 --- November 17, 2000, \\
       August 27, 2001, January 20, 2002 (minor change), \\
       See changeLog about changes after this date.}
*/

\maketitle

\noindent ChangeLog

/*&eg  


\noindent
2004-3-8: We add a new stackmacine command {\tt SM\_executeFunctionWithOptionalArgument}. \\
2005-3-4: Added a description about a byte order negotiation to send floating 
point numbers.  {\tt CMO\_64BIT\_MACHINE\_DOUBLE} \\
2016-08-22: Added a description on the bigfloat32. Old bigfloat is removed.
Refer to codes of oxpari. \\
2016-08-22: option {\tt no\_ox\_reset}
*/



/*&jp


\noindent
2004-3-8: 新しい SM コマンド {\tt SM\_executeFunctionWithOptionalArgument} を加えた. \\
2005-3-4: 浮動小数点数を送るための byte order negotiation についての
 記述を加えた. {\tt CMO\_64BIT\_MACHINE\_DOUBLE} \\
2016-08-22: bigfloat32 の記述を加えた. 古い bigfloat の記述は削除した.
oxpari のコードも参照. \\
2016-08-22: option {\tt no\_ox\_reset}
*/

\bigbreak


\newpage
\tableofcontents
\newpage
\def\noroa#1{  }
\def\remove#1{ }

//&C \noindent Draft for protocol version 1.1.3.  \\
//&C 1.1.3 is encoded as 001001003 in mathcap. \\


/*&jp
{\tt kxx/openxxx.tex}. {\bf Draft} 1997, 11/20 --- .
この文書は open XM の設計用のメモとしての役目もあるので,
一部のファイル名は開発者のみが参照できる.
*/

/*&C
%% $OpenXM: OpenXM/doc/OpenXM-specs/introduction.tex,v 1.2 2000/01/24 07:36:35 noro Exp $
//&jp \section{�Ϥ����}
//&eg \section{Introduction}

/*&jp
OpenXM ��, ���ʤ������פޤ��ϰۤʤ륿���פο��إץ������֤�
��å������Τ��Ȥ�ε���Ǥ���.
��ȯ��ư����, ����(�ޤ��ϸ���Ū��)���إ��եȤ���߾������μ¸�
�����ʬ���׻��μ�����
���Ǥ��ä���, ���������إ��եȴ֤����Ǥʤ�, ��ץ����եȤ�,
���󥿥饯�ƥ��֤ʿ�����,
����ˤϿ��إǥ�������ʪ���ѤΥ��եȤ����ε���˽���, 
���إ��եȤ�ƤӽФ����Ȥʤɤˤ����ѤǤ���.
���̤���ɸ�� OpenXM ���إ��եȥѥå��������뤳�ȤǤ���.
����Ϥ��ޤ��ޤʿ��إ��եȤ��ĤΥѥå������Ȥ���
��ñ�˹����ʸ�����Ȥ���褦�ˤ���ץ��������ȤǤ���.
���� OpenXM.tar.gz �ˤ�,
asir, sm1, phc, gnuplot, tigers �����äƤ���.
OpenXM ���إ��եȥѥå������ץ��������Ȥ�
���äǤ���褦�� CMO ��
������ĥ���Ƥ�������ε���������ΤȤ���.
*/
/*&eg
OpenXM is a free, or Open Source, infrastructure for mathematical
software systems.
It provides methods and protocols
for interactive distributed computation and
for integrating mathematical software systems.
OpenXM package is a set of software systems that support OpenXM protocols.
It is currently a collection of softwares
{\tt Risa/Asir} \cite{asir}, {\tt Kan/sm1} \cite{kan}, {\tt PHC} pack \cite{phc}
, {\tt GNUPLOT},
{\tt Mathematica} interface, and
{\tt OpenMath}/XML \cite{openmath} translator.
*/

/*&jp
��������Ū�ʿ��إ��եȤ�¿����ȯ
����Ƥ��뤬, ������ο��إ��եȤ�¾�Υ��եȤ�ꥵ�֥롼����Ȥ���
�ƤӽФ��뵡ǽ���Ĥ٤��Ǥ���.
���Τ褦�˶�Ĵ��ǽ�ˤ����줿�ץ�������񤯤�����߷פ�
�����ɥ饤��, �ץ�������ˡ�Τ褦�ʤ�Τ���Ƥ���Τ�, 
���Υץ��������Ȥ���Ū�Ǥ���.
���Τ褦�ʥ����ɥ饤��ˤ��äƥץ�����ह�뤳�Ȥˤ��,
���إ��르�ꥺ�༫�Τ˽���Ǥ���Ķ����¸��Ǥ��뤳�Ȥ�
���Ԥ��Ƥ���.

�߷פ����ˤȤ���, (1) ñ�� (2) ��ĥ�� (3) �����δ��ؤ� (4) ������(5) 
�⿮����(robustness),�˽Ť��򤪤��Ƥ���.  

OpenXM �Ϥʤˤ�ͤ����˴�ñ����³�Ǥ��륷���ƥ������,�Ȥ����ޤ��
Ū�ǤϤʤ�.  ����Ū�� object �ϰ����ǤϤ����ʤ���, ���������쵬�ʤ�
�Ĥ���Ȥ����Τϵ����󤯤ʤ�Ż��Ǥ���. ���Τ����, ����ꤹ����������
�����ä��ǡ����򴹤�ʬ�������ƥ๽�ۤλŻ���ڤˤ������Ȥ����Τ�������
������1��ɸ�Ǥ���.  �ޤ�, Mathematica �� Maple �Ȥ��ä���������祽��
�Ȥ�, Free Mathematical Softwares �Ȥ��Ƥ���켫�Ȥμ�ǤĤ��äƤ���
����δ��äǤ⤢��.
*/

/*&eg
We have been profited from increasing number
of mathematical software systems.
These are usually ``expert'' systems in one area of mathematics
such as ideals, groups, numbers, polytopes, and so on.
They have their own interfaces and data formats,
which are fine for intensive users of these systems.
However, a unified system will be more convenient
for users who want to explore a new area of mathematics with these
software systems or users who need these systems only occasionally.
It is also wonderful for developpers to have various software components
that can be used from the target system.
 
OpenXM provides not only data representation and communication protocols
but also programming guidelines to develop cooperative applications.
One will be able to concentrate on
developing mathematical algorithms with such guidelines.
Our design goals are (1) simpleness, (2) extensibility, (3) 
easiness of implementation, (4) practicality, and (5) robustness.

We believe that an open integrated system is a future of mathematical
softwares.  However, it might be just a dream without realizability.
We want to build a prototype of such an open system by using existing
standards, technologies and several mathematical softwares.  We want
to see how far we can go with this approach.
*/

/*&jp
����Ū�� Object ��ɤΤ褦��ɽ������Τ�, �ɤ�������Τ���ͤ��뤳�Ȥ�
�褷�ƤĤޤ�ʤ�����ǤϤʤ�.
���Τ褦�������, ���������ص������¤��������Ȼ��Ƥ��뤫�⤷��ʤ�.
�桹��, ������ $0$ ��ޤ��10�ʿ���ɽ����,
��ʬ�� $dx$ �Ƚ�, ������ $ \longrightarrow $ �Ǥ���魯.
�����ε���ˡ����ɤ����¿�������פ����Ƥ��뤫, �פ���Ϥ����ߤ���.
*/
/*&eg
It is not an obvious problem to consider how mathematical objects
are represented and communicated.
It may be similar to trying to create new mathematical symbols.
We have the decimal notation to represent numbers, the symbol $dx$
to represent a differential, and $ \longrightarrow $ to represent
a mapping. One should imagine how we are benefited from these notations.
*/

%% �ѿ�̾��ɤ����뤫Ǻ��Τˤ���Ƥ�.

/*&jp
OpenXM ������Ū�ˤ�
OX ��, SM ��, CMO �ؤˤ狼���.
OX Object ��, �ǡ���, ���ޥ�ɤ����̤Ǥ���.
�ǡ����Ϥ��Υץ��������ȤΥ��ꥸ�ʥ�Υǡ�������
�Ǥ��� CMO (Common Mathematical Object Format) ��
���Ф��Ƥ�褤��, MP �� Open MATH �ʤɤη������Ѥ��Ƥ�褤.
SM �ؤ� (�����å��ޥ���)�����Ф����椹�뤿���̿���
���Ĥޤ�Ǥ���, �����Ф���Ʊ��Ū��ư����뤳�Ȥ���ǽ�Ǥ���.
���������⤤�Τ�, IMC �ʤɤΥ�⡼�ȥץ��������㥳����Ϥ�
�ץ��ȥ���⥨�ߥ�졼�ȤǤ���.
OX ��å������� DTD ���Ѥ��������ǽ�Ǥ���, XML ���Ѥ���
���ҤǤ���.
*/
/*&eg 
In OpenXM, communication is an exchange of messages. 
The messages are classified
into three types: DATA, COMMAND, and SPECIAL.  They are called OX
(OpenXM) messages.  Among the three types, {\it OX data messages} wrap
mathematical data.  We use standards of mathematical data formats such
as OpenMath and MP as well as our own data format {\it CMO} ({\it
Common Mathematical Object format}).  Servers, which provide services
to other processes, are stack machines.  The stack machine is called
the {\it OX stack machine}.  Existing mathematical software systems are
wrapped with this stack machine.  OX stack machines work in the
asynchronous mode which is similar to X servers. That is, OpenXM
server won't send messages to the client unless it is requested
to send them. It is also possible to emulate RPC and a web server for MCP
\cite{iamc} on our asynchronous OX stack machines.
OX messages can be defined by DTD and be expressed by XML.
We call it OpenXM/XML.
*/
/*&jp
OpenXM �˽�򤷤������ƥ� xxx ��, open xxx �Ȥ��.
���Ȥ���, OpenXM �б��� asir �� open asir �Ǥ���,
OpenXM �б��� kan/sm1 �� open sm1 �Ǥ���.
*/
/*&eg
A system xxx complient to the OpenXM protocol is called open xxx.
For example Asir complient to the OpenXM protocol is called open Asir, and
kan/sm1 complient to the OpenXM protocol is called open sm1.
*/

/*&jp
OpenXM ��, �ǡ������򤭤�Ƥ�����ʬ��,
���̥����å���������Ƥ�����ʬ�ˤ狼���.
�褤���Ȥ����ɤ����������ʤ���,
OpenXM ����������Ȥ�, unicode �������Ȥ˻��Ƥ�����ʬ�⤢��.
���Ȥ���, �������쥢�����Ƕ��̤˻��Ѥ���Ƥ��뤬, �������ñ�̤�
�����������Ťİ㤦��Τ�����.
unicode �Ǥ�, ����������Ʊ�� code �ˤ��������.
OpenXM �Ǥ�, �����ƥ���ˤ��Ȥʤ뤬���Ƥ���ǡ�������
Ʊ�����Ȥ��Ƥ��Ĥ���.
���Ȥ���, ``ʬ��¿�༰'' �ΰ�̣��, asir �� kan/sm1 �ǰۤʤ뤬,
OpenXM �Ǥ�, Ʊ�����Ȥ��Ƥ��Ĥ�����.
����������Ǥ��Խ�ʬ�ʾ�礬����Τ�,�����ƥ��ͭ�Υǡ������ⰷ����
�褦�ʻ��Ȥߤ��Ѱդ��Ƥ���.
(���ä��Τ��Ȥ��Ǥ�, �������, unicode �Ǥʤ�, ISO ��ʸ�����Ϥ��б����뤫.)
���̥����å������, ���������ζ��̥��ޥ�ɤ�Τ���
����Ū�˥����ƥ���˸�ͭ�δؿ��ˤ��¹Ԥ����.
����ˤ��, open xxx �Υ����Фμ�������ӥޥ˥奢��ε��Ҥ�
�������뤷����ñ�ˤʤ�.
�����ƥ���˸�ͭ�δؿ����̤δؿ�̾�ˤ���ץ��������Ȥ�
�ͤ��Ƥ��뤬, ���̤δؿ�̾�� OpenMath �Τ�Τ����Ѥ���ͽ��Ǥ���.
*/
/*&eg
% not translated yet.
*/




%% $OpenXM: OpenXM/doc/OpenXM-specs/cmo-basic0.tex,v 1.4 2000/02/02 02:03:40 takayama Exp $
//&jp \section{CMO Primitive object}
//&eg \section{CMO Primitive object}
\label{sec:basic0}
/*&jp
CMO (Common Mathematical Object format) ���롼�� Primitive �� Object ��, 
������٥�Υǡ�����, {\tt int} , {\tt string}
�ʤɤ��б����� object �Ǥ���.
���� CMO �Ϥ��٤Ƥ� OpenXM ���������Ƥ���Ȳ��ꤵ���.
������Ǥ�, ����ȥ����������Ȥ���, ���롼�� Primitive ��°���� CMObject
(Common Mathematical Object) �����Ū����ˡ��Ĥ��鷺Ƴ�����褦.
*/
/*&eg
Objects in CMO (Common Mathematical Object format) group Primitive 
are primitive data such as {\tt int}, {\tt string}.
All OpenXM compliant systems should implement all data types
in the group Primitive.
In this section, as an introduction, we will introduce
CMObject (Common Mathematical Object) of the group Primitive without 
using the Backus-Nauer form.
*/
/*&jp
���Υ��롼�פ�������̾����,
CMObject/Primitive  �Ǥ���.
�ʲ�, {\tt int32} ��2�����ɽ�����줿
32 bit integer �򤢤�魯
(����Ϥ褯�Ȥ���׻����Ǥ� C ����� int ������ɽ��).
{\tt byte} �� 8 bit �ǡ����򤢤�魯.
*/
/*&eg
The canonical name of this group is
CMObject/Primitive.
In the sequel,
{\tt int32} means the signed 32 bit integer expressed by two's complement
(internal expressions of {\tt int} of the language C usually use 
this expression).
{\tt byte} means 8 bit data.
*/

//&C  
/*&jp
CMObject �� TCP/IP �Ѥμ����Ǥ�,
CMO �� object �� \\
\begin{tabular}{|c|c|}
\hline
{\tt cmo\_tag}& {\tt cmo\_body} \\ 
\hline
\end{tabular} \\
�ʤ���򤷤Ƥ���.
������, {\tt cmo\_tag} ��, ����
{\tt int32} ��ɽ�������Τȵ��󤹤�.
*/
/*&eg
In our encoding of the CMO's for TCP/IP,
any CMObject consists of a tag and a body: \\
\begin{tabular}{|c|c|}
\hline
{\tt cmo\_tag}& {\tt cmo\_body} \\ 
\hline
\end{tabular} \\
{\tt cmo\_tag} should be given by a positive
{\tt int32}.
*/

/*&C  

*/
/*&jp
{\tt cmo\_tag} �� object �Υ����פ򤢤�魯�����Ǥ���,
�ʲ��Τ褦�˷��Ƥ���.
*/
//&eg The following is a list of tags of CMObject/Primitive.
/*&C
@../SSkan/plugin/cmotag.h
\begin{verbatim}
#define LARGEID  0x7f000000   /* 2130706432 */
#define CMO_ERROR2 (LARGEID+2)
#define CMO_NULL   1
#define CMO_INT32  2
#define CMO_DATUM  3
#define CMO_STRING 4
#define CMO_MATHCAP 5
#define CMO_LIST 17
\end{verbatim}
*/

/*&jp
�ʲ�, �� object �� �ե����ޥåȤ���������.
������, ���饤����ȤϤ��٤Ƥ� object �� CMO �����򥵥ݡ��Ȥ���ɬ�פϤʤ���,
{\tt CMO\_ERROR2}, {\tt CMO\_NULL},
{\tt CMO\_INT32}, {\tt CMO\_STRING}, {\tt CMO\_MATHCAP}, {\tt CMO\_LIST}
�ϺǤ����Ū�ʥǡ����Ǥ���,
�ޤ����ƤΥ�����, ���饤����Ȥ��������٤� CMO �ǡ����Ǥ���.
*/
/*&eg
We will explain each object format.
Servers and clients do not need to implement all CMO's.
However, 
{\tt CMO\_ERROR2}, {\tt CMO\_NULL},
{\tt CMO\_INT32}, {\tt CMO\_STRING}, {\tt CMO\_MATHCAP}, {\tt CMO\_LIST}
are primitive data and
all servers and clients have to implement them.
*/

/*&C

\medbreak \noindent
*/
//&jp CMObject Error2 �� \\
//&eg CMObject Error2 is of the form \\
/*&C
\begin{tabular}{|c|c|}
\hline
{\tt int32 CMO\_ERROR2} & {\sl CMObject} {\rm ob} \\ 
\hline
\end{tabular} \\
*/
/*&jp
�ʤ����ɽ������.  
���顼�λ��� push ���� object �Ǥ���, {\it CMObject} ob ��
��ʬ�˾ܺ٤ʥ��顼���󤬤Ϥ���.
ob �ϥꥹ�ȤǤ���,  TCP/IP �ˤ�륹�ȥ꡼�෿��³�ξ��,
�ǽ����ʬ�ϥ��顼�򵯤����� OX ��å�����(���)
�Υ��ꥢ���ֹ�Ǥʤ��Ȥ����ʤ�.
���ꥢ���ֹ�� Integer32 ��ɽ������.

\noindent
Remark: ���Ū����ͳ�ˤ��, ���顼���֥������Ȥ� CMO ������ {\it
CMO\_ERROR2} ��̾�դ����Ƥ���. �����Ǥλ��ͽ�Ǥ� {\tt CMO\_ERROR}
���ѹ�����ͽ��Ǥ���. 
*/
/*&eg
It is an object used when a server makes an error.
{\it CMObject} ob carries error informations.
The instance ob is a list and in case of a stream connection like TCP/IP 
the first element must be the serial number of the OX message 
that caused the error.
The serial number is given by the data type Integer32.

\noindent
Remark: For a historical reason the CMO tag of the error object is 
named {\tt CMO\_ERROR2}. In the next vesion of OpenXM specification
we wll rename it {\tt CMO\_ERROR}.
*/

/*&C

\medbreak \noindent
*/

//&jp CMObject Null �� \\
//&eg CMObject Null has the format \\
/*&C
\begin{tabular}{|c|c|}
\hline
{\tt int32 CMO\_NULL}  \\ 
\hline
\end{tabular} \\
*/
/*&jp
�ʤ����ɽ������.
*/

/*&C

\noindent
*/

//&jp 32 bit integer n �� CMObject �Ȥ��Ƥ� Integer32 �ȸƤФ�, \\
//&eg 32 bit integer n is called Integer32 as a CMObject and has the format \\
/*&C
\begin{tabular}{|c|c|}
\hline
{\tt int32 CMO\_INT32}& {\tt int32} {\rm n}  \\ 
\hline
\end{tabular} \\
*/
//&jp �ʤ����ɽ������.

/*&C

\medbreak \noindent
*/


//&jp Ĺ�� n �� �Х����� data �� CMObject �Ȥ��Ƥ�, Datum ���Ȥ�Ф� \\
//&eg A byte array of the length n is called Datum as a CMObject and has the format \\
/*&C
\begin{tabular}{|c|c|c|c|}
\hline
{\tt int32 CMO\_DATUM}& {\tt int32} {\rm n} & {\tt byte} {\rm  data[0]} 
& {\tt byte} {\rm  data[1]} \\
\hline
$\cdots$ & {\tt byte} {\rm  data[n-1]} \\ 
\cline{1-2}
\end{tabular} \\
*/
//&jp ��ɽ������.

/*&C

\noindent
*/

//&jp Ĺ�� n �� ʸ���� data ��, CMObject �Ȥ��Ƥ�, Cstring ���Ȥ�Ф� \\
//&eg String of n bytes is called Cstring as CMObject and has the format \\
/*&C
\begin{tabular}{|c|c|c|c|}
\hline
{\tt int32 CMO\_STRING}& {\tt int32} {\rm n} & {\tt byte} {\rm data[0]} 
& {\tt byte} {\rm data[1]}  \\ 
\hline
$\cdots$ & {\tt byte} {\rm data[n-1]} \\ 
\cline{1-2}
\end{tabular} \\
*/
/*&jp
��ɽ������.  C ����������Ѥ�����, ʸ����Τ����� {\tt 0} ��ʸ����
�˴ޤ�ʤ�. 
*/

/*&C	

\noindent
*/
//&jp CMObject Mathcap �� \\
//&eg CMObject Mathcap  has the format \\
/*&C
\begin{tabular}{|c|c|}
\hline
{\tt int32 CMO\_MATHCAP} & {\it CMObject} {\rm ob} \\ 
\hline
\end{tabular} \\
*/
/*&jp
�ʤ����ɽ������.  
{\tt ob} �ϥꥹ�ȤǤ��꾯�ʤ��Ȥ�3�Ĥ����Ǥ���.
0 ���ܤ����Ǥ�, Integer32 ��ɽ���� OpenXM protocol version number ��,
Cstring ��ɽ���������ƥ�̾, Server version, CPU type, ����¾�ξ���
�Υꥹ�ȤǤ���.
1 ���ܤ����Ǥ�, �����ƥ� xxx ���������Ȥβ�ǽ��
SM ������, Integer32 ��ɽ��������Τ򽸤᤿�ꥹ�ȤǤ���.
3 ���ܤ����Ǥ�, �����ƥ� xxx �����Ĥ������Ȥβ�ǽ��
�ǡ��������򤢤Ĥ᤿�ꥹ�ȤǤ���.
�ܺ٤� mathcap �������������.
*/
/*&eg
ob is a list of which length is more than or equal to three.
The first element is a list of
OpenXM protocol version number in Integer32,
the server name in Cstring,
the server version and CPU type in Cstring,
and extra informations.
The second element is a list of SM tags in Integer 32.
The third element is a list of data type tags which the server or the client
can understand.
The details will be explained in the section on mathcap.
*/

/*&C

\medbreak \noindent
*/
//&jp Ĺ�� m �Υꥹ�Ȥ� \\
//&eg A list of the length m has the form \\
/*&C
\begin{tabular}{|c|c|c|c|c|}
\hline
{\tt int32 CMO\_LIST}& {\tt int32} {\rm m} & {\tt CMObject}\, ob[0] & $\cdots$ &
{\tt CMObject}\, ob[$m-1$] \\
\hline
\end{tabular}\\
*/
//&jp ��ɽ������.


%% $OpenXM: OpenXM/doc/OpenXM-specs/formal-expression.tex,v 1.5 2000/09/08 18:57:49 takayama Exp $
//&jp \section{ CMO �η���Ūɽ����ˡ }
//&eg \section{ A formal expression of CMO }

/*&jp
������� CMO ��ɽ����ˡ�����Ū���������, 
CMO ��Primitive ��ɽ��ˡ������������,
�����Ǥ�, CMO �� Lisp ��ɽ�� (Lisp-like expression)
�Ǥ���
CMOexpression
������������������ CMO ��ɸ�� encoding ˡ��⤦������������.
% (�����ξ�ά��ˡ���ۤ���.)
*/
/*&eg
In the previous section, we have explained the format of CMO's in the
Primitive group.
In this section, we will introduce CMOexpression which is like the 
bracket expression of Lisp. 
We again explain a standard encoding method of CMO,
which we have already explained in the previous section.
*/

/*&jp

�ޤ�, CMOexpression �����Ū�� ��ĥ BNF ��ˡ���Ѥ���������褦.
�����ץ饤���ե���ȤǤ����줿����Ͻ�ü������̣����.
``:'' ��������̣����. ``$|$'' ��''�ޤ���''���̣����.
\{ X \} �� X �� 0 ��ʾ�η����֤���ɽ��.
[ x ] �� X �� 0 ��ޤ��� 1 ��и����뤳�Ȥ�ɽ��.
���ε�ˡ���Ѥ���� CMOexpression �ϼ��Τ褦������Ǥ���.

*/
/*&eg

Let us define CMOexpression by the extended BNF expression.
Symbols in the type writer fonts mean terminals.
``:'' means a definition.
``$|$'' means ''or''.
\{ X \} is a repetition of X of more than or equal to 0 times.
[ x ] stands for X or nothing.
By using this notation, CMOexpression is defined as follows.

*/

/*&C
\begin{eqnarray*}
\mbox{CMOexpression}  
&:& \quad
\mbox{\tt (} \mbox{\tt cmo\_tag} \ 
\{ \mbox{ expression} \} \mbox{\tt )}\\
\mbox{expression}
&:& \quad  \mbox{CMOexpression} \\
&   &|\   \mbox{\tt int32}  \\
&   &|\   \mbox{\tt string} \\
&   &|\   \mbox{\tt byte} \\
\end{eqnarray*}
*/
/*&jp
��ü���� {\tt int32} ��, 32 bit integer ��ɽ��, 10 �ʤޤ��� 16 �ʤο�����
��Ǥ���.
��ü���� {\tt string} ��''ʸ��''����Ǥ���.
��ü���� {\tt byte} �� 8 bit �ǡ�����ɽ��, 10 �ʤޤ��� 16 �ʤο�������Ǥ���.
*/
/*&eg
Terminal {\tt int32} is signed 32 bit integer. 
Terminal {\tt string} is a byte array which usually expresses a string.
Terminal {\tt byte} is 8 bit data.
*/

/*&jp
CMOexpression �ˤ����������Ǥ���ڤ뤿��� {\tt ,} (�����) ���Ѥ��Ƥ�褤.
{\tt cmo\_tag} �� {\tt CMO\_} �ǻϤޤ�����Ǥ���.
CMOexpression ��ɽ������� object �� CMObject �ȸƤ�.

*/
/*&eg
The comma ({\tt ,}) may be used to separate each element in CMOexpressions.
{\tt cmo\_tag} is a constant that starts with {\tt CMO\_}.

*/

/*&jp
����ɽ��ˡ CMOexpression �����Ѥ���, CMO Primitive �� object �򵭽�
���Ƥߤ褦.
Object ���Τι�¤���������뤿��,
BNF ��⤦��������ĥ����, ��ü����, ��ü����̾�Τߤʤ餺, �ѿ���
̾����񤯤��Ȥˤ���. �������뤳�Ȥˤ��, object �ΰ�̣���������ưפˤʤ�
����Ǥ���. �ޤ� ``---'' �ǥ����ȤΤϤ��ޤ��ɽ����ΤȤ���. 
*/
/*&eg
Let us describe CMO's in the Primitive group.
In order to explain the meaning of objects,
we may also put variable names to CMOexpressions.
The start of comments are denoted by ``---''.
*/
/*&jp
���Ȥ���, (CMObject ��) 32 bit integer �Ǥ��� integer32 ��
BNF����������,
\begin{center}
Integer32 \  : \ ({\tt CMO\_INT32}, {\tt int32})
\end{center}
�Ƚ񤯤Τ�����ν����ˤ�뵭ˡ�Ǥ��뤬, �����Ǥ�,
\begin{eqnarray*}
\mbox{Integer32} \  &:& \ ({\tt CMO\_INT32}, {\sl int32}\  n) \\
& & \  \mbox{--- 32 bit integer $n$ ��ɽ��. } \\
\end{eqnarray*}
�Ƚ񤯤��Ȥ�������Ȥˤ���.
���Τ褦�˽񤯤��Ȥˤ��, ��ü����  Integer32 ��,
\begin{center}
Integer32 \  : \ ({\tt CMO\_INT32}, {\tt int32})
\end{center}
�Τ褦��, ��ü���� {\tt CMO\_INT32} �� {\tt int32} ����ʬ�ˤ��,
CMObject �� 
({\tt CMO\_INT32}, {\sl int32}\ n)
��, 
32 bit integer $n$ ��ɽ�����Ƥ������Ȥ������Ȥ�, 1 �ԤǤ狼��.
*/
/*&jp
���ε�ˡ���Ѥ���, �����Ƴ������, Primitive �� CMObject ��
����Ū��������褦.
*/
/*&eg
By using this notation, let us define formally CMObjects in the group
Primitive.
*/

/*&C

\bigbreak
\noindent
Group CMObject/Primitive  requires nothing. \\
Error2, Null, Integer32, Datum, Cstring, Mathcap, List $\in$ CMObject/Primitive. \\
Document of CMObject/Primitive is at {\tt http://www.math.kobe-u.ac.jp/OpenXM}
(in English and Japanese) \\
\begin{eqnarray*}
\mbox{Error2}&:& ({\tt CMO\_ERROR2}, {\sl CMObject}\, \mbox{ob}) \\
\mbox{Null}  &:& ({\tt CMO\_NULL}) \\
\mbox{Integer32}
&:& ({\tt CMO\_INT32}, {\sl int32}\ \mbox{n}) \\
\mbox{Datum} &:& ({\tt CMO\_DATUM}, {\sl int32}\, \mbox{n}, {\sl byte}\, 
\mbox{data[0]}, 
\ldots , {\sl byte}\, \mbox{data[n-1]}) \\
\mbox{Cstring}&:& ({\tt CMO\_STRING},{\sl int32}\,  \mbox{ n}, 
{\sl string}\, \mbox{s}) \\
\mbox{Mathcap}&:& ({\tt CMO\_MATHCAP},{\sl CMObject}\,  \mbox{ob} ) \\
\mbox{List} &:& 
\mbox{({\tt CMO\_LIST}, {\sl int32}\, m, {\sl CMObject}\, ob[0], $\ldots$,
{\sl CMObject}\, ob[m-1])} \\
& & \mbox{---  m is the length of the list.} 
\end{eqnarray*}

*/


//&jp Cstring ��, {\sl string} s ����ʬ�� {\tt byte} ��ʬ�򤹤��,
//&eg In the definition of ``Cstring'', if we decompose  ``{\sl string} s'' into bytes, then  ``Cstring'' should be defined as
/*&C
\begin{eqnarray*}
\mbox{Cstring}&:& ({\tt CMO\_STRING},{\sl int32}\,  \mbox{ n}, 
{\sl byte}\, \mbox{s[0]},
\ldots, {\sl byte}\ \mbox{s[n-1]})
\end{eqnarray*}
*/
//&jp �Ȥʤ�.
/*&jp
�ޤ�, 
``Group CMObject/Primitive  requires nothing''
��, �ʲ���, ���롼�� CMObject/Primitive ������Ǥ���,
���Υ��롼�פ� CMObject ���������Τ�, ��������� CMObject �Υ��롼�פ�
�ʤ����Ȥ򼨤�.
``Error2, Null, Integer32, Datum, Cstring, Mathcap, List
$\in$ CMObject/Primitive''
��, ���롼�� CMObject/Primitive �ˤ�,  Error2, Null, Integer32,
Datum, Cstring �ʤ륯�饹�� object ��°���뤳�Ȥ򼨤�.
*/
/*&eg
``Group CMObject/Primitive  requires nothing''
means that there is no super group to define CMO's in the group Primitive.
``Error2, Null, Integer32, Datum, Cstring, Mathcap, List
$\in$ CMObject/Primitive''
means that
Error2, Null, Integer32, Datum, Cstring
are members of the group CMObject/Primitive.
*/

/*&C

*/

/*&jp 
�Ǥ�, �ºݤΥǡ�����ɽ�������ߤƤߤ褦.
���Ȥ���, 32 bit integer �� 1234 ��,
*/
/*&eg
Let us see examples.
32 bit integer 1234 is expressed as
*/
/*&C
\begin{center}
({\tt CMO\_INT32}, 1234)
\end{center}
*/
/*&jp
�Ȥ���.
ʸ���� ``Hello''  ��
*/
/*&eg
The string ``Hello'' is expressed as
*/
/*&C
\begin{center}
({\tt CMO\_STRING}, 5, "Hello")
\end{center}
*/
//&jp �Ƚ�.

/*&C

*/
/*&jp
CMOexpression ��, CMObject �ζ��̤����򤷤Ƥ����ΤϽ��פǤ���.
���Ȥ���
\begin{center}
({\tt CMO\_INT32}, 234, "abc",({\tt CMO\_STRING}))
\end{center}
�� CMOexpression �ǤϤ��뤬, CMObject �ǤϤʤ�.
�����, ʸˡŪ�ˤ��������ץ���������, �ʤˤ���Τ�������������
�ץ������Ȼ��Ƥ���.

*/

/*&jp
����, Open math �� (\cite{openmath}) 
�� XML ɽ��ˡ���ǽ�Ǥ���, ���ξ���, �����Ĥ���ϼ��Τ褦��
��.
*/
/*&eg
It is possible to express CMO by XML like Open Math (\cite{openmath}).
See example below.
*/

/*&C

\begin{verbatim}
<cmo>
 <cmo_int32>
   <int32> 1234 </int32>
 </cmo_int32>

 <cmo_string>
   <int32 for="length"> 5 </int32>
   <string> "Hello" </string>
 </cmo_string>
</cmo>
\end{verbatim}
*/

//&jp \noindent cmo\_string �ϼ��Τ褦�ˤ���路�Ƥ�褤.
//&eg \noindent cmo\_string can also be expressed as follows.
/*&C
\begin{verbatim}

<cmo>
 <cmo_string>
   <int32 for="length"> 5 </int32>
   <byte> 'H' </byte> <byte> 'e' </byte>    <byte> 'l' </byte>
   <byte> 'l' </byte> <byte> 'o' </byte>
 </cmo_string>
</cmo>
\end{verbatim}
*/

//&jp \noindent ���ξ��� cmo\_string �� DTD �ˤ������ϼ��Τ褦�ˤʤ�. \\
//&eg \noindent In this case, the DTD for cmo\_string is as follows; \\
//&C \verb+  <!ELEMENT cmo_string (int32, byte*)>  +
/*&C

\bigbreak
*/

/*&jp
����, ɸ�� encoding ˡ���������褦.
ɸ�� encoding ˡ�Ǥ�, cmo\_tag �� �ͥåȥ���Х��ȥ���������
32 bit integer {\tt int32} ��,
����¾�Υե�����ɤ�, ����˵��Ҥ���Ƥ���ǡ������˽���,
byte �ǡ��� {\tt byte} ���ޤ��� 
�ͥåȥ���Х��ȥ��������� 32 bit integer {\tt int32} ��, �Ѵ�����.
*/
/*&eg
Let us explain the standard encoding method.
All {\tt int32} data are encoded into network byte order 32 bit integers
and byte data are encoded as it is.
*/

/*&C

*/

/*&jp
��®���̿���ˡ���Ѥ���
��Ψ��Ż뤹����³�ξ��ˤ�, {\tt int32} �� network byte order
���Ѵ��������������ʥ����ХإåɤȤʤ뤳�Ȥ�
��𤵤�Ƥ���.
100Mbps ���̿�ϩ�� 12Mbytes �� {\tt CMO\_ZZ} ��ž���Ǥ� 
�� 90\% �λ��֤� network byte order �ؤ��Ѵ��ˤĤ��䤵��Ƥ���Ȥ���
�¸��ǡ����⤢��.
��Ψ��Ż뤷�� encoding ˡ�ˤĤ��Ƥϸ�Ҥ���.
*/
/*&eg
When we are using a high speed network,
the translation from the internal expression of 32 bit integers to
network byte order may become a bottle neck.
There are experimental data which presents that 90 percents of the transmission
time are
for the translation to the network byte order to send {\tt CMO\_ZZ} of size
12M bytes on a 100Mbps network.
In a later section, we will discuss a protocol to avoid the translation.
*/

/*&C

*/

/*&jp
ɸ�� encoding �� CMOexpression �δ֤��Ѵ����ưפǤ���.
������Ѥ����ǡ�����ɽ��ˡ,
���Ȥ���, 
*/
/*&eg
The translation between the standard encoding and CMOexpression
is easy.
For example,
*/
/*&C
\begin{center}
\begin{tabular}{|c|c|}
\hline
{\tt int32 CMO\_INT32}& {\tt int32 1234}  \\ 
\hline
\end{tabular} 
\end{center}
*/
/*&jp
��, CMOexpression 
*/
/*&eg
is the encoding of the CMOexpression
*/
/*&C
\begin{center}
({\tt CMO\_INT32}, 1234)
\end{center}
*/
/*&jp
�� ɸ�� encoding ˡ�ˤ��ɽ���Ǥ���.
*/

/*&C

\bigbreak

*/

/*&jp
(�¸�Ū)
CMO ����� OX packets �� XML ���ʤ˽�򤷤Ƥ���.
XML ���ʤ� Attribute �� binary encode ���뤿���
���̤ʥ��� \\
*/
/*&eg
(Experimental)
CMO and OX packets are complient to XML specification.
In order to encode ``Attribute'' in XML in a binary format,
we have a tag: \\
*/
//&C \verb! #define CMO_ATTRIBUTE_LIST  (LARGEID+3) !  \\
/*&jp
���Ѱդ��Ƥ���.
*/
/*&jp
���Ȥ��� Attribute {\tt font="Times-Roman" }  �� \\
*/
/*&eg
For example, the attibute {\tt font="Times-Roman" } is encoded as \\
*/
/*&C
\begin{verbatim}
 (CMO_ATTRIBUTE (CMO_LIST 
                  (CMO_LIST (CMO_STRING,"font") (CMO_STRING, "Times-Roman"))))
\end{verbatim}
*/
/*&jp
�� encoding �����.
*/
//&C  
/*&jp
�������̤� CMO tag {\tt CMO\_ATTRIBUTE\_LIST} �ʳ���,
XML ɽ���Ǥ� XML �Υ����Ȥ������򤵤��.
*/
/*&eg
All tags except this special CMO tag {\tt CMO\_ATTRIBUTE\_LIST}
are XML tags in the CMO/XML expression.
*/


%% $OpenXM: OpenXM/doc/OpenXM-specs/communication-model.tex,v 1.2 2000/01/20 09:22:01 noro Exp $
//&jp \section{ Open XM ���̿���ǥ�}
//&eg \section{ Communication model of Open XM}  (This part has not yet been translated)

/*&jp
������, ���إץ���������å�������
�򴹤��ʤ���׻����ʹԤ��Ƥ����Ȥ�����ǥ�����ꤷ���߷פ򤹤���Ƥ���.
�ƥץ������ϥ����å��ޥ���Ǥ���, ����� OX �����å��ޥ���Ȥ��.
���إץ������δ֤��̿�ϩ�γ��ݤλ����Ȥ��Ƥϰʲ��Τ褦��
���������ʼ¸���ˡ�����ꤷ�Ƥ���.
\begin{enumerate}
\item �ե������𤷤��̿�����.
\item Library �Ȥ���ľ�ܥ�󥯤����̿�����.
\item TCP/IP �����åȤ�����.
\item Remote Procedure call ������.
\item �ޥ������åɤ�����.
\item PVM �饤�֥�������.
\item MPI �饤�֥�������.
\end{enumerate}

�̿��Ȥϥץ������֤Υ�å������Τ��Ȥ�Ǥ���.
��å�����������Ū�˼��Τ褦�ʹ�¤����: 
*/
/*&eg
In our model of comutation, mathematical processes proceed 
a computation by exchanging messages. Each process is a stack machine,
which is called an OX stack machine.
The following methods are possible to realize communications between
mathematical processes.
\begin{enumerate}
\item Communication by files.
\item Linking as a subroutine library.
\item TCP/IP streams.
\item Remote Procedure call.
\item Muitithread.
\item PVM library.
\item MPI library.
\end{enumerate}

In OpenXM Communication is exechange of messages between processes.
A message has the following structure:
*/
/*&C
\begin{center}
\begin{tabular}{|c|c|c|}
\cline{1-2}
{\tt destination} & {\tt origin} &  \multicolumn{1}{}{}  \\ \hline
{\tt extension}&{\tt ox message\_tag}&{\tt message\_body} \\ 
\hline
\end{tabular}
\end{center}
*/
/*&jp
���Υ�å�������, OX Message (Open XM message object) �Ȥ��.
OX Message �ϥȥåץ�٥�Υ�å����� object �Ǥ���,
���ͽ�Ǥ�, ���ޤ��ޤʥ��롼�פ�°���� object ���о줹��.
���롼��̾��, ���Ȥ���, OX Message/TCPIP/Basic0 �ʤɤȽ�.
{\tt message\_body} ����ʬ�λ��ͤ�, OX Message
�ξ�̤˰��֤�����ʬ�Ǥ���,  SMobject �ޤ��� CMObject ������.
������ object �ϥ�������, ���Υ�������� {\tt SM\_} �ޤ���
{\tt CMO\_} �ǤϤ��ޤ�. 
SMobject ��, �����å��ޥ��󥳥ޥ�ɥ�å����� object �Ǥ���,
��Ϥ�, ���롼��ʬ������Ƥ���.
�ƥ��롼��̾��,
SMobject/Basic0,  SMobject/Basic1 �ʤɤȽ�.
SMobject ����
�����Х����å��ޥ������Ǿܤ�����������.
CMObject �ˤĤ��ƤϤ��Ǥ� Basic0 �� CMObject �������򤷤���,
���Ȥ� CMObject ��٥� 1�������򤹤�.
OX Message ��
{\tt ox message\_tag} ������� {\tt OX\_} �ǻϤޤ�.
*/
/*&eg
We call it an OX message (OpenXM message object).
OX Message is the top level message object.
The OX messages are classified into three types: DATA, COMMAND,
and SPECIAL. They are distinguished by {\tt ox message\_tag}.
The name of an ox message tag begins with  {\tt OX\_}.
Typical OX message tags are {\tt OX\_COMMAND} followed by
SMobject and {\tt OX\_DATA} followed by CMOobject.
Each message object also has its tag. For SMobject, the name
of a tag begins with {\tt SM\_}. For CMOobject, the name of
a tag begins with {\tt CMO\_}.
An SMobject represents a stack machine command and categorized
into several groups such as SMobject/Basic0, SMobject/Basic1.
The details of SMobjects will be explained in Section \ref{sec:stackmachine}.
We have already explained the Basic0 CMOobjects.
We will describe the Basic1 CMOobjects later.
*/
//&jp \subsection{  OX Message �� ɽ����ˡ }
//&eg \subsection{  OX Messages }

/*&jp
Open XM �dzƥץ�������
\begin{center}
(OXexpression �����򤹤륹���å��ޥ���) $+$ (xxx �����ư�����󥸥�)
\end{center}
�ʤ�ϥ��֥�åɹ����Ǥ���.
���Υץ�������, OX �����å��ޥ���ȸƤ�.
�����Ǥ�, OX �����å��ޥ���Ȥ��Ȥꤹ���å������Ǥ���,
OX Message ��ɽ�����뤿��� OXexpression, �����,
�����å��ޥ���� operator ���б�����, SMobject ��ɽ�����뤿��� SMexpression
��������褦.
OX Message �� �����å��ޥ��󥳥ޥ��,
SMobject �� �����å��ޥ��󥪥ڥ졼���Ȥ���.
*/

/*&eg
In Open XM, each process may have a hybrid interface; 
it may accept and execute not only stack machine commands,
but also its original command sequences.
We call such a process an OX stack machine.
Here we introduce OXexpression and SMexpression 
to express OX messages and SM objects respectively.
*/

/*&C
\begin{eqnarray*}
\mbox{OXexpression}  
&:& \quad
\mbox{\tt (} \mbox{\tt OX\_tag} \ 
[\mbox{ expression}]  \mbox{\tt )}\\
\mbox{expression}
&:& \quad  \mbox{SMexpression} \\
&   &|\   \mbox{CMOexpression} \\
\mbox{SMexpression}
&:&  \mbox{\tt (} \mbox{\tt SM\_tag} \ 
\{ \mbox{CMOexpression} \} \mbox{\tt )}\\
\end{eqnarray*}
*/
/*&jp
expression �γ����Ǥ���ڤ뤿��� {\tt ,} (�����) ���Ѥ��Ƥ�褤.
{\tt OX\_tag} �� {\tt OX\_} �ǻϤޤ�����Ǥ���.
{\tt SM\_tag} �� {\tt SM\_} �ǻϤޤ륹���å��ޥ��󥪥ڥ졼�����̤�������Ǥ���.
ȯ���� AAA, ������ BBB ���ɬ�פ�����Ȥ���,
From AAA, To BBB, �� OXexpression �����˽�.
ɬ�פʤ���о�ά����.

���Ȥ���, ���Ȥ���������, CMO string ``Hello'' �� �����å��˥ץå��夹��
ɽ���ϼ��Τ褦�˽�:
*/

/*&eg
A comman `{\tt ,}' may be used to separate elements in an expression.
{\tt OX\_tag} is a constant which denotes an OX message tag.
{\tt SM\_tag} is a constant which denotes an SM command tag.
If a sender AAA or a receiver BBB has to be specified,
'From AAA' or 'To BBB' is written before the OXexpression.

For example the following expression means a request to
push a CMO string ``Hello''.
*/
/*&C
\begin{center}
(OX\_DATA, (CMO\_STRING, 5, "Hello"))
\end{center}
*/


/*&jp
The following expression means a request to execute
a local function ``hoge''.
*/

/*&C
\begin{center}
(OX\_DATA, (CMO\_STRING, 5, "hoge")) 
\end{center}
\begin{center}
(OX\_COMMAND, SM\_executeStringByLocalParser)
\end{center}
*/

/*&jp
In our standard encoding method, each tag is expressed as
a 32 bit (4 byte) integer with the network byte order.
*/

//&jp \subsection{OXexpression �� ɸ�� encoding �� TCP/IP �����åȤˤ�����ˡ}
//&eg \subsection{Standard enconding of OXexpressions and an implementation by TCP/IP sockets}
/*&jp
�̿��μ¸���ˡ���̿�ϩ�ΤȤ꤫���ˤ�꤫��뤬,
������¤������Ū�ˤ��Ĥ���ʤ��Ȥ����ʤ�.
OXexpression �Ϥ���������¤�򵭽Ҥ��Ƥ���.

�����Ǥ� OXexpression ��ɸ�� encoding �γ�ά����������.
���� encoding ˡ��TCP/IP �����å��Ѥ� encoding ˡ�Ȥ���
����¸�ߤ��Ƥ��륵���Ф˻��Ѥ���Ƥ���.
�����OX �����å��ޥ���η׻����֤����椹�뤿���, ����ȥ������å�������
�Ĥ��Ƥ���������.


{\tt destination}, {\tt origin} ����ʬ��, �����åȤˤ��
peer to peer ����³�ʤΤǾ�ά����.
{\tt extension} �ե�����ɤ�
{\tt message\_tag} �ե�����ɤμ��ˤ���.
{\tt extension} �ե�����ɤ� OX �ѥ��åȤΥ��ꥢ���ֹ椬�Ϥ���.
���ꥢ���ֹ�� {\tt int32} �Ǥ���.
�����ֹ��, �����Ф����顼�򵯤��������, ���顼�򤪤�����,
OX �ѥ��åȤ��ֹ���᤹�Τ˼�����Ѥ����.
�ʲ� {\tt extension} �ե�����ɤ�, {\tt message\_tag} ��
�˴ޤޤ������� {\tt extension} �ե�����ɤϾ�ά����.
�������äƥѥ��åȤ�
���Τ褦�˵��Ҥ���
*/

/*&eg
The logical structure of OX messages are independent of implementations
of communication. The OXexpression represents the logical structure.
Here we explain an outline of the standard encoding scheme of OXexpression.
This encoding scheme is used to implement OpenXM protocols on TCP/IP sockets.
In addition, we also explain the control messages to control stack machines.

As the socket connection is peer to peer, {\tt destination} and {\tt origin}
are omitted.
The {\tt extension} field is placed after the {\tt message\_tag} field.
The {\tt extension} field consists of the serial number for OX message,
which is {\tt int32}.
The serial number is used to identify an OX message which caused
an error on a server.
In the following we regard the {\tt extension} as a component of
the {\tt message\_tag} field and omit the {\tt extension} field.
Thus OX messages are represented as follows.
*/
/*&C
\begin{center}
\begin{tabular}{|c|c|}
\hline
{\tt ox message\_tag}&{\tt message\_body} \\ 
\hline
\end{tabular}
\end{center}
*/
//&jp ��, ��äȤ��ޤ��������,
//&eg More precisely it has the following representation.
/*&C
\begin{center}
\begin{tabular}{|c|c|}
\hline
{\tt ox message\_tag}, \ {\tt packet number}&{\tt message\_body} \\ 
\hline
\end{tabular}
\end{center}
*/
/*&jp
�ȤʤäƤ���.

���롼�� OX Message/TCPIP/Basic0 ��
{\tt ox message\_tag} �Ȥ��Ƥϼ��Τ�Τ��Ѱդ���Ƥ���.
*/
//&eg As {\tt ox message\_tag} the following are provided.

/*&C
@plugin/oxMessageTag.h
\begin{verbatim}
#define   OX_COMMAND         513
#define   OX_DATA            514

#define   OX_DATA_WITH_LENGTH  521
#define   OX_DATA_OPENMATH_XML 523
#define   OX_DATA_OPENMATH_BINARY 524
#define   OX_DATA_MP           525

#define   OX_SYNC_BALL       515
\end{verbatim}
*/

/*&jp
�̿�ϩ�� 2 ���Ѱդ���.
1���ܤ��̿�ϩ��
\verb+  OX_COMMAND +
�����
\verb+   OX_DATA +
���ʤ����.
2���ܤ��̿�ϩ ({\tt control}�ȸƤ�) �ˤ�,
\verb+  OX_COMMAND + ����Ӥ����³������ȥ����륳�ޥ��
\verb+  SM_control_* +
�ޤ��ϥ���ȥ�����ط��Υǡ���, �Ĥޤ� header
\verb+   OX_DATA + �ǤϤ��ޤꤽ���³�� CMO �ǡ���
���ʤ����.
�����򥳥�ȥ������å���������ӥ���ȥ������å������η��
��å������ȸƤ�.
����ץ륵���ФǤ�, ���� 2 �Ĥ��̿�ϩ��, 2 �ĤΥݡ��Ȥ��Ѥ���
�¸����Ƥ���.


\verb+ OX_COMMAND + ��å������ϼ��η��Υѥ��åȤ�ɽ�������: \\
*/
/*&eg
Two streams are used for communication between a client and a server.
One is the stream to exchange data and to send stack machine commands.
The other is the stream to control stack machines.
Messages on the latter stream are called control messages and
results of control messages. The sample server implements
the above two streams by using two ports on TCP/IP.

The stack machine command message has the following forms: \\
*/
/*&C
\begin{tabular}{|c|c|}
\hline
{\tt OX\_COMMAND} & {\tt int32 function\_id} \\  \hline
{\it message\_tag} & {\it message\_body}
\\ \hline
\end{tabular}, \quad
({\tt OX\_COMMAND}, ({\tt SM\_*}))
\\
*/

//&jp \verb+ OX_DATA + ��å������ϼ��η��Υѥ��åȤ�ɽ�������: \\
//&eg CMO data message has the following form:\\
/*&C
\begin{tabular}{|c|c|}
\hline
{\tt OX\_DATA} &  {\tt CMO data} \\  \hline
{\it message\_tag} & {\it message\_body}\\ \hline
\end{tabular}, \quad
({\tt OX\_DATA}, {\sl CMObject} data)
\\
*/
//&jp ����ȥ������å������ϼ��η��Υѥ��åȤ�ɽ�������: \\
//&eg The control message has the following form:\\
/*&C
\begin{tabular}{|c|c|}
\hline
{\tt OX\_COMMAND} & {\tt int32 function\_id}  \\  \hline
\end{tabular},  \quad
({\tt OX\_COMMAND},({\tt SM\_control\_*}))
\\
*/
/*&jp
����ȥ������å�������, �׻������Ǥ�����, debug �Ѥ� ����åɤ�ư����,
debug �⡼�ɤ�ȴ������, �ʤɤ����Ӥ����Ѥ���.
*/
/*&eg
The control message is used to interrupt a computation, to invoke
debugging threads, or to exit form the debugging mode.
*/

//&jp ����ȥ������å������η�̥�å������ϼ��η��Υѥ��åȤ�ɽ�������: \\
//&eg The result of a cotrol message has the following form:\\
/*&C
\begin{tabular}{|c|c|l|}
\hline
{\tt OX\_DATA} & {\tt CMO\_INT32} & {\tt int32 data} \\  \hline
\end{tabular}, \quad
({\tt OX\_DATA}, {\sl Integer32 } n)
\\
*/


/*&jp
{\tt int32 function\_id}
����ʬ��, �����Х����å��ޥ��� �� operator ���б������ֹ椬�Ϥ���.
���롼�� SMobject/Basic0 ����� SMobject/Basic1 ��°����
�����Ȥ��ưʲ��Τ�Τ�����. 
*/
/*&eg
{\tt int32 function\_id} is the value of a stack machine command.
SM tags in SMobject/Basic0 and SMobject/Basic1 and corresponding
values are as follows.
*/
/*&C
@plugin/oxFunctionId.h
\begin{verbatim}
#define SM_popSerializedLocalObject 258
#define SM_popCMO 262
#define SM_popString 263 

#define SM_mathcap 264
#define SM_pops 265
#define SM_setName 266
#define SM_evalName 267 
#define SM_executeStringByLocalParser 268 
#define SM_executeFunction 269
#define SM_beginBlock  270
#define SM_endBlock    271
#define SM_shutdown    272
#define SM_setMathCap  273
#define SM_executeStringByLocalParserInBatchMode 274
#define SM_getsp       275
#define SM_dupErrors   276


#define SM_control_kill 1024
#define SM_control_reset_connection  1030
\end{verbatim}
*/

//&jp ���Ȥ���,
//&eg For example
/*C
\begin{center}
(OX\_COMMAND, SM\_pops)
\end{center}
*/
//&jp ��
//&eg is encoded as follows.
/*&C
\begin{center}
\begin{tabular}{|c|c|}
\hline
{\tt int32} 513  &  {\tt int32} 265 \\
\hline
\end{tabular}
\end{center}
*/
//&jp �ȥ��󥳡��ɤ����.

/*&jp
operator �ξܺ٤ϼ��������������.
�����������̾���ϥ���ץ���ȤΤȤ�û�̷���ɽ�����Ƥ�褤.
*/
/*&eg
The details of the operators are described in Section \ref{sec:stackmanine}.
Names of these constants may be represented by abbrebiated forms.
*/


//&jp \section{ OX �����å��ޥ��� }
//&eg \section{ OX stackmachine }  (This section has not yet been translated.)

/*&jp
������Ǥ�, OX �����å��ޥ��� operator ������
(TCP/IP �����åȾ�Ǥ�ɸ�� encoding ˡ ���Ѥ���),
�����, ����ץ륵���Фȥ�󥯤�����ޤ���
open XM �饤�֥��Ȥ��ƥ�󥯤��ƻ��Ѥ������
����� C �δؿ��λ��ͤ���������.

����������, OX �����Х����å��ޥ����ư��θ�§��
�������Ƥ���.
�����Х����å��ޥ����,
{\tt SM\_pop*} �ϤΥ����å��ޥ��󥳥ޥ�ɤ����ʤ�������,
��ȯŪ�˥�å��������������뤳�ȤϤʤ�.
���θ�§�˴�Ť���ʬ���׻��Υץ�����ߥ󥰤򤪤��ʤ�.
���٥�ȥɥ�֥�ʥץ������ˡ�ȤϤ��������Ȥ�
���դ��褦.
*/

/*&eg 
In this section we describe the OX stack machine operators.  In
the descriptions OX messages are represented by th standard encoding
scheme on TCP/IP sockets.  In principle, an OX stack machine never
sends data to the output stream unless it receives {\tt SM\_pop*}
commands.  Note that the programming style should be different from
that for event-driven programming.
*/

//&jp \subsection{�����Х����å��ޥ��� }
//&eg \subsection{Server stack machine}

/*&jp
����ץ륵���ФǤ��� {\tt oxserver00.c}
�ϰʲ��λ��ͤ� C �δؿ����Ѱդ���, 
{\tt nullstackmachine.c } ���֤�������а��ư���Ϥ��Ǥ���.
*/
/*&eg
{\tt oxserver00.c} is implemented as a sample server.
If you want to implement you own server,
write the following functions and use them instead of
those in {\tt nullstackmachine.c }.
*/

//&jp \subsubsection{�����Х����å��ޥ���Υ��롼�� SMobject/Basic0 ��°���륪�ڥ졼��}
//&eg \subsubsection{Operators in the group SMobject/Basic0}

/*&jp
\noindent
�����Х����å��ޥ���Ϻ����1�ܤΥ����å�
\begin{verbatim}
Object xxx_OperandStack[SIZE];
\end{verbatim}
����.  ������, {\tt Object} �Ϥ��Υ����ƥ��ͭ�� Object ���ǹ���ʤ�.
CMObject �ϳƥ����и�ͭ�Υ������륪�֥������Ȥ��Ѵ����ƥ����å��إץ�
���夷�Ƥ褤.  �������Ѵ�, ���Ѵ������������ΤϹ��������Ǥ��뤳�Ȥ�
�Τ��ޤ���.  CMObject ��ɤΤ褦�� (local) Object ���Ѵ����뤫, Object 
�������դ����å������������,�ƥ����ƥब�ȼ��ˤ����ʸ�񲽤��Ƥ���
��ΤȤ���.  �Ĥޤꤹ�٤ƤΥ�å�������, private �Ǥ���.  ���Ȥ���,
{\tt add } �Τ褦�ʴ���Ū�� ��å������ˤ������Ƥ�, OX �����å��ޥ���
�Ϥʤˤ⤭��Ƥ��ʤ�.  ����Ū�ˤ� open math \cite{openmath} �Τ褦��
CMObject ���Ф����������Ū�ʥ�å������λ��ͤ�content dictionary
(CD) ������������.

�ʲ�, \verb+ xxx_ + �ϸ���ζ��줬�ʤ��Ȥ��Ͼ�ά����.
\verb+ xxx_ + �� local �����Х����ƥ�˸�ͭ�μ��̻ҤǤ���.
{\tt Asir} �ξ��� \verb+ Asir_ + ���Ѥ���.  {\tt kan/sm1} �ξ��� 
\verb+ Sm1_ + ���Ѥ���.  �ؿ�̾, ����̾��Ĺ���ΤǾ�ά�����Ѥ��Ƥ�褤.

�ʲ��Ǥϼ��Τ褦�˥ѥ��åȤ򵭽Ҥ���.  �ƥե�����ɤ�,
\fbox{�ǡ����� \quad  �ǡ���} �ʤ����
�ǽ�.  ���Ȥ���, {\tt int32 OX\_DATA} �� 32 bit network byte order 
�ο��� {\tt OX\_DATA}�Ȥ�����̣�Ǥ���.  ``������å��ǽ񤫤�Ƥ���ե���
��ɤ�,������̤ΤȤ����Ǥʤ���Ƥ��뤫���˸���Τʤ��褦�ʼ�������
����������Ƥ��� object ��ɽ��.''  ���Ȥ���, {\it String commandName} 
��, String �ǡ������� local object {\it commandName} ���̣����.  (����
�Х����å��ޥ����� object ��, CMO ������ object�Ȥϸ¤�ʤ����Ȥ���
��.  CMO �����ǽ񤤤Ƥ��äƤ�, ����ϥ����Х����å��ޥ����local ����
�ǥ����å���ˤ���Ȳ�ᤷ�Ʋ�����.)

���٤ƤΥ����Х����å��ޥ���ϰʲ��δؿ���������Ƥ��ʤ��Ȥ����ʤ�.

*/

/*&eg
\noindent
Any OX stack machine has at least one stack.
\begin{verbatim}
Object xxx_OperandStack[SIZE];
\end{verbatim}
Here {\tt Object} may be local to the system {\tt xxx} wrapped by the stack
machine. 
That is, the server may translate CMObjects into local its
objects and push them onto the stack.  However, it is preferable that
the composition of such a translation and its inverse is equal to the
identity map. The translation scheme is called the phrase book of the
server and it should be documented for each stack machine.  In OpenXM,
any message is private.  In future we will provide a content
dictionary (CD; see OpenMath \cite{openmath}) for basic specifications
of CMObjects.

In the following, \verb+ xxx_ + may be omitted if no confusion occurs.
As the names of functions and tags are long, one may use abbreviated
names.  Message packets are represented as follows.  Each field is
shown as \fbox{data type \quad data}.  For example {\tt int32
OX\_DATA} denotes a number {\tt OX\_DATA} which is represented by a 32
bit network byte order.  If a field is displayed by italic characters,
it should be defined elsewhere or its meaning should be clear.  For
example {\it String commandName} denotes a local object {\it
commandName} whose data type is String.  Note that an object on the
stack may have a local data type even if it is represented as CMO.

Any server stack machine has to implement the following operations.
*/

\begin{enumerate}
\item
/*&jp
CMObject/Basic0 �� CMO �ǡ����Τ���ɬ�ܤΤ��, {\tt CMO\_ERROR2}, {\tt
CMO\_NULL}, {\tt CMO\_INT32}, {\tt CMO\_STRING}, {\tt CMO\_LIST}������
�����褿��礽��򥹥��å��� push ����.  ���Ȥ���, {\tt CMO\_NULL} 
���뤤�� {\tt CMO\_String} �ξ�缡�Τ褦�ˤʤ�.
*/
/*&eg
Any server should accept CMObjects in the group CMObject/Basic0.
The server pushes such data onto the stack.
The following examples show the states of the stack after receiving
{\tt CMO\_NULL} or {\tt CMO\_String} respectively.
*/

Request:
\begin{tabular}{|c|c|}  \hline
{\tt int32 OX\_DATA} & {\tt int32 CMO\_NULL} \\
\hline
\end{tabular}

Stack after the request:
\begin{tabular}{|c|}  \hline
{\it NULL} \\
\hline
\end{tabular}

Result:  none.

Request:
\begin{tabular}{|c|c|c|c|c|c|}  \hline
{\tt int32 OX\_DATA} & {\tt int32 CMO\_String} &{\tt int32} {\rm size} 
&{\tt byte} {\rm s1} & $\cdots$ &{\tt byte} {\rm ssize}\\
\hline
\end{tabular}

Stack after the request:
\begin{tabular}{|c|}  \hline
{\it String s} \\
\hline
\end{tabular}

Result:  none.

//&jp CMO �ǡ����μ������˼��Ԥ������Τ�  \\
//&eg If the server fails to receive a CMO data,
\begin{tabular}{|c|c|c|}  \hline
{\tt int32 OX\_DATA} & {\tt int32 CMO\_ERROR2} & {\it CMObject} ob\\
\hline 
\end{tabular}
\\
/*&jp
�򥹥��å��� push ����.
���ߤΤȤ���, ob �ˤ�, \\
\centerline{
[{\sl Integer32} OX �ѥ��å��ֹ�, {\sl Integer32} ���顼�ֹ�, 
{\sl CMObject} optional ����]
}
�ʤ�ꥹ�Ȥ������ (CMO �����Ǥ����Ƥ��뤬, ����ϥ������ȼ��η����Ǥ褤.
CMO �Ȥ������Ф����Ȥ����Τ褦�ʷ����Ǥʤ��Ȥ����ʤ��Ȥ�����̣�Ǥ���.)
*/
/*&eg
is pushed onto the stack.
Currently ob is a list\\
\centerline{
[{\sl Integer32} OX serial number, {\sl Integer32} error code, 
{\sl CMObject} optional information]
}
*/

\item
\begin{verbatim}
SM_mathcap
\end{verbatim}
/*&jp
���Υ����Ф� mathcap ���ɤ� (termcap �Τޤ�).  �����ФΥ�����, ����
�Х����å��ޥ����ǽ�Ϥ��Τ뤳�Ȥ��Ǥ���.  C ����Ǽ����������,
mathCap �ι�¤�Τ򥷥��ƥ���ˤ�����ΤȤ�,���δؿ��Ϥ��ι�¤�Τؤ�
�ݥ��󥿤��᤹.  (open sm1 �Ǥ� {\tt struct mathCap} ���Ѥ��Ƥ���.
*/
/*&eg
It requests a server to push the mathcap of the server. 
The mathcap is similar to the termcap. One can know the server type
and the capability of the server from the mathcap. 
*/
@plugin/mathcap.h)

Request:
\begin{tabular}{|c|c|}  \hline
{\tt int32 OX\_COMMAND} & {\tt int32 SM\_mathcap}  \\
\hline
\end{tabular}

Stack after the request: 
\begin{tabular}{|c|c|}  \hline
{\tt int32 OX\_DATA} & {\sl Mathcap}  mathCapOb \\
\hline
\end{tabular}

Result: none.

\item
\begin{verbatim}
SM_setMathCap
\end{verbatim}
/*&jp
������ä� Mathcap {\tt m} ��ʬ�Υ����ƥ�����ꤷ��, ���¦��������
ǽ�� CMO �򤪤���ʤ��褦�ˤ���.  C ����Ǽ����������, mathCap �ι�
¤�Τ򥷥��ƥ���ˤ�����ΤȤ�,���δؿ��Ϥ��ι�¤�ΤؤΥݥ��󥿤��
���Ȥ���.  (open sm1 �Ǥ� {\tt struct mathCap} ���Ѥ��Ƥ���.
*/
/*&eg
It requests a server to register the peer's mathcap {\tt m} in the server.
The server can avoid to send OX messages unknown to its peer.
*/
@plugin/mathcap.h)

Request:
\begin{tabular}{|c|c|}  \hline
{\tt int32 OX\_DATA} & {\sl Mathcap} m  \\ \hline
{\tt int32 OX\_COMMAND} & {\tt int32 SM\_setMathCap}  \\
\hline
\end{tabular}

Result:  none.
/*&jp
\noindent
����: mathcap �ϰ��̤˥��饤����ȼ��Τ����ꤹ��.
���饤����Ȥ������Ф� {\tt SM\_mathcap} �򤪤���,
������¦�� mathcap ������.
�����, ���饤����ȤϤ��Υ����Ф��տ路�� mathcap �Ȥ���
���ꤹ��.
����, ���饤����Ȥϥ����Ф˼�ʬ�� mathcap ��
{\tt SM\_setMathCap} �Ǥ�����, ��ʬ�� mathcap �����ꤵ����.
*/
/*&eg
\noindent
Remark: In general the exchange of mathcaps is triggered by a client.
A client sends {\tt SM\_mathcap} to a server and obtains the server's
mathcap. Then the client registers the mathcap. Finally the client
sends its own mathcap by {\tt SM\_setMathCap} and the server
registers it.
*/

\item
\begin{verbatim}
SM_executeStringByLocalParser
\end{verbatim}
/*&jp
ʸ���� {\tt s} �� stack ���� pop ��, 
����ʸ����򥷥��ƥ��ͭ��ʸˡ(�����Х����å��ޥ�����Ȥ߹��ߥ�����
�����)�ˤ������ä����ޥ�ɤȤ��Ƽ¹Ԥ���.  ���ޥ�ɤμ¹Ԥη�
�̤κǸ������ͤ�����Ȥ���, {\tt OperandStack} ������ͤ� push ����.
OpenXM �Ǥ�, ���ߤΤȤ����ؿ�̾��ɸ�ಽ�Ϥ����ʤäƤ��ʤ�.  
���δؿ������ {\tt popString} �ε�ǽ��¸������, ����¤� open XM ��
�����Фˤʤ��.  �����Ǥ�, �ޤ�������Ĥδؿ��ε�ǽ��¸����٤��Ǥ���.
*/
/*&eg
It requests a server to pop a character string {\tt s}, to
parse it by the local parser of the stack machine,  and
to interprete by the local interpreter.
If the exececution produces a result, it is pushed onto 
{\tt OperandStack}.
If an error has occured,  Error2 Object is pushed onto the stack.
OpenXM does not provide standard function names.
If this operation and {\tt SM\_popString} is implemented, the stack machine
can be used as an OX server.
*/

Stack before the request: 
\\
\begin{tabular}{|c|}  \hline
{\it String commandString} \\
\hline 
\end{tabular} 

Request: 
\begin{tabular}{|c|c|}  \hline
{\tt int32 OX\_COMMAND}& {\tt int32 SM\_executeStringByLocalParser} \\
\hline 
\end{tabular}

Result:  none.
/*&jp
\noindent
����: \  �¹����Υ����å��Υǡ�����,
{\it String commandString} �ʤ� local stackmachine �� object �Ȥ��ƥ�
���å���ˤ��뤬, TCP/IP ���̿�ϩ�Ǥ�, ���Τ褦�ʥǡ������ޤ��ʤ����
{\it commandName} ��������� push �����:
*/
/*&eg
\noindent
Remark: Before this request, one has to push {\it String commandString}
onto the stack. It is done by sending the following OX data message.
*/
\begin{tabular}{|c|c|c|}  \hline
{\tt int32 OX\_DATA} & {\tt int32 CMO\_string} & {\it size and the string commandString} \\
\hline 
\end{tabular}

\item
\begin{verbatim}
SM_executeStringByLocalParserInBatchMode
\end{verbatim}
/*&jp
�����å��������Ѥ��ʤ�(�����å��ˤ������Ƥʤ�����⤷�ʤ�)���Ȥ����
��Ȥޤä���Ʊ���ؿ��Ǥ���.  ���顼�λ��Τ�, Error2 Object �򥹥��å�
�إץå��夹��.
*/
/*&eg
This is the same request as {\tt SM\_executeStringByLocalParser}
except that it does not modify the stack. It pushes an Error2 Object
if an error has occured.
*/
\item
\begin{verbatim}
SM_popString
\end{verbatim}
/*&jp
{\tt OperandStack} ��� Object �� pop ��, ����� xxx �ν��ϵ�§�ˤ�������ʸ
���󷿤��Ѵ�������������.  �����å������ΤȤ���, {\tt (char *)NULL} ���᤹.
ʸ����� TCP/IP stream �� CMO �Υǡ�
���Ȥ�����������.  ���顼�ξ��� {\tt CMO\_ERROR2} ���᤹�٤��Ǥ���.
*/
/*&eg
It requests a server to pop an object from {\tt OperandStack},
to convert it into a character string according to the output format
of the local system, and to send the character string via TCP/IP stream.
{\tt (char *)NULL} is returned when the stack is empty.
The returned strings is sent as a CMO string data.
{\tt CMO\_ERROR2} should be returned if an error has occured.
*/

Stack before the request:
\begin{tabular}{|c|}  \hline
{\it Object} \\
\hline 
\end{tabular}

Request:
\begin{tabular}{|c|c|}  \hline
{\tt int32 OX\_COMMAND} & {\tt int32 SM\_popString} \\
\hline 
\end{tabular}

Result: 
\begin{tabular}{|c|c|c|}  \hline
{\tt int32 OX\_DATA} & {\tt int32 CMO\_STRING} & {\it size and the string s} \\
\hline 
\end{tabular}

\item
\begin{verbatim}
SM_getsp
\end{verbatim}
/*&jp
���ߤΥ����å��ݥ��󥿤ΰ��֤��ɤ�.  �������Ȼ����Ǥΰ��֤� 0 �Ǥ���,
object �� push ���줿�Ф���, 1 �Ť��������ΤȤ���.
*/
/*&eg
It requests a server to push the current stack pointer onto the stack.
The stack pointer is represented by a non-negative integer.
Its initial value is 0 and a push operation increments the
stack pointer by 1.
*/

Stack before the request:
\begin{tabular}{|c|}  \hline
{\it Object} \\
\hline 
\end{tabular}

Request:
\begin{tabular}{|c|c|}  \hline
{\tt int32 OX\_COMMAND} & {\tt int32 SM\_getsp} \\
\hline 
\end{tabular}

Stack after the request:
\begin{tabular}{|c|c|c|}  \hline
{\tt int32 OX\_DATA} & {\tt int32 CMO\_INT32} & {\it stack pointer value} \\
\hline 
\end{tabular}

Result:  none.

\item
\begin{verbatim}
SM_dupErrors
\end{verbatim}
/*&jp
�����å���Υ��顼���֥������Ȥ�ꥹ�Ȥˤ����᤹.  �����å����Τ��Ѳ�
�����ʤ�.
*/
/*&eg
It requests a server to push a list object containing all error objects on the stack.
*/

Stack before the request:
\begin{tabular}{|c|}  \hline
{\it Object} \\
\hline 
\end{tabular}

Request:
\begin{tabular}{|c|c|}  \hline
{\tt int32 OX\_COMMAND} & {\tt int32 SM\_dupErrors} \\
\hline 
\end{tabular}

Stack after the request:
\begin{tabular}{|c|c|c|}  \hline
{\tt int32 OX\_DATA} & {\sl CMObject} \ a list of errors\\
\hline 
\end{tabular}

Result: none.
\end{enumerate}

\medbreak
\noindent
/*&jp
{\bf ��}: \ 
mathcap ���䤤��碌���Ф���, {\tt ox\_sm1} �ϼ��Τ褦��������.
*/
/*&eg
{\bf Example}: \ 
{\tt ox\_sm1} returns the following data as its mathcap.
*/
%%Prog: [(cmoMathCap)] extension ::
\begin{verbatim}
Class.mathcap 
 [ [199909080 , $Ox_system=ox_sm1.plain$ , $Version=2.990911$ , 
    $HOSTTYPE=i386$ ]  , 
   [262 , 263 , 264 , 265 , 266 , 268 , 269 , 272 , 273 , 275 , 276 ]  , 
   [[514] , [2130706434 , 1 , 2 , 4 , 5 , 17 , 19 , 20 , 22 , 23 , 24 , 
             25 , 26 , 30 , 31 , 60 , 61 , 27 , 33 , 40 , 34 ]]]
\end{verbatim}

/*&jp
mathcap �� 3�Ĥ����Ǥ��ĥꥹ�ȤǤ���.  �ޤ�, �ǽ�����Ǥ򸫤褦.
Ox\_system �� open xxx �����ƥ�̾�Ǥ���.  �ɤ߹���饤�֥�꤬�����ä�
����, �ؿ�̾(�ޤ��� ����ܥ�)�ΰ�̣���������Ȥ��Ϥ���̾���⤫����. ��
�Ȥ���, open math �� basic content dictionary �б��δؿ�����ޥ�������
�ߤ���� sm1 ��, ox\_sm1\_basicCD �ʤ�̾���ˤ���.  HOSTTYPE�ͤ�, CPU 
�μ���򤢤�路unix �ǤϴĶ��ѿ�\verb+$HOSTTYPE+ ���ͤǤ���.  2 ����
�����Ǥ� ���Ѳ�ǽ�� SM ���ޥ�ɤ򤢤Ĥ᤿�ꥹ�ȤǤ���.  3 ���ܤΥꥹ��
��, ������ǽ�ʿ��إǡ����η���, �����CMO�ξ��ʤ������ǽ��CMO�Υ���
�Υꥹ�Ȥ�³��.  �����Ǥ�, 514 �� {\tt OX\_DATA} �򤢤�路, ���إǡ�
���Υե��ޥåȤ�(����������ʤ���) CMO �Ǥ��뤳�Ȥ򼨤�.
*/
/*&eg
A mathcap has three components. The first one contains informations
to identify the system and hosts on which the application runs.
In the above example, Ox\_system denotes the system name.
HOSTTYPE represents the OS type and taken from \verb+$HOSTTYPE+
enviroment variable.
The second component consists of avaiable SM commands.
The third component is a list of pairs. Each pair consists
of an OX message tag and the available message tags.
Again in the above example, 514 is the value of {\tt OX\_DATA}
and it indicates that the server accepts CMO (without size information)
as mathematical data messages. In this case the subsequent
list represents available CMO tags.
*/

\medbreak
\noindent
//&jp {\bf ��}: \ 
//&eg {\bf Example}: \ 
%%Prog: (ox.sm1) run  sm1connectr  [(oxWatch) ox.ccc] extension
%%Prog: ox.ccc (122345; ) oxsubmit ;
//&jp {\tt message\_body} �μ���򤢤���.   ���ꥢ���ֹ����Ͻ����Ƥ���.
//&eg We show examples of {\tt message\_body}. Serial numbers are omitted.
\begin{enumerate}
\item  {\tt executeStringByLocalParser("12345 ;");} 
/*&jp
�ϼ��Τ褦�ʥѥ��åȤ��Ѵ������. �ƿ����� 16��1�Х��Ȥ򤢤�魯.
{\tt xx(yy)} �Τʤ��� {\tt (yy)} ���б����륢�����������ɤ򤢤�餹.
*/
/*&eg
is converted into the following packet. Each number denotes
one byte in hexadecimal representation.
{\tt (yy)} in {\tt xx(yy)} represents the corresponding ASCII code.
*/
\begin{verbatim}
0   0   2   2   0   0   0   4   0   0   0   7  
31(1)  32(2)  33(3)  34(4)  35(5)  20  3b(;)   
0   0   2   1   0   0   1   c 
\end{verbatim}
/*&jp
���줾��Υǡ����ΰ�̣�ϼ��ΤȤ���Ǥ���.
*/
/*&eg
Each data has the following meaning.
*/

\begin{verbatim}
0   0   2   2  (OX_DATA) 0   0   0   4  (CMO_STRING)
0   0   0   7  (size)
31(1)  32(2)  33(3)  34(4)  35(5)  20  3b(;)   (data)
0   0   2   1  (OX_COMMAND) 
0   0   1   c  (SM_executeStringByLocalParser)
\end{verbatim}
//&jp ����� OXexpression ��ɽ������ȼ��Τ褦�ˤʤ�.
//&eg This is expressed by OXexpression as follows.
\begin{center}
(OX\_DATA, (CMO\_STRING, 7, "12345 ;"))
\end{center}
\begin{center}
(OX\_COMMAND, (SM\_executeStringByLocalParser))
\end{center}

//&jp \item  {\tt popString()}  ���������������:
//&eg \item  A message which requests {\tt SM\_popString}:
\begin{verbatim}
0   0   2   1  (OX_COMMAND) 
0   0   1   7  (SM_popString)
\end{verbatim}
//&jp OXexpression �Ǥ�
//&eg In OXexpression it is represented as 
(OX\_COMMAND, (SM\_popString)).

\noindent
//&jp ����ˤ������Ƽ���������å�����������.
//&eg The server returns the following reply message:
\begin{verbatim}
0   0   2   2   (OX_DATA) 
0   0   0   4   (CMO_STRING) 0   0   0   5  (size)
31(1)  32(2)  33(3)  34(4)  35(5) 
\end{verbatim}
//&jp OXexpression �Ǥ�����,
//&eg In OXexpression it is represented as
(OX\_DATA, (CMO\_STRING, 7, "12345 ;")).
\end{enumerate}

//&jp \subsubsection{���롼�� SMobject/Basic1 ��°���륪�ڥ졼��}
//&eg \subsubsection{Operators in the group SMobject/Basic1}

\begin{enumerate}
\item
\begin{verbatim}
SM_pops
\end{verbatim}
/*&jp
operand stack ���, {\it n} �Ĥθ� ({\it obj1, obj2, $\ldots$, objn}) 
�� pop ���ƼΤƤ�.
*/
/*&eg
It requests a server to pop {\it n} and to discard elements {\it obj1, obj2,
$\ldots$, objn}) from the stack.
*/

//&jp Stack before the request: (���� stack �ΥȥåפǤ���.) \\
//&eg Stack before the request: (The rightmost one is the top of the stack.) \\
\begin{tabular}{|c|c|c|c|c|}  \hline
{\it obj1} & {\it  obj2}  & $\cdots$ & {\it objn}  &{\it INT32 n} \\
\hline
\end{tabular}

Request:
\begin{tabular}{|c|c|}  \hline
{\tt int32 OX\_COMMAND} & {\tt int32 SM\_pops } \\
\hline
\end{tabular}

Result:  none.


\item
\begin{verbatim}
int SM_setName
\end{verbatim}
/*&jp
{\tt OperandStack} ��� {\it name} �� pop ��, �Ĥ���{\tt OperandStack} 
��� {\it obj} �� pop ��, ����򸽺ߤ�̾�����֤��ѿ� {\it name} �� 
bind ����.  ���ェλ�ʤ� 0 ��, �۾ェλ�ʤ� -1 ���ɤ�.  TCP/IP �ˤ�
���̿��Ǥ�, �۾ェλ�λ��Τ�, {\tt CMO\_ERROR2} ��stack �� push ����.
*/
/*&eg
It requests a server to pop {\it name}, to pop {\it obj}, and to
bind {\it obj} to a variable {\it name} in the current name space.
If an error has occured {\tt CMO\_ERROR2} is pushed to the stack.
*/
//&jp Stack before the request: (���� stack �� top.)
//&eg Stack before the request: (The rightmost one is the top of the stack.)
\begin{tabular}{|c|c|}  \hline
{\it obj} & {\it String name}  \\
\hline
\end{tabular}

Request:
\begin{tabular}{|c|c|}  \hline
{\tt int32 OX\_COMMAND} & {\tt int32 SM\_setName} \\
\hline
\end{tabular}

Result: none.

\item
\begin{verbatim}
SM_evalName
\end{verbatim}

/*&jp
���ߤ�̾�����֤��ѿ� {\it name} ��ɾ������.  ɾ���η�� {\it
resultObj} �򥹥��å����᤹.  �ؿ����Τ����ェλ�ʤ� 0 ��, �۾ェλ��
�� -1 ���ɤ�.  TCP/IP �ξ��, �۾ェλ�ξ��Τ� {\tt CMO\_ERROR2} 
�� stack �� push ����.
*/

/*&eg
It requests a server to pop {\it name} and to evaluate a variable
{\it name} in the current name space. The result of the evaluation
{\it resultObj} is pushed to the stack.
If an error has occured {\tt CMO\_ERROR2} is pushed to the stack.
*/

//&jp Stack before the request: (���� stack �� top.)
//&eg Stack before the request: (The rightmost one is the top of the stack.)
\begin{tabular}{|c|}  \hline
{\it String name}  \\
\hline
\end{tabular}

Request:
\begin{tabular}{|c|c|}  \hline
{\tt int32 OX\_COMMAND} & {\tt int32 SM\_evalName} \\
\hline
\end{tabular}

//&jp Stack after the request: (���� stack �� top.)
//&eg Stack after the request: (The rightmost one is the top of the stack.)
\begin{tabular}{|c|}  \hline
{\it resultObj} \\
\hline
\end{tabular}

Result:  none.

\item
\begin{verbatim}
SM_executeFunction
\end{verbatim}
/*&jp
�����å���� {\it n} �ĤΥǡ����� pop ����, �����ФΥ�������ؿ�{\it
s} ��¹Ԥ���.  ���顼�ΤȤ��Τ� {\tt CMO\_ERROR2} �� stack �� push ��
��.
*/
/*&eg
It requests a server to pop {\it s} as a function name, 
to pop {\it n} as the number of arguments and to execute
a local function {\it s} with {\it n} arguments popped from
the stack.
If an error has occured {\tt CMO\_ERROR2} is pushed to the stack.
*/

//&jp Stack before the request: (���� stack �� top.) \\
//&eg Stack before the request: (���� stack �� top.) \\
\begin{tabular}{|c|c|c|c|c|}  \hline
{\it objn} & $\cdots$ & {\it obj1} & {\it INT32 n} & {\it String s} \\
\hline 
\end{tabular}

Request:
\begin{tabular}{|c|c|}  \hline
{\tt int32 OX\_COMMAND} & {\tt int32 SM\_executeFunction}  \\
\hline
\end{tabular}

//&jp Stack after the request: �ؿ��¹Ԥη��.
//&eg Stack after the request: The result of the execution.

Result: none.

\item
\begin{verbatim}
SM_popSerializedLocalObject
\end{verbatim}

/*&jp
�������� pop ���� object �� local ������ serialization ����
OX message �Ȥ��� stream �ؽ��Ϥ���. OX message tag �Ȥ��Ƥ�, 
local �������б�������Τ��������Ƥ��뤳�Ȥ�ɬ�פǤ���. 
���δؿ��Ϥ����, homogeneous ��ʬ�������ƥ���Ѥ���.
*/
/*&eg
It requests a sever to pop an object, to convert it into a
serialized form according to a local serialization scheme, and
to send it to the stream as an OX message. 
An OX message tag corresponding to
the local data format must be sent prior to the serialized data
itself.
This operation is used mainly on homogeneous distributed systems.
*/

\item
\begin{verbatim}
SM_popCMO
\end{verbatim}

/*&jp
{\tt OperandStack} ��� object �� pop �� CMO ������ serialized object �� 
stream �� header {\tt OX\_DATA} ��Ĥ��Ƥʤ���. 
*/
/*&eg
It requests a server to pop an object from the stack, to convert
it into a serialized form according to the standard CMO encoding scheme,
and to send it to the stream with the {\tt OX\_DATA} header.
*/

Request:
\begin{tabular}{|c|c|}  \hline
{\tt int32 OX\_COMMAND} & {\tt int32 OX\_popCMO}  \\
\hline 
\end{tabular}

Result:  
\begin{tabular}{|c|c|}  \hline
{\tt int32 OX\_DATA} &   {\it Serialized CMO} \\
\hline 
\end{tabular}

\end{enumerate}


%% $OpenXM: OpenXM/doc/OpenXM-specs/project.tex,v 1.3 2000/01/24 12:32:49 noro Exp $
//&jp \section{���߸�Ƥ��ε�ǽ}
//&eg \section{Projects in work in progress}

//&jp \subsection{ OX DATA with Length �� }
//&eg \subsection{ OX DATA with Length }

/*&jp
Digital signature �դ� {\tt OX\_DATA} ��
�إå� {\tt OX\_DATA\_WITH\_LENGTH }��, �Ϥ��ޤ�, CMO ������,
���줫��, ����Υޡ�������ӥǥ������̾������.
���η��Υǡ����� {\it secured OX DATA} �ȸƤ�.
*/
/*&eg
{\tt OX\_DATA\_WITH\_LENGTH } is the OX tag for 
OX data message with a digital signature.
It is followed by the serial number, CMO, an end mark and a digital signature.
This type of OX data message is called {\it secured OX DATA}.
*/

\begin{verbatim}
#define OX_SECURED_DATA               521
\end{verbatim}

\noindent
\begin{tabular}{|c|c|c|c|c|}  \hline
{\tt int32 OX\_DATA\_WITH\_LENGTH} & {\tt int32} {\rm serial} 
& {\tt int32} {\rm size}
& {\sl CMObject} {\rm o} & {\it tail} \\
\hline 
\end{tabular}

//&jp {\tt size} �ե�����ɤ� �� -1 �����äƤ����礳�ξ����̵�뤹��.
//&eg If {\tt size} is equal to -1, then it is ignored.

//&jp {\it tail } �ϼ��Τ褦���������.
//&eg {\it tail } is defined as follows.
\\ \noindent
\begin{tabular}{|c|c|c|}  \hline
{\tt int32 CMO\_START\_SIGNATURE} & {\tt int32} {\rm size}
& {\it signature} \\
\hline 
\end{tabular}

/*&jp
������, {\tt size} �ϥХ����� {\it signature} ����Ĺ��.
{\it signature} ��, Hash �ؿ����Ѥ���, {\it CMO data}
�ˤ�������, �ǥ������̾�򤤤�, ������ serialized object 
�򸡽Ф���.
Tail �� {\tt size} �ե�����ɤ� 0 �ξ��, �ǥ������̾���Ϥʤ�.
*/
/*&eg
Here {\tt size} is the length of {\it signature}.
{\it signature} is a digital signature of {\it CMO data} by
a Hash function and is used to detect invalid serialized objects.
If {\tt size} of Tail is equal to 0, then it has no digital signature.
*/

//&jp ���饤�����, �����Фμ����ˤϼ���4�Ĥ����򤬤���.
//&eg Currently there are four modes of communicating data.
/*&jp
\begin{enumerate}
\item {\tt OX\_DATA} �Τߤ��Ѥ��� CMObject ������ (mathcap ��).
\item {\tt OX\_SECURED\_DATA} �Τߤ��Ѥ��� CMObject ������.
\item {\tt OX\_DATA} �����
{\tt OX\_SECURED\_DATA} �򺮺ߤ���
���ѤǤ���褦�ˤ���.
\item {\tt OX\_DATA} �Τߤ��Ѥ��Ƥ��� mathcap ���Ѥ��ʤ�(���ֹ�®).
\end{enumerate} 
*/
/*&eg
\begin{enumerate}
\item Only {\tt OX\_DATA} is used with checking by mathcap.
\item Only {\tt OX\_SECURED\_DATA} is used.
\item Both {\tt OX\_DATA} and {\tt OX\_SECURED\_DATA} can be used.
\item Only {\tt OX\_DATA} is used without checking by mathcap.
\end{enumerate} 
*/
/*&jp
1 ��û���, mathcap �μ������Խ�ʬ��,����Ǥ��ʤ� CMObject �򤦤��Ȥ�
��,�ʸ�� CMObject ������Ǥ��ʤ��ʤ�.  1 ��Ĺ���, Ĺ���ե�����ɤη�
���򤪤��ʤ�ʤ�����, �ǡ���������Ф�¦����ô��������ʤ����ȤǤ���.
2 ��Ĺ���, mathcap �μ������Խ�ʬ��,����Ǥ��ʤ� CMObject �򤦤��Ȥ�
�Ƥ�,Ĺ���ե�����ɤ��Ѥ���, �̿������Υ�������ɤ����Ȥ��Ǥ���.  2 ��
û���, Ĺ���ե�����ɤη׻��򤪤��ʤ�����, �ǡ���������Ф�¦����ô��
�����뤳�ȤǤ���.
*/
/*&eg
Suppose that the mathcap handling is incomplete and an application
has received unknown CMObject.
In mode 1, the application cannot detect the end of the CMObject
and it will not be able to understand the subsequent messages.
In mode 2, the application can detect the end of the unknown CMObject
from the size information. However, in mode 2, additional cost is
required on the sender to compute the total length of CMObjects.
*/

//&jp ���ߤΤ��٤ƤΥ���ץ륵���Ф�, 1, 4 �Τߤ�������Ƥ���.
//&teg Currently all sample servers implements only 1 and 4.

/*&jp
mathcap �θ򴹤ϥ��å����γ��ϻ�����ɬ�ܤȤ������ǤϤʤ����Ȥ����դ�
�줿��.  ���Ȥ���,�⡼�� 4 ���̿�����,���줫��, mathcap ��򴹤���,�⡼
�� 1 �ذܹԤ��뤳�Ȥ��ǽ�ʤ褦�˼������٤��Ǥ���.
*/
/*&eg
Note that the exchange of mathcaps are not necessary at the start
of a session. Any server should be implemented so that it can
change the communication mode dynamically, say, from 4 to 1.
*/

//&jp \subsection{�����Х����å��ޥ���ϥ�������ʳ�ĥ��ǽ���äƤ褤}
//&eg \subsection{Local extension on server stack machines}

\begin{verbatim}
#define  CMO_PRIVATE   0x7fff0000  /* 2147418112 */
\end{verbatim}

/*&jp
{\tt CMO\_PRIVATE} = {\tt OX\_PRIVATE} = {\tt SM\_PRIVATE} 
���Ϥ��ޤ�, 0x10000 �Ĥ� ID ��, private �ΰ�Ȥ���ͽ�󤵤�Ƥ���.
�ƥ����Фγ�ȯ�Դ֤�ɽ����ˡ�ˤĤ��ޤ���դ��ʤ��褦��, CMObject, 
OX Message, SMobject ��ɽ�����뤿��˻��Ѥ���.
*/
/*&eg
0x10000 ID's beginning from 
{\tt CMO\_PRIVATE} = {\tt OX\_PRIVATE} = {\tt SM\_PRIVATE} 
are reserved for private use.
They can be used to represent OX tags, CMObjects, SMobjects
which are not authorized yet.
*/


//&jp \subsection{MathLink �� OpenMath �ʤɤ��̿�����μ���}
//&eg \subsection{Implementation of other protocols such as MathLink and OpenMath}

/*&jp
Open asir, open sm1 �� Mathematica ���Ѥ��Ƥ��� MathLink �� 
Open Math (\cite{openmath})
�ץ��ȥ�����Ѵ����뤿���
�饤�֥��䥵���Ф��Ѱդ����, {\tt asir} �� {\tt kan/sm1} ��
������¤�����򤹤뤳�Ȥʤ���, �����˽�򤷤������ƥ���̿��Ǥ���.
*/
/*&eg
If we provide a library or a server for protocol conversion
between CMO and ``foreign'' protocols such as MathLink or OpenMath,
a client conforming to such protocols can communicate with 
Asir or kan/sm1 without knowing their internal structures.
*/

//&jp \subsection{���̥����å��ޥ������}
//&eg \subsection{Common operations on stack machines}

/*&jp
CMO ���տ路��, ���Ȥ���, Integer 32 ���Ф���,
add, sub, mul, �ʤɤδ���Ū�ʷ׻��� {\tt SM\_executeFunction}
�����ƤΥ����ƥ�Ǽ¹ԤǤ���褦�˸�Ƥ���Ƥ���.
�����å��ޥ�������湽¤�ˤĤ��Ƥ⸡Ƥ���Ƥ���.
*/
/*&eg
Fundamental operations such as {\tt add}, {\tt sub}, {\tt mul} should be
executed on any server by {\tt SM\_executeFunction}.
Control structures on stack machines such as {\tt if} and {\tt for}
are also being considered.
*/


%% $OpenXM: OpenXM/doc/OpenXM-specs/control.tex,v 1.12 2016/08/27 02:11:01 takayama Exp $
\section{Session Management}

\subsection{Control server}
/*&jp
OpenXM では, 次に述べるような単純かつロバストなサーバの制御方法
を採用している. 

OpenXM サーバは論理的に 2 つの I/O channel をもつ: 一方は計算データ
用であり, 他方は計算制御用である. 制御 channel はサーバを制御する
ためのコマンドを送るために使われる. 
サンプルサーバ ({\tt oxmain.c}) では, そのようなコントロールメッセージ
は別のプロセスが行っている. 以下, そのプロセスをコントロールサーバ
と呼ぶ. これに対して, 計算用サーバをエンジンと呼ぶ. 
コントロールサーバとエンジンは同一のマシン上で動作する. 
このため, コントロールサーバからエンジンに signal を送ることは容易である. 
コントロールサーバ自体も OX スタックマシンであり
{\tt SM\_control\_*} コマンドを受け取る. それらはエンジンへの
signal 送信, engine process の終了などの request のためのコマンドである. 
*/

/*&eg
In OpenXM we adopted the following simple and robust method to 
control servers.

An OpenXM server has logically two I/O channels: one for exchanging
data for computations and the other for controlling computations. The
control channel is used to send commands to control execution on the
nserver. The sample server ({\tt oxmain.c}) processes such control
messages on another process. We call such a process a {\it
control server}. In contrast, we call a server for computation an {\it
engine}. As the control server and the engine runs on the
same machine, it is easy to send a signal from the control server. 
A control server is also an
OpenXM stack machine and it accepts {\tt SM\_control\_*} commands
to send signals to a server or to terminate a server.
*/

\subsection{New OpenXM control servers}
/*&jp
OpenXM RFC 101 Draft を見よ
\htmladdnormallink{http://www.math.kobe-u.ac.jp/OpenXM/OpenXM-RFC.html}{http://www.math.kobe-u.ac.jp/OpenXM/OpenXM-RFC.html}.
*/
/*&eg
See OpenXM RFC 101 Draft.
\htmladdnormallink{http://www.math.kobe-u.ac.jp/OpenXM/OpenXM-RFC.html}{http://www.math.kobe-u.ac.jp/OpenXM/OpenXM-RFC.html}.
*/

\subsection{OpenXM reset protocol}

/*&jp
クライアントはコントロールサーバ経由でいつでもエンジンに signal を
送ることができる. しかし, I/O 操作は通常バッファリングされている
ため, トラブルが生ずる場合がある. エンジンを安全にリセットするため
次が必要である. 

\begin{enumerate}
\item 全ての OX メッセージは Java の意味で synchronized object である. 

\item エンジンのリセット後に送られるクライアントからの計算要求メッセージと
エンジンからの返答が正しく対応していなければならない. 
\end{enumerate}

{\tt SM\_control\_reset\_connection} は, エンジンの安全なリセットを
行う一連の手続きを開始するための SM コマンドである. 
クライアントから {\tt SM\_control\_reset\_connection} がコントロール
サーバに送られると, コントロールサーバは {\tt SIGUSR1} をエンジンに
送る. 以後の手続きは次の通りである. 

\vskip 2mm
\noindent
{\it クライアント側} 
\begin{enumerate}
\item {\tt SM\_control\_reset\_connection} をコントロールサーバに
送った後, クライアントはリセット状態に入る. リセット状態では, 
{\tt OX\_SYNC\_BALL} を受け取るまですべてのメッセージを読みとばす. 
\item {\tt OX\_SYNC\_BALL} を受け取ったあと, クライアントは
{\tt OX\_SYNC\_BALL} をエンジンに送り, 通常状態に戻る. 
\end{enumerate}

\noindent
{\it エンジン側}
\begin{enumerate}
\item 
{\tt SIGUSR1} をコントロールサーバから受け取ったら, エンジンは
リセット状態に入る. {\tt OX\_SYNC\_BALL} をクライアントに送る. 
この時点でクライアントは既にリセット状態にあるので, この送信が
ブロックされることはない. 
\item エンジンは
{\tt OX\_SYNC\_BALL} を受け取るまですべてのメッセージを読みとばす. 
{\tt OX\_SYNC\_BALL} を受け取ったら通常状態に戻る. 
\end{enumerate}
*/
/*&eg
A client can send a signal to an engine by using the control channel 
at any time. However, I/O operations are usually buffered,
which may cause troubles.
To reset an engine safely the following are required.

\begin{enumerate}
\item Any OX message must be a synchronized object in the sense of Java.

\item After restarting an engine, a request from a client 
must correctly corresponds to the response from the engine.
\end{enumerate}

{\tt SM\_control\_reset\_connection} is a stack machine command to
initiate a safe resetting of an engine.
The control server sends {\tt SIGUSR1} to the engine if it receives
{\tt SM\_control\_reset\_connection} from the client.
Under the OpenXM reset protocol, an engine and a client act as follows.

\vskip 2mm
\noindent
{\it Client side} 
\begin{enumerate}
\item After sending {\tt SM\_control\_reset\_connection} to the
control server, the client enters the resetting state. It discards all {\tt
OX} messages from the engine until it receives {\tt OX\_SYNC\_BALL}.
\item After receiving {\tt OX\_SYNC\_BALL} the client sends 
{\tt OX\_SYNC\_BALL} to the engine and returns to the usual state.
\end{enumerate}

\noindent
{\it Engine side}
\begin{enumerate}
\item 
After receiving {\tt SIGUSR1} from the control server,
the engine enters the resetting state.
The engine sends {\tt OX\_SYNC\_BALL} to the client.
The operation does not block because
the client is now in the resetting state.
\item The engine discards all OX messages from the engine until it
receives {\tt OX\_SYNC\_BALL}. After receiving {\tt OX\_SYNC\_BALL} 
the engine returns to the usual state.
\end{enumerate}
*/

/*&eg
Figure \ref{reset} illustrates the flow of data.
{\tt OX\_SYNC\_BALL} is a special OX message and
is used to mark the end of data remaining in the
I/O streams. After reading it, it is assured that each stream is empty
and that the subsequent request from a client correctly 
corresponds to the response from the engine.
*/
/*&jp
図 \ref{reset} はデータの流れを示す. 
{\tt OX\_SYNC\_BALL} は特殊な OX メッセージであり, 
I/O stream に残るデータの終りを示す. 
{\tt OX\_SYNC\_BALL} を読んだ後, それぞれの stream は空であり, 
後に続くクライアントからのリクエストと, エンジンからの返答が
正しく対応する. 
*/
\begin{figure}[htbp]
\epsfxsize=10cm
\begin{center}
\epsffile{reset.eps}
\end{center}
\caption{OpenXM reset procedure}
\label{reset}
\end{figure}

\subsection{Control message (SMObject/TCPIP/Control)}

\begin{enumerate}
\item 
\begin{verbatim}
SM_control_reset_connection 
\end{verbatim}
/*&jp
コントロールサーバに, {\tt SIGUSR1} をエンジンに送るよう要求する. 
*/
/*&eg
It requests a control server to send {\tt SIGUSR1} to the engine.
The control server should immediately reply an acknowledgment to
the client.
*/
Request:
\begin{tabular}{|c|c|}  \hline
{\tt int32 OX\_COMMAND} & {\tt int32 SM\_control\_reset\_connection}  \\
\hline 
\end{tabular}
Result:   none. \\
/*&jp
  すべてエンジンは reset protocol を実装することが推奨されるが,
実装していない場合は, mathcap の第4引数の option list で
{\tt no\_ox\_reset} を送信すべきである (参照: oxpari). \\
*/
/*&eg
  All engines are encouraged to install the reset protocol,
but when it is not implemented, 
{\tt no\_ox\_reset} option should be included in the fourth argument
(option list) of the mathcap (ref: oxpari). \\
*/
/*&jp
注意:  古い実装(2000年以前)の control server, client では,
次の形式の result code が戻ることを仮定している場合がある.
これら古い実装は更新することが必要である. \\
*/
/*&eg
Note: Some old implementations of control servers and clients (before 2000)
assume the result code of the following format.
These obsolete implementations should be updated.\\
*/
\begin{tabular}{|c|c|}  \hline
{\tt int32 OX\_DATA} & {\tt CMO\_INT32} {\rm result} \\
\hline 
\end{tabular}\\

\item
\begin{verbatim}
SM_control_kill
\end{verbatim}
/*&jp
サーバはこのメッセージを受信したら
%ただちに返答をおくり, 
すべてのファイルをクローズして終了する.
*/
/*&eg
It requests a control server to terminate the engine and the control server
itself. 
%The control server should immediately reply an acknowledgment to
%the client.
All files and streams should be closed before the termination of servers.
*/
Request:
\begin{tabular}{|c|c|}  \hline
{\tt int32 OX\_COMMAND} & {\tt int32 SM\_control\_kill}  \\
\hline 
\end{tabular}\\
Result: none.
\end{enumerate}

\medbreak
\noindent
//&jp {\bf 例}: (シリアル番号は省略してある.) 
//&eg {\bf Example}: (serial numbers are omitted.)
\begin{verbatim}
0  0 2 01 (OX_COMMAND) 
0  0 4 06 (SM_control_reset_connection)
\end{verbatim}

%//&jp Reset に対する返事.
%//&eg Reply to the reset request
%\begin{verbatim}
%0  0 2 02 (OX_DATA)
%0  0 0  2 (CMO_INT32)
%0  0 0  0 (  0   )
%\end{verbatim}


//&jp 第1のチャンネルでは次の {\tt OX\_SYNC\_BALL} が交換されて同期が取られる.
//&eg {\tt OX\_SYNC\_BALL} are exchanged on the data channel for synchronization.

\begin{verbatim}
0   0   2   03   (OX_SYNC_BALL)
\end{verbatim}

\subsection{Notification from servers}

/*&jp
OpenXM サーバは, 可能であるかぎり寡黙である.
たとえばエラーをおこしても, エラーはサーバのエンジンスタックにつまれる
だけであり, サーバはクライアントが {\tt pop\_cmo} メッセージをおくらない
かぎり何も送信しない.
*/
/*&eg
OpenXM servers try to be as quiet as possible.
For example, engine errors of a server are only put on the engine stack and
the engine does not send error packets unless the client sends the message
{\tt pop\_cmo}.
*/

/*&jp
OpenXM はこの原則をやぶる例外的な方法を一つ用意している.
コントロールサーバは,
{\tt OX\_NOTIFY} ヘッダおよびそれにつづく {\tt OX\_DATA} パケットを送る
ことができる.
この機能は mathcap で禁止することも可能である.
*/
/*&eg
OpenXM provides a method to notify events.
Control server may send {\tt OX\_NOTIFY} header and an {\tt OX\_DATA} packet.
This transmission may be prohibited by mathcap.
*/

/*&jp
この機能をどのように使うか例をあげて説明しよう.
{\tt Asir} の {\tt ox\_plot} サーバは, quit ボタンをもっている.
quit ボタンがおされると canvas が消滅するが, エンジン自体は存在を
つづける.  この状態で描画命令がくると,
エンジンスタックに, ``canvas does not exist'' というエラーがつまれる.
エンジンがこのエラーが生じたことを緊急に知らせたいときに
{\tt OX\_NOTIFY} を用いる.
*/
/*&eg
Let us explain how to use {\tt OX\_NOTIFY} by an example.
The {\tt ox\_plot} server of {\tt asir} has a quit button.
If the quit button is pressed, the canvas dissappears, but the engine
does not terminate.
If the client sends drawing messages without the canvas, 
then the engine pushes
error packets ``canvas does not exist'' on the engine stack.
If the engine wants to notify the error to the client immediately,
the {\tt OX\_NOTIFY} message should be used.
*/

/*&jp
ここで, {\tt OX\_NOTIFY} をおくるのは, コントロールプロセスで
あることに注意しよう.
したがってエンジンはなんらかの方法で, コントロールサーバに
{\tt OX\_NOTIFY} をおくることを依頼しないといけない.
この方法は, OS によりいろいろな方法が可能だか,
たとえば, unix では
ファイル {\tt /tmp/.ox\_notify.pid} に touch することでこれを
一つの実現することが可能である.
ここで {\tt pid} はエンジンのプロセス番号である.
コントロールサーバはファイル {\tt /tmp/.ox\_notify.pid} が
touch されたことを検出したら, クライアントに
{\tt OX\_NOTIFY} パケットおよび {\tt OX\_DATA} で {\tt cmo\_null} を送る.
エンジンはファイルを用いてコントロールサーバに急を知らせる以外に,
共有メモリやシグナルを用いてしらせてもよい.
*/
/*&eg
Let us note that only the control process is allowed to send {\tt OX\_NOTIFY}.
Therefore, the engine must ask the control server to send
{\tt OX\_NOTIFY}.
Methods to ask the control process from the engine
depends on operating system.
In case of unix, one method is the use of a file;
for instance,
if the engine touches the file
{\tt /tmp/.ox\_notify.pid}, then the control server sends
the {\tt OX\_NOTIFY} header and the {\tt OX\_DATA} packet 
of {\tt cmo\_null}.
Here, {\tt pid} is the process id of the engine.
Engines and control processes may use a shared memory or a signal
instead of the file {\tt /tmp/.ox\_notify.pid}.
*/

%% $OpenXM$
//&jp \section{TCP/IP �ǤΥ��å����Υ�������}
//&eg \section{How to start a session on TCP/IP} (This section has not yet been translated.)

/*&jp
����ȥ�����ץ�����, �׻��ץ�����
�Ȥ��, ��ưľ���
1 byte �Υǡ�����񤭽Ф� flush ����.
���Τ���, 1 byte �Υǡ������ɤ߹���.
���饤����Ȥϥ���ȥ�����ץ�����, �׻��ץ�����
�ˤĤʤ���ե�����ǥ�������ץ���ξ������
�ޤ� 1 byte �Υǡ������ɤ�.
������
1 byte �Υǡ�����񤭽Ф� flush ����.
1 byte �Υǡ�����,
{\tt 0}, {\tt 1}, {\tt FF} �Τɤ줫�Ǥ���,
{\tt 0} �ϰʲ����̿��ˤ����� {\tt int32} �򤪤���Τ�,
network byte order ����Ѥ�����,
{\tt 1} �ϰʲ����̿��ˤ����� {\tt int32} �򤪤���Τ�,
little endian ����Ѥ�����,
{\tt FF} �ϰʲ����̿��ˤ����� {\tt int32} �򤪤���Τ�,
big endian ����Ѥ�����,
�Ȥ�����̣�Ǥ���.
ξ�Ԥδ�˾�����פ��ʤ����ϤĤͤ� {\tt 0} (network byte order)
����Ѥ���.
Network byte order �����������Ƥ��ʤ������ƥ�Ǥ�,
{\tt 0} �򤪤���Ф褤.
��������Ψ������Ȥʤ��̿��ˤ�����, network byte order �ؤ��Ѵ���
�������ʥܥȥ�ͥå��Ȥʤ뤳�Ȥ����뤳�Ȥ�λ�򤷤Ƥ����٤��Ǥ���.

\begin{verbatim}
#define OX_BYTE_NETWORK_BYTE_ORDER    0
#define OX_BYTE_LITTLE_ENDIAN         1
#define OX_BYTE_BIG_ENDIAN         0xff
\end{verbatim}


����: {\tt OpenXM/src/kxx} �˴ޤޤ��, {\tt ox} (����ȥ�����ץ�����,
�׻��ץ������򤿤���������������)��, ɸ��� One Time Password
�ε�ǽ���äƤ��ޤ�.
���ε�ǽ�� OFF �ˤ���ˤ� {\tt -insecure} option ����Ѥ��Ʋ�����.
One Time Password �� 0 �ǽ�λ����Х�����Ǥ���,
����ȥ�����, �׻������Υץ������򤿤�������ޤ���,
{\tt ox} �ϥ���ȥ�����, �׻������Υץ��������б�����,
�դ��ĤΥݡ��Ȥ�
One Time Password �Х���������Ф��Ƥ��ޤ�.

{\tt ox} (�������� {\tt oxmain.c}, {\tt kan96xx/plugin/oxmisc.c})�ˤ����Ƥ�
{\tt oxTellMyByteOrder()} ��, �����ФˤĤ��Ƥ� byte order �����
����, �ɤ߹��ߤ��äƤ���.
���饤����ȤˤĤ��Ƥ�,
{\tt oxSetByteOrder()} ��, byte order ������ɤ߹���, ���Ф򤪤��ʤäƤ���.

One time �ѥ���ɤϰ������̿�ϩ�����������ɬ�פ�����ޤ�.
�ޤ�, ���ߤ� {\rm ox} �μ����Ǥ�, One time �ѥ���ɤ�
������, ���饤����Ȥ� login ���Ƥ���ͤϤ��٤Ƹ��뤳�Ȥ�
�Ǥ��ޤ��Τ�, ������, ���饤����Ȥˤϰ��դΤ���ͤϤ��ʤ���
���ꤷ�ʤ��Ȥ����ޤ���.
One time �ѥ���ɤ������������, ��⡼�ȥޥ���� {\rm ox}
��Ω���夲��ˤ�
���Ȥ��� 
{\tt ssh} �� {\tt -f } ���ץ������Ѥ��ƻ��Ѥ��ޤ�.

�ʲ��� {\rm sm1} �Ǥμ���, ư����Ǥ�.
�����Ǥ�, {\tt yama} �� {\tt sm1} ��� {\tt dc1} �� {\tt ox} 
��Ω���夲�Ƥ��ޤ�.
{\footnotesize
\begin{verbatim}
yama% sm1
sm1>(ox.sm1) run ;
ox.sm1, --- open sm1 protocol module 10/1,1999  (C) N.Takayama. oxhelp for help
sm1>[(dc1.math.kobe-u.ac.jp) (taka)] sm1connectr-ssh /ox.ccc set ;
Hello from open. serverName is yama.math.kobe-u.ac.jp and portnumber is 0
Done the initialization. port =1024
Hello from open. serverName is yama.math.kobe-u.ac.jp and portnumber is 0
Done the initialization. port =1025
[    4 , 1025 , 3 , 1024 ] 
Executing the command : ssh -f dc1.math.kobe-u.ac.jp -l taka 
"/home/taka/OpenXM/bin/oxlog /usr/X11R6/bin/xterm -icon 
-e /home/taka/OpenXM/bin/ox -reverse -ox /home/taka/OpenXM/bin/ox_sm1 
-host yama.math.kobe-u.ac.jp -data 1025 -control 1024 -pass 518158401   "
[ 
taka@dc1.math.kobe-u.ac.jp's password: 
Trying to accept... Accepted.
Trying to accept... Accepted.

Control port 1024 : Connected.

Stream port 1025 : Connected.
Byte order for control process is network byte order.
Byte order for engine process is network byte order.
\end{verbatim}
}

*/


%% $OpenXM$
//&jp \section{ ���֥������Ȥ�ʸ����ɽ�� }
//&eg \section{ String expression of objects}

/*&jp
ʸ����ɽ����, �����ƥ� xxx �Υޥ˥奢��˵��Ҥ���Ƥ���켡��Ū�����Ϸ�����
��������.

*/
/*&eg
The string expression of objects of the system xxx is also used
for a string expression for the OX xxx server.

*/



%% $OpenXM: OpenXM/doc/OpenXM-specs/cmo-basic1.tex,v 1.5 2000/01/24 07:36:35 noro Exp $
//&jp \section{ ��, ¿�༰ ��  CMO ɽ�� }
//&eg \section{ CMOexpressions for numbers and polynomials }
\label{sec:basic1}
/*&C
@../SSkan/plugin/cmotag.h
\begin{verbatim}
#define     CMO_MONOMIAL32  19
#define     CMO_ZZ          20 
#define     CMO_QQ          21
#define     CMO_ZERO        22
#define     CMO_DMS_GENERIC  24
#define     CMO_DMS_OF_N_VARIABLES  25
#define     CMO_RING_BY_NAME   26
#define     CMO_DISTRIBUTED_POLYNOMIAL 31
#define     CMO_RATIONAL       34


#define     CMO_INDETERMINATE  60
#define     CMO_TREE           61
#define     CMO_LAMBDA         62    /* for function definition */
\end{verbatim}

*/

/*&jp
�ʲ�, ���롼�� CMObject/Basic, CMObject/Tree 
����� CMObject/DistributedPolynomial
��°���� CMObject �η�������������.

\noindent
{\tt OpenXM/src/ox\_toolkit} �ˤ��� {\tt bconv} ���������
CMO expression �� binary format ���Ѵ��Ǥ���Τ�,
����򻲹ͤˤ���Ȥ���.
*/
/*&eg
In the sequel, we will explain on the groups
CMObject/Basic, CMObject/Tree 
and CMObject/DistributedPolynomial.

\noindent
The program {\tt bconv} at {\tt OpenXM/src/ox\_toolkit}
translates 
CMO expressions into binary formats.
It is convinient to understand the binary formats explained in
this section.
*/

/*&C
\noindent Example:
\begin{verbatim}
bash$ ./bconv
> (CMO_ZZ,123123);
00 00 00 14 00 00 00 01 00 01 e0 f3 
\end{verbatim}
*/
/*&jp

\bigbreak
\noindent
Group CMObject/Basic requires CMObject/Primitive. \\
ZZ, QQ, Zero, Rational, Indeterminate,$\in$ CMObject/Basic. \\
\begin{eqnarray*}
\mbox{Zero} &:& ({\tt CMO\_ZERO}) \\ 
& & \mbox{ --- ��˥С������ ������ɽ��. } \\
\mbox{ZZ}         &:& ({\tt CMO\_ZZ},{\sl int32}\, {\rm f}, {\sl byte}\, \mbox{a[1]}, \ldots
{\sl byte}\, \mbox{a[m]} ) \\
&:& \mbox{ --- bignum �򤢤�魯. a[i] �ˤĤ��ƤϤ��Ȥ�����}\\
\mbox{QQ}        &:& ({\tt CMO\_QQ}, {\sl ZZ}\, {\rm a}, {\sl ZZ}\, {\rm b}) \\
& & \mbox{ --- ͭ���� $a/b$ ��ɽ��. } \\
\mbox{Rational}        &:& ({\tt CMO\_RATIONAL}, {\sl CMObject}\, {\rm a}, {\sl CMObject}\, {\rm b}) \\
& & \mbox{ ---  $a/b$ ��ɽ��. } \\
\mbox{Indeterminate}        &:& ({\tt CMO\_INDETERMINATE}, {\sl Cstring}\, {\rm v}) \\
& & \mbox{ --- �ѿ�̾ $v$ . } \\
\end{eqnarray*}
*/
/*&eg

\bigbreak
\noindent
Group CMObject/Basic requires CMObject/Primitive. \\
ZZ, QQ, Zero, Rational, Indeterminate,$\in$ CMObject/Basic. \\
\begin{eqnarray*}
\mbox{Zero} &:& ({\tt CMO\_ZERO}) \\ 
& & \mbox{ --- Universal zero } \\
\mbox{ZZ}         &:& ({\tt CMO\_ZZ},{\sl int32}\, {\rm f}, {\sl byte}\, \mbox{a[1]}, \ldots
{\sl byte}\, \mbox{a[m]} ) \\
&:& \mbox{ --- bignum. The meaning of a[i] will be explained later.}\\
\mbox{QQ}        &:& ({\tt CMO\_QQ}, {\sl ZZ}\, {\rm a}, {\sl ZZ}\, {\rm b}) \\
& & \mbox{ --- Rational number $a/b$. } \\
\mbox{Rational}        &:& ({\tt CMO\_RATIONAL}, {\sl CMObject}\, {\rm a}, {\sl CMObject}\, {\rm b}) \\
& & \mbox{ ---  Rational expression $a/b$. } \\
\mbox{Indeterminate}        &:& ({\tt CMO\_INDETERMINATE}, {\sl Cstring}\, {\rm v}) \\
& & \mbox{ --- Variable name $v$ . } \\
\end{eqnarray*}
*/
/*&C

*/

/*&jp
Indeterminate ���ѿ�̾�򤢤�魯.
v �ϥХ�����Ǥ���Фʤˤ��Ѥ��Ƥ�褤��,
�����ƥ�����ѿ�̾�Ȥ����Ѥ�����Х���������¤�����.
�ƥ����ƥ� xxx ��Ǥ�դ�ʸ�����ƥ����ƥ��ͭ���ѿ�̾��1��1���Ѵ��Ǥ���褦��
�������ʤ��Ȥ����ʤ�.
(�����
{\tt Dx} �� {\tt \#dx} ���Ѵ�����ʤɤ�
escape sequence ���Ѥ��Ƽ¸�����Τ�, ̵��������褦�Ǥ���.
�ơ��֥���������ɬ�פ�����Ǥ�����.)
*/
/*&eg
Indeterminate is a name of a variable.
v may be any sequence of bytes, but each system has its own
restrictions on the names of variables.
Indeterminates of CMO and internal variable names must be translated
in one to one correspondence.
*/

/*&jp 

\noindent
Group CMObject/Tree requires CMObject/Basic. \\
Tree, Lambda $\in$ CMObject/Basic. \\
\begin{eqnarray*}
\mbox{Tree}        &:& ({\tt CMO\_TREE}, {\sl Cstring}\, {\rm name},
 {\sl Cstring}\, {\rm cdname}, {\sl List}\, {\rm leaves}) \\
& & \mbox{ --- ̾�� name ������ޤ��ϴؿ�. �ؿ���ɾ���Ϥ����ʤ�ʤ�. } \\
& & \mbox{ --- cdname �϶�ʸ����Ǥʤ���� name �ΰ�̣����������Ƥ��� }\\
& & \mbox{ --- OpenMath CD (content dictionary) ��̾��. } \\
\mbox{Lambda}        &:& ({\tt CMO\_LAMBDA}, {\sl List}\, {\rm args},
                          {\sl Tree} {\rm body}) \\                          
& & \mbox{ --- body �� args ������Ȥ���ؿ��Ȥ���. } \\
& & \mbox{ --- optional �ʰ�����ɬ�פʤȤ���, leaves �θ�ؤĤŤ���.} \\
\end{eqnarray*}
*/
/*&eg 

\noindent
Group CMObject/Tree requires CMObject/Basic. \\
Tree, Lambda $\in$ CMObject/Basic. \\
\begin{eqnarray*}
\mbox{Tree}        &:& ({\tt CMO\_TREE}, {\sl Cstring}\, {\rm name},
 {\sl Cstring}\, {\rm cdname}, {\sl List}\, {\rm leaves}) \\
& & \mbox{ --- A function or a constant of name. Functions are not evaluated. } \\
& & \mbox{ --- cdname may be a null. If it is not null, it is the name of}\\
& & \mbox{ --- the OpenMath CD (content dictionary). } \\
\mbox{Lambda}        &:& ({\tt CMO\_LAMBDA}, {\sl List}\, {\rm args},
                          {\sl Tree} {\rm body}) \\                          
& & \mbox{ --- a function with the arguments body. } \\
& & \mbox{ --- optional arguments come after leaves.} \\
\end{eqnarray*}
*/

/*&jp
������������륷���ƥ�Ǥ�, Tree ��¤�����̤ˤ��������.
���Ȥ���, $\sin(x+e)$ ��,
{\tt (sin, (plus, x, e))}
�ʤ� Tree �Ǥ���魯�Τ�����Ū�Ǥ���.
Tree ��ɽ���� �����å��ޥ���Υ�٥�Ǥ����ʤ��Ȥ����,
{\tt ox\_BEGIN\_BLOCK}, {\tt ox\_END\_BLOCK} ��ɾ������������Τ�
��Ĥ���ˡ�Ǥ��� (cf. Postscript �� {\tt \{ }, {\tt \} }).
���Ȥ��о����ˡ�Ǥ� 
{\tt x, e, plus, sin } �� begin block, end block �Ǥ�����Ф������.
�����ϥ����å��ޥ���μ�����ʤ�٤���ñ�ˤ���Ȥ���Ω���Ȥꤿ��,
�ޤ����إ��֥������Ȥ� OX �����å��ޥ���� CMObject �򺮺ߤ���ɽ��������
�ʤ�.
�������ä�,
Tree ��¤�� Open Math ����ɽ���������� CMO ��Ƴ�����뤳�Ȥˤ���.
�ޤ����Τۤ���, ���������ꤹ�륷���ƥ� xxx �ˤ�����, Open XM �б���
�Ϥ뤫���ưפǤ���.
�ʤ�, Tree ��, Open Math �Ǥ�, Symbol, Application �Υᥫ�˥������������.
*/
/*&eg
In many computer algebra systems, mathematical expressions are usually
expressed in terms of a tree structure.
For example,
$\sin(x+e)$ is expressed as
{\tt (sin, (plus, x, e))}
as a tree.
Tree may be expressed by putting the expression between
{\tt SM\_beginBlock} and {\tt SM\_endBlock}, which are
stack machine commands for delayed evaluation.
(cf. {\tt \{ }, {\tt \} } in PostScript).
However it makes the implementation of stack machines complicated.
It is desirable that CMObject is independent of OX stack machine.
Therefore we introduce an OpenMath like tree representation for CMO 
tree object.
This method allows us to implement tree structure easily 
on individual OpenXM systems.
Note that CMO Tree corresponds to Symbol and Application in OpenMath.
*/


/*&C

*/
/*&jp
Lambda �ϴؿ���������뤿��δؿ��Ǥ���.
Lisp �� Lambda ɽ����Ʊ��.
*/
/*&eg
Lambda is used to define functions.
It is the same as the Lambda expression in Lisp.
*/

\noindent
//&jp ��: $sin(x+e)$ ��ɽ��.
//&eg Example: the expression of $sin(x+e)$.
\begin{verbatim}
(CMO_TREE, (CMO_STRING, "sin"), (CMO_STRING, "basic"),
    (CMO_LIST,[size=]1, 
        (CMO_TREE, (CMO_STRING, "plus"), (CMO_STRING, "basic"),
            (CMO_LIST,[size=]2, (CMO_INDETERMINATE,"x"),
//&jp                  (CMO_TREE,(CMO_STRING, "e"),  ���������
//&eg                  (CMO_TREE,(CMO_STRING, "e"),  the base of natural logarithms
                            (CMO_STRING, "basic"))
        ))
    )
)
\end{verbatim}

\noindent
Example:
\begin{verbatim}
sm1> [(plus) (Basic) [(123).. (345)..]] [(class) (tree)] dc ::
Class.tree [    $plus$ , $Basic$ , [    123 , 345 ]  ] 
\end{verbatim}



\bigbreak
//&jp ����, ʬ��ɽ��¿�༰�˴ط����륰�롼�פ�������褦.
/*&eg
Let us define a group for distributed polynomials. In the following
DMS stands for Distributed Monomial System.
*/

\medbreak
\noindent
Group CMObject/DistributedPolynomials requires CMObject/Primitive,
CMObject/Basic. \\
Monomial, Monomial32, Coefficient, Dpolynomial, DringDefinition,
Generic DMS ring, RingByName, DMS of N variables $\in$ 
CMObject/DistributedPolynomials. \\
/*&jp
\begin{eqnarray*}
\mbox{Monomial} &:& \mbox{Monomial32}\, |\, \mbox{Zero} \\
\mbox{Monomial32}&:& ({\tt CMO\_MONOMIAL32}, {\sl int32}\, n,
{\sl int32}\, \mbox{e[1]}, \ldots,
{\sl int32}\, \mbox{e[n]}, \\
& & \ \mbox{Coefficient}) \\
& & \mbox{ --- e[i] ��, $n$ �ѿ� monomial 
$x^e = x_1^{e_1} \cdots x_n^{e_n}$ �γƻؿ� $e_i$
�򤢤�魯.} \\
\mbox{Coefficient}&:& \mbox{ZZ} | \mbox{Integer32} \\
\mbox{Dpolynomial}&:& \mbox{Zero} \\
& & |\ ({\tt CMO\_DISTRIBUTED\_POLYNOMIAL},{\sl int32} m, \\
& & \ \ \mbox{DringDefinition},
[\mbox{Monomial32}|\mbox{Zero}], \\
& &\ \ 
\{\mbox{Monomial32}\}) \\
& &\mbox{--- m �ϥ�Υߥ���θĿ��Ǥ���.}\\
\mbox{DringDefinition}
&:& \mbox{DMS of N variables} \\
& & |\ \mbox{RingByName} \\
& & |\ \mbox{Generic DMS ring} \\
& & \mbox{ --- ʬ��ɽ��¿�༰�Ĥ����. } \\
\mbox{Generic DMS ring}
&:& \mbox{({\tt CMO\_DMS\_GENERIC}) --- ���ǤϤ�����}\\
\mbox{RingByName}&:& ({\tt CMO\_RING\_BY\_NAME}, {\sl Cstring}\  {\rm s}) \\
& & \mbox{ --- ̾�� s ��, ��Ǽ���줿 ring ���.} \\
\mbox{DMS of N variables}
&:& ({\tt CMO\_DMS\_OF\_N\_VARIABLES}, \\
& & \ ({\tt CMO\_LIST}, {\sl int32}\, \mbox{m},
{\sl Integer32}\,  \mbox{n}, {\sl Integer32}\,\mbox{p} \\
& & \ \ [,{\sl object}\,\mbox{s}, {\sl Cstring}\,\mbox{c}, 
          {\sl List}\, \mbox{vlist},
{\sl List}\, \mbox{wvec}, {\sl List}\, \mbox{outord}]) \\
& & \mbox{ --- m �Ϥ��Ȥ�³�����Ǥο�} \\
& & \mbox{ --- n ���ѿ��ο�, p �� ɸ��} \\
& & \mbox{ --- s �� ring ��̾��} \\
& & \mbox{ --- c �Ϸ�����, QQ, ZZ �ξ���ʸ����� QQ, ZZ �Ƚ�.} \\
& & \mbox{ --- vlist �� Indeterminate �Υꥹ��(����). ¿�༰�Ĥ��ѿ��ꥹ��} \\
& & \mbox{ --- wvec �� order �򤭤�� weight vector,} \\
& & \mbox{ --- outord �Ͻ��Ϥ���Ȥ����ѿ����.} \\
\end{eqnarray*}
*/
/*&eg
\begin{eqnarray*}
\mbox{Monomial} &:& \mbox{Monomial32}\, |\, \mbox{Zero} \\
\mbox{Monomial32}&:& ({\tt CMO\_MONOMIAL32}, {\sl int32}\, n,
                      {\sl int32}\, \mbox{e[1]}, \ldots,
                      {\sl int32}\, \mbox{e[n]}, \\
                 & & \ \mbox{Coefficient}) \\
                 & & \mbox{ --- e[i] is the exponent $e_i$ of the monomial 
                      $x^e = x_1^{e_1} \cdots x_n^{e_n}$. } \\
\mbox{Coefficient}&:& \mbox{ZZ} | \mbox{Integer32} \\
\mbox{Dpolynomial}&:& \mbox{Zero} \\
                 & & |\ ({\tt CMO\_DISTRIBUTED\_POLYNOMIAL},{\sl int32} m, \\
                 & & \ \ \mbox{DringDefinition}, [\mbox{Monomial32}|\mbox{Zero}], \\
                 & &\ \ 
                    \{\mbox{Monomial32}\})  \\
                 & &\mbox{--- m is equal to the number of monomials.}\\
\mbox{DringDefinition}
                 &:& \mbox{DMS of N variables} \\
                 & & |\ \mbox{RingByName} \\
                 & & |\ \mbox{Generic DMS ring} \\
                 & & \mbox{ --- definition of the ring of distributed polynomials. } \\
\mbox{Generic DMS ring}
                 &:& ({\tt CMO\_DMS\_GENERIC}) \\
\mbox{RingByName}&:& ({\tt CMO\_RING\_BY\_NAME}, {\sl Cstring} s) \\
                 & & \mbox{ --- The ring definition referred by the name ``s''.} \\
\mbox{DMS of N variables}
                 &:& ({\tt CMO\_DMS\_OF\_N\_VARIABLES}, \\
                 & & \ ({\tt CMO\_LIST}, {\sl int32}\, \mbox{m},
                  {\sl Integer32}\,  \mbox{n}, {\sl Integer32}\, \mbox{p} \\
                 & & \ \ [,{\sl Cstring}\,\mbox{s}, {\sl List}\, \mbox{vlist},
                          {\sl List}\, \mbox{wvec}, {\sl List}\, \mbox{outord}]) \\
                 & & \mbox{ --- m is the number of elements.} \\
                 & & \mbox{ --- n is the number of variables, p is the characteristic} \\
                 & & \mbox{ --- s is the name of the ring, vlist is the list of variables.} \\
                 & & \mbox{ --- wvec is the weight vector.} \\
                 & & \mbox{ --- outord is the order of variables to output.} \\
\end{eqnarray*}
*/

/*&jp
RingByName �� DMS of N variables �Ϥʤ��Ƥ�, DMS ������Ǥ���.
�������ä�, ������������Ƥʤ������ƥ�� DMS �򰷤���Τ�
���äƤ⤫�ޤ�ʤ�.

�ʲ�, �ʾ�� CMObject  �ˤ�������,
xxx = asir, kan �ο��񤤤򵭽Ҥ���.
*/
/*&eg
Note that it is possible to define DMS without RingByName and 
DMS of N variables.

In the following we describe how the above CMObjects 
are implemented on Asir and Kan.
*/

\subsection{ Zero}
/*&jp
CMO �Ǥ� ������ɽ��ˡ���ʤ�Ȥ���⤢�뤬,
�ɤΤ褦�ʥ����򤦤��ȤäƤ�,
�����ƥ�Υ������Ѵ��Ǥ���٤��Ǥ���.
*/
/*&eg
Though CMO has various representations of zero,
each representation should be translated into zero
in the system.
*/


//&jp \subsection{ ���� ZZ }
//&eg \subsection{ Integer ZZ }

\begin{verbatim}
#define     CMO_ZZ          20 
\end{verbatim}

/*&jp
������Ǥ�Open xxx ����ˤ�����Ǥ�դ��礭��������(bignum)�ΰ����ˤĤ�
����������.  Open XM ����ˤ�����¿������������ɽ���ǡ����� CMO\_ZZ �� 
GNU MP�饤�֥��ʤɤ򻲹ͤˤ����߷פ���Ƥ���, ����դ�������ɽ������
���Ƥ���.  (cf. {\tt kan/sm1} �����ۥǥ��쥯�ȥ�Τʤ��� {\tt
plugin/cmo-gmp.c}) CMO\_ZZ �ϼ��η�����Ȥ�.
*/
/*&eg
We describe the bignum (multi-precision integer) representation in OpenXM.
In OpenXM {\tt CMO\_ZZ} is used to represent bignum. Its design is similar
to that in GNU MP. (cf. {\tt plugin/cmo-gmp.c} in the {\tt kan/sm1} 
distribution). CMO\_ZZ is defined as follows.
*/

\begin{tabular}{|c|c|c|c|c|}
\hline
{\tt int32 CMO\_ZZ} & {\tt int32 $f$} & {\tt int32 $b_0$} & $\cdots$ &
{\tt int32 $b_{n}$} \\
\hline
\end{tabular}

/*&jp
$f$ ��32bit�����Ǥ���.  $b_0, \ldots, b_n$ �� unsigned int32 �Ǥ���.
$|f|$ �� $n+1$ �Ǥ���.  ���� CMO ������ $f$ ����������.  ���Ҥ�
���褦��, 32bit����������� 2 �����ɽ����ɽ�����.

Open xxx ����ǤϾ�� CMO �ϰʲ����������̣����. ($R = 2^{32}$)
*/
/*&eg
$f$ is a 32bit integer. $b_0, \ldots, b_n$ are unsigned 32bit integers.
$|f|$ is equal to $n+1$. 
The sign of $f$ represents that of the above CMO. As stated in Section
\ref{sec:basic0}, a negative 32bit integer is represented by
two's complement.

In OpenXM the above CMO represents the following integer. ($R = 2^{32}$.)
*/

\[
\mbox{sgn}(f)\times (b_0 R^{0}+ b_1 R^{1} + \cdots + b_{n-1}R^{n-1} + b_n R^n).
\]

/*&jp
{\tt int32} �� network byte order ��ɽ��
���Ƥ���Ȥ����,�㤨��, ���� $14$ �� CMO\_ZZ ��ɽ�魯��,
*/
/*&eg
If we express {\tt int32} by the network byte order,
a CMO\_ZZ $14$ is expressed by
*/
\[
\mbox{(CMO\_ZZ, 1, 0, 0, 0, e)},
\]
//&jp ��ɽ�魯. ����ϥХ�����Ǥ�
//&eg The corresponding byte sequence is
\[
\mbox{\tt 00 00 00 14 00 00 00 01 00 00 00 0e}
\]
//&jp �Ȥʤ�.


//&jp �ʤ� ZZ �� 0 ( (ZZ) 0 �Ƚ� ) ��, {\tt (CMO\_ZZ, 00,00,00,00)} ��ɽ������.
//&eg Note that CMO\_ZZ 0 is expressed by {\tt (CMO\_ZZ, 00,00,00,00)}.


//&jp \subsection{ ʬ��ɽ��¿�༰ Dpolynomial }
//&eg \subsection{ Distributed polynomial Dpolynomial }

/*&jp
�ĤȤ����°����¿�༰�ϼ��Τ褦�ʹͤ������Ǥ��Ĥ���.

Generic DMS ring ��°���븵��,
�ѿ��� $n$ �Ļ��� Ŭ���ʷ������� $K$ �����¿�༰�� $K[x_1, \ldots, x_n]$ 
�θ��Ǥ���.
�������� $K$ ���ʤˤ���, �ºݥǡ������ɤ߹���, Coefficient �򸫤��ʳ���
�狼��.
���δĤ�°����¿�༰�� CMO �����Ǥ����Ȥä����, �ƥ����ФϤ���
�����Ф��б����� Object  ���Ѵ����ʤ��Ȥ����ʤ�. 
�����Ѵ��λ�����, �ƥ�������ˤ����.

Asir �ξ���, $K[x_1, \ldots, x_n]$ �θ���ʬ��ɽ��¿�༰���Ѵ������.
\noroa{ �Ǥ�, order �Ϥɤ��ʤ��? }

{\tt kan/sm1} �ξ��ϻ����ʣ���Ǥ���.
{\tt kan/sm1} ��, Generic DMS ring �ˤ����� ���饹��⤿�ʤ�.
�Ĥޤ�, Default ��¸�ߤ���, $n$ �ѿ���ʬ��ɽ��¿�༰�Ĥ�¸�ߤ��ʤ��櫓�Ǥ���.
�������ä�, {\tt kan/sm1} �Ǥ�, DMS of N variables ���褿���,
����� CurrentRing �θ��Ȥ����ɤ߹���.  CurrentRing ���ѿ��ο��� $n'$
��, $n' < n$ ���ȿ�����¿�༰�Ĥ��������ƥǡ������ɤ߹���.
Order ����¾�� optional ����Ϥ��٤�̵�뤹��.

DMS �� 2 ���ܤΥե�����ɤ�,
Ring by Name ���Ѥ������, ���ߤ�̾�����֤��ѿ� yyy �˳�Ǽ���줿 ring object
�θ��Ȥ���, ����¿�༰���Ѵ����ʤ����Ȥ�����̣�ˤʤ�.
{\tt kan/sm1} �ξ��, �Ĥ������ ring object �Ȥ��Ƴ�Ǽ����Ƥ���,
���� ring object �� �ѿ� yyy �ǻ��Ȥ��뤳�Ȥˤ�� CMO �Ȥ��Ƥ����Ȥä�
¿�༰�򤳤� ring �θ��Ȥ��Ƴ�Ǽ�Ǥ���.
*/

/*&eg
We treat polynomial rings and their elements as follows.

Generic DMS ring is an $n$-variate polynomial ring $K[x_1, \ldots, x_n]$,
where $K$ is some coefficient set. $K$ is unknown in advance
and it is determined when coefficients of an element are received.
When a server has received an element in Generic DMS ring,
the server has to translate it into the corresponding local object
on the server. Each server has its own translation scheme.
In Asir such an element are translated into a distributed polynomial.
In {\tt kan/sm1} things are complicated.
{\tt kan/sm1} does not have any class corresponding to Generic DMS ring.
{\tt kan/sm1} translates a DMS of N variables into an element of
the CurrentRing. 
If the CurrentRing is $n'$-variate and $n' < n$, then
an $n$-variate polynomial ring is newly created. Optional informations such as
the term order are all ignored.

If RingByName ({\tt CMO\_RING\_BY\_NAME}, yyy) 
is specified as the second field of DMS,
it requests a sever to use a ring object whose name is yyy
as the destination ring for the translation.
This is done in {\tt kan/sm1}.
*/

\medbreak \noindent
//&jp {\bf Example}: (���٤Ƥο���ɽ���� 16 ��ɽ��)
//&eg {\bf Example}: (all numbers are represented in hexadecimal notation)
{\footnotesize \begin{verbatim}
Z/11Z [6 variables]
(kxx/cmotest.sm1) run
[(x,y) ring_of_polynomials ( ) elimination_order 11 ] define_ring ;
(3x^2 y). cmo /ff set ;
[(cmoLispLike) 1] extension ;
ff ::
Class.CMO CMO StandardEncoding: size = 52, size/sizeof(int) = 13, 
tag=CMO_DISTRIBUTED_POLYNOMIAL 

  0  0  0 1f  0  0  0  1  0  0  0 18  0  0  0 13  0  0  0  6
  0  0  0  0  0  0  0  2  0  0  0  0  0  0  0  0  0  0  0  1
  0  0  0  0  0  0  0  2  0  0  0  3

ff omc ::
 (CMO_DISTRIBUTED_POLYNOMIAL[1f],[size=]1,(CMO_DMS_GENERIC[18],),
  (CMO_MONOMIAL32[13],3*x^2*y),),
\end{verbatim} }
/*&jp
$ 3 x^2 y$ �� 6 �ѿ���¿�༰�Ĥ� ���Ȥ��Ƥߤʤ���Ƥ���.
*/
/*&eg
$3 x^2 y$ is regarded as an element of a six-variate polynomial ring.
*/


//&jp \subsection{�Ƶ�ɽ��¿�༰�����} 
//&eg \subsection{Recursive polynomials} 

\begin{verbatim}
#define CMO_RECURSIVE_POLYNOMIAL        27
#define CMO_POLYNOMIAL_IN_ONE_VARIABLE  33
\end{verbatim}

Group CMObject/RecursivePolynomial requires CMObject/Primitive, CMObject/Basic.\\
Polynomial in 1 variable, Coefficient, Name of the main variable,
Recursive Polynomial, Ring definition for recursive polynomials
$\in$ CMObject/RecursivePolynomial \\

/*&jp
\begin{eqnarray*}
\mbox{Polynomial in 1 variable} &:& 
\mbox{({\tt CMO\_POLYNOMIAL\_IN\_ONE\_VARIABLE},\, {\sl int32}\, m, } \\
& & \quad \mbox{ Name of the main variable }, \\
& & \quad \mbox{ \{ {\sl int32} e, Coefficient \}} \\
& & \mbox{ --- m �ϥ�Υߥ���θĿ�. } \\
& & \mbox{ --- e, Coefficieint �ϥ�Υߥ����ɽ�����Ƥ���. } \\
& & \mbox{ --- ����ι⤤��ˤʤ�٤�. ���̤϶Ҥι⤤��.} \\
& & \mbox{ ---  e �� 1�ѿ�¿�༰�ζҤ򤢤�魯. } \\
\mbox{Coefficient} &:& \mbox{ ZZ} \,|\, \mbox{ QQ } \,|\, 
\mbox{ integer32  } \,|\,
\mbox{ Polynomial in 1 variable } \\
& & \quad \,|\, \mbox{Tree} \,|\, \mbox{Zero} \,|\,\mbox{Dpolynomial}\\
\mbox{Name of the main variable } &:& 
\mbox{ {\sl int32} v }   \\
& & \mbox{ --- v �� �ѿ��ֹ� (0 ����Ϥ��ޤ�) ��ɽ��. } \\
\mbox{Recursive Polynomial} &:& 
\mbox{ ( {\tt CMO\_RECURSIVE\_POLYNOMIAL}, } \\
& & \quad \mbox{ RringDefinition, } \\
& & \quad
\mbox{ Polynomial in 1 variable}\, | \, \mbox{Coefficient}   \\
\mbox{RringDefinition} 
& : &  \mbox{ {\sl List} v } \\
& & \quad \mbox{ --- v ��, �ѿ�̾(indeterminate) �Υꥹ��. } \\
& & \quad \mbox{ --- ����ι⤤��. } \\
\end{eqnarray*}
*/
/*&eg
\begin{eqnarray*}
\mbox{Polynomial in 1 variable} &:& 
\mbox{({\tt CMO\_POLYNOMIAL\_IN\_ONE\_VARIABLE},\, {\sl int32}\, m, } \\
& & \quad \mbox{ Name of the main variable }, \\
& & \quad \mbox{ \{ {\sl int32} e, Coefficient \}} \\
& & \mbox{ --- m is the number of monomials. } \\
& & \mbox{ --- A pair of e and Coefficient represents a monomial. } \\
& & \mbox{ --- The pairs of e and Coefficient are sorted in the } \\
& & \mbox{ \quad decreasing order, usually with respect to e.} \\
& & \mbox{ ---  e denotes an exponent of a monomial with respect to } \\
& & \mbox{ \quad the main variable. } \\
\mbox{Coefficient} &:& \mbox{ ZZ} \,|\, \mbox{ QQ } \,|\, 
\mbox{ integer32  } \,|\,
\mbox{ Polynomial in 1 variable } \\
& & \quad \,|\, \mbox{Tree} \,|\, \mbox{Zero} \,|\,\mbox{Dpolynomial}\\
\mbox{Name of the main variable } &:& 
\mbox{ {\sl int32} v }   \\
& & \mbox{ --- v denotes a variable number. } \\
\mbox{Recursive Polynomial} &:& 
\mbox{ ( {\tt CMO\_RECURSIVE\_POLYNOMIAL}, } \\
& & \quad \mbox{ RringDefinition, } \\
& & \quad
\mbox{ Polynomial in 1 variable}\, | \, \mbox{Coefficient}   \\
\mbox{RringDefinition} 
& : &  \mbox{ {\sl List} v } \\
& & \quad \mbox{ --- v is a list of names of indeterminates. } \\
& & \quad \mbox{ --- It is sorted in the decreasing order. } \\
\end{eqnarray*}
*/
\bigbreak
\noindent
Example:
\begin{verbatim}
(CMO_RECURSIEVE_POLYNOMIAL, ("x","y"),
(CMO_POLYNOMIAL_IN_ONE_VARIABLE, 2,      0,  <--- "x"
  3, (CMO_POLYNOMIAL_IN_ONE_VARIABLE, 2, 1,  <--- "y"
       5, 1234,
       0, 17),
  1, (CMO_POLYNOMIAL_IN_ONE_VARIABLE, 2, 1,  <--- "y"
       10, 1,
       5, 31)))
\end{verbatim}
//&jp �����,
//&eg This represents
$$   x^3 (1234 y^5 + 17 ) +  x^1 (y^{10} + 31 y^5)  $$
/*&jp
�򤢤�魯.
��Ĵ�¿�༰�⤳�η����Ǥ���路�����Τ�, �Ѥν�����Τ褦��
���뤳��. �Ĥޤ�, ���ѿ������뷸���ν���.
*/
/*&eg
We intend to represent non-commutative polynomials with the
same form. In such a case, the order of products are defined
as above, that is a power of the main variable $\times$ a coeffcient.
*/

\noindent
\begin{verbatim}
sm1
sm1>(x^2-h). [(class) (recursivePolynomial)] dc /ff set ;
sm1>ff ::
Class.recursivePolynomial h * ((-1)) + (x^2  * (1))
\end{verbatim}

//&jp \subsection{CPU��¸�� double } 
//&eg \subsection{CPU dependent double} 

\begin{verbatim}
#define CMO_64BIT_MACHINE_DOUBLE   40
#define CMO_ARRAY_OF_64BIT_MACHINE_DOUBLE  41
#define CMO_128BIT_MACHINE_DOUBLE   42
#define CMO_ARRAY_OF_128BIT_MACHINE_DOUBLE  43
\end{verbatim}

\noindent
Group CMObject/MachineDouble requires CMObject/Primitive.\\
64bit machine double, Array of 64bit machine double
128bit machine double, Array of 128bit machine double
$\in$ CMObject/MachineDouble \\

/*&jp
\begin{eqnarray*}
\mbox{64bit machine double} &:& 
\mbox{({\tt CMO\_64BIT\_MACHINE\_DOUBLE}, } \\
& & \quad \mbox{ {\sl byte} s1 , \ldots , {\sl byte}} s8)\\
& & \mbox{ --- s1, $\ldots$, s8 �� {\tt double} (64bit). } \\
& & \mbox{ --- ����ɽ����CPU��¸�Ǥ���.}\\
&&  \mbox{\quad\quad mathcap �� CPU ������ղä��Ƥ���.} \\
\mbox{Array of 64bit machine double} &:& 
\mbox{({\tt CMO\_ARRAY\_OF\_64BIT\_MACHINE\_DOUBLE}, {\sl int32} m, } \\
& & \quad \mbox{ {\sl byte} s1[1] , \ldots , {\sl byte}}\, s8[1], \ldots , {\sl byte}\, s8[m])\\
& & \mbox{ --- s*[1], $\ldots$ s*[m] �� m �Ĥ� double (64bit) �Ǥ���. } \\
& & \mbox{ --- ����ɽ����CPU��¸�Ǥ���.}\\
& & \mbox{ \quad\quad mathcap �� CPU ������ղä��Ƥ���.} \\
\mbox{128bit machine double} &:& 
\mbox{({\tt CMO\_128BIT\_MACHINE\_DOUBLE}, } \\
& & \quad \mbox{ {\sl byte} s1 , \ldots , {\sl byte}} s16)\\
& & \mbox{ --- s1, $\ldots$, s16 �� {\tt long double} (128bit). } \\
& & \mbox{ --- ����ɽ����CPU��¸�Ǥ���.}\\
&&  \mbox{\quad\quad mathcap �� CPU ������ղä��Ƥ���.} \\
\mbox{Array of 128bit machine double} &:& 
\mbox{({\tt CMO\_ARRAY\_OF\_128BIT\_MACHINE\_DOUBLE}, {\sl int32} m, } \\
& & \quad \mbox{ {\sl byte} s1[1] , \ldots , {\sl byte}} s16[1], \ldots , {\sl byte} s16[m])\\
& & \mbox{ --- s*[1], $\ldots$ s*[m] �� m �Ĥ� long double (128bit) �Ǥ���. } \\
& & \mbox{ --- ����ɽ����CPU��¸�Ǥ���.}\\
& & \mbox{ \quad\quad mathcap �� CPU ������ղä��Ƥ���.} 
\end{eqnarray*}
*/
/*&eg
\begin{eqnarray*}
\mbox{64bit machine double} &:& 
\mbox{({\tt CMO\_64BIT\_MACHINE\_DOUBLE}, } \\
& & \quad \mbox{ {\sl byte} s1 , \ldots , {\sl byte}} s8)\\
& & \mbox{ --- s1, $\ldots$, s8 �� {\tt double} (64bit). } \\
& & \mbox{ --- This depends on CPU.}\\
&&  \mbox{\quad\quad Add informations on CPU to the mathcap.} \\
\mbox{Array of 64bit machine double} &:& 
\mbox{({\tt CMO\_ARRAY\_OF\_64BIT\_MACHINE\_DOUBLE}, {\sl int32} m, } \\
& & \quad \mbox{ {\sl byte} s1[1] , \ldots , {\sl byte}}\, s8[1], \ldots , {\sl byte}\, s8[m])\\
& & \mbox{ --- s*[1], $\ldots$ s*[m] are 64bit double's. } \\
& & \mbox{ --- This depends on CPU.}\\
& & \mbox{\quad\quad Add informations on CPU to the mathcap.} \\
\mbox{128bit machine double} &:& 
\mbox{({\tt CMO\_128BIT\_MACHINE\_DOUBLE}, } \\
& & \quad \mbox{ {\sl byte} s1 , \ldots , {\sl byte}} s16)\\
& & \mbox{ --- s1, $\ldots$, s16 �� {\tt long double} (128bit). } \\
& & \mbox{ --- This depends on CPU.}\\
& & \mbox{\quad\quad Add informations on CPU to the mathcap.} \\
\mbox{Array of 128bit machine double} &:& 
\mbox{({\tt CMO\_ARRAY\_OF\_128BIT\_MACHINE\_DOUBLE}, {\sl int32} m, } \\
& & \quad \mbox{ {\sl byte} s1[1] , \ldots , {\sl byte}} s16[1], \ldots , {\sl byte} s16[m])\\
& & \mbox{ --- s*[1], $\ldots$ s*[m] are 128bit long double's. } \\
& & \mbox{ --- This depends on CPU.}\\
& & \mbox{\quad\quad Add informations on CPU to the mathcap.} \\
\end{eqnarray*}
*/

\bigbreak
//&jp ���� IEEE ���� float ����� Big float ��������褦.
//&eg We define float and big float conforming to the IEEE standard.
\begin{verbatim}
#define CMO_BIGFLOAT   50
#define CMO_IEEE_DOUBLE_FLOAT 51
\end{verbatim}

/*&jp
IEEE ���� float �ˤĤ��Ƥ�, IEEE 754 double precision floating-point
format (64 bit) ������򸫤�.
*/
/*&eg
See IEEE 754 double precision floating-point (64 bit) for the details of 
float conforming to the IEEE standard.
*/

\noindent
Group CMObject/Bigfloat requires CMObject/Primitive, CMObject/Basic.\\
Bigfloat
$\in$ CMObject/Bigfloat \\

\begin{eqnarray*}
\mbox{Bigfloat} &:& 
\mbox{({\tt CMO\_BIGFLOAT}, } \\
& & \quad \mbox{ {\sl ZZ} a , {\sl ZZ} e})\\
& & \mbox{ --- $a \times 2^e$. } \\
\end{eqnarray*}

%% $OpenXM$
//&C \section{ OX Local Data }

/*&jp
�����ƥ��ͭ�� Object ������ɽ����,
\begin{verbatim}
#define   OX_LOCAL_OBJECT       0x7fcdef30
\end{verbatim}
��������Ƥ������ǥ����Ť�����.
0x200000 ���ΰ����դ뤳�ȤȤ���.
�����ϥ����ƥऴ�Ȥ˳�꿶��.
\begin{verbatim}
#define OX_LOCAL_OBJECT_ASIR   (OX_LOCAL_OBJECT+0)
#define OX_LOCAL_OBJECT_SM1    (OX_LOCAL_OBJECT+1)
\end{verbatim}
������, {\tt OX\_DATA} ������Ȥ��ƻ��ѤǤ���.
*/

%% $OpenXM$
//&C \section{ CMO ERROR2 }

/*&jp
���顼�ֹ� (�����ֹ�)
\begin{verbatim}
#define  Broken_cmo    1
#define  mathcap_violation  2
\end{verbatim}
*/


%% $OpenXM: OpenXM/doc/OpenXM-specs/redefine.tex,v 1.1.1.1 2000/01/20 08:52:46 noro Exp $
//&jp \section{メンバの再定義に関する注意}

/*&jp
われわれは,
Coefficient を Group DistributedPolynomials の メンバとして定義した.
これを新しい group のメンバーとして定義を拡張することができる.
たとえば, Coefficient を, 再帰表現多項式のメンバとしても定義できる.
この場合副作用として, DistributedPolynomials のメンバとしての Coeffients
の意味も拡張される.
システムがそのあたらしい DistributedPolynomials をすくなくとも CMO としては
理解できないといけないが, それを実際にそのシステムのローカルな Object に
変換できるかはシステム依存である.
このような再定義をゆるすのは, 一般的な実装法がやはりこのような仕組みだから
である.
mathcap にデータ転送の保護機能はこのような再帰的表現もチェックできないと
いけない.
*/

%% $OpenXM$
//&jp \section{ ����, �ǥХå�, ���� }

//&jp \subsection{ ������ }

/*&jp
���ե� xxx ��, open XM �б��ˤ���Τˤϰʲ��Τ褦��
����դ�ȳ�ȯ���ưפǤ�����.
\begin{enumerate}
\item[Step 1.]
{\tt executeStringByLocalParser}
����� {\tt popString} �ε�ǽ��¸�����,
xxx ��饤�֥��Ȥ��ƤޤȤ�, ¾�Υ��եȤȥ�󥯤���
���ѤǤ��뤫�ƥ��Ȥ���.
C �Ǥμ¸��ξ�� �����ߤμ谷�����դ��פ���.
��������μ¸��Ǥ�, ����ץ륵����
({\tt nullserver00.c}) �ȥ�󥯤����,
open XM �б��Υ��饤����Ȥ��̿��Ǥ���.
���饤�����¦�Ǥ�, ���Υ����ƥ���б�������ǽ�ƤӽФ�
�ץ��������.
\item[Step 2.]
����, CMO ���ǽ�ʸ¤��������.
open sm1 �� open asir ��, CMO �����Υǡ�����
��������, �ɤ߹���뤫�ƥ��Ȥ���.
\item[Step 2'.]
{\tt kan/sm1} �� plugin �Ȥ����Ȥ߹���ȥ����Фγ�ȯ���ڤ��⤷��ʤ�.
{\tt kan/sm1} �Υ������ե�����Υǥ��쥯�ȥ� {\tt plugin} �򸫤�.
\item[Step 3.]
CMO �� stream �ؤ�ž��, stream ����ž����,
����ǡ��������������ˤ��������ڤǤ���.
��������������ץ륵���Фȥ�󥯤���.
\end{enumerate}


\subsection{���}
kan -- asir �֤Ǥ�ʾ�Τ褦�˳�ȯ���������.

Risa/Asir �γ�pȯ�����Ť��ٻΥե������Ǥ����ʤ��Ƥ���
����, �䤬���Ť�, 1996ǯ, 1��19����ˬ�䤷,
{\tt Asir\_executeString()}
�ε�ǽ����Ϥ����˽񤤤Ƥ��ä�, kan ��� asir ��ʸ����ǸƤӽФ�
��ǽ����Ӥ��εդ�¸������Τ����Ȥ�ȯü�Ǥ���. 
���Ȥ���, asir ��� kan/sm1 �ε�ǽ��ƤӽФ��ˤ�,
\begin{verbatim}
F = x; G = y;
Ans = kan("(%p).  (%p).  mul toString",F,G)
\end{verbatim}
�����Ϥ���Ф褤.
{\tt x} �� {\tt y} ���Ѥ� kan �Dz��¹Ԥ����, ��̤�
��ɤ�.
���Υ�٥�η��Ǥ� kan/sm1 ��, ��¢���󥿥ץ꥿�դ�
�饤�֥��Ȥ������ѤǤ��Ƥ����֤������Ǥ���.
���δؿ� {\tt kan} �� {\tt builtin/parif.c} ���Ȥ߹����.
{\tt asir} �� {\tt pari} �δؿ����Ȥ߹���Ǥ��뤬, �����Ȥ߹��ߤ�
�ᥫ�˥�������Ѥ�����ưפ˥�󥯤Ǥ���.
����: {\tt  noro/src/Old1/parif.c}.

����, CMO Basic0 �ε�ǽ��
1997, 5 ��, 6 ��˼¸�����.
���θ�, 1997ǯ 7 ���, SMObject/Basic1 �μ���,
1997ǯ 7 ��ˤ�, ��Ϥ�������ꥢ�� CoCoA �Υ������åפˤ�����,
�簤�פ� b-function �� stratification ���ߤǷ׻�����׻��ץ�������
asir, kan ��Ϣ�����ƥǥ⤷��. ���ΤȤ���, {\tt Sm1\_executeStringByLocalParser}
�ؿ����Ѥ���, �饤�֥��Ȥ��ƥ�󥯤��ƤĤ��ä�.
1997 ǯ 11 ��� TCP/IP �ˤ��, �����Х����å��ޥ���֤��̿���
�����򤪤��ʤäƤ���.
�̿��μ����ƥ��ȤΤ����, Java ����� C �� null server , null client
���������. �ʲ�, ����ˤĤ��Ҥ٤�.


\subsection{ ����ץ륵����, ���饤����� }

Open XM �Ǥ�, ���ߤΤȤ���,
����ץ륵���ФȤ���  {\tt oxserver00.c} ���󶡤��Ƥ���.
���Υ����Ф��Ȥˤ���, {\tt asir} ����� {\tt kan/sm1} 
�� open XM �����Ф���Ǥ��� ({\tt ox\_asir}, {\tt ox\_sm1}).
{\tt ox\_sm1} �Ǥ�, {\tt sm1stackmachine.c} ��
open XM �����å��ޥ����¸����Ƥ���.
����ץ륯�饤����Ȥ�, �ͥåȥ���˥ǡ��������Ф����
�������뵡ǽ�Τߤ���,  {\tt testclient.c} ����
���Ƥ���. 
{\tt asir} ����� {\tt kan/sm1} �ˤ��ܳ�Ū��
���饤����ȵ�ǽ(open XM �����Ф�ƤӽФ�
��ǽ)���Ȥ߹���Ǥ���.
�����Ф�ư����ץ�������, {\tt kan/sm1} ���롼�פǤ�,
{\tt ox} �ʤ�̾����, {\tt asir} ���롼�פǤ�,
{\tt ox\_lauch} �ʤ�̾���Ǥ��뤬, ��ǽ��Ʊ���Ǥ���.
{\tt ox} �Υ������� {\tt oxmain.c} �Ǥ���.

\subsubsection{OpenXM/src/ox\_toolkit �ˤ��륵��ץ����}
���Υǥ��쥯�ȥ�β���ʸ��򸫤�.

\subsubsection{ ox\_null }

Basic0 �Υ����å��ޥ���Υ����åȤˤ�����. 
�����å��ޥ���� {\tt nullstackmachine.c} �˼�������Ƥ���,
{\tt oxserver00.c} �˥�󥯤��ƥ����ФȤʤ�.
����ץ륵���ФǤ���, ����� CMO Basic0 ���ͤδؿ����礹���,
��� �����Ф�ư���Ϥ��Ǥ���.
�����å��ˤ�,CMO �� Basic0 �� object �ؤΥݥ��󥿤����Τޤ� push �����.
����ȥ����뵡ǽ�ʤ�. 1997/11/29 �Ǥ�ꥳ��ȥ����뵡ǽ�ɲ�.
@Old/nullserver00.c.19971122c,
@Old/mytcpip.c.19971122c

���ߤϤ��Υ����Фϥ��ƥʥ󥹤���Ƥ��ʤ�
(object �ط��δؿ����ɲä��ʤ��ȥ���ѥ���Ǥ��ʤ�.)

\subsubsection{ testclient }


Java �ˤ�����:
@Old/client.java.19971122c
����⸽�ߤϤդ뤤.
OX �ѥ��åȤΥ��ꥢ���ֹ���б����Ƥ��ʤ�.
�����������˲�������ͽ��.
{\tt executeString} ����� {\tt popString} ����������ǽ�Ϥ��������ʤ�.
������ ����å� {\tt listner} �������ʤäƤ���.
������ byte �ǡ�����ɽ������ΤߤǤ���.
����åɤ�ͥ���٤򤦤ޤ��Х�󥹤Ȥ�ʤ���, �����ǡ���������Τ�
ɽ�����ʤ��ä��ꤹ��.

C �ˤ�� {\tt testclient}
Ʊ���褦�ʵ�ǽ���ĥץ������μ����⤢��.
{\footnotesize \begin{verbatim}
./ox -ox ox_sm1 -host localhost -data 1300 -control 1200  (�����Ф�Ω���夲)
./testclient     (testclient ��Ω���夲)
\end{verbatim}}
����⸽�ߤϤդ뤤.

��¼ ({\tt tamura@math.kobe-u.ac.jp}) �ˤ�� Java �ؤο���������������
(1999, 3��, ������ؽ�����ʸ).

\subsubsection{ {\tt ox} }
{\tt ox} �� ox �����Ф򤿤������뤿��Υץ������Ǥ���.
���饤����Ȥ�ꥵ���Ф���³����ˤ���Ĥ���ˡ������.
��Ĥ� {\tt ox} �� �ǡ����ȥ���ȥ������Ѥ���Ĥ�
�ݡ����ֹ������Ū�˵�ư��, ���饤����Ȥ����Υݡ��ȤؤĤʤ���
������ˡ�Ǥ���.
�⤦��Ĥ�, ���饤�����¦�Ǥޤ�, �����Ƥ���ݡ��Ȥ����
������, ���줫�� {\tt ox} �� {\tt -reverse} �ǵ�ư����
������¦�����饤����ȤˤĤʤ��ˤ�����ˡ�Ǥ���.
���ξ��, {\tt ox} �Ϥ��Ȥ��м��Τ褦�˵�ư�����.
{\footnotesize \begin{verbatim}
/home/nobuki/kxx/ox -reverse -ox /home/nobuki/kxx/ox_sm1 
-data 1172 -control 1169 -pass 1045223078 
\end{verbatim} }

{\tt ox} ��, �Ҥɤ�ץ������Ȥ���, {\tt ox\_asir}, {\tt ox\_sm1}
�ʤɤ�ư����ΤǤ��뤬,
�����Υץ�������
3 ���OX �ǡ���, ���ޥ�ɤ��ɤ߹���, 4 �� OX �ǡ�����񤭽Ф�.
���ߤμ����Ǥ� 3 �� 4 �� dup ����Ʊ��뤷�Ƥ��ޤäƤ���.
{\tt ox} ��TCP/IP �Υǡ���ž���Υݡ��Ȥ�, 3, 4 �ؤ�ꤢ�Ƥ�,
�Ҥɤ�ץ�������ư����.
{\footnotesize \begin{verbatim}
close(fdControl);   /* close(0); dup(fdStream); */
dup2(fdStream,3);
dup2(fdStream,4);  
\end{verbatim}}


\subsubsection{ {\tt ox\_asir} phrase book}

[ ������ε��ҤϸŤ�]
CMObject �� asir �� object �ϼ��ε�§�ˤ������ä��Ѵ������.
�ʤ� asir �� object �Υ�����ߤ�ˤϴؿ� {\tt type} ���Ѥ���.
\begin{enumerate}
\item Null :  0 �Ȥ��ƻ��Ѥ����.
\item Integer32 : ����Ū�ˤΤ߻��Ѥ����. ���Ф�, ZZ ���Ѵ������.
  -1 �� (unsigned int) -1 ���Ѵ�����Ƥ���, ZZ ���Ѵ������Τ�,
  ���ο��Ȥʤ�.
\item Cstring : ʸ���� (type = 7) ���Ѵ������.
\item ZZ : �� (type = 1 ) ���Ѵ������.
\item QQ : �� (type = 1 ) ���Ѵ������.
\item List : �ꥹ�� (type = 4) ���Ѵ������.
\item Dpolynomial : ʬ��ɽ��¿�༰ (type = 9) ���Ѵ������.
order �Ϥ����Ȥä���Υߥ���Υꥹ�Ȥ�Ʊ�� order �Ǥ���.
\item RecursivePolynomial : �Ƶ�ɽ��¿�༰���Ѵ������.
��������˼�ư�Ѵ������.
\item Indeterminate : ���긵���Ѵ������.
\end{enumerate}
���ҤΤʤ� CMObject �˴ؤ��Ƥ�, ���ѤǤ��ʤ� (cf. mathcap ).

\noindent
������: 0 �ΰ����λ��ͤ��ޤ������ޤäƤ��ʤ�.
Null ���� (type = 1) �� 0 ���Ѵ�������Ǥ⤢��.

\medbreak
\noindent
��: 
ʬ��ɽ��¿�༰�� $x^2-1$ �ΰ���ʬ��� asir �ˤ�äƤ�餦 
OXexpression ����򤢤���.
{\footnotesize
\begin{verbatim}
(OX_DATA, (CMO_LIST, 4, CMO_DMS,CMO_DMS_GENERIC,
(CMO_MONOMIAL32,1,2,(CMO_ZZ,1)),
(CMO_MONOMIAL32,1,0,(CMO_ZZ,-1)))),
(OX_DATA, (CMO_INT32,1))
(OX_DATA, (CMO_STRING,"ox_dtop"))
(OX_COMMAND,(SM_executeString))

(OX_DATA, (CMO_INT32,1))
(OX_DATA, (CMO_STRING,"fctr"))
(OX_COMMAND,(SM_executeString))


(OX_DATA, (CMO_INT32,1))
(OX_DATA, (CMO_STRING,"ox_ptod"))
(OX_COMMAND,(SM_executeString))

(OX_COMMAND,(SM_popCMO))
\end{verbatim}}

������, ZZ �θ������̤�����ɽ���Ǥ���路��.
{\tt dtop1} ����� {\tt ptod1} �Ϥ��줾��, ʬ��ɽ��¿�༰��, Asir �κƵ�ɽ��
¿�༰��, �դ�, Asir �κƵ�ɽ��¿�༰��, ʬ��ɽ��¿�༰���Ѵ�����,
�桼������� 1 �����ؿ��Ǥ���. %% kxx/oxasir.sm1
�����δؿ��� Asir �� �ꥹ�Ȥˤ���Ѥ����뤳�Ȥ���ǽ�Ǥ���, ���ξ���
���ǤȤ��ƤǤƤ���,  ʬ��ɽ��¿�༰ �ޤ��� Asir �κƵ�ɽ��¿�༰
��ɬ�פʷ����Ѵ�����.
{\tt fctr} �ϰ���ʬ��򤹤��Ȥ߹��ߴؿ��Ǥ���.

{\tt kxx/oxasir.asir} ������.
{\footnotesize \begin{verbatim}
OxVlist = [x,y,z]$

def ox_ptod(F) {
extern OxVlist;
if (type(F) == 4) return(map(ox_ptod,F));
else if (type(F) == 2) return(dp_ptod(F,OxVlist));
else return(F);
}

def ox_dtop(F) {
extern OxVlist;
if (type(F) == 4) return(map(ox_dtop,F));
else if (type(F) == 9) return(dp_dtop(F,OxVlist));
else return(F);
}

end$
\end{verbatim}}

\subsubsection{ {\tt ox\_sm1} phrase book }

[ ������ε��ҤϸŤ�]
CMObject �� kan/sm1 �� object �ϼ��ε�§�ˤ��������Ѵ������.
�ʤ�, kan/sm1 �� object �Υ�����ߤ�ˤ�, {\tt tag} �ޤ��� {\tt etag}
���Ѥ���.
\begin{enumerate}
%% \item Error : Error (etag = 257) ���Ѵ������.
\item Error2 : Error (etag = 257) ���Ѵ������.
\item Null : null (tag = 0) ���Ѵ������.
\item Integer32 : integer (tag = 1) ���Ѵ������.
\item Cstring : ʸ���� (type = 5) ���Ѵ������.
\item ZZ : universalNumber (type = 15 ) ���Ѵ������.
\item QQ : rational (tag = 16 ) ���Ѵ������.
\item List : array (tag = 6) ���Ѵ������.
\item Dpolynomial : ¿�༰ (tag = 9) ���Ѵ������.
\end{enumerate}


\noroa{ {\tt SS475/memo1.txt} �⸫��.}

����: {\tt ReverseOutputOrder = 1} (ɸ��)
�ΤȤ�, {\tt xn, ..., x0, dn, ..., d0} �ν��֤�
({\tt show\_ring} �η���) Dpolynomial ���Ѵ������
(������������,
{\tt xn} �� {\tt e}, {\tt d0} �� {\tt h},
{\tt x0} �� {\tt E}, {\tt dn} �� {\tt H}).
���Ȥ���,
{\tt ox\_send\_cmo(id,<<1,0,0,0,0,0>>)}  ��,
{\tt x2} ���Ѵ������.
{\tt ox\_send\_cmo(id,<<0,0,1,0,0,0>>)}  ��,
{\tt x0} ���Ѵ������.

{\tt OxVersion} �ѿ��� openXM �Υץ��ȥ���� version ��ɽ��.

\subsubsection{ {\tt ox\_sm1} ���Ѥ������饤����ȤΥƥ�����ˡ }
�ޤ������Ƥʤ�.

\subsubsection{ {\tt Asir} ���Ѥ��������ФΥƥ�����ˡ }

\subsection{ �Ǿ��� TCP/IP ���饤����Ȥ��� }

Java �ޤ��� M2 �ˤ�륽���������ɤ�Ǻܤ�ͽ��.

\subsection{ ���饤����� asir, sm1 }

sm1 �ˤĤ��Ƥ�, ox.sm1 , oxasir.sm1 �����饤����ȥѥå�����.
{\tt ox}, {\tt ox\_asir}, {\tt ox\_sm1} ��¸�ߤ���ѥ�,
����� sm1 ���ƤӽФ������ asir �δؿ�����Ǥ���
{\tt oxasir.asir} �Τ���ѥ���
�����Υѥå������˽񤭹���Ǥ���ɬ�פ�����.

{\tt ox\_asir} ��, {\tt asir} �ʤ�̾���Ǥ�Ф���
asir �Ȥ���ư�, {\tt ox\_asir} �ʤ�̾���Ǥ�Ф���,
open XM �����ФȤ���ư���.
{\tt /usr/local/lib/asir} �ޤ���
{\tt ASIR\_LIBDIR} �� {\tt ox\_asir} ���Τ򤪤�,
{\tt ox\_launch} �򤪤ʤ��ǥ��쥯�ȥ�� {\tt ox\_asir} �ؤΥ���ܥ�å����
�Ȥ��ƺ�������.
���ޥ�ɥ������ѥ��ˤ���ǥ��쥯�ȥ�� {\tt asir} �� {\tt ox\_asir}
�ؤΥ���ܥ�å���󥯤Ȥ��ƺ�������.
{\footnotesize
\begin{verbatim}

This is Asir, Version 990413.
Copyright (C) FUJITSU LABORATORIES LIMITED.
3 March 1994. All rights reserved.
0
[324] ox_launch(0,"/usr/local/lib/asir/ox_asir");
1      <=== ���줬�����Ф��ֹ�.
[326] ox_execute_string(1,"fctr(x^10-1);");
0
[327] ox_pop_local(1);
[[1,1],[x-1,1],[x+1,1],[x^4+x^3+x^2+x+1,1],[x^4-x^3+x^2-x+1,1]]
[328] ox_execute_string(1,"dp_ptod((x+y)^5,[x,y]);");
0
[329] ox_pop_cmo(1);
(1)*<<5,0>>+(5)*<<4,1>>+(10)*<<3,2>>+(10)*<<2,3>>+(5)*<<1,4>>+(1)*<<0,5>>
[330] ox_rpc(1,"fctr",x^10-y^10);
0
[331] ox_pop_local(1);
[[1,1],[x^4-y*x^3+y^2*x^2-y^3*x+y^4,1],[x^4+y*x^3+y^2*x^2+y^3*x+y^4,1],[x+y,1],[x-y,1]]
[332] ox_rpc(1,"fctr",x^1000-y^1000);   ox_cmo_rpc �⤢��.
0
[333] ox_flush(1);
1
[334] ox_pop_local(1);
0

ox_sync(1);   --- sync ball ������.

\end{verbatim}}

\subsection{��ȯ��Υ�����, ����󥢥��}

Mathematica ������, ���饤����� : ����.
Java ���饤�����, Open Math proxy : ��¼.
Gnuplot ������, Macaulay 2 ���饤�����, ������,
����¾���������ե� (Toric GB, Grobner fan)�Υ�����,
Algebraic equation solver by numerical method: �⻳.
Open Asir �ޥ˥奢��, ����ץ�: ����, �⻳.

����Ū�ˤ��⤷���������ºݤˤ��Ĥ��äƤߤʤ���
�狼��ʤ��������⤪��������Ȼפ�.
����, ${\cal A}$-Ķ�����ؿ��β�Υ����ɽ��,
�ѥ�᡼������ʬ�Υ����ɽ���Υ��եȤ򤫤����Ȥ�
������������򤵤��뤳�Ȥ�
�ײ褷�Ƥ���.
����֥�deformation �ˤ��¿�༰��, ͭ�����Ƴ��(���������������������,
��������ʬ�����������ȯ���������������������ˤʤ�)
�� OpenXM Ū�ʥץ������Τ��⤷������������.
Java ������׻�����ǽ�Ϥ�Ĥ��ä� ox �����Ф�Ȥ��Τ⤪�⤷����.

������οͤˤĤ��äƤ�館�뵬�ʤˤ���ˤ�
�ޤ��ޤ��¸��ȷи��򤫤��ͤʤ��Ȥ����ʤ�.
Free Mathematical Software Initiative ����٤�������.

\subsection{ Change log }
\begin{enumerate}
\item 1997/11/20 : ���� document �κǽ�� version �����ޤ줿.
  kxx/openxxx.tex �ʤ�̾���Ǥ��ä�.
\item 1999/07 : {\tt CMO\_ZZ} �η������Ѥ���.
\item 1999/09/7, 9/8 : ʬ��ɽ��¿�༰, Mathcap, RecursivePolynomial,
�η������Ѥ���. asir, sm1 �˼�������. ���顼�����Τ����,
dupErrors, getsp �� SM ���ޥ�ɤ˲ä���.
\end{enumerate}

\subsection{ }
Java �� sm1 �����Ф򤹤٤ƽ񤤤Ƥߤ뤫?
*/

%% $OpenXM$
//&jp \section{ �ɤΤ褦�˶�Ĵ��ǽ�ˤ����줿 Math Soft ��񤯤Τ�? }

/*&jp
�����Ĥ��ηи�Ū�ʥץ�������ˡ����Ȥ��ƤޤȤ�Ƥ���.
\begin{enumerate}
\item ʸ������ɤ߹���, �׻�����, �ޤ�ʸ������̤Ȥ��ƽ��Ϥ���
�ץ�����प��Ӵؿ��϶�Ĵ��ǽ�ˤ����줿�ץ������δ��ܤǤ���.
����ե����륤�󥿡��ե������ΤߤΥץ�������¾�Υ��եȤȶ�Ĵ
������Τ��ऺ������.
\item Output ���ɤ�ΤϿʹ֤Ȥϸ¤�ʤ�. 
Output ���ɼԤ�, ¾�Υ��եȤ��⤷��ʤ�.
�������ä�, Output ���Ϥ�, ʸˡ�ˤ������äƽ��Ϥ��٤��Ǥ���.
���顼���Ϥ�ʸˡ�ˤ��������٤��Ǥ���.
\end{enumerate}

\subsection{open gnuplot ���}
�񤤤Ƥʤ�.

\subsection{open phc ���}
�񤤤Ƥʤ�.

\subsection{open M2 ���}
�񤤤Ƥʤ�.

���������­��,
{\tt this/open/ohp} �򸫤� (Open XM day, 1999, 7/12 Mon �θ���)
*/

%% $OpenXM: OpenXM/doc/OpenXM-specs/cmo-register.tex,v 1.1.1.1 2000/01/20 08:52:46 noro Exp $
//&jp \section{ ������ CMO ����Ͽ }
//&eg \section{ Registering a new CMO }

//&jp \subsection{������ CMO ����Ȥ�����«}
//&eg \subsection{Requirement for a new CMO}

/*&jp
CMO �Ǥ�, �Ƶ�Ū�ʥǡ���ɽ�����뤷�Ƥ���Τ�,
Object �����䤷�����, ��Ȥ��餢�� CMO ��ư����ĥ�����.
���Υ����ƥब������ư���Τ���ǧ����ɬ�פ�����.
�׻����뤳�ȤϤǤ��ʤ��Ƥ�, �̿������Ǥʤ���,
�ǡ����μ��Ϥ���Ǥ��ʤ��Ȥ����ʤ�.
*/
/*&eg
It is allowed to define CMO data types recursively.
Thus, if one introduces a new CMO, then one has to modify
existing implementations which may handle the new CMO.
Note that the new CMO should be at least sent or received without
error even if it cannot be used for computation.
*/

//&jp \subsection{OpenXM �ץ��������Ȥ˻��ä���ˤ�?}
//&eg \subsection{How to join OpenXM project}

/*&jp
OpenXM �˥ѥå�������ä��Ƥ����Τ��紿�ޤǤ�.
takayama@math.kobe-u.ac.jp��Ϣ��������.
������ CMO ��ɬ�פʤ�ä��Ʋ�����.
�������
\begin{enumerate}
\item ������� CMObject �η���Ū�����������.
\item �����ƥ� xxx ��, ���� CMObject �˴ؤ��ƤɤΤ褦�˿��񤦤�������.
\item ���� CMObject, �����ƥ� xxx �˴ؤ��� URL.
\end{enumerate}
�򤪤��äƲ�����.
ɬ�פʤ�ǥ������å����򤪤��ʤ�, �����Ǥλ��ͤ������, ���ꤷ���ʳ���
CMO �Υ�����ȯ�Ԥ�, ���� CMObject �˴�Ϣ���� URL ��
OpenXM �ۡ���ڡ��� \cite{openxxx} ����󥯤��ޤ�.
*/

/*&eg
You are welcome to add packages to OpenXM.
Ask takayama@math.kobe-u.ac.jp for details.
You may introduce new CMO's if necessary.
If you have defined a new CMO, send 
\begin{enumerate}
\item the formal definition and an explanation of the CMO,
\item an explanation of the behavior of a system xxx for the CMO,
\item URL's related to the CMO or xxx.
\end{enumerate}
After discussing on the new CMO, we will fix the specification.
Then we will issue the tag for the new CMO and create links to
the URL related to the CMO from the OpenXM home page \cite{openxxx}.
*/

%%% $OpenXM$
//&C \section{Appendix: English translation}

/*&C
\noindent
(This section has not been updated.)

\subsection{Common Mathematical Object format}

\begin{verbatim}
#define LARGEID  0x7f000000
#define CMO_ERROR2 (LARGEID+2)
#define CMO_NULL   1
#define CMO_INT32  2
#define CMO_DATUM  3
#define CMO_STRING 4
#define CMO_LIST 17
\end{verbatim}

\bigbreak
\noindent
Group CMObject/Basic0  requires nothing. \\
Error2, Null, Integer32, Datum, Cstring, List $\in$ CMObject/Basic0. \\
Document of CMObject/Basic0 is at {\tt http://www.math.kobe-u.ac.jp/openxxx}
(in Japanese) \\
\begin{eqnarray*}
\mbox{Error2}&:& ({\tt CMO\_ERROR2}, {\sl CMObject}\, \mbox{ob}) \\
\mbox{Null}  &:& ({\tt CMO\_NULL}) \\
\mbox{Integer32}
             &:& ({\tt CMO\_INT32}, {\sl int32}\ \mbox{n}) \\
\mbox{Datum} &:& ({\tt CMO\_DATUM}, {\sl int32}\, \mbox{n}, {\sl byte}\, 
                 \mbox{data[0]}, 
                  \ldots , {\sl byte}\, \mbox{data[n-1]}) \\
\mbox{Cstring}&:& ({\tt CMO\_STRING},{\sl int32}\,  \mbox{n}, 
                   {\sl string}\, \mbox{s}) \\
\mbox{List} &:& 
\mbox{({\tt CMO\_LIST}, {\sl int32}\, m, {\sl CMObject}\, ob[1], $\ldots$,
                                       {\sl CMObject}\, ob[m])} \\
             & & \mbox{---  m is the length of the list.} \\
\end{eqnarray*}

In the definition of ``Cstring'', if we decompose  ``{\sl string} s'' into 
bytes, then  ``Cstring'' should be defined as
\begin{eqnarray*}
\mbox{Cstring}&:& ({\tt CMO\_STRING},{\sl int32}\,  n, 
                  {\sl byte}\, \mbox{s[0]},
                  \ldots, {\sl byte}\ \mbox{s[n-1]})
\end{eqnarray*}

\noindent
Example:
\begin{center}
({\tt CMO\_INT32}, 1234)
\end{center}
Example:
\begin{center}
({\tt CMO\_STRING}, 5, "Hello")
\end{center}

\begin{verbatim}
#define     CMO_MONOMIAL32  19
#define     CMO_ZZ          20 
#define     CMO_QQ          21
#define     CMO_ZERO        22
#define     CMO_DMS          23   /* Distributed monomial system */
#define     CMO_DMS_GENERIC  24
#define     CMO_DMS_OF_N_VARIABLES  25
#define     CMO_RING_BY_NAME   26
\end{verbatim}
\bigbreak
\noindent
Group CMObject/Basic1 requires CMObject/Basic0. \\
ZZ, QQ, Zero $\in$ CMObject/Basic1. \\
\begin{eqnarray*}
\mbox{Zero} &:& ({\tt CMO\_ZERO}) \\ 
            & & \mbox{ --- universal zero } \\
\mbox{ZZ}         &:& ({\tt CMO\_ZZ},{\sl int32}\, f, {\sl byte}\, \mbox{a[1]}, \ldots
                           {\sl byte}\, \mbox{a[m]} ) \\
                 &:& \mbox{ --- bignum }\\
\mbox{QQ}        &:& ({\tt CMO\_QQ}, {\sl ZZ}\, a, {\sl ZZ}\, b) \\
                 & & \mbox{ --- rational number $a/b$. } \\
\end{eqnarray*}

\bigbreak
Let us define a group for distributed polynomials.

\medbreak
\noindent
Group CMObject/DistributedPolynomials requires CMObject/Basic0,
CMObject/Basic1. \\
Monomial, Monomial32, Coefficient, Dpolynomial, DringDefinition,
Generic DMS ring, RingByName, DMS of N variables $\in$ 
CMObject/DistributedPolynomials. \\
\begin{eqnarray*}
\mbox{Monomial} &:& \mbox{Monomial32}\, |\, \mbox{Zero} \\
\mbox{Monomial32}&:& ({\tt CMO\_MONOMIAL32}, {\sl int32}\, n,
                      {\sl int32}\, \mbox{e[1]}, \ldots,
                      {\sl int32}\, \mbox{e[n]}, \\
                 & & \ \mbox{Coefficient}) \\
                 & & \mbox{ --- e[i] is the exponent $e_i$ of the monomial 
                      $x^e = x_1^{e_1} \cdots x_n^{e_n}$. } \\
\mbox{Coefficient}&:& \mbox{ZZ} | \mbox{Integer32} \\
\mbox{Dpolynomial}&:& \mbox{Zero} \\
                 & & |\ ({\tt CMO\_LIST},{\sl int32} m, \\
                 & & \ \ {\tt CMO\_DMS}, \mbox{DringDefinition},
                    [\mbox{Monomial32}|\mbox{Zero}], \\
                 & &\ \ 
                    \{\mbox{Monomial32}\})  \\
                 & &\mbox{--- Distributed polynomial is a sum of monomials}\\
                 & &\mbox{--- m is equal to the number of monomials $+2$.}\\
\mbox{DringDefinition}
                 &:& \mbox{DMS of N variables} \\
                 & & |\ \mbox{RingByName} \\
                 & & |\ \mbox{Generic DMS ring} \\
                 & & \mbox{ --- definition of the ring of distributed polynomials. } \\
\mbox{Generic DMS ring}
                 &:& ({\tt CMO\_INT32, CMO\_DMS\_GENERIC}) \\
\mbox{RingByName}&:& ({\tt CMO\_RING\_BY\_NAME}, {\sl Cstring} s) \\
                 & & \mbox{ --- The ring definition refered by the name ``s''.} \\
\mbox{DMS of N variables}
                 &:& ({\tt CMO\_DMS\_OF\_N\_VARIABLES}, \\
                 & & \ ({\tt CMO\_LIST}, {\sl int32}\, \mbox{m},
                  {\sl Integer32}\,  \mbox{n}, {\sl Integer32}\,\mbox{p} \\
                 & & \ \ [,{\sl Cstring}\,\mbox{s}, {\sl List}\, \mbox{vlist},
                          {\sl List}\, \mbox{wvec}, {\sl List}\, \mbox{outord}]) \\
                 & & \mbox{ --- m is the number of elements.} \\
                 & & \mbox{ --- n is the number of variables, p is the characteristic} \\
                 & & \mbox{ --- s is the name of the ring, vlist is the list of variables.} \\
                 & & \mbox{ --- wvec is the weight vector.} \\
                 & & \mbox{ --- outord is the order of variables to output.} \\
\end{eqnarray*}


\subsection{ Stackmachine commands}

\begin{verbatim}
#define SM_popSerializedLocalObject 258
#define SM_popCMO 262
#define SM_popString 263 

#define SM_mathcap 264
#define SM_pops 265
#define SM_setName 266
#define SM_evalName 267 
#define SM_executeStringByLocalParser 268 
#define SM_executeFunction 269
#define SM_beginBlock  270
#define SM_endBlock    271
#define SM_shutdown    272


#define SM_control_kill 1024
#define SM_control_reset_connection  1030
\end{verbatim}


\subsection{OX messages}

Top level messages are OX messages.
These messages start with one of the following tags.

\begin{verbatim}
#define   OX_COMMAND         513        // for stackmachine commands
#define   OX_DATA            514        // for CMO 
#define   OX_SECURED_DATA    521

#define   OX_SYNC_BALL       515
\end{verbatim}

\noindent
Example:
\begin{center}
 (OX\_COMMAND, SM\_popCMO)
\end{center}

\noindent
Example:
\begin{center}
 (OX\_DATA, ({\tt CMO\_STRING}, 5, "Hello"))
\end{center}

*/



%% $OpenXM: OpenXM/doc/OpenXM-specs/library.tex,v 1.2 2000/01/24 00:57:11 noro Exp $
/*&jp
\section{ OX �����Ф��Ф��� C �饤�֥�ꥤ�󥿥ե����� }

������OX �����ФǤ� C �Υ饤�֥��Ȥ��ƥ�󥯤��ƻ��Ѥ����Ǥ���.
�饤�֥��Ȥ��ƻ��Ѥ��뤿��� C �δؿ���
Asir �� OX �������ѥ��饤����ȴؿ��˻������󥿥ե����������.

�饤�֥��ؿ��ˤ�, CMO �� binary encoding �����Ϥ�.
�饤�֥��ؿ������, CMO �� binary encoded form �����.

*/
/*&eg
\section{ OX servers as a C library}

In some OX servers, one can use the OX server as a C library.
The interface functions of the C library
are similar to Asir OX client functions such as
{\tt ox\_push\_cmo()}, {\tt ox\_pop\_cmo()}.

CMO should be converted into the binary encoded form to call these functions.
*/
/*&C

\medbreak
\begin{verbatim}
  int xxx_ox_init(int type)
\end{verbatim}
*/
/*&eg
  This function initializes the library interface.
  {\tt type} specifies the byte order to send int32 to the OX server xxx.
  If type is equal to 0, the network byte order will be used.
  If type is equal to 1, the little endian order will be used.
  In case of error, -1 will be returned.
*/
/*&jp
  ���δؿ��ϥ饤�֥�ꥤ�󥿥ե������ν������Ԥ�. 
  OX ������ xxx �� int32 �����뤿��� byte order �� type �ǻ��ꤹ��.
  type = 0 �ξ�� network byte order �����ꤵ���. 
  type = 1 �ξ�� little endian order �����ꤵ���. 
  ���Ԥ������, -1 ���᤹.
*/

/*&C

\smallskip
\begin{verbatim}
  void xxx_ox_push_cmo(void *cmo)
\end{verbatim}
*/
/*&eg
Push the binary encoded CMO {\tt cmo} onto the stack of the OX server xxx.
*/
/*&jp
Binary encoded ���줿 CMO ��, OX ������ xxx
�� stack �� push ����. 
*/
/*&C

\smallskip
\begin{verbatim}
  int xxx_ox_pop_cmo(void cmo, int limit)
\end{verbatim}
*/
/*&eg
Pop a binary encoded CMO from the OX server xxx
and write it at {\tt cmo}.
The return value is the size of the CMO in bytes.
In case of the stack underflow, the return value is 0.
If the size exceeds the {\tt limit}, -1 will be returned
and the CMO is not popped and will not be written to {\tt cmo}.
*/
/*&jp
Binary encoded ���줿 CMO (�������� size) ��,  OX �����Ф��
pop ����, cmo �˽񤭹���.
����ͤ� CMO �Υ������� byte ���᤹.
Stack underflow �ξ�������ͤ� 0 �Ǥ���.
������, limit ����礭���������� CMO �Ͻ񤭹��ޤ�ʤ�.
���ξ��, ����ͤ� -1 �Ȥʤ�.
*/

/*&C

\smallskip
\begin{verbatim}
  int xxx_ox_peek_cmo_size()
\end{verbatim}
*/
/*&eg
Return the size of the CMO at the top of the stack.
*/
/*&jp
stack �ΰ��־�ˤ��� CMO �Υ������� byte ���᤹.
*/

\smallbreak
\begin{verbatim}
 void xxx_ox_push_cmd(int cmd)
\end{verbatim}
/*&eg
This function sends a stack machine command 
({\tt OX\_COMMAND},int32 cmd) to a server.
*/
/*&jp
({\tt OX\_COMMAND},int32 cmd) �򥵡��Ф�����. 
*/

\begin{verbatim}
 void xxx_ox_execute_string(char *s) 
\end{verbatim}
/*&eg
These function requests a server to execute a string {\tt s}.
{\tt s} should be acceptable by the parser of the server.
*/
/*&jp
ʸ���� {\tt s} �򥵡��Ф˼¹Ԥ�����. {\tt s} �ϥ����ФΥѡ���
��������ǽ�ʤ�ΤǤʤ���Фʤ�ʤ�. 
*/


*/

/*&C

\begin{thebibliography}{99}
\bibitem{GKW}
Gray, S., Kajler, N. and Wang, P. S.,
Design and Implementation of MP, a Protocol for Efficient
  Exchange of Mathematical Expressions,
  {\sl Journal of Symbolic Computation}, 1996.
\bibitem{gap}
Linton, S. and Solomon, A.,
OpenMath, IAMC and {\tt GAP},
preprint, 1999.
\bibitem{asir}
Noro, M. et al.,
A Computer Algebra System {\tt Risa/Asir},  1993, 1995, 2000\\
\htmladdnormallink{ftp://archives.cs.ehime-u.ac.jp/pub/asir2000/}{ftp://archives.cs.ehime-u.ac.jp/pub/asir2000/}
\bibitem{openmath}  \htmladdnormallink{{\tt http://www.openmath.org}}{http://www.openmath.org}
\bibitem{openxxx}   \htmladdnormallink{{\tt http://www.math.kobe-u.ac.jp/OpenXM}/}{http://www.math.kobe-u.ac.jp/OpenXM/}  \rm
\bibitem{openasir-intro}  Ohara, Takayama, Noro: Introduction to Open Asir , 
1999, (in Japanese),
Suushiki-Shyori, Vol 7, No 2, 2--17. (ISBN4-87243-086-7, SEG , Tokyo).
\bibitem{kan}
Takayama, N.,
{\em Kan: A system for computation in
algebraic analysis,} 1991 version 1,
1994 version 2, the latest version is 2.991106.
\htmladdnormallink{ftp://ftp.math.kobe-u.ac.jp/pub/kan}{ftp://ftp.math.kobe-u.ac.jp/pub/kan}
\bibitem{phc}
Verschelde, J.,
PHCpack: A general-purpose solver for polynomial systems by
homotopy continuation.  ACM Transaction on Mathematical Softwares, 25(2)
251-276, 1999.
\bibitem{iamc}
Wang, P.,
Design and Protocol for Internet Accessible Mathematical Computation.
Technical Report ICM-199901-001, ICM/Kent State University, 1999.
\bibitem{xml}
XML
\htmladdnormallink{\tt http://www.w3c.org}{http://www.w3c.org}
\end{thebibliography}
*/

/*&C
\bigbreak
\bigbreak
\bigbreak

\small{
\noindent
    \rightline{ Masayuki Noro,}
    \rightline{ FUJITSU LABORATORIES LTD., Kawasaki, Japan; (by Aug., 2000)}
 \rightline{{\tt noryo@flab.fujitsu.co.jp}}
   \rightline{Current Address: Department of Mathematics, Kobe University, 
               Rokko, Kobe, 657-8501, Japan;}
\rightline{{\tt noro@math.kobe-u.ac.jp}}

\vskip .5cm

\noindent
    \rightline{ Nobuki Takayama,}
   \rightline{Department of Mathematics, Kobe University, 
               Rokko, Kobe, 657-8501, Japan;}
\rightline{{\tt takayama@math.kobe-u.ac.jp}}
}

\end{document}
*/
