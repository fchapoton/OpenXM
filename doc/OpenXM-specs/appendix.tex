%% $OpenXM$
//&C \section{Appendix: English translation}

/*&C
\noindent
(This section has not been updated.)

\subsection{Common Mathematical Object format}

\begin{verbatim}
#define LARGEID  0x7f000000
#define CMO_ERROR2 (LARGEID+2)
#define CMO_NULL   1
#define CMO_INT32  2
#define CMO_DATUM  3
#define CMO_STRING 4
#define CMO_LIST 17
\end{verbatim}

\bigbreak
\noindent
Group CMObject/Basic0  requires nothing. \\
Error2, Null, Integer32, Datum, Cstring, List $\in$ CMObject/Basic0. \\
Document of CMObject/Basic0 is at {\tt http://www.math.kobe-u.ac.jp/openxxx}
(in Japanese) \\
\begin{eqnarray*}
\mbox{Error2}&:& ({\tt CMO\_ERROR2}, {\sl CMObject}\, \mbox{ob}) \\
\mbox{Null}  &:& ({\tt CMO\_NULL}) \\
\mbox{Integer32}
             &:& ({\tt CMO\_INT32}, {\sl int32}\ \mbox{n}) \\
\mbox{Datum} &:& ({\tt CMO\_DATUM}, {\sl int32}\, \mbox{n}, {\sl byte}\, 
                 \mbox{data[0]}, 
                  \ldots , {\sl byte}\, \mbox{data[n-1]}) \\
\mbox{Cstring}&:& ({\tt CMO\_STRING},{\sl int32}\,  \mbox{n}, 
                   {\sl string}\, \mbox{s}) \\
\mbox{List} &:& 
\mbox{({\tt CMO\_LIST}, {\sl int32}\, m, {\sl CMObject}\, ob[1], $\ldots$,
                                       {\sl CMObject}\, ob[m])} \\
             & & \mbox{---  m is the length of the list.} \\
\end{eqnarray*}

In the definition of ``Cstring'', if we decompose  ``{\sl string} s'' into 
bytes, then  ``Cstring'' should be defined as
\begin{eqnarray*}
\mbox{Cstring}&:& ({\tt CMO\_STRING},{\sl int32}\,  n, 
                  {\sl byte}\, \mbox{s[0]},
                  \ldots, {\sl byte}\ \mbox{s[n-1]})
\end{eqnarray*}

\noindent
Example:
\begin{center}
({\tt CMO\_INT32}, 1234)
\end{center}
Example:
\begin{center}
({\tt CMO\_STRING}, 5, "Hello")
\end{center}

\begin{verbatim}
#define     CMO_MONOMIAL32  19
#define     CMO_ZZ          20 
#define     CMO_QQ          21
#define     CMO_ZERO        22
#define     CMO_DMS          23   /* Distributed monomial system */
#define     CMO_DMS_GENERIC  24
#define     CMO_DMS_OF_N_VARIABLES  25
#define     CMO_RING_BY_NAME   26
\end{verbatim}
\bigbreak
\noindent
Group CMObject/Basic1 requires CMObject/Basic0. \\
ZZ, QQ, Zero $\in$ CMObject/Basic1. \\
\begin{eqnarray*}
\mbox{Zero} &:& ({\tt CMO\_ZERO}) \\ 
            & & \mbox{ --- universal zero } \\
\mbox{ZZ}         &:& ({\tt CMO\_ZZ},{\sl int32}\, f, {\sl byte}\, \mbox{a[1]}, \ldots
                           {\sl byte}\, \mbox{a[m]} ) \\
                 &:& \mbox{ --- bignum }\\
\mbox{QQ}        &:& ({\tt CMO\_QQ}, {\sl ZZ}\, a, {\sl ZZ}\, b) \\
                 & & \mbox{ --- rational number $a/b$. } \\
\end{eqnarray*}

\bigbreak
Let us define a group for distributed polynomials.

\medbreak
\noindent
Group CMObject/DistributedPolynomials requires CMObject/Basic0,
CMObject/Basic1. \\
Monomial, Monomial32, Coefficient, Dpolynomial, DringDefinition,
Generic DMS ring, RingByName, DMS of N variables $\in$ 
CMObject/DistributedPolynomials. \\
\begin{eqnarray*}
\mbox{Monomial} &:& \mbox{Monomial32}\, |\, \mbox{Zero} \\
\mbox{Monomial32}&:& ({\tt CMO\_MONOMIAL32}, {\sl int32}\, n,
                      {\sl int32}\, \mbox{e[1]}, \ldots,
                      {\sl int32}\, \mbox{e[n]}, \\
                 & & \ \mbox{Coefficient}) \\
                 & & \mbox{ --- e[i] is the exponent $e_i$ of the monomial 
                      $x^e = x_1^{e_1} \cdots x_n^{e_n}$. } \\
\mbox{Coefficient}&:& \mbox{ZZ} | \mbox{Integer32} \\
\mbox{Dpolynomial}&:& \mbox{Zero} \\
                 & & |\ ({\tt CMO\_LIST},{\sl int32} m, \\
                 & & \ \ {\tt CMO\_DMS}, \mbox{DringDefinition},
                    [\mbox{Monomial32}|\mbox{Zero}], \\
                 & &\ \ 
                    \{\mbox{Monomial32}\})  \\
                 & &\mbox{--- Distributed polynomial is a sum of monomials}\\
                 & &\mbox{--- m is equal to the number of monomials $+2$.}\\
\mbox{DringDefinition}
                 &:& \mbox{DMS of N variables} \\
                 & & |\ \mbox{RingByName} \\
                 & & |\ \mbox{Generic DMS ring} \\
                 & & \mbox{ --- definition of the ring of distributed polynomials. } \\
\mbox{Generic DMS ring}
                 &:& ({\tt CMO\_INT32, CMO\_DMS\_GENERIC}) \\
\mbox{RingByName}&:& ({\tt CMO\_RING\_BY\_NAME}, {\sl Cstring} s) \\
                 & & \mbox{ --- The ring definition refered by the name ``s''.} \\
\mbox{DMS of N variables}
                 &:& ({\tt CMO\_DMS\_OF\_N\_VARIABLES}, \\
                 & & \ ({\tt CMO\_LIST}, {\sl int32}\, \mbox{m},
                  {\sl Integer32}\,  \mbox{n}, {\sl Integer32}\,\mbox{p} \\
                 & & \ \ [,{\sl Cstring}\,\mbox{s}, {\sl List}\, \mbox{vlist},
                          {\sl List}\, \mbox{wvec}, {\sl List}\, \mbox{outord}]) \\
                 & & \mbox{ --- m is the number of elements.} \\
                 & & \mbox{ --- n is the number of variables, p is the characteristic} \\
                 & & \mbox{ --- s is the name of the ring, vlist is the list of variables.} \\
                 & & \mbox{ --- wvec is the weight vector.} \\
                 & & \mbox{ --- outord is the order of variables to output.} \\
\end{eqnarray*}


\subsection{ Stackmachine commands}

\begin{verbatim}
#define SM_popSerializedLocalObject 258
#define SM_popCMO 262
#define SM_popString 263 

#define SM_mathcap 264
#define SM_pops 265
#define SM_setName 266
#define SM_evalName 267 
#define SM_executeStringByLocalParser 268 
#define SM_executeFunction 269
#define SM_beginBlock  270
#define SM_endBlock    271
#define SM_shutdown    272


#define SM_control_kill 1024
#define SM_control_reset_connection  1030
\end{verbatim}


\subsection{OX messages}

Top level messages are OX messages.
These messages start with one of the following tags.

\begin{verbatim}
#define   OX_COMMAND         513        // for stackmachine commands
#define   OX_DATA            514        // for CMO 
#define   OX_SECURED_DATA    521

#define   OX_SYNC_BALL       515
\end{verbatim}

\noindent
Example:
\begin{center}
 (OX\_COMMAND, SM\_popCMO)
\end{center}

\noindent
Example:
\begin{center}
 (OX\_DATA, ({\tt CMO\_STRING}, 5, "Hello"))
\end{center}

*/


